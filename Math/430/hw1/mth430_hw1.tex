\documentclass{article}
\usepackage{times}
\usepackage{amssymb, amsmath, amsthm}
\usepackage[margin=1in]{geometry}
\usepackage{graphicx}

\begin{document}

\title{MTH 430 Homework 1}
\author{Philip Warton}
\date{\today}
\maketitle

\section*{Problem 1}
Let $f:X\rightarrow Y$ be a function.
\subsection*{(a)}
Show that for all $A_1, A_2 \subset X, \ \ \ \ f(A_1 \cup A_2) = f(A_1) \cup f(A_2)$.
\begin{proof}
    Let $A_1, A_2 \subset X$.
    \\\\
    \fbox{$\subset$}
    Let $y \in Y$ such that $y \in f(A_1 \cup A_2)$.
    Then $\exists x \in A_1 \cup A_2$ such that $f(x) = y$.
    If $x \in A_1$, then $f(x) = y \in f(A_1) \subset f(A_1) \cup f(A_2)$.
    If $x \notin A_1$ then $x \in A_2$, and similarly it follows that $y \in f(A_1) \cup f(A_2)$.
    \\\\
    \fbox{$\supset$}
    Now, let $y \in Y$ such that $y \in f(A_1) \cup f(A_2)$.
    Then either $y \in f(A_1)$ or $y \in f(A_2)$.
    If $y \in f(A_1)$ then $\exists a_1 \in A_1$ such that $f(a_1) = y$.
    Thus, $a_1 \in A_1 \cup A_2$ and $f(a_1) = y \in f(A_1 \cup A_2)$
    Otherwise, $y \in f(A_2)$, and then $\exists a_2 \in A_2 : f(a_2) = y$, and thus $f(a_2) = y \in f(A_1 \cup A_2)$.
\end{proof}
\subsection*{(b)}
Show that for all $A_1, A_2 \subset X, \ \ \ \ f(A_1 \cap A_2) \subset f(A_1) \cap f(A_2)$.
\begin{proof}
    Let $A_1, A_2 \subset X$ be arbitrary.
    Let $y \in Y$ such that $y \in f(A_1 \cap A_2)$.
    Then, there exists $a \in A_1 \cap A_2$ such that $f(a) = y$.
    Since $a \in A_1, f(a) \in f(A_1)$, and similarly $f(a) \in f(A_2)$.
    Thus $f(a) = y \in f(A_1) \cap f(A_2)$.
\end{proof}

\section*{Problem 2}
\subsection*{(a)}
Show that for all $B_1, B_2 \in Y, \ \ \ \ f^{-1}(B_1 \cup B_2) = f^{-1}(B_1) \cup f^{-1}(B_2)$
\begin{proof}
    Let $B_1, B_2 \subset Y$.
    \\\\
    \fbox{$\subset$}
    Let $x \in X$ such that $x \in f^{-1}(B_1 \cup B_2)$.
    Then $f(x) \in B_1 \cup B_2$.
    If $f(x) \in B_1$ then $x \in f^{-1}(B_1) \subset f^{-1}(B_1) \cup f^{-1}(B_2)$.
    Otherwise, $f(x) \in B_2$ thus $x \in f^{-1}(B_2) \subset f^{-1}(B_1) \cup f^{-1}(B_2)$.
    \\\\
    \fbox{$\supset$}
    Let $x \in X$ such that $x \in f^{-1}(B_1) \cup f^{-1}(B_2)$.
    If $x \in f^{-1}(B_1)$, then $f(x) \in B_1 \subset B_1 \cup B_2$, thus $x \in f^{-1}(B_1 \cup B_2)$.
    Otherwise, $x \in f^{-1}(B_2)$, and it follows that $f(x) \in B_2 \subset B_1 \cup B_2$ so $x \in f^{-1}(B_1 \cup B_2)$.
\end{proof}
\subsection*{(b)}
Show that for all $B_1, B_2 \in Y, \ \ \ \ f^{-1}(B_1 \cap B_2) = f^{-1}(B_1) \cap f^{-1}(B_2)$
\begin{proof}
    Let $B_1, B_2 \subset Y$.
    \\\\
    \fbox{$\subset$}
    Let $x \in X$ such that $x \in f^{-1}(B_1 \cap B_2)$.
    Then $f(x) \in B_1 \cap B_2$, thus $f(x) \in B_1$ and $f(x) \in B_2$.
    Since $f(x) \in B_1, x \in f^{-1}(B_1)$, and similarly $x \in f^{-1}(B_2)$.
    Therefore $x \in f^{-1}(B_1) \cap f^{-1}(B_2)$.
    \\\\
    \fbox{$\supset$}
    Let $x \in X$ such that $x \in f^{-1}(B_1) \cap f^{-1}(B_2)$.
    Since $x \in f^{-1}(B_1), f(x) \in B_1$, and since $x \in f^{-1}(B_2), f(x) \in B_2$.
    Since $f(x) \in B_1$ and $f(x) \in B_2$ and thus $f(x) \in B_1 \cap B_2$, it follows that $x \in f^{-1}(B_1 \cap B_2)$.
\end{proof}
\section*{Problem 3}
\subsection*{(b)}
We wish to show that $(a)$ and $(b)$ are equivalent.
\begin{proof}
    Let $A \subset X$ be arbitrary.
    \\\\
    "$\Rightarrow$" Assume $f$ is injective.
    Let $a \in A$, then $f(a) \in f(A)$.
    Since $f(a) \in f(A)$, by definition $a \in f^{-1}(f(A))$.
    Thus $\forall a \in A, a \in f^{-1}(f(A))$ and we say $A \subset f^{-1}(f(A))$.
    Now let $a \in f^{-1}(f(A))$ be arbitrary, then $f(a) \in f(A)$.
    Since $f(a) \in f(A)$, then $\exists a_0 \in A$ such that $f(a_0) = f(a)$.
    We know that $f$ is injective therefore $a = a_0 \in A$.
    Thus $f^{-1}(f(A)) \subset A$, and $f^{-1}(f(A)) \supset A$, so $f^{-1}(f(A)) = A$.
    \\\\
    "$\Leftarrow$" Assume that $f^{-1}(f(A)) = A \ \ \forall A \subset X$.
    Let $a,b \in X$ such that $f(a) = f(b)$ and let $A = \{a\}$.
    Then $f(A) = \{f(a)\}$ and since $f(a) = f(b)$ it follows that $f(b) \in f^{-1}(f(A))$.
    Therefore by our assumption that $f^{-1}(f(A))=A$, we have $b \in A$, and thus $b=a$.
\end{proof}

\subsection*{(c)}
We wish to show that $(a)$ and $(c)$ are equivalent.
\begin{proof}
    A function $f$ is injective if and only if $f(A \cap B) = f(A) \cap f(B)$.
    \\\\
    "$\Rightarrow$" Assume that $f$ is injective.
    We wish to show that $f(A \cap B) = f(A) \cap f(B)$.
    Let $y \in f(A \cap B)$, then $\exists x \in A \cap B$ such that $f(x) = y$.
    Since $x \in A, f(x) = y \in f(A)$.
    Similarly $y \in f(B)$, thus $y \in f(A) \cap f(B)$, and we say $f(A \cap B) \subset f(A) \cap f(B)$.
    \\\\
    Now let $y \in f(A) \cap f(B)$, then $\exists x_1 \in A : f(x_1) = y$.
    Similarly $\exists x_2 \in B : f(x_2) = y$.
    Since $f$ is an injection we can say $x_1 = x_2 = x$.
    Thus $x \in A$ and $x \in B$ so $x \in A \cap B$ and it follows that $y = f(x) \in f(A \cap B)$.
    \\\\
    "$\Leftarrow$" Assume that $f(A \cap B) = f(A) \cap f(B) \ \ \ \ \forall A,B \subset X$.
    Let $a,b \in X$ such that $f(a) = y = f(b)$.
    Let $A = \{a\}$ and $B = \{b\}$, then $f(A) = \{y\} = f(B) = f(A) \cap f(B) = f(A \cap B)$.
    Since $y \in f(A \cap B)$, then there must exist some $x \in A \cap B$ such that $f(x) = y$.
    Therefore $x \in \{a\} \cap \{b\}$ and $a = x = b$.
\end{proof}

\subsection*{(d)}
We wish to show that $(c)$ and $(d)$ are equivalent.
\begin{proof}
    We want to show $f(A \cap B) = f(A) \cap f(B)$ if and only if $A \cap B = \O \Rightarrow f(A) \cap f(B) = \O$.
    \\\\
    "$\Rightarrow$" Assume that for all $A,B \subset X$ that $f(A \cap B) = f(A) \cap f(B)$.
    Let $A,B \subset X$ such that $A \cap B = \O$.
    Then $f(A \cap B) = \O = f(A) \cap f(B)$, and the desired implication holds.
    \\\\
    "$\Leftarrow$" Assume that for all $A,B\subset X$ that $A \cap B = \O \Rightarrow f(A) \cap f(B) = \O$.
    Let $a \in \{a\} = A \subset X$ and $b \in \{b\} = B \subset X$, and suppose $a \neq b$.
    Then $A \cap B = \O = f(A) \cap f(B)$, which means that $f(a) \neq f(b)$.
    Since $a \neq b \Rightarrow f(a) \neq f(b)$, it follows that $f$ is injective, which is equivalent to $f(A \cap B) = f(A) \cap f(B)$ for all $A,B \subset X$.
\end{proof}

\subsection*{(e)}
We wish to show that $f$ is injective if and only if $\forall B\subset A \subset X, f(A \setminus B) = f(A) \setminus f(B)$.
\begin{proof}
    "$\Rightarrow$" Assume that $f$ is injective. Let $B \subset A \subset X$ be arbitrary.
    We want to show that $f(A \setminus B) = f(A) \setminus f(B)$.
    \\\\
    \fbox{$\subset$}
    Let $y \in Y : y \in f(A \setminus B)$.
    Since $f$ is injective, there exists a unique $x \in A \setminus B$ such that $f(x) = y$.
    Since $x \in A, f(x) = y \in f(A)$.
    We know that for all $b \in B, f(b) \neq y$, because our unique solution $x \notin B$.
    Since $\nexists b \in B : f(b) = y, y \notin f(B)$.
    Then with $y \in f(A)$ and $y \notin f(B)$, $y \in f(A) \setminus f(B)$.
    \\\\
    \fbox{$\supset$}
    Let $y \in Y$ such that $y \in f(A) \setminus f(B)$.
    Then $\nexists b \in B : f(b) = y$, and $\exists x \in A : f(x) = y$.
    It follows that if $x \in A$ and $f(x) = y$ that $x \in A \setminus B$.
    Therefore $f(x) = y \in f(A \setminus B)$.
    \\\\
    "$\Leftarrow$" Assume that for all $B \subset A \subset X \ \ \ \ f(A \setminus B) = f(A) \setminus f(B)$.
    Let $a, b \in X$ such that $f(a) = y = f(b)$.
    We want to show that $a = b$.
    Suppose by contradiction that $a \neq b$.
    Let $B = \{b\} \subset A = \{a, b\} \subset X$.
    Then $A \setminus B= \{a\}$, and then $f(A \setminus B) = \{f(a)\} = \{y\}$.
    However, we also know that $f(A) = \{y\}$ and $f(B) = \{y\}$ so then $f(A) \setminus f(B) = \O$.
    By assumption $f(A \setminus B) = f(A) \setminus f(B)$, therefore $\{y\} = \O$ (contradiction).
    It must then be the case that $a = b$.
\end{proof}

\section*{Problem 4}
\begin{proof}
    To show that this set, denote $\tau$ is a topology on $X$, we must check 3 things.
    \\\\
    \fbox{$(i)$ : $\O$ and $x \in \tau$}
    By construction we know that $\O \in \tau$.
    Since $\forall p \in X, \exists B \in \beta : p \in B$, the union of all $B \in \beta$ must be equal to $X$, thus $X \in \beta$.
    \\\\
    \fbox{$(ii)$ : $\forall T_1, T_2 \in \tau, T_1 \cup T_2 \in \tau$}
    Let $T_1, T_2 \in \tau$ be arbitrary.
    If $T_1 = \O$ or $T_2 = \O$, the $T_1 \cup T_2 \in \tau$ trivially.
    Otherwise, we can denote
    \[ T_1 = B_{1_1} \cup B_{1_2} \cup \cdots, \ \ \ \ \ \ T_2 = B_{2_1} \cup B_{2_2} \cup \cdots \]
    Then $T_1 \cup T_2 = B_{1_1} \cup B_{2_1} \cup B_{2_1} \cup B_{2_2} \cup \cdots$ which will be an element of $\tau$.
    \\\\
    \fbox{$(iii)$ : Any intersection of a finite subcollection of members of $\tau$ is in $\tau$}
    We want to show that $\forall T_1, T_2 , \cdots T_k \in \tau, \ \ \ \ \ \ T_1 \cap T_2 \cap \cdots \cap T_k \in \tau$.
    If $T_1 \cap T_2 \in \tau$ it follows that any finite intersection $\bigcap_{i = 1}^kT_i \in \tau$.
    So let $T_1, T_2 \in \tau$ be arbitrary, we wish to show that $T_1 \cap T_2 \in \tau$.
    If either $T_1 = \O$ or $T_2 = \O$, the intersection is empty and thus in $\tau$.
    Otherwise, we can denote
    \begin{align*}
        T_1 \cap T_2 
        & = (B_{1_1} \cup B_{1_2} \cup B_{1_3} \cdots ) \cap (B_{2_1} \cup B_{2_2} \cup B_{2_3} \cdots ) \\ 
        & = \bigcup_{i \in \mathbb{N}} \bigcup_{j \in \mathbb{N}}(B_{1_i} \cap B_{2_j})
    \end{align*}
    Thus if for all $B_1, B_2 \in \beta, B_1 \cap B_2 \in \tau$, then by $(ii)$ their union will be in $\tau$, and we are done.
    Let $B_1, B_2 \in \beta$ arbitrarily.
    Then, since for all $p \in B_1 \cap B_2$, there exists $B_p \subset B_1 \cap B_2 : p \in B_p$, it follows that
    \[ \bigcup_{p \in B_1 \cap B_2}B_p \ \ = \ \  B_1 \cap B_2\]
    Since this is a union of elements of $\beta$, we say that $B_1 \cap B_2 \in \tau$, thus any $T_1 \cap T_2 \in \tau$.
    It follows from logic before that there intersection any finite subcollection of in $\tau$ will be a member of $\tau$.
\end{proof}

\section*{Problem 5}
\subsection*{(a)}
The function $f$ is continuous.
Let $O$ be an arbitrary open set in $\mathbb{R}_{\tau} : O = \{ [a, b) | a < b\}$.
Then we write
\begin{align*}
    f^{-1}(O)
    & = \{ x \in \mathbb{R} | f(x) \in O \} \\
    & = \{ x \in \mathbb{R} | a \leqslant f(x) < b \} \\
    & = \{ x \in \mathbb{R} | a \leqslant 2x - 5 < b \} \\
    & = \left\{ x \in \mathbb{R} \bigg| \frac{a + 5}{2} \leqslant x < \frac{a+5}{2}\right\}
\end{align*}
Choose $a_0 = \dfrac{a + 5}{2}$ and $b_0 = \dfrac{a+5}{2}$, and then $f^{-1}(O)$ is in $\mathbb{R}_{\tau}$, thus $f$ is continuous.
\subsection*{(b)}
The function $f(x) = -x$ is not continuous.
\\\\
\fbox{Counterexample}
Let $O = \{[0,1)\}$.
Then, $f^{-1}(O) = \{x \in \mathbb{R} | x \in O\} = \{ (-1, 0]\}$.
Since the interval is open on the left and closed on the right, it is not in $\mathbb{R}_{\tau}$.

\subsection*{(c)}
The function $f(x) = x^2$ is not continuous.
\\\\
\fbox{Counterexample} Let $O = \{[0, 1)\}$.
Then, $f^{-1}(O) = \{x \in \mathbb{R} | x \in O\} = \{x \in \mathbb{R} | x^2 \in O\}$.
For $x^2 \in O$, we must have $0 \leqslant x^2 < 1$, which holds for any $x$ in $(-1,1)$.
Since this interval is open on the left, it is not in $\mathbb{R}_{\tau}$, therefore $f$ is not continuous.

\end{document}