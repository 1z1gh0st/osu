\documentclass{article}

\usepackage{times}
\usepackage{amssymb, amsmath, amsthm}
\usepackage[margin=1in]{geometry}
\usepackage{graphicx}

\begin{document}

\title{MTH 430 Homework 2}
\author{Philip Warton}
\date{\today}
\maketitle

\section*{Problem 1}
$X = \mathbb{R} \cup \{p\}$
\\
$\beta = \{ \text{open intervals in $\mathbb{R}$ and neighborhoods of p} \}$
\\
$V(p) = (a, 0) \cup \{p \} \cup (0, b) : a < 0 < b$
\subsection*{(a)}

Show that $\beta$ is a basis for a topology.
\begin{proof}
    To prove that $\beta$ is a basis, there are two requirements.
    \begin{align*}
        (\text{i}) & \ \ \ \ \forall x \in X \ \ \ \ \exists B \in \beta : x \in B \\
        (\text{ii}) & \ \ \ \ \forall B_1,B_2 \in \beta, \forall x \in B_1 \cap B_2 \ \ \exists B \in \beta : x \in B \subset B_1 \cap B_2
    \end{align*}
    We want to show that $\forall sx \in X, \exists B \in \beta : x \in B$.
    Let $x \in \mathbb{R}$ be arbitrary.
    If $x = p$, then any neighborhood of $p$ is automatically in $\beta$.
    Otherwise, $x \neq p \Longrightarrow x \in \mathbb{R}$.
    Choose a real open interval $(a,b) \subset \mathbb{R} : a < x < b.$
    \\\\
    Now let $B_1, B_2 \in \beta$.
    We want to show that $\forall x \in B_1 \cap B_2, \ \ \exists B \in \beta$ such that $x \in B \subset B_1 \cap B_2$.
    Let $x \in B_1 \cap B_2$.
    If $x = p$, then $p \in B_1$ and $p \in B_2$, and we can write
    \[
        B_1 = (a_1, 0) \cup \{p\} \cup (0, b_1) \ \ \ \ \ \ \ 
        B_2 = (a_2, 0) \cup \{p\} \cup (0, b_2)
    \]
    Then let $a = \max \{a_1,a_2\}$ and $b = \min \{b_1,b_2\}$ and we have a neighborhood around $p$, $B = (a,0) \cup \{p\} \cup (0,b)$ where $B \subset B_1 \cap B_2$.
    \\\\
    If $x \neq p$, then $x \in \mathbb{R}$.
    Denote our intervals as
    \[
        B_1 = (a_1, b_1) \ \ \ \ \ \ \ B_2 = (a_2, b_2)
    \]
    Take $a$ and $b$ as done previously and we have $a \geqslant a_1,a_2$ and $b \leqslant b_1,b_2$, which gives us $(a,b) = B \subset B_1$ and $B \subset B_2$ so then $B \subset B_1 \cap B_2$.
\end{proof}
\subsection*{(b)}
Show that $\forall U,V \subset X$ such that $0 \in U, p \in V$ and both sets are open, that $U \cap V \neq \O$.
\begin{proof}
    Let $U \subset X$ be an open set containing $0$.
    Let $V \subset X$ be an open set containing $p$.
    We say $\exists U' \subset U : 0 \in U'$ where $U'$ is of the form 
    \[
        U' = (-a,a) = (-a,0) \cup \{0\} \cup (0,a) : a > 0
    \]
    Similarly there exists $V' \subset V$ such that
    \[
        V' = (-b,0) \cup \{p\} \cup (0,b) : b > 0
    \]
    If such subsets do not exist, then either $U$ or $V$ is not an open set containing the point $0$ or $p$ respectively.
    Let $c = \min \{a,b\}$ then $V' \cap U' = (-c, 0) \cup (0,c)$ and then $\O \neq V' \cap U' \subset V \cap U$.
\end{proof}
\subsection*{(c)}
Show that $\mathbb{Q} \subset \mathbb{R}$ is dense in $X$.
\begin{proof}
    We say that $\mathbb{Q}$ is dense in $X$ if $\overline{\mathbb{Q}} = X$.
    Note that $\overline{\mathbb{Q}} = \mathbb{Q} \cup \mathbb{Q}'$ where $\mathbb{Q}' = \{\text{all limit points of } Q\}$.
    We know that each irrational is a limit point of $\mathbb{Q}$, and so it is in the closure of $\mathbb{Q}$.
    This leaves us with only the point $p$.
    Let $V(p)$ be a neighborhood of $p$ arbitrarily.
    We say $V(p) = (a,0) \cup \{p\} \cup (0,b)$.
    Since there exists a rational number between any two real numbers, $\exists c \in \mathbb{Q} : a < c < 0$.
    Therefore $V(p) \cap \mathbb{Q} \neq \O$ and so the set $\mathbb{Q}$ is dense in $X$.
\end{proof}
\subsection*{(d)}
Let $f: X \rightarrow \mathbb{R}$ be a function such that $f(x) = x$ if $x \in \mathbb{R}$, otherwise $f(x) = 0$.
Show that $f$ is continuous on $X$.
\begin{proof}
    Let $O \subset \mathbb{R}$ be and open set in $\mathbb{R}$.
    We want to show that $f^{-1}(O)$ is open in $X$.
    \\\\
    \fbox{Case 1: $0 \in O$} If $0 \in O$ then $f^{-1}(O) = O \cup \{p\}$.
    Since $O$ is an open set containing $0$, there exists $a \in mathbb{R}$ and $A_1, A_2 \subset O$ such that 
    \[
        A_1 = (-a, 0), \ \ \ \ \ \ A_2 = (0, a)
    \]
    Then since $O= O \cup A_1 \cup A_2$, we can rewrite 
    \[f^{-1}(O) = O \cup \{p\} = (O \cup A_1 \cup A_2) \cup \{p\} = O \cup (A_1 \cup \{p\} \cup A_2)\]
    Then we have the union of the open set $O$ and the $a$-neighborhood of $p$.
    Since the union of two open sets is open, $f^{-1}(O)$ is open in $X$.
    \\\\
    \fbox{Case 2: $0 \notin O$} If $0 \notin O$, then it follows that $f^{-1}(O) = O$, and is open in $X$.
\end{proof}
\section*{Problem 2}
\subsection*{(a)}
Let $T_1, T_2$ be topologies on $X$, show that the intersection $T_1 \cap T_2$ is also a topology on $X$.
\begin{proof}
    We want to show that $\O, X \in T_1 \cap T_2$.
    Since $T_1, T_2$ are both topologies $\O \in T_1$ and $\O \in T_2$, so $\O \in T_1 \cap T_2$.
    Similarly, since both are topologies on $X$, the set $X$ must be in both, and we say $X \in T_1 \cap T_2$.s
    \\\\
    Now we want to show that $\bigcup_{\alpha \in A}O_\alpha \in T_1 \cap T_2$.
    We know that for each open set in our intersection, $O_\alpha \in T_1$ and $O_\alpha \in T_2$.
    Since $T_1$ is a topology on $X$, we know that any union of sets in $T_1$ will also be in $T_1$, so $\bigcup_{\alpha \in A}O_\alpha \in T_1$.
    Similarly, all open sets in our intersection are sets in $T_2$, and we can say that $\bigcup_{\alpha \in A}O_\alpha \in T_2$.
    If this arbitrary union is a set in both $T_1$ and $T_2$ it must be the case that $\bigcup_{\alpha \in A}O_\alpha \in T_1 \cap T_2$.
    \\\\
    Finally we must show that any finite intersection of sets in $T_1 \cap T_2$ is also a set in $T_1 \cap T_2$.
    The argument is very similar to that regarding unions.
    For every $\alpha \in A$, $O_\alpha \in T_1$.
    Since $T_1$ is a topology on $X$, any finite intersection of sets in $T_1$ is also in $T_1$, so $\bigcap_{k=1}^n O_k \in T_1$.
    Similarly we say $\bigcap_{k=1}^n O_k \in T_2$.
    With our finite intersection being a set in both $T_1$ and $T_2$, it must be in the intersection $T_1 \cap T_2$.
\end{proof}
\subsection*{(b)}
This argument is similar to that of the construction of the closure of a set.
Take the collection of all topologies that contain the collection $\beta$, if we take the intersection of all of these, then we have a smallest topology.
This is the smallest topology because for every toplogy $\tau_0$ containing $\beta, \tau \subset \tau_0$.
As we have shown in part $(a)$, the intersection of two topologies on $X$ is also a topology on $X$, and for this reason the arugment holds.
\subsection*{(c)}
\begin{proof}
    Let $A = \{\text{every finite intersection of sets in $\beta$, and every set in $\beta$}\}$.
    Then let $\tau = \{\text{every union of sets in $A$}\} \cup \{\O\}$.
    From part $(b)$ we know that this is the smallest topology on $X$ containing $\beta$ if $\tau = $ the intersection of all topologies containing $\beta$.
    If we can show that for every topology containing $\beta, \tau$ is a subset of it, then it is the smallest one.
    \\\\
    \fbox{$\tau$ is a topology on $X$}
    By construction we have $\O$ and $\bigcup_{i \in I} B_i = X \in \tau$.
    Since our topology is constructed only from unions of sets in $A$, it follows that unions of sets in $\tau$
    will simply be unions of sets in $A$ which are in $\tau$ by construction.
    For intersections, let $G,H \in \tau$.
    Then $\exists G_0 \subset A$ and $H_0 \subset A$ such that
    \[
        G = \left\{\bigcup G_0 \right\} \ \ \ \ \ \ H = \left\{\bigcup H_0\right\}
    \]
    Then their intersection can be written as 
    \[
        G \cap H = \left\{\bigcup G_0\right\} \cap \left\{\bigcup H_0\right\} = \bigcup \left(G_0 \cap H_0\right)
    \]
    Recall that $G_0, H_0 \subset A$, which means that they are finite intersections of sets in $\beta$.
    For this reason, their intersection is also a finite intersection and $G_0 \cap H_0 = \{ B_{1_g} \cap B_{2_g} \cap \cdots \cap B_{n_g} \} \cap \{B_{1_h} \cap B_{2_h} \cap \cdots \cap B_{m_h}\}$.
    Thus $G_0 \cap H_0 \subset A$, and so the union of elements in this intersection will be in the topology by construction.
    It follows that by induction any finite intersection will be in $\tau$.
\\\\
    \fbox{$\tau$ is the smallest topology containing $\beta$}
    Since topologies must be closed under finite intersections, the smallest one must contain every set in $A$.
    Since topologies must also be closed under infinite unions, any topology containing $\beta$ must contain at least every element in $\tau$, and we conclude that
    for any topology $\tau_0$ such that $B \in \beta \Rightarrow B \in \tau_0$, $\tau \subset \tau_0$.
\end{proof}
\section*{Problem 3}
\subsection*{(a)}
Let $X = \mathbb{R}$ with $\tau$ consisting of all subsets $B \subset \mathbb{R}$ such that $B^c$ contains finitely many elements or $B^c = \mathbb{R}$.
Show that $\tau$ is a topology on $X$.
\begin{proof}
    First we want to show that the empty set and the entire set are in the topology.
    Since $\O \subset X$, and $\O^c = \mathbb{R}, \O \in \tau$.
    For $X$ we have $X \subset X$, and also $X^c = \O$, and we say that the empty set has finitely many elements therefore $X \in \tau$.
    \\\\
    Now we wish to prove that any union of sets in $\tau$ is also a set in $\tau$.
    For any set $B \in \tau$, we say there is some natural number $n$ for which $B^c = \{x_1, x_2, \cdots, x_n\}$.
    Take the complement of the union $\left(\bigcup_{i\in I}B_i\right)^c$ and if it has finitely many elements, the union must be a set in $\tau$.
    Let $x \in \left(\bigcup_{i\in I}B_i\right)^c$.
    Then $\forall i \in I, x \notin B_i$ which is equivalent to saying $\forall i \in I, x \in B_i^c$.
    Then since $x$ is in the complement of each $B_i$, $x \in \bigcap_{i \in I}B_i^c$.
    This gives us the result $\left(\bigcup_{i\in I}B_i\right)^c \subset \bigcap_{i \in I}B_i^c$ (both sets are in fact equal but we need not prove this here).
    Since each $B_i^c$ has finitely many terms, it follows logically that $\bigcap_{i \in I}B_i^c$ must have at most finitely many terms.
    Since $\left(\bigcup_{i\in I}B_i\right)^c \subset \bigcap_{i \in I}B_i^c$ and the superset has finitely many terms, the subset must also have finitely many terms.
    With the complement having finitely many terms, $\bigcup_{i\in I}B_i \in \tau$.
    \\\\
    Finally we must show that any finite intersection of sets in $\tau$ is also in $\tau$.
    Take the complement of the union $\left(\bigcap_{k=1}^nB_k\right)^c$.
    Let $x \in \left(\bigcap_{k=1}^nB_k\right)^c$.
    Then $x$ is not in every $B_k$, which means that $\exists k : x \in B_k^c$.
    Since there is some set where $x$ is in the complement, we say that $x \in \bigcup_{k=1}^n B_k^c$.
    This gives us the result $\left(\bigcap_{k=1}^nB_k\right)^c \subset \bigcup_{k=1}^n B_k^c$.
    Now we have a finite union from $1$ to $n$ of sets with finitely many elements (the complements of $B \in \tau$).
    It follows that this union must have finitely many elements, and since it is a superset of the complement of $\bigcap_{k=1}^nB_k$, the complement of said set has finitely many elements so the set is in $\tau$.
\end{proof}
\subsection*{(b)}
Let $X = \{a,b,c\}$ and $\tau = \{\O, \{c\}, \{a,c\}, \{b,c\}, \{a,b,c\}\}$.
Show that $\tau$ is a topology on $X$.
\begin{proof}
    We know that $\O, X \in \tau$ by construction.
    Let us check if the union of each pair of nonempty sets in $\tau$ is also in $\tau$.
    \begin{align*}
        \{c\}\cup \{a,c\} & = \{a,c\} \\
        \{c\} \cup \{b,c\} & = \{b,c\} \\
        \{c\} \cup \{a,b,c\} & = \{a,b,c\} \\
        \{a,c\} \cup \{b,c\} & = \{a,b,c\} \\
        \{a,c\} \cup \{a,b,c\} & = \{a,b,c\} \\
        \{b,c\} \cup \{a,b,c\} & = \{a,b,c\}      
    \end{align*}
    Any finite union can be broken up by associativity to unions of pairs of sets and every pair of sets in $\tau$ has a union in $\tau$ therefore any finite union of sets in $\tau$ is a set in $\tau$.
    \\\\
    We make a similar argument for a finite intersection of sets in $\tau$, and we look at the intersection of every pair of non-empty sets.
    \begin{align*}
        \{c\}\cap \{a,c\} & = \{c\} \\
        \{c\} \cap \{b,c\} & = \{c\} \\
        \{c\} \cap \{a,b,c\} & = \{c\} \\
        \{a,c\} \cap \{b,c\} & = \{c\} \\
        \{a,c\} \cap \{a,b,c\} & = \{a,c\} \\
        \{b,c\} \cap \{a,b,c\} & = \{b,c\}
    \end{align*}
    And since all pairs result in sets that belong to $\tau$ we say that any finite intersection of sets in $\tau$ belongs to $\tau$.
    No infinite cases exist since our topology is a finite set.
\end{proof}
\section*{Problem 4}
Show that $\overline{A} = A^\circ \cup A^b$.
\begin{proof}
    \fbox{$A^\circ \cup A^b \subset \overline{A}$} If a point $p$ is in $A^\circ \subset A \subset \overline{A}$, then it is in $\overline{A}$.
    Otherwise, suppose $p \in A^b$.
    Assume by contradiction that the closed set $\overline{A}$ did not contain $p$.
    Then the complement of $\overline{A}$ would both contain $p$.
    Since $(\overline{A})^c$ is the complement of a closed set, it is open.
    Since $A \subset \overline{A}$ we say that $(\overline{A})^c \cap A = \O$.
    Therefore there exists a nieghborhood of $p$ that does not intersect $A$, and $p$ is not a boundary point (contradiciton).
    So $p \in A^b \Longrightarrow p \in \overline{A}$.
    \\\\
    \fbox{$\overline{A}\subset A^\circ \cup A^b$}
    Let $p \in \overline{A}$.
    Suppose by contradiction that $p \notin A^\circ \cup A^b$.
    Then both $p \notin A$ and $p$ is not a limit point of $A$.
    Therefore there exists a neighborhood of $p$, $O(p)$ that does not intersect $A$.
    The complement of $O(p)$ would then be a closed set containing $A$ that did not contain $p \Longrightarrow p \notin \overline{A}$ (contradiction).
\end{proof}
\section*{Problem 5}
Show that $Cl(Int(Cl(Int(A)))) = Cl(Int(A))$.
\begin{proof}
    Note that $Cl(A)$ is closed set, so it must be equal to its closure, i.e $Cl(A) = Cl(Cl(A))$.
    Similarly an interior $Int(A)$ is an open set, and therefore must be equal to its interior which means $Int(A) = Int(Int(A))$.
    Also recall that $Int(A) \subset A \subset Cl(A)$.
    \\\\
    \fbox{$\subset$} Let $A''$ be a set.
    We can say 
    \begin{align*}
        Int(A'') & \subset A'' \\
        Cl(Int(A'')) & \subset Cl(A'')
    \end{align*}
    Now let $A'' = Cl(A')$ and it follows that
    \[        Cl(Int(Cl(A'))) \subset Cl(Cl(A')) = Cl(A')
    \]
    Finally let $A' = Int(A)$ and we get
    \[
        Cl(Int(Cl(Int(A)))) \subset Cl(Int(A))
    \]
    \fbox{$\supset$}
    \begin{align*}
        A' & \subset Cl(A') \\
        \Longrightarrow Int(A') & \subset Int(Cl(A'))
    \end{align*}
    Let $A' = Int(A)$, then
    \begin{align*}
        Int(Int(A)) = Int(A) & \subset Int(Cl(Int(A)))\\
        Cl(Int(A)) & \subset Cl(Int(Cl(Int(A))))
    \end{align*}
\end{proof}
\end{document}