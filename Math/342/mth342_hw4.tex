\documentclass{article}

\usepackage{times}
\usepackage{amssymb, amsmath, amsthm}
\usepackage[margin=1in]{geometry}

\begin{document}

\title{Mth 342 Homework 4}
\author{Philip Warton}
\date{\today}
\maketitle

\section*{1.}
	\subsection*{a.}
		To get the transformation matrix $[T]_B$ let us take the image of each basis vector and write it in the coordinates of $B$. We have 
		\begin{align*}
			T(1) & = 0 = \begin{bmatrix}0\\0\\0\end{bmatrix}_B \\
			T(x) & = 2x = \begin{bmatrix}0\\2\\0\end{bmatrix}_B \\
			T(x^2) & = 8x^2+4x = \begin{bmatrix}0\\4\\8\end{bmatrix}_B \\
		\end{align*}
		Putting these together in a matrix, we get
		\[ [T]_B =  \begin{bmatrix}0&0&0\\0&2&4\\0&0&8\end{bmatrix} \]

	\subsection*{b.}
		To get the characteristic polynomial we take the determinant of the matrix minus some multiple of the identity $\lambda I_3$.
		\begin{align*}
		det([T]_B - \lambda I_3) & = det( \begin{bmatrix}0-\lambda&0&0\\0&2-\lambda&4\\0&0&8-\lambda\end{bmatrix} ) \\
		& = -\lambda(2-\lambda)(8-\lambda) \\ 
		 \Longrightarrow & \lambda_1 = 0, \\
 		& \lambda_2 = 2, \\
		& \lambda_3 = 8
		\end{align*}
		Thus our eigenvalues are $\lambda_1, \lambda_2,$ and $\lambda_3$.
	
	\subsection*{c.}
		Let us take the null space of our matrix $[T]_B - \lambda I_3$ for each value of lambda. For $\lambda_1 = 0$ we get 
		\[ null\left( \begin{bmatrix}0&0&0\\0&2&4\\0&0&8\end{bmatrix} \right) =  \left \{ x \begin{bmatrix}1\\0\\0\end{bmatrix} : x \in \mathbb{R} \right\} \]
		With $\lambda_2 = 2$ we have 
		\[ null\left( \begin{bmatrix}-2&0&0\\0&0&4\\0&0&6\end{bmatrix} \right) =  \left \{ x \begin{bmatrix}0\\1\\0\end{bmatrix} : x \in \mathbb{R} \right\} \]
		And finally with $\lambda_3 = 8$ we get 
		\[ null\left( \begin{bmatrix}-8&0&0\\0&-6&4\\0&0&0\end{bmatrix} \right) =  \left \{ x \begin{bmatrix}0\\2\\3\end{bmatrix} : x \in \mathbb{R} \right\} \]

\section*{2.}
	\subsection*{a.}
		We want to show $\lambda \in \mathbb{R}$ is an eigenvalue of $T$ if and only if $\lambda$ is an eigenvalue of $S \circ T \circ S^{-1}$.
		\begin{proof}
			We wish to show that such an implication holds in both directions. \\

			"$\Rightarrow$" Assume that $\lambda \in \mathbb{R}$ is an eigenvalue for $T$.	By definition, this means that $\exists v \in V : T(v) = \lambda v$.
			Let $w \in V$ such that $w = S(v)$. By taking $S^{-1}$ of both sides we have $S^{-1}w = v$. From there we can write
			\begin{align*}
				T(S^{-1}(w)) & = \lambda S^{-1}(w) \ \ \ \ \ \ \ \ \ \ \ \ \ \ \ \ \ \ \ \ \ \ \ \ \ \ \ \ \ \ \ \ ( \text{since $v = S^{-1}w$} ) \\
				S(T(S^{-1}(w))) & = S(\lambda S^{-1}(w)) \\
				& = \lambda S(S^{-1}(w)) \\
				& = \lambda w
			\end{align*} 
			Thus $\lambda$ is an eigenvalue for $S \circ T \circ S^{-1}$. \\ 

			"$\Leftarrow$" Assume that $\lambda$ is an eigenvalue for $S \circ T \circ S^{-1}$. We know that $\exists w \in V : S \circ T \circ S^{-1}(w) = \lambda w$.
			Let $v \in V$ such that $v = S^{-1}(w)$. We can write
			\begin{align*}
				S(T(S^{-1}(w))) & = \lambda w \\
				S^{-1}(S(T(S^{-1}(w)))) & = S^{-1}(\lambda w )\\
				T(S^{-1}(w)) & = \lambda S^{-1}(w) \\
				T(v) & = \lambda v
			\end{align*}
			And thusforth $\lambda$ is an eigenvalue for $T$. We have now shown that $\lambda \in \mathbb{R}$ is an eigenvalue of $T$ if and only if $\lambda$ is an eigenvalue of $S \circ T \circ S^{-1}$.

		\end{proof}

	\subsection*{b.}
		Given a specific $\lambda \in \mathbb{R}$, if $v \in V$ is an eigenvector for $T$, then $S(v)$ is an eigenvector for $STS^{-1}$.
\section*{3.}
	\subsection*{a.}
		True. Two similar matricies have the same eigenvalues. As we just showed in $\fbox{2a.}$ if $\lambda$ is an eigenvalue for $B$, then it is also an eigenvalue for $PBP^{-1}$. Since $A = PBP^{-1}$, it is an eigenvalue for $A$.

	\subsection*{b.}
		False. There exists two matricies with the same eigenvalues that are not similar. Suppose we have \[ A = \begin{bmatrix}1&1\\0&1\end{bmatrix}, I = \begin{bmatrix}1&0\\0&1\end{bmatrix} \]
		Then both have $\lambda = 1$ as their only eigenvalues, but if we take 
		\begin{align*}
		A & = PIP^{-1} \\
		A & = PP^{-1} \\
		A & = I
		\end{align*}
		This is a contradiction, therefore the equation is false, and the matricies are not similar.

\section*{4.}
	We want to show that for $A,B \in M_{n \times n} (F)$, $trace(AB) = trace(BA)$.
	\begin{proof}
		Suppose $A,B \in M_{n \times n} (F)$. Let $AB = C$ and $BA = C'$. By definition of matrix multiplication we have 
		\[ c_{ii} = \sum_{k=1}^n a_{ik} b_{ki} \ \ \ \ \ \ \text{and}\ \ \ \ \ \ \ c'_{ii} =\sum_{k=1}^n b_{ik} a_{ki}  \]
		Therefore the trace of $C$ is
		\begin{align*}
		trace(C) & = \sum_{i = 1}^n \sum_{k = 1}^n a_{ik} b_{ki} \\
		& = \sum_{k = 1}^n \sum_{i = 1}^n a_{ik} b_{ki} \\
		& = \sum_{k = 1}^n \left( \sum_{i = 1}^n b_{ki} a_{ik} \right) \\
		& =  \sum_{k = 1}^n c'_{kk} \\
		& = trace(C')
		\end{align*}
	\end{proof}

\section*{5.}
	We have \[ A = \begin{bmatrix}1&2&3&4\\2&4&6&8\\3&6&9&12\\4&8&12&16\end{bmatrix} \] To get the eigenvalues, we will use matlab with the following command:
	\begin{verbatim}
>> eig(A)
	\end{verbatim}
	This gives us eigenvalues $\lambda_1 = 0$ with algebraic mutliplicity 3 and $\lambda_2 = 30$. We can use another matlab command to get the null space of $A - 0I$ and $A-30I$.
	\begin{verbatim}
>> null(A)

ans = 
        0.1144    -0.9765          0
       -0.2553     0.0383    -0.8944
        0.8130     0.1977          0
       -0.5107     0.0767     0.4472

>> null(A - (30 * eye(4)))

ans =
        0.1826
        0.3651
        0.5477
        0.7303
	\end{verbatim}
	Thus we can write that \[ E_{\lambda = 0} = \left \{ \begin{bmatrix}0.1144\\-0.2553\\0.8130\\-0.5107\end{bmatrix},  \begin{bmatrix}-0.9765\\0.0383\\0.1977\\0.0767\end{bmatrix},
	\begin{bmatrix}0\\-0.8944\\0\\0.4472\end{bmatrix} \right \} \] and \[ E_{\lambda = 30} = \left \{ \begin{bmatrix}1\\2\\3\\4\end{bmatrix} \right \} \]
	Since $dim(E_{\lambda = 0}) = 3$ and $dim(E_{\lambda = 30}) = 1$, $A$ is diagonalizable. We have \[ D = \begin{bmatrix}0&0&0&0\\0&0&0&0\\0&0&0&0\\0&0&0&30\end{bmatrix} \]
	and the diagonalizable matrix \[ Q = \begin{bmatrix}-2&-3&-4&1\\1&0&0&2\\0&1&0&3\\0&0&1&4\end{bmatrix} \]
\end{document}