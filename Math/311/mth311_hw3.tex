% simple.tex 

\documentclass{article}

\usepackage{times}
\usepackage{amssymb, amsmath, amsthm}
\usepackage{tasks}
\usepackage[margin=1in]{geometry}

\begin{document}

\title{MTH 311 Homework 3}
\author{Philip Warton}
\date{\today}
\maketitle

\section*{2.2.2}
	\subsection*{c.)}
		Show that $\lim \dfrac{sin(n^2)}{\sqrt[3]{n}} = 0$.
		\begin{proof}
			Choose $\epsilon > 0$, arbitrarily. Let $N_\epsilon > \dfrac{1}{\epsilon^3}$.
			We can therefore write that for all $n \geqslant N_\epsilon$ we have $n > \dfrac{1}{\epsilon^3}$ which is equivalent to stating that $\epsilon^3n > 1$.
			By properties of the sine function we also know that $|sin(n^2)| \leqslant 1$ and therefore $|sin(n^2)|^3 \leqslant 1^3 = 1$.
			From there, we have by ordering that $|sin(n^2)|^3 < \epsilon^3n$.
			We can divide this expression by $n$ to get the inequality $\dfrac{|sin(n^2)|^3}{n} < \epsilon^3$.
			Since both sides are positive, this is equivalent to $\dfrac{|sin(n^2)|}{\sqrt[3]{n}} < \epsilon$.
			With both the numerator and denominator positive, we write $\left| \dfrac{sin(n^2)}{\sqrt[3]{n}} \right| < \epsilon$.
			Trivially subtracting a zero we get $\left| \dfrac{sin(n^2)}{\sqrt[3]{n}} - 0\right| < \epsilon$.
			By definition of convergence we have $\lim \dfrac{sin(n^2)}{\sqrt[3]{n}} = 0$.
	
		\end{proof}

\section*{2.2.5}
	\subsection*{a.)}
		Let $[[x]]$ be the floor function, where $[[x]]$ is equal to the greatest integer less than or equal to $x$. Show that $\lim \left[\left[\dfrac{5}{n}\right]\right]= 0$.
		\begin{proof}
			Let $\epsilon > 0$, chosen arbitrarily. Choose $N_\epsilon > \dfrac{5}{\epsilon}$.
			From there we have for all $n \geqslant N_\epsilon$ that $n > \dfrac{5}{\epsilon}$, which is equivalent to stating $\epsilon > \dfrac{5}{n}$.
			With $n$ positive, we have $\left[\left[\dfrac{5}{n}\right]\right] \leqslant \dfrac{5}{n} < \epsilon$, and with all terms being always positive this is equivalent to $\left|\left[\left[\dfrac{5}{n}\right]\right] - 0 \right| < \epsilon$.
			Therefore, we have $\lim \left[\left[\dfrac{5}{n}\right]\right]= 0$.

		\end{proof}

\section*{2.2.7}
	\subsection*{a.)}
		Is the sequence $(-1)^n$ eventually or frequently in the set $\{ 1 \}$? \\

		The sequence $(-1)^n$ is not eventually in the set $\{ 1 \}$. In other words, for all $N \in \mathbb{N}, \ \exists n \geqslant N : (-1)^n \notin \{ 1 \}$.
		\begin{proof}
			Let $N \in \mathbb{N}$, arbitrarily.
			There are two cases, $(-1)^N = (-1)$, and $(-1)^N = 1$. \\\\
			\fbox{Case 1: $(-1)^N = (-1)$} Let $n = N$.
			We have $(-1)^n = (-1)^N = (-1)$, and $(-1) \notin \{ 1 \}$. \\\\
			\fbox{Case 2: $(-1)^N = 1$} Let $n = N +1$.
			From there we can write $(-1)^n = (-1)^{N+1}$, equivalent to $(-1)^N(-1)^1$.
			Thus, we have $1(-1) = -1 \notin \{ 1 \}$.

		\end{proof}
		
		The sequence $(-1)^n$ is frequently in the set $\{ 1 \}$ (ie. $\forall N \in \mathbb{N}, \ \exists n \geqslant N : (-1)^n \in \{ 1 \}$).
		\begin{proof}
			Let $N \in \mathbb{N}$, arbitrarily.
			There are once again two cases, $(-1)^N = (-1)$, and $(-1)^N = 1$. \\\\
			\fbox{Case 1: $(-1)^N = (-1)$} Let $n = N + 1$.
			We have $(-1)^n = (-1)^{N+1}$ which is equal to $(-1)^N(-1) = 1$.
			Since $1 \in \{ 1 \}$, we have  $(-1)^n \in 1$ for some $n \geqslant N$.\\\\
			\fbox{Case 2: $(-1)^N = 1$} Let $n = N$.
			From there we can write $(-1)^n = 1 \in \{ 1 \}$.

		\end{proof}
	\subsection*{b.)}
		Eventually is stronger than frequently.
		We can state that eventually implies frequently, but frequently does not imply eventually.\\\\
		Suppose we have a sequence $a_n$ eventually in the set $S$.
		Eventaully implies frequently because if there exists $N_e$ where $\forall n_e \geqslant N_e$, $a_{n_e} \in S$, then for every $N_f \in \mathbb{N}$, choose $n_f \geqslant \max \{N_e, N_f \}$.
		Then we have $a_{n_f} \in S$ since $n_f \geqslant N_e$. \\\\
		This does not hold in the other direction, since frequently only defines that there must exist one index greater than any $N$ that is in the proposed set.
		Therefore it cannot be said that all indecies greater than some $N$ will be in the set.
	\subsection*{c.)}
		A sequence $(a_n)$ converges to $a$ if, given any $\epsilon$-neighborhood $V_e(a)$ of $a$, $(a_n)$ is eventually in $V_e(a)$.
	\subsection*{d.)}
		Suppose an infinite number of terms of a sequence $x_n$ are equal to 2.
		This does not imply that $x_n$ is eventually in the interval $(1.9, 2.1)$.
		The counterexample would be suppose $x_n = \{ 1, 2, 1, 2, 1, 2, ... \}$. \\\\
		However, having an infinite number of terms be equal to 2 does imply that $x_n$ would be frequently in the aforementioned interval.
		Suppose that this implication was not true, then there would be some $N$ after which all $x_n$ would not be in the interval, and therefore not equal to two.
		Thus there would be a finite number of terms equal to 2, which is a contradiction.

\section*{2.3.9}
	Suppose $a_n$ bounded and $\lim b_n = 0$. Show that $\lim a_n b_n = 0$.
	\begin{proof}
		Since $a_n$ is bounded we know there exists $M > 0$ such that $|a_n| \leqslant M$ for all $n \in \mathbb{N}$.
		Also since $b_n$ converges to 0 we have $\forall \epsilon > 0 \ \exists N_{\epsilon B} : \forall n \geqslant N_{\epsilon B}, \ \ |b_n - 0| < \epsilon$. \\\\
		Let $\epsilon_j > 0$ be arbitrary. Choose $N_{\epsilon A B} = N_{\epsilon B}$.
		We know that for all $n \geqslant N_{\epsilon A B}$, the following hold:
		\begin{align*}
			|a_n| & \leqslant M \\
			|b_n - 0| & < \epsilon_j \ \ \ \ \ \ \text{(with all terms positive we can multiply these inequalities resulting in strict inequality)}\\
			\Longrightarrow |a_n||b_n-0| & < M (\epsilon_j) \\
			|a_n||b_n| = |a_nb_n-0| & < M (\epsilon_j)
		\end{align*}
		Since we chose $\epsilon_j$ arbitrarily, we can write this inequality for any other $\epsilon_k > 0$ by letting $\epsilon_k = M \epsilon_j$.
		Therefore we can say that for all $\epsilon_k > 0$, there exists $N_{\epsilon A B}$ such that for all $n \geqslant N_{\epsilon A B}$, $|a_nb_n-0| < \epsilon_k$.
		Thus $\lim a_n b_n = 0$.

	\end{proof}
\end{document}