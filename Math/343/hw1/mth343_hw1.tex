\documentclass{article}

\usepackage{times}
\usepackage{amssymb, amsmath, amsthm}
\usepackage[margin=1in]{geometry}
\usepackage{graphicx}

\begin{document}

\title{MTH 343 Homework 1}
\author{Philip Warton}
\date{\today}
\maketitle

\section*{1.3}
\subsection*{13}
\begin{proof}
    \fbox{$A \setminus (B \cup C) \subset (A \setminus B) \cap (A \setminus C)$}
    \\\\
    Let $x \in A \setminus (B \cup C)$.
    We know that $x \in A$ and $x \notin B \cup C$, thus $x \notin B$ and $x \notin C$.
    Since $x \in A$ and $x \notin B$, $x \in A \setminus B$.
    Similarly since $x \notin C, x \in A \setminus C$, thus $x \in (A \setminus B) \cap (A \setminus C)$, and thus $A \setminus (B \cup C) \subset (A \setminus B) \cap (A \setminus C)$.
    \\\\
    \fbox{$A \setminus (B \cup C) \supset (A \setminus B) \cap (A \setminus C)$}
    \\\\
    Let $x \in (A \setminus B) \cap (A \setminus C)$.
    Then $x \in A$ and $x \notin B$ and $x \notin C$. Thus $x \notin B \cup C$, and it follows that $x \in A \setminus (B \cup C)$.
    Therefore $A \setminus (B \cup C) \supset (A \setminus B) \cap (A \setminus C)$.
    And we say that the two sets are equal.
\end{proof}

\subsection*{18}
\subsubsection*{(a)}
Let $f$ be a function $f : \mathbb{R} \rightarrow \mathbb{R}$ where $f(x) = e^x$.
\\\\
\fbox{1 : 1}
Let $x, y \in \mathbb{R}$ such that $f(x) = f(y)$.
Then $e^x = e^y$, and we can take the natural log of both sides which gives $x = y$.
Thus $f$ is one-to-one.
\\\\
\fbox{Onto}
For $f$ to be onto, for all $y \in \mathbb{R}$ there must exist some $x \in \mathbb{R}$ such that $f(x) = y$.
Let $y = -1$, then there should be some $x$ such that $f(x) = e^x = -1$.
Since this equation has no solutions, $f$ is not onto.
If $y > 0$ then $\exists x : f(x) = y$, so we say that the range of $f$ is $(0, \infty)$.
\subsubsection*{(b)}
Let $f$ be a function $f : \mathbb{Z} \rightarrow \mathbb{Z}$ where $f(n) = n^2 + 3$.
\\\\
\fbox{1 : 1}
Let $m,n \in \mathbb{N}$ such that $f(m) = f(n)$.
Then we say that $m^2 + 3 = n^2 + 3$, which is equivalent to saying that $m^2 = n^2$.
This does not guarentee that $m = n$, because the case where $m = -n$ is a also a solution, therefore $f$ is not one-to-one.
\\\\
\fbox{Onto}
Let $f(n) = 0 \in \mathbb{Z}$, then 
\begin{align*}
    n^2 + 3 & = 0 \\
    n^2 & = -3 \\
    n & = \sqrt{-3}
\end{align*}
Since this has no solutions, $f$ is not onto.
The range of $f$ is $[3, \infty) \cap \mathbb{Z}$.
\subsubsection*{(c)}
Let $f$ be a function $f : \mathbb{R} \rightarrow \mathbb{R}$ where $f(x) = \sin(x)$.
\\\\
\fbox{1 : 1}
Let $x = 0$ and $y = 2 \pi$, then $f(x) = f(y) = 0$, but $x \neq y$.
Therefore $f$ is not one-to-one.
\\\\
\fbox{Onto}
Since $-1 \leqslant \sin(x) \leqslant 1$, $f$ is not onto and its range is $[-1, 1]$.
\subsubsection*{(d)}
Let $f$ be a function $f : \mathbb{Z} \rightarrow \mathbb{Z}$ where $f(n) = n^2$.
\\\\
\fbox{1 : 1}
Choose $m = 1, n = -1$, then $f(m) = f(n)$ but $m \neq n$, so $f$ is not one-to-one.
\\\\
\fbox{Onto}
We know that $n^2 \geqslant 0$ for all $n \in \mathbb{Z}$, so $f$ is not onto and its range is $\{n \in \mathbb{Z} \ \ | \ \ \sqrt{n} \in \mathbb{Z} \}$

\subsection*{22}
Let $f: A \rightarrow B$ and $g: B \rightarrow C$.
\subsubsection*{(a)}
Suppose $f$ and $g$ are one-to-one.
Show $g \circ f$ is one-to-one.
\begin{proof}
    Let $a_1, a_2 \in A$ such that $g \circ f (a_1) = g \circ f (a_2)$.
    Since $g$ is one-to-one, we know that $f(a_1) = f(a_2)$.
    Since $f$ is one-to-one, it follows that $a_1 = a_2$, therefore $g \circ f$ is one-to-one as well
\end{proof}

\subsubsection*{(b)}
Show that $g \circ f$ is onto $\Longrightarrow g$ is onto.
\begin{proof}
    Suppose that $g \circ f$ is onto.
    Then for all $c \in C$ there exists some $a \in A$ such that $g \circ f (a) = c$.
    Let $c \in C$ be arbirtrary.
    Then, there $\exists a \in A$ such that $c=g(f(a))$.
    We know that $f : A \rightarrow B$, so $f(a) \in B$.
    Thus, there exists $b = f(a) \in B$ such that $g(b) = c$, therefore $g$ is onto.
\end{proof}
\subsubsection*{(c)}
Show that $g \circ f$ is one-to-one $\Longrightarrow f$ is one-to-one.
\begin{proof}
    Assume that $g \circ f$ is one-to-one.
    If $g(f(a_1)) = g(f(a_2))$ then $a_1 = a_2$ for any $a_1, a_2 \in A$.
    We want to show that $x \neq y \Longrightarrow f(x) \neq f(y)$.
    Let $x,y \in A$ such that $x \neq y$.
    Then, by assumption, $g(f(x)) \neq g(f(y))$.
    Suppose by contradiction that $f(x) = f(y)$, then since $g$ is a function it follows that $g(f(x)) = g(f(y))$ (contradiction).
    Therefore $f(x)$ must not equal $f(y)$, and we say that $f$ is one-to-one.
\end{proof}
\subsubsection*{(d)}
Show that $g \circ f$ is one-to-one and $f$ is onto $\Longrightarrow g$ is one-to-one.
\begin{proof}
    Assume that $g \circ f$ is one-to-one and that $f$ is onto.
    We want to show that $g(b_1) = g(b_2) \Longrightarrow b_1 = b_2 \ \ \forall b_1, b_2 \in B$.
    Let $b_1, b_2 \in B$ such that $g(b_1) = g(b_2)$ without loss of generality.
    Then since $f$ is onto, we know that $\exists a_1, a_2 \in A$ such that $f(a_1) = b_1$ and $f(a_2) = b_2$.
    Therefore, $g(f(a_1)) = g(f(a_2))$, and since $g \circ f$ is one-to-one, it follows that $a_1 = a_2$.
    Since $g$ is well-defined and $a_1 = a_2$, $b_1 = b_2$ therefore $g$ is one-to-one.
\end{proof}
\subsubsection*{(e)}
Show that $g \circ f$ is onto and $g$ is one-to-one $\Longrightarrow f$ is onto.
\begin{proof}
    Assume that $g \circ f$ is onto and $g$ is one-to-one.
    We want to show that for all $b \in B$, there exists $a \in A$ such that $f(a) = b$.
    Let $b \in B$ be arbitrary, thus $g(b) \in C$.
    Since $g \circ f$ is onto, this means that there exists $a \in A$ such that $g(f(a)) = c$.
    Since $g$ is one-to-one and $c = g(f(a)) = g(b)$, this means that $f(a) = b$.
    Thus for all $b \in B$, there exists $a \in A$ such that $f(a) = b$.
\end{proof}
\section*{2.3}
\subsection*{1}
Prove that
\[ 1^2 + 2^2 + \cdots + n^2 = \frac{n(n+1)(2n+1)}{6}  \ \ \ \ \ \ \ \  
\ \ \ \ \ \ \ \ \forall n \in \mathbb{N}\]
\begin{proof}
    We must show the base case and the inductive step in order to show that the statement holds for all natural numbers.
    \\\\
    \fbox{Base Case}
    Let $n = 1$, then
    \[ 1^2 = \frac{1(1+1)(2(1)+1)}{6} \]
    This holds.
    \\\\
    \fbox{Inductive Step}
    We want to show that if the equation holds for $n$, then it will hold for $n + 1$.
    Assume that \[ 1^2 + 2^2 + \cdots + n^2 = \dfrac{n(n+1)(2n+1)}{6} \]
    Then, adding $(n+1)^2$ to both sides we get
    \begin{align*}
        1^2 + 2^2 + \cdots + n^2 + (n+1)^2 & = \frac{n(n+1)(2n+1)}{6} + (n+1) ^2 \\
        & = \frac{n(n+1)(2n+1) + 6(n+1)^2}{6} \\
        & = \frac{(2n^3 + 3n^2 + n) + (6n^2 + 12n + 6)}{6} \\
        & = \frac{2n^3 + 9n^2 +11n + 6}{6} \\
        & = \frac{(n+1)(n+2)(2(n+1)+1)}{6}
    \end{align*}
    Thus, the statement is true for all $n \in \mathbb{N}$.
\end{proof}

\subsection*{18}

\end{document}