\documentclass{article}

\usepackage{times}
\usepackage{amssymb, amsmath, amsthm}
\usepackage[margin=1in]{geometry}
\usepackage{graphicx}

\begin{document}

\title{MTH 343 Homework 2}
\author{Philip Warton}
\date{\today}
\maketitle
\section*{(1) 3.4.6}
Create a multiplication table for $U(12)$.
The integers that are co-prime to 12 are $\{1, 5, 7, 11\}$ and their respective equivalence classes.
We now compute the multiplication table.
\begin{center}
    \begin{tabular}{ c| c | c | c | c |}
        & 1 & 5 & 7 & 11 \\
       \hline
       1 & 1 & 5 & 7 & 11 \\ 
       \hline
       5 & 5 & 1 & 11 & 7 \\ 
       \hline
       7 & 7 & 11 & 1 & 5 \\ 
       \hline
       11 & 11 & 7 & 5 & 1 \\ 
       \hline
    \end{tabular}
\end{center}

\section*{(2) 3.4.7}
Let $S = \mathbb{R} \setminus \{ -1 \} $ and define a binary operation on $S$ by $a*b = a+b+ab$.
Prove that $(S, *)$ is an abelian group.
\begin{proof}
    \fbox{Commutativity}
    We want to show that $(S, *)$ is commutative.
    If $a * b = b * a \forall a,b \in S$, then the group is commutative.
    Let $a,b \in S$ be arbitrary.
    Then,
    \begin{align*}
        a * b & = a + b + ab \\
        & = b + a + ab \\
        & = b + a + ba \\
        & = b * a
    \end{align*}
    The group $(S, *)$ is commutative.
    \\\\
    \fbox{Associativity}
    To show associativity, we must show that $(a * b) * c = a * (b * c) \forall a,b,c \in S$.
    Let $a,b,c \in S$, then,
    \begin{align*}
        (a * b) * c & = (a + b + ab) * c \\
        & = (a + b + ab) + c + (a + b + ab)c \\
        & = a + b + c + ab + ac + bc + abc \\
        & = a + (b + c + bc) + a(b + c + bc) \\
        & = a * (b + c + bc) \\
        & = a * (b * c)
    \end{align*}
    The group $(S, *)$ is associative.
    \\\\
    \fbox{Identity}
    We want to show that $\exists e \in S$ such that $e * a = a * e = a \ \ \ \ \forall a \in S$.
    Let $x \in (S = \mathbb{R} \setminus \{ 1 \})$.
    Denote the identity element $e$, then $x * e = x$.
    This can be written as
    \begin{align*}
        x * e = x & = x + e + xe \\
        \Longrightarrow 0 & = e + xe \\
        0 & = e(1 + x) \\
        \Longrightarrow e & = 0
    \end{align*}
    Since $0 \in S$, and for any $s \in S, 0 * s = s = s * 0$, we have an identity.
    \\\\
    \fbox{Inverse}
    We want to show that $\forall a \in S, \exists a' : s * s' = e$.
    Let $a \in S$ be arbitrary, then we want to find $a'$ such that.
    \begin{align*}
        a * a' & = e \\
        a + a' + aa' & = 0 \\
        a'(1 + a) & = -a \\
        a' & = \frac{-1}{1+a}
    \end{align*}
    If $a = -1$, then no $a'$ exists, but since $-1 \notin S$, this situation will never occur.
    The inverse of 0 is $-1 \notin S$, but since 0 is the identity, it need not have an inverse.
    \\\\
    \fbox{Closure}
    We want to show that $\forall a,b \in S, a * b \in S$.
    Let $a,b \in S$. Then,
    \[ a * b = a + b + ab \]
    Since $\mathbb{R}$ is closed under addition and multiplication, the only thing we have to check is that it does not ever produce $-1$.
    Suppose that $a * b = -1$.
    Then
    \begin{align*}
        a + b + ab & = -1 \\
        a(1 + b) + b & = -1 \\
        a & = \frac{-1-b}{1+b} \\
        a & = \frac{-(1+b)}{(1+b)} \\
        a & = -1
    \end{align*}
    This is a contradiction since $a = -1 \notin S$.
    Therefore $a * b$ cannot equal -1.
    Since all of these conditions are met, we say that $(S, *)$ is an abelian group.
\end{proof}
\section*{(3) 3.4.21}
For each $a \in \mathbb{Z}_n$ find an element $b \in \mathbb{Z}_n$ such that $a + b \equiv 0 \text{  (mod n)}$.
\\\\
Let $a \in \mathbb{Z}_n$. Then $a \in \{ 0, 1, \cdots n-1 \} = \{ n - n, \cdots, n-2, n-1\}$.
We can write $a = n - m$ where $m \in \mathbb{N} : 1 \leqslant m \leqslant n$.
Define $b = m \in \mathbb{Z}_N$.
Then $a + b = n - m + m = n \equiv 0 \text { (mod n)}$.
\section*{(4) 3.4.40}
Let $G$ consist of $2 \times 2$ matricies of the form
\[ 
    \begin{bmatrix}
        \cos(\theta) & -\sin (\theta) \\
        \sin(\theta) & \cos(\theta)
    \end{bmatrix}
\]
where $\theta \in \mathbb{R}$. Prove that $G \leq SL_2(\mathbb{R})$.
\begin{proof}
    Since $G \leq SL_2(\mathbb{R}) \Longleftrightarrow ab^{-1} \in G \ \ \ \ \forall a,b \in G$, we will show the right hand side.
    Let $A,B \in G$ such that
    \[A = \begin{bmatrix}
        \cos(a) & -\sin(a) \\
        \sin(a) & \cos(a)
    \end{bmatrix}, \ \ \ \ \ \ B =
    \begin{bmatrix}
        \cos(b) & -\sin(b) \\
        \sin(b) & \cos(b)
    \end{bmatrix}\]
    Then we know by properties of rotation matricies that
    \[B^{-1}= 
    \begin{bmatrix}
        \cos(-b) & -\sin(-b) \\
        \sin(-b) & \cos(-b)
    \end{bmatrix}\]
    Computing $AB^{-1}$ we get the following result,
    \begin{align*}
        AB^{-1} & = \begin{bmatrix}
            \cos(a) & -\sin(a) \\
            \sin(a) & \cos(a)
        \end{bmatrix}
        \begin{bmatrix}
            \cos(-b) & -\sin(-b) \\
            \sin(-b) & \cos(-b)
        \end{bmatrix}\\ & = \begin{bmatrix}
            \cos(a)\cos(-b)-\sin(a)\sin(-b) & -\cos(a)\sin(-b)-\sin(a)\cos(-b) \\
            \sin(a)\cos(-b)+\cos(a)\sin(-b) & \cos(a)\cos(-b)-\sin(a)\sin(-b)
        \end{bmatrix}
        \\ & = \begin{bmatrix}
            \cos(a-b) & -\sin(a-b) \\
            \sin(a-b) & \cos(a-b)
        \end{bmatrix} \in G
    \end{align*}
    Therefore $G \leq SL_2(\mathbb{R})$.
\end{proof}
\section*{(5) 3.4.41}
Let $G = \{ a + b \sqrt{2} : a,b \in \mathbb{Q}, \ \ a \neq 0 \text{ or } b \neq 0\}$.
Show $G \leq \mathbb{R}^*$ under multiplication.
\begin{proof}
    Let $\alpha, \beta \in G$.
    We want to show that $\alpha \beta^{-1} \in G$.
    Denote
    \[\alpha = a_1 + b_1 \sqrt{2}, \ \ \ \ \beta = a_2 + b_2 \sqrt{2}\]
    Then $\beta^{-1} = \dfrac{1}{a_2 + b_2\sqrt{2}}$.
    We can compute the product
    \begin{align*}
        \alpha \beta^{-1} & = (a_1 + b_1\sqrt{2})\left(\frac{1}{a_2 + b_2 \sqrt{2}}\right)\\
        & = \frac{a_1 + b_1 \sqrt{2}}{a_2 + b_2 \sqrt{2}} \\
        & = \frac{a_1 + b_1 \sqrt{2}}{a_2 + b_2 \sqrt{2}} \cdot \frac{a_2 - b_2 \sqrt{2}}{a_2 - b_2 \sqrt{2}} \\
        & = \frac{a_1a_2 - a_1b_2\sqrt{2} + a_2b_1\sqrt{2} - 2b_1b_2}{a_2^2-2b_2^2}\\
        & = \frac{a_1a_2 - 2b_1b_2}{a_2^2-2b_2^2}(1) + \frac{a_2b_1-a_1b_2}{a_2^2-2b_2^2}(\sqrt{2})
    \end{align*}
    We say that this final result is an element of $G$.
    Suppose the denominator was equal to 0, then 
    \begin{align*}
        a_2^2-2b_2^2 & = 0 \\
        a_2^2 & = 2b_2^2 \\
        |a_2| & = \sqrt{2}|b_2| \\
        \Longrightarrow a_2 \notin \mathbb{Q} & \text{ or } b_2 \notin \mathbb{Q} \ \ \ \ \ \text{    (contradiction)}
    \end{align*}
    If both numerators are 0, it would mean $\alpha \beta^{-1} = 0$, since neither $\alpha = 0$ or $\beta = 0 = \beta^{-1}$ this is impossible.
\end{proof}
\section*{(6) 3.4.45}
    Show that the intersection of two subgroups is also a subgroup.
    \begin{proof}
        Let $H, K \leq G$.
        We want to show that $H \cap K \leq G$.
        Let $a, b \in H \cap K$, then we need to show that $ab^{-1} \in H \cap K$.
        We know that $a,b \in H$, and since $H$ is a subgroup $b^{-1} \in H$ as well.
        Therefore $ab^{-1} \in H$.
        Similaryly $a,b^{-1} \in K$, with the same inverse as $G$ and as $H$, and thus $ab^{-1} \in K$.
        Therefore $ab^{-1} \in H \cap K$.
    \end{proof}
\section*{(7) 3.4.46}
If $H,K \leq G$, it is not implied that $H \cup K \leq G$.
\begin{proof}
    Let $H,K \leq G$ where neither $H \subset K$ or $K \subset H$.
    Then $\exists a \in H \setminus K$ and $\exists b \in K \setminus H$.
    Suppose by contradiction that $H \cup K$ is a subgroup of $G$.
    Then $ab^{-1} \in H \cup K$.
    This means that either $ab^{-1} \in H$ or $ab^{-1} \in K$.
    If $ab^{-1} \in H$, since $a \in H$ we must have $a^{-1} \in H$.
    This would mean $a^{-1}ab^{-1} \in H \Longrightarrow b \in H$ (contradiction).
    Otherwise $ab^{-1} \in K$ therefore $ab^{-1}b \in K \Longrightarrow a \in K$ (contradiction).
    Hence $H \cup K$ is not a subgroup of $G$.
\end{proof}
\section*{(8) 4.4.1}
\subsection*{(a)}
Prove or disprove that all generators of $\mathbb{Z}_{60}$ are prime.
\begin{proof}
    Take the number $49 \in \mathbb{Z}_{60}$.
    Since $\gcd(49,60) = 1$, we know that $\langle 49 \rangle = \mathbb{Z}_{60}$.
    However, 49 is not prime.
\end{proof}
\subsection*{(b)}
Prove or disprove that $U(8)$ is cyclic.
\begin{proof}
    If $U(8) = \{1,3,5,7\}$ then there exists $a \in U(8)$ such that $\langle a \rangle = U(8)$.
    Let us check each element,
    \begin{align*}
        \langle 1 \rangle & = \{ 1 \} \\
        \langle 3 \rangle & = \{1, 3\} \\
        \langle 5 \rangle & = \{1,5\}\\
        \langle 7 \rangle & = \{1,7\}
    \end{align*}
    Since none generate $U(8)$, the group is not cyclic.
\end{proof}
\subsection*{(e)}
\section*{(9) 4.4.2}
Find the order of the element in the group.
\subsection*{(a)}
$5 \in \mathbb{Z}_{12}$
\\\\
We know that the least common multiple of 5 and 12 is $60 = 5(12)$.
Therefore 5 is order 12.
\subsection*{(b)}
$\sqrt{3} \in \mathbb{R}$
\\\\
Since $\mathbb{R}$ is infinite, there is no natural degree which will result in the identity, so we say that the order is infinite.
Order of $\sqrt{3} = \infty$.
\subsection*{(d)}
$-i \in \mathbb{C}^*$
\\\\
We know that
\begin{align*}
    -i & = -i \\
    -i^2 & = -1 \\
    -i^3 & = i \\
    -i^4 & = 1
\end{align*}
Therefore $i$ is order 4.
\section*{(10) 4.4.3}
List every...
\subsection*{(a)}
Element of $7\mathbb{Z}$.
\[ 7\mathbb{Z} = \{ \cdots, -14, -7, 0, 7, 14, \cdots\} \]
\subsection*{(b)}
Element generated by $15 \in \mathbb{Z}_{24}$.
\[\langle 15 \rangle= \{0,3,6,9,12,15,18,21\}\]
\subsection*{(c)}
Subgroups of $\mathbb{Z}_{12}$
\begin{align*}
    & \{0\} \\
    & \{0,6\} \\
    & \{0,4,8\} \\
    & \{0,3,6,9\} \\
    & \{0,2,4,6,8,10\} \\
    & \mathbb{Z}_{12}
\end{align*}
\subsection*{(d)}
Subgroups of $\mathbb{Z}_{60}$.
\begin{align*}
    & \{0\} \\
    & \{0,30\} \\
    & \{0,20,40\} \\
    & \{0,15,30,45\} \\
    & \{0,12,24,\cdots,48\} \\
    & \{0,10,20,30,\cdots,50\} \\
    & \{0,6,12,18, \cdots, 54\} \\
    & \{0,5,10,15, \cdots, 55\} \\
    & \{0,4,8,12, \cdots, 56\} \\
    & \{0,3,6,9, \cdots, 57\} \\
    & \{0,2,4,6, \cdots, 58\} \\
    & \mathbb{Z}_{60}
\end{align*}
\subsection*{(e)}
Subgroups of $\mathbb{Z}_{13}$.
\begin{align*}
    & \{0\} \\
    & \mathbb{Z}_{13}
\end{align*}
\subsection*{(f)}
Subgroups of $\mathbb{Z}_{48}$.
\begin{align*}
    & \{0\} \\
    & \{0,24\} \\
    & \{0,16,32\} \\
    & \{0,12,24,36\} \\
    & \{0,8,16, \cdots , 40\} \\
    & \{0,6,12, \cdots, 42\} \\
    & \{0,4,8,12, \cdots, 40,44\} \\
    & \{0,3,6,9, \cdots, 42,45\} \\
    & \{0,2,4,6, \cdots, 44,46\} \\
    & \mathbb{Z}_{48}
\end{align*}
\subsection*{(g)}
The subgroup generated by $3 \in U(20)$.
\[\langle 3 \rangle = \{1,3,7,9\}\]
\end{document}