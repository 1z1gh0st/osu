\documentclass{article}

\usepackage{times}
\usepackage{amssymb, amsmath, amsthm}
\usepackage[margin=1in]{geometry}

\begin{document}

\title{MTH 338 - Paper Proposal}
\author{Philip Warton}
\date{\today}
\maketitle
%\tableofcontents
%\newpage

\section{Introduction}

When talking about geometry, there are many rules, theorems, and axioms that one takes for granted. Because of this, one could assume that geometry as a whole is quite complex. In this sense, finite geometries are the minimalist geometries, where the rules are as simple as possible. This paper looks to explore the finite geometries, and find the more complex analogues of finite projective geometries such as Fano's projective plane. First, we will introduce finite geometries that we know and understand already, and introduce these to the reader to give a basis for understanding, and then go beyond this. \\\\
To go further means to first do some counting. Assume axioms, and generalize their theorems for $n$ points. We will start by doing this for axioms of the four-point geometry, and then ultimately make an attempt to do this for the Fano and Young plane geometries. There may be some experimentation with which axioms we choose to keep, remove, or alter in order to find generalized patterns. Once this process is complete we will present the results of the more complex projective geometries, and choose a particular one of these to examine. The goal is to then build a robust and elegant model for this geometry, that will then be presented to the reader.

\clearpage

\subsection{Incidence Geometries}
\subsection{Known Cases}
\subsection{Desired Results}

\section{Patterns}

\subsection{Minimalist Incedence Geometries}
\subsubsection{Parrallel Lines}
\subsubsection{Intersections}

\subsection{Incedence Geometries with Three Colinear Points}
\subsubsection{Parallel LInes}
\subsubsection{Intersections}

\subsection{N Colinear Points}

\subsection{Projective Planes}

\section{Finding the More Complex Projective Plane}

\subsection{Fano Plane}
\subsection{Desired Plane}

\section{Building a Model}

\subsection{Different Possible Models}
\subsection{The Final Most Elegant Model}

\section{Generalize $\dots$ ?}

\end{document}