\documentclass{article}

\usepackage{times}
\usepackage{amssymb, amsmath, amsthm}
\usepackage[margin=1in]{geometry}

\begin{document}

\title{MTH 351 HW 6}
\author{Philip Warton}
\date{\today}
\maketitle

\section*{1}
\subsection*{a}
We can set this system of equations up as
\begin{align}
P(2)&=a(2^3)+b(2^2)+c(2) +d = 1 \\
P(1) &=a(1^3)+b(1^2)+c(1)+d = 0\\
P(3) &= a(3^3)+b(3^2)+c(3)+d=-1\\
P(0) &= d= 2
\end{align}

Then we can row reduce on the matrix to get 
\[ rref\left( \begin{bmatrix}8&4&2&1&1\\1&1&1&1&0\\27&9&3&1&-1\\0&0&0&1&2\end{bmatrix}\right) = \begin{bmatrix}1&0&0&0&-1\\0&1&0&0&4.5\\0&0&1&0&-5.5\\0&0&0&1&2\end{bmatrix} \]
Therefore we can write
\[ P(x) = -x^3 + \frac{9}{2}x^2-\frac{11}{2}x+2 \]

\subsection*{b}
For the Lagrange method, we use our Matlab code. We get 
\[ L_1 = \frac{-(t)(t-1)(t-3)}{2}, L_2 = \frac{(t)(t-2)(t-3)}{2}, L_3 = \frac{(t)(t-1)(t-2)}{6}, L_4 = \frac{-(0.5t-1)(t-1)(t-3)}{3} \]
Then we once we simplify we get the polynomial 
\[ P(x) = -x^3 + \frac{9}{2}x^2-\frac{11}{2}x+2 \]

\subsection*{c}
Using the diagram for the Newton method, we get
\[ c_0 = 1, c_1 = 1, c_2 = \frac{-3}{2}, c_3 = -1 \]
Then we have 
\[ P(x) = 1 + 1(x-2) -\frac{3}{2}(x-2)(x-1) - (x-2)(x-1)(x-3) =-x^3 + \frac{9}{2}x^2-\frac{11}{2}x+2 \]

\section*{2}
To get the divided difference, we compute 
\[f[1,2,3] = \frac{f[2,3]-f[1,2]}{3-1} \]
Then we can compute the two divided differences in the numerator 
\[f[2,3] = \frac{f[3]-f[2]}{3-2} = -1 \]
\[f[1,2] = \frac{f[2]-f[1]}{2-1} = -2 \]
Therefore plugging those back into our original divided difference computation we get 
\[ f[1,2,3] = \frac{-1 + 2}{2} = \frac{1}{2} \]

\section*{3}

\end{document}