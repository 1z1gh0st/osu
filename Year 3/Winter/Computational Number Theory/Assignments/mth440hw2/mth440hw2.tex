\documentclass{article}

\usepackage{times}
\usepackage{amssymb, amsmath, amsthm}
\usepackage[margin=.5in]{geometry}
\usepackage{graphicx}
\usepackage[linewidth=1pt]{mdframed}

\usepackage{import}
\usepackage{xifthen}
\usepackage{pdfpages}
\usepackage{transparent}

\newcommand{\incfig}[1]{%
    \def\svgwidth{\columnwidth}
    \import{./figures/}{#1.pdf_tex}
}

\newtheorem{theorem}{Theorem}[section]
\newtheorem{lemma}{Lemma}[section]
\newtheorem*{remark}{Remark}
\theoremstyle{definition}
\newtheorem{definition}{Definition}[section]

\begin{document}

\title{Number Theory - Homework 2}
\author{Philip Warton}
\date{\today}
\maketitle
\section*{Problem 8}
Find every equivalence class mod 7 that satisfies the condition.
\subsection*{(a)}
\fbox{$|x| \leqslant 3$}
\[
    \{ 0,1,2,3 \}
\]
\subsection*{(b)}
\fbox{$x$ is odd}
\[
    \{ 1, 3, 5, 6\}
\]
\subsection*{(c)}
\fbox{$x$ is divisible by 3}
\[
    \{ 0, 3, 6 \}
\]
\subsection*{(d)}
\fbox{$x$ is prime}
\[
    \{ 2, 3, 5 \}
\]
\section*{Problem 9}
\subsection*{(a)}
\fbox{$k \in \mathbb{Z} \Longrightarrow k^2 \equiv 0$ (mod 4) or $k^2 \equiv 1$ (mod 4)}
\begin{proof}
    Let $k \in \mathbb{Z}$ be arbitrary. Then we know that mod 4, $k$ is equivalent to either
    0, 1, 2, or 3. Then we can square each of these equivalence classes, and find that the only 
    results will be 0 or 1.
    \begin{align*}
        0^2 & \equiv 0 \\
        1^2 & \equiv 1 \\
        2^2 & \equiv 0 \\
        3^2 & \equiv 1
    \end{align*}
\end{proof}
\subsection*{(b)}
\fbox{If $m \equiv 3$ (mod 4) then $m$ cannot be expressed as the sum of two squares in $\mathbb{Z}$}\\\\
\begin{proof}
Suppose that $m$ can be expressed as the sum of two squares, that is,
\[
    a^2 + b^2 = m
\]
However, from there we know $a^2$ and $b^2$ are equivalent to 0 or 1, thus their sum mod 4 will be equivalent to one of the following
\begin{align*}
    0 + 0 &= 0 \\
    0 + 1 &= 1 \\
    1 + 0 &= 1 \\
    1 + 1 &= 2
\end{align*}
Since none of these are 3, we say that $m$ is not equivalent to 3 (mod 4).
\end{proof}
\section*{Problem 10}
\fbox{Find $35^{-1}\in\mathbb{Z}_{97}$}\\\\
First we assume that 35 is a unit element mod 97, and is therefore invertible.
Then we use the extended euclidean algorithm to find the multiplicative inverse.
\begin{align*}
    97 & = (2)35 + 27 \\
    35 & = (1) 27 + 8 \\
    27 & = (3)8 + 3 \\
    8 & = (2)3 + 2 \\
    3 & = (1)2 + 1 \\\\
    3 - 2 & = 1 \\
    3 - (8 - (2)3) & = 1 \\
    (3)3 - 8 & = 1\\
    (3)(27 - (3)8) - 8 & = 1 \\
    (3)27 - (10)8 & = 1 \\
    (3)27 - (10)(35 - 27) & = 1 \\
    (13)27 - (10)35 & = 1 \\
    (13)(97 - (2)35) - (10)35 & = 1 \\ 
    (13)97 - (36)35 & = 1
\end{align*}
From here we can say $(-36)(35) = (-13)97 + 1 \Longrightarrow (-36)(35) \equiv 1$ (mod 97).
So then we say that $-36 \equiv 61$ is the multiplicative inverse of 35 mod 97. 
\section*{Problem 11}
    \begin{mdframed}
        Find the $1 \leqslant x \leqslant 10$, find the order of $x$ mod 11. Which of these $x$ are primitive roots?
    \end{mdframed}
    \begin{align*}
        1^1 & \equiv 1 \text{ (mod } 11\text{)}\\
        2^{10} & \equiv 1 \ \ \ \ \ \ \ \vdots \\
        3^5 & \equiv 1 \\
        4^5 & \equiv 1 \\
        5^5 & \equiv 1 \\
        6^{10} & \equiv 1 \\
        7^{10} & \equiv 1 \\
        8^{10} & \equiv 1 \\
        9^5 & \equiv 1 \\
        10^2 & \equiv 1 
    \end{align*}
    We have primitive roots $\{2,6,7,8,10\}$.
\section*{Problem 12}
    \begin{mdframed}
        Show that for every natural number $n, 3^{2n + 5} + 2^{4n + 1}$ is divisible by 7.
    \end{mdframed}
    \begin{proof}
        We use the property that $a \equiv b$ (mod $n$) implies that $ac \equiv bc$ (mod $n$). Then,
        looking at $\mathbb{Z}_7$ we write the following:
        \begin{align*}
            3^{2n + 5} + 2^{4n + 1} &\equiv (3^5)3^{2n} + (2)2^{4n} \\
            & \equiv (5)3^{2n} + (2)2^{4n} \\
            & \equiv (5)(3^2)^n + (2)(2^4)^n \\
            & \equiv (5)2^n + (2)2^2 \\
            & \equiv 7(2^n) \equiv 0 \text{ (mod 7)}
        \end{align*}
    \end{proof}
\section*{Problem 13}
    \begin{mdframed}
        Code:
        \begin{verbatim}
def smallest_prime_factor(n):
    if n % 2 == 0:
        return 2
    else:
        x = 3
        while True:
            if n % x == 0:
                return x
            x = x + 2

print(smallest_prime_factor(594088117))
print(smallest_prime_factor(346132737927421))
        \end{verbatim}
        Output:
        \begin{verbatim}
7
592759
        \end{verbatim}
    \end{mdframed}
\section*{Problem 14}
    \begin{mdframed}
        Code:
        \begin{verbatim}
def compute_order(a, n):
    if a % n == 1:
        return 1
    for i in range(2,n):
        if a^i % n == 1:
            return i

print(compute_order(17, 100))
print(compute_order(100001, 11111))
        \end{verbatim}
        Output:
        \begin{verbatim}
20
540
        \end{verbatim}
    \end{mdframed}
\section*{Problem 15}
    \begin{mdframed}
        Code:
        \begin{verbatim}
primes = [101,103,107]

def find_smallest_primitive_root(n):
    for a in range(1, n):
        if compute_order(a, n) == n - 1:
            return a

for p in primes:
    print(find_smallest_primitive_root(p))
        \end{verbatim}
        Output:
        \begin{verbatim}
2
5
2
        \end{verbatim}
    \end{mdframed}
\end{document}