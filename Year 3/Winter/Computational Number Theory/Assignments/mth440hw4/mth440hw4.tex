\documentclass{article}

\usepackage{times}
\usepackage{amssymb, amsmath, amsthm}
\usepackage[margin=.5in]{geometry}
\usepackage{graphicx}
\usepackage[linewidth=1pt]{mdframed}

\usepackage{import}
\usepackage{xifthen}
\usepackage{pdfpages}
\usepackage{transparent}

\newcommand{\incfig}[1]{%
    \def\svgwidth{\columnwidth}
    \import{./figures/}{#1.pdf_tex}
}

\newtheorem{theorem}{Theorem}[section]
\newtheorem{lemma}{Lemma}[section]
\newtheorem*{remark}{Remark}
\theoremstyle{definition}
\newtheorem{definition}{Definition}[section]

\begin{document}

\title{Computational Number Theory - Homework 4}
\author{Philip Warton}
\date{\today}
\maketitle
\section*{Problem 24}
\section*{Problem 25}
\subsection*{a)}
\[
    x^2 \equiv 17 \mod{67}
\]
\begin{align}
    \left(\frac{17}{67}\right) & = \left(\frac{67}{17}\right)\\
    & = \left(\frac{16}{17}\right) \\
    & = \left(\frac{2}{17}\right)^4 \\
    & = 1
\end{align}
\subsection*{b)}
\[
    x^2 \equiv 3 \mod{67}
\]
\begin{align}
    \left(\frac{3}{67}\right)
    & = (-1)\left(\frac{67}{3}\right) \\
    & = (-1)\left(\frac{1}{3}\right) \\
    & = -1
\end{align}
\subsection*{c)}
\[
    2x^2 + 5x + 1 \equiv 0   
\]
\subsection*{d)}
\[
    x^2 \equiv 65 \mod{101}    
\]
\begin{align}
    \left(\frac{65}{101}\right) 
    & = (-1)\left(\frac{101}{65}\right)\\
    & = (-1)\left(\frac{31}{65}\right)\\
    & = (-1)\left(\frac{65}{31}\right)\\
    & = (-1)\left(\frac{3}{31}\right)\\
    & = \left(\frac{31}{3}\right)\\
    & = \left(\frac{1}{3}\right) = 1
\end{align}
\subsection*{e)}
\[
    x^2 \equiv 5 \mod{2\cdots 1}
\]
\begin{align}
    \left(\frac{5}{2\cdots 1}\right)
    & = (-1)\left(\frac{2\cdots 1}{5}\right)\\
    & = (-1)\left(\frac{1}{5}\right) \\
    & = -1
\end{align}
\section*{Problem 26}
\begin{mdframed}[]
    \fbox{Code:}
    \begin{verbatim}
def solve_quadratic(b, n):
    for a in range(1, n):
        if a^2 % n == b:
            return a
        else:
            a = a + 1
    return 0

print(solve_quadratic(17, 67))
print(solve_quadratic(65, 101))
    \end{verbatim}
    \fbox{Output:}
    \begin{verbatim}
33
41
    \end{verbatim}
\end{mdframed}
The answer to \fbox{a.} is 33, and the answer to \fbox{d.} is 41.
\section*{Problem 27}
\begin{mdframed}
    \fbox{Code:}
    \begin{verbatim}
def discrete_log(n, a, b):
    k = 1
    while(a^k % n != b):
        k = k + 1
    return k

def diffie_hellman(p, g, g_a, g_b):
    a = discrete_log(p, g, g_a)
    b = discrete_log(p, g, g_b)
    s = g^(a*b) % p
    return s

print(diffie_hellman(49253, 2, 558, 32288))
    \end{verbatim}
    \fbox{Output:}
    \begin{verbatim}
43739
    \end{verbatim}
\end{mdframed}
\section*{Problem 28}
\begin{mdframed}
    \fbox{Code:}
    \begin{verbatim}
def fermat_factorization(n):

    #--STEP 1--
    t_0 = ceil(sqrt(n))
    
    #--STEP 2--
    l = 0
    while (not is_square((t_0 + l)^2 - n)):
        l = l + 1
    t = t_0 + l
    s = sqrt(t^2 - n)
    
    #--STEP 3--
    return (t - s, t + s)

print(fermat_factorization(41156989185107))
    \end{verbatim}
    \fbox{Output:}
    \begin{verbatim}
(6409511, 6421237)
    \end{verbatim}
\end{mdframed}
\section*{Problem 29}
\[
x^7 \equiv 17792272918826 \mod{41156989185107}
\]
\[
    6409511 \cdot 6421237 = 41156989185107
\]
\begin{align}
    x^7 &\equiv 17792272918826 \mod{6421237}\\
    x^7 &\equiv 17792272918826 \mod{6409511}
\end{align}
We get the solution 9546903516023, but this cannot be right since there is no 95th or 54th
character.
\end{document}