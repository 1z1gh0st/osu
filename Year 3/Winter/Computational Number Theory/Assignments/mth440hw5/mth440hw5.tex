\documentclass{article}

\usepackage{times}
\usepackage{amssymb, amsmath, amsthm}
\usepackage[margin=.5in]{geometry}
\usepackage{graphicx}
\usepackage[linewidth=1pt]{mdframed}

\usepackage{import}
\usepackage{xifthen}
\usepackage{pdfpages}
\usepackage{transparent}

\newcommand{\incfig}[1]{%
    \def\svgwidth{\columnwidth}
    \import{./figures/}{#1.pdf_tex}
}

\newtheorem{theorem}{Theorem}[section]
\newtheorem{lemma}{Lemma}[section]
\newtheorem*{remark}{Remark}
\theoremstyle{definition}
\newtheorem{definition}{Definition}[section]

\begin{document}

\title{Computational Number Theory - Homework 5}
\author{Philip Warton}
\date{\today}
\maketitle
\section*{Problem 31}
Let $p$ be an odd prime.
\subsection*{a)}
\begin{mdframed}
    Prove that $-1$ is a quadractic residue mod $p$ if and only if $p \equiv 1 \mod{4}$
\end{mdframed}
\begin{proof}
    We know that the legendre symbol of $-1$ is $\left(\frac{-1}{p}\right)$. By Euler's criterion, we have
    \begin{align*}
        \left(\frac{-1}{p}\right) & \equiv (-1)^{(p-1)/2} \mod{p} \\
    \end{align*}
    Suppose $p = 4k + 3$, then 
    \[
        (-1)^{(4k + 3 - 1)/2} = (-1)^{2k + 1} = -1
    \]
    Thus $-1$ cannot be a quadractic residue mod $p$. Then suppose $p = 4k + 1$,
    \[ (-1)^{(4k + 1 - 1)/2} = (-1)^{4k} = 1 \]
    So it must be the case that $-1$ is a quadratic residue mod $p$.
    These are the only options for $p$, as $p$ is an odd prime.
\end{proof}
\subsection*{b)}
\begin{mdframed}
    Assuming $p \geqslant 5$, prove that $-3$ is a quadratic residue mod $p$ if and only if $p \equiv 1 \mod{3}$.
\end{mdframed}
\begin{proof}
    We begin by writing the legendre symbol for $-3$,
    \[
        \left(\frac{-3}{p}\right) = \left(\frac{-1}{p}\right)\left(\frac{3}{p}\right) = (-1)^{(p-1)/2}\left(\frac{3}{p}\right)
    \]
    Suppose that $p \equiv 1$ mod ${4}$,
    then by quadratic reciprocity we have 
    \[
        (-1)^{(p-1)/2}\left(\frac{3}{p}\right) = (-1)^{(4k + 1 - 1)/2}\left(\frac{3}{p}\right) = (-1)^{2k}\left(\frac{p}{3}\right) = \left(\frac{p}{3}\right)
    \]
    Then suppose we have $p \equiv 3$ mod $4$, it follows that we get
    \[
        (-1)^{(4k + 3 - 1)/2} \left(\frac{3}{p}\right)  = (-1)^{2k + 1}(-1)\left(\frac{p}{3}\right) = (-1)(-1)\left(\frac{p}{3}\right) = \left(\frac{p}{3}\right)
    \]
    In either scenario, we have the end result of $\left(\frac{p}{3}\right)$. \\\\
    Then we know that $\left(\frac{p}{3}\right) = \left(\frac{p \text{ mod } 3}{3}\right)$.
    So if $p \equiv 1$ mod $3$, clearly the legendre symbol of $-3$ is $\left(\frac{1}{3}\right) = 1$ and we say that $-3$ is a quadratic residue mod $p$.
    Then if $p \not\equiv 1$ mod $3$, we must have $p \equiv 2$ mod $3$, and so the legendre symbol of $-3$ is $\left(\frac{2}{3}\right) = -1$ so $-3$ is a quadractic non-residue mod $p$.
\end{proof}
\section*{Problem 32}
\begin{mdframed}
    Code:
    \begin{verbatim}
def QuadraticResidues(p):
    QR = []
    for a in range(0, p):
        if legendre_symbol(a, p) == 1:
            QR.append(a)
    return(QR)

print(QuadraticResidues(17))
print(QuadraticResidues(53))
    \end{verbatim}
    Output:
    \begin{verbatim}
[1, 2, 4, 8, 9, 13, 15, 16]
[1, 4, 6, 7, 9, 10, 11, 13, 15, 16, 17, 24, 25, 28, 29, 36, 37, 38, 40, 42, 43, 44, 46,
 47, 49, 52]        
    \end{verbatim}
\end{mdframed}
\section*{Problem 33}
\begin{mdframed}
    Code:
    \begin{verbatim}
def DiscreteLog(a, n):
    for k in range(0, n):
        if 2^k % n == a % n:
            return k
    return 0

print(DiscreteLog(452, 1019))
    \end{verbatim}
    Output:
    \begin{verbatim}
632     
    \end{verbatim}
\end{mdframed}
\section*{Problem 34}
\begin{mdframed}
    Euclid's GCD Algorithm:
    \begin{verbatim}
def GCD(a, b):
    if a > b:
        temp = b
        b = a
        a = temp
    r = b % a
    if r == 0:
        return a
    else:
        return GCD(a, r)

# Tests
print(GCD(87444, 238))
print(GCD(28464, 812))
    \end{verbatim}
Output:
\begin{verbatim}
14
4
\end{verbatim}
\end{mdframed}
\end{document}