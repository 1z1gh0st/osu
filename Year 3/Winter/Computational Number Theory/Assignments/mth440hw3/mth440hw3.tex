\documentclass{article}

\usepackage{times}
\usepackage{amssymb, amsmath, amsthm}
\usepackage[margin=.5in]{geometry}
\usepackage{graphicx}
\usepackage[linewidth=1pt]{mdframed}

\usepackage{import}
\usepackage{xifthen}
\usepackage{pdfpages}
\usepackage{transparent}

\newcommand{\incfig}[1]{%
    \def\svgwidth{\columnwidth}
    \import{./figures/}{#1.pdf_tex}
}

\newtheorem{theorem}{Theorem}[section]
\newtheorem{lemma}{Lemma}[section]
\newtheorem*{remark}{Remark}
\theoremstyle{definition}
\newtheorem{definition}{Definition}[section]

\begin{document}

\title{Computational Number Theory - Homework 3}
\author{Philip Warton}
\date{\today}
\maketitle
\section*{Problem 17}
\begin{mdframed}[]
    Show that if $n$ is a positive integer and $n \equiv 2 \mod{4}$, then $8^n + 9^n$ is divisible by 5.
\end{mdframed}
\begin{proof}
    Since $n \equiv 2 \mod{4}$ of course we have some $k \in \mathbb{Z}$ such that $n = 4k + 2$.
    Then we write \[8^n + 9^n = 8^{4k + 2} + 9^{4k + 2} = 8^2 8^{4k} + 9^2 9^{4k}\]
    Observe the following facts modulo 5,
    \begin{align}
        8^2 & \equiv 64 \equiv 4 \\
        9^2 & \equiv 81 \equiv 1 \\\\
        8^4 & \equiv 4^2 \equiv 16 \equiv 1 \\
        9^4 & \equiv 1^2 \equiv 1 
    \end{align}
    So then we can say 
    \begin{align}
        8^n + 9^n \equiv (4)1^k + (1)1^k \equiv 4 + 1 \equiv 0 \mod{5}
    \end{align}
\end{proof}
\section*{Problem 18}
\begin{mdframed}
    Show that if $p \geqslant 5$ is prime and $a,b\in\mathbb{Z}$, then $ab^p - a^pb$ is divisible by $6p$.
\end{mdframed}
\begin{proof}
    Assume that $p \geqslant 5$ is some prime number, and let $a, b$ be integers. We want to show that 
    $ab^p -a^pb$ is divisible by $6p$. We write 
    \[
        ab^p - a^pb = (ab)(b^{p-1} - a^{p-1})
    \]
    Then if a number is divisible by $6p$ it must be divisible by 6 and by $p$. We know that $p \nmid 6$ it follows that this 
    proof may involve Fermat's Little Theorem.
    So we know that $6^{p-1} \equiv 1 \mod{p}$. We can write $6p = (2)(3)(p)$.
    Possibly we can invoke Fermat's Little Theorem to say that $b^{p-1} - a^{p-1}$ must be equivalent to 0 mod 2, mod 3, or mod $p$.
    If both $a$ and $b$ are divisible by $2,3$ and $p$ then trivially the term $ab^p - a^pb$ is divisible by $6p$.
    If both are not divisible by 2, 3, or $p$ then of course $b^{p-1} - a^{p-1} \equiv 0 \mod{p}$ and therefore the number
    $ab^p -a^pb$ is divisible by $p$ and therefore $6p$. Of course if we have any other combinations of $a$ and $b$'s prime factors that covers $6p$
    it follows that $ab$ will be divisible by $6p$.
\end{proof}
\section*{Problem 19}
\begin{mdframed}
    Let $n \geqslant 1$ and let $m = 2^n - 1$. Show that (a.) if $m$ is prime then $n$ is prime, and that (b.) if $n$ is prime then $m$ is either prime or base 2 psuedo-prime.
\end{mdframed}
\subsection*{(a)}
\subsection*{(b)}
\section*{Problem 20}
\begin{mdframed}
    
\end{mdframed}
\section*{Problem 21}
\section*{Problem 22}
\end{document}