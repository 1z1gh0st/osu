\documentclass{article}

\usepackage{times}
\usepackage{amssymb, amsmath, amsthm}
\usepackage[margin=.5in]{geometry}
\usepackage{graphicx}
\usepackage[linewidth=1pt]{mdframed}

\usepackage{import}
\usepackage{xifthen}
\usepackage{pdfpages}
\usepackage{transparent}

\newcommand{\incfig}[1]{%
    \def\svgwidth{\columnwidth}
    \import{./figures/}{#1.pdf_tex}
}

\newtheorem{theorem}{Theorem}[section]
\newtheorem{lemma}{Lemma}[section]
\newtheorem*{remark}{Remark}
\theoremstyle{definition}
\newtheorem{definition}{Definition}[section]

\begin{document}

\title{Computational Number Theory - Homework 1}
\author{Philip Warton}
\date{\today}
\maketitle
\section*{Problem 1}
\subsection*{(a)}
    \[
        64 = (5)11 + 9
    \]
\subsection*{(b)}
    \[
        -50 = (-8)7 + 6
    \]
\subsection*{(c)}
    \[
        91 = (7)13 + 0
    \]
\subsection*{(d)}
    \[
        11 = (0)15 + 11
    \]
\section*{Problem 2}
    Prove that $6 | n^3 - n$ for all $n \in \mathbb{N}$.
    \begin{proof}
        \begin{align*}
            0^3 - 0 & = 0 \\
            1^3 - 1 & = 1 \\
            2^3 - 2 & = 6 \\
            3^3 - 3 & = 24 \\
            4^3 - 4 & = 60 \\
            5^3 - 5 & = 120
        \end{align*}
        Now let $n > 5$. We write $n = q6 + r$.
        Then we write 
        \begin{align*}
            n^3 - n & = (q6 + r)^3 - (q6 + r) \\
            & = (6^2q^2 + 6(2)rq + r^2)(6q + r) - (q6 + r) \\
            & = (6^3q^3 + 6^2(2)rq^2 + 6r^2q + 6^2q^2r + 6(2)r^2q + r^3) - (6q + r) \\
            & = (6)(6^2q^3 + 6(2)rq^2 + r^2q + 6q^2r + (2)r^2q - q) + (r^3 - r)
        \end{align*}
        Then the first term is clearly divisible by $6$, and since $r \in \{0,1,2,3,4,5\}$ we know that the second term is also divisible by 6.
    \end{proof}
\section*{Problem 3}
    \subsection*{(a)}
        Let $p$ be a prime number which is not 2 or 3. Show that when $p$ is divided by 6, the 
        remainder is either 1 or 5.
        \begin{proof}
            Suppose that the remainder is 2, then it follows that $2 | p$ since the number is even. Suppose that the remainder is $3$ then it follows
            that $3 | p$. If the remainder is $4$, then $2 | p$ since $p$ must be even.
        \end{proof}
    \subsection*{(b)}
        Show that the product of two numbers of the form $6x + 1$ is also of the form $6x + 1$.
        \begin{proof}
            Let $x,y \in \mathbb{Z}$. We say that $(6x + 1)(6y + 1) = 6^2xy + 6y + 6x + 1 = 6(6xy + y + x) + 1$, which is 
            of the described form $6k + 1$.
        \end{proof}
    \subsection*{(c)}
        Show that if $k$ is a positive integer, then $6k + 5$ has a prime factor $p$ of the form $p = 6x + 5$.
        \begin{proof}
            Let $k \in \mathbb{Z}^+$. We want to show that $6k + 5$ has a prime factor $p$ of the form
            $p = 6x + 5$. Suppose that each prime factor is not of the form $6x + 5$. Then they must all be of the form $6x + 1$, 
            since any prime number that is not 2 or 3 has remainder of either 1 or 5.
            But in this case, their product would be of the form $6x + 1$. Therefore 
            it must be the case that there is some prime factor of $6k + 5$ that is of that same form.
        \end{proof}
    \subsection*{(d)}
        Suppose there is a finite number of primes of the form $6x + 5$.
        Construct a set containing each of these denoted by $\{q_1,q_2,q_3, \cdots q_k\}$.
        Then let $Q = \Pi_{i = 1}^k q_i$. If $k$ is even, $Q \equiv [1]_6$, otherwise $Q \equiv [5]_6$. \\\\
        \fbox{Case 1: $k$ is even} Let $p = Q + 4 \equiv [5]_6$. In other words, $\exists k \in \mathbb{N}$ such that
        $p = 6k + 5$. Therefore it must have a prime factor of the form $p' = 6x + 5$. It must be equal to some $q_i$,
        since it is a prime of the form $6x + 5$. Therefore it must divide $Q$. Thus $p' | Q$ and $p' | Q + 4$.
        However, it cannot be the case that $p'$ divides 4, since 4's only prime factor 2 is not of the form $6x + 5$.
        This means that $p' | p - Q = 4$ but $p' \nmid 4$ (contradiction).\\\\
        \fbox{Case 2: $k$ is odd} Let $p = Q + 6 \equiv [5]_6$. Then it must be of the form $6k + 5$ for some $k \in \mathbb{N}$.
        Therefore it must have some prime factor $p'$ of the form $6x + 5$. Then it must be the case that $p' | Q$ and that $p' | p$.
        However this means $p' | p - Q = 6$, but 6 is not of the form $6x + 5$ (contradiction).
\section*{Problem 4}
    \subsection*{(a)}
    \begin{proof}
        Suppose that $\gcd(a,b) = 1$, and $a | c$ and $b | c$.
        Then $\exists x,y \in \mathbb{Z}$ such that $ax + by = 1$.
        We want to show that there exists $k$ such that $c = k(ab)$.
        There exists $k_a, k_b \in \mathbb{Z}$ such that $c = k_a(a), c = k_b(b)$.
        So we can write the following:
        \begin{align*}
            c & = c(1) \\
            &= c(ax + by) \\
            &= cax + cby \\
            &= k_b(b)(ax) + k_a(a)(by) \\
            &= ab(k_b x) + ab(k_a y) \\
            &= ab(k_b x + k_a y)
        \end{align*}
    \end{proof}
    \subsection*{(b)}
    \begin{proof}
        Suppose that $\gcd(a,b) = 1$ and that $a | bc$.
        Then write $bc = k_a(a)$ and $ax + by = 1$. We have the following:
        \begin{align*}
            c & = c(1) \\
            &= c(ax + by) \\
            &= cax + cby \\
            &= cax + (bc) y \\
            &= cax + k_a(a) y \\
            &= a(cx + k_a y)
        \end{align*}
    \end{proof}
    \subsection*{(c)}
    \begin{proof}
        Suppose that $p$ is prime and that $p | ab$.
        If $\gcd(p,a) = 1$, then $p | b$. Otherwise, it must be the case that 
        $\gcd(p,a) = p$ (since $p$ has no factors other than 1 and $p$) and therefore $p | a$.
    \end{proof}
    \subsection*{(d)}
    \begin{proof}
        Let $x \in \mathbb{Z}$. Suppose $\gcd(6x +5, 5x + 4) \neq 1$.
        Then there must be some number not equal to 1 that divides both.
    \end{proof}
\section*{Problem 5}
Use the extended Euclidean algorithm to solve $ax + by = \gcd(a,b)$.
\subsection*{(a)}
Let $a=-23,b=16$.
\begin{align*}
    -23 &= -2(16) + 9 \\
    16 &= 1(9) + 7 \\
    9 &= 1(7) + 2 \\
    7 &= 3(2) + 1 \\\\
    7 - 3(2) &= 1 \\
    4(7) - 3(9) &= 1 \\
    4(16) - 7(9) &= 1 \\
    -10(16) - 7(-23) &= 1
\end{align*}
\subsection*{(b)}
Let $a = 111, b=442$.
\begin{align*}
    442 &= 3(111) + 109 \\
    111 &= 1(109) + 2 \\
    109 &= 54(2) + 1 \\\\
    109 - 54(2) & = 1 \\
    109 - 54(111 - 109) & = 1\\
    55(109) - 54(111) &= 1\\
    55(442 - 3(111)) - 54(111) &= 1\\
    55(442) -219(111) &= 1
\end{align*}
\section*{Problem 6}
\subsection*{(i)}
To count the number of primes we run the following:
\begin{mdframed}
Code:
\begin{verbatim}
    count = 0
    for n in range(100,999):
        if is_prime(n):
            count = count + 1
    print(count)
    \end{verbatim}
Output:
\begin{verbatim}
    143
\end{verbatim}
\end{mdframed}
There are a total of 143 prime numbers that have 3 digits.
\subsection*{(ii)}
To find the smallest 3 primes with 10 digits we run the following:
\begin{mdframed}
    Code:
    \begin{verbatim}
count = 0
num = 1000000000
while count < 3:
    if is_prime(num):
        print(num)
        count = count + 1
    num = num + 1
    \end{verbatim}
    Output:
    \begin{verbatim}
1000000007
1000000009
1000000021
    \end{verbatim}
\end{mdframed}
\subsection*{(iii)}
To list and count all primes of the form $10^3 \leqslant n^2 + 1 < 10^4$  we run the
following Sage code:
\begin{mdframed}
    Code:
    \begin{verbatim}
start = floor(sqrt(1000)) - 1
end = ceil(sqrt(10000)) + 1
count = 0
for number in range(start, end):
    x = number^2 + 1
    if is_prime(x):
        print(x)
    count = count + 1
print("\nCOUNT: ", count)
    \end{verbatim}
    Output:
    \begin{verbatim}
1297
1601
2917
3137
4357
5477
7057
8101
8837

COUNT:  9
    \end{verbatim}
\end{mdframed}
\end{document}