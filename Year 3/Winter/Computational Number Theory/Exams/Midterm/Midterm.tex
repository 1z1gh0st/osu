\documentclass{article}

\usepackage{times}
\usepackage{amssymb, amsmath, amsthm}
\usepackage[margin=.5in]{geometry}
\usepackage{graphicx}
\usepackage[linewidth=1pt]{mdframed}

\usepackage{import}
\usepackage{xifthen}
\usepackage{pdfpages}
\usepackage{transparent}

\newcommand{\incfig}[1]{%
    \def\svgwidth{\columnwidth}
    \import{./figures/}{#1.pdf_tex}
}

\newtheorem{theorem}{Theorem}[section]
\newtheorem{lemma}{Lemma}[section]
\newtheorem*{remark}{Remark}
\theoremstyle{definition}
\newtheorem{definition}{Definition}[section]

\begin{document}

\title{Computational Number Thoery - Midterm Exam}
\author{Philip Warton}
\date{\today}
\maketitle
\section*{Problem 1}
\begin{align*}
    101 & = 8 \cdot 12 + 5 \\
    77 & = 11 \cdot 7 + 0 \\
    -40 & = -4 \cdot 11 + 4
\end{align*}
\section*{Problem 2}
Use the extended Euclidean algorithm to show that gcd$(14,89) = 1$ and to find
the smallest positive integer $x$ satisfying $14x \equiv 1 $ (mod 89).
\begin{align*}
    89 & = (6)14 + 5 \\
    14 & = (2)5 + 4 \\
    5 & = (1)4 + 1 \\
    4 & = (4)1 + 0 \\\\
    5 - 4 & = 1 \\
    5 - (14 - (2)5) & = 1 \\
    (3)5 - 14 & = 1 \\
    (3)(89 - (6)14) - 14 & = 1 \\
    (3)89 - (19)14 & = 1
\end{align*}
Then we say that $14(-19) \equiv 1$ (mod 89) or equivalently $14(70) \equiv 1$ (mod 89).
\section*{Problem 3}
Find the orders of 2 and 3 mod 13 Are either primitive roots? Recall first that if $a \equiv b$ (mod 13) then of course
$(c)a \equiv (c)b$ (mod 13).
\begin{align*}
    2^1 & \equiv 2 \\
    2^2 & \equiv 4 \\
    2^3 & \equiv 8 \\
    2^4 & \equiv 16 \equiv 3 \\
    2^5 & \equiv 6 \\
    2^6 & \equiv 12 \\
    2^7 & \equiv 24 \equiv 11 \\
    2^8 & \equiv 22 \equiv 9 \\
    2^9 & \equiv 18 \equiv 5 \\
    2^{10} & \equiv 10 \\
    2^{11} & \equiv 20 \equiv 7 \\
    2^{12} & \equiv 14 \equiv 1
\end{align*}
We can compute these powers for the number 3 as well:
\begin{align*}
    3^1 \equiv 3 \\
    3^2 \equiv 9 \\
    3^3 \equiv 1
\end{align*}
Then since 2 has an order of 12, it is a primitive root mod 13. The number 3 has an order of 3 and is not a primitive root mod 13.
\section*{Problem 4}
\subsection*{(a)}
\begin{proof}
    By assumption, we say that $a$ is order 3 mod $p$. This means, $a^3 \equiv 1 \mod{p}$. Then,
    since $p$ is prime and $a \nmid p$, by Fermat's Little Theorem we have $a^{p - 1} \equiv 1 \mod{p}$.
    We know that the following pattern will be generated by multiplying $a$ mod $p$:
    \begin{align*}
        a & \equiv x \mod{p} \\
        a^2 &\equiv y \mod{p} \\
        a^3 &\equiv 1 \mod{p} \\
        a^4 &\equiv x \mod{p} \\
        a^5 &\equiv y \mod{p}\\
        a^6 &\equiv 1 \mod{p}\\
        a^7 &\equiv x \mod{p}\\
        &\vdots \\
        a^{p-1} &\equiv 1 \mod{p}\\
    \end{align*}
    Since $\langle a \rangle \cong U_3$ and cycles every 3 powers, it follows that $p-1$ must be of the form $3k$ for some $k \in \mathbb{Z}$.
    So of course $p - 1 = 3k \Longrightarrow p = 3k + 1$ therefore $p \equiv 1 \mod{3}$.
\end{proof}
\subsection*{(b)}
\begin{proof}
    We want to show that $a^2 + a + 1 \equiv 0 \mod{p}$.
    \begin{align*}
        & (a - 1)(a^2 + a + 1) = a^3 - 1 \\
        & a^3 \equiv 1 \mod{p} \\
        & a^3 - 1 \equiv 0 \mod{p} \\
        \Longrightarrow \ \ & (a - 1)(a^2 + a + 1) \equiv 0 \mod{p}
    \end{align*}
    Then it must be the case that either $a^2 + a + 1 \equiv 0 \mod{p}$ or that $a - 1 \equiv 0 \mod{p}$.
    If $a - 1 \equiv 0 \mod{p}$ then $a \equiv 1 \mod{p}$ and $a$ would be order 1 mod $p$ (contradiction, $a$ is order 3 mod $p$).
    So it must be the case that $a^2 + a + 1 \equiv 0 \mod{p}$.
\end{proof}
\subsection*{(c)}
\begin{proof}
    We want to show that $a + 1$ is order 6 mod $p$. Let us check each power $1,2,
    \cdots,5,6$. Suppose that $a + 1 \equiv 1 \mod{p}$ then we would have $a \equiv 0 \mod{p}$,
    which by the order of $a$ being 3 mod $p$ we know to be false. So we check 
    $(a+1)^2$, and in this case we say
    \[
        (a+1)^2 = a^2 + 2a + 1 = a + (a^2 + a + 1) \equiv a + 0 \not\equiv 0 \mod{p}
    \]
    Now we can check $(a+1)^3$, which can be rewritten as $(a+1)(a+1)^2$. Then
    \[
        (a+1)^3 = (a+1)(a+1)^2 \equiv (a+1)a = a^2 + a \equiv -1 \mod{p}
    \]
    We know this last equivalence by \fbox{b}. Then since $a$ has order 3, we know that $p \neq 2$ and thus $1 \not\equiv -1 \mod{p}$.
    Since $(a+1)^2 \equiv a \mod{p}$, and since $a$ is order 3, we can write
    \[ (a+1)^4 = ((a+1)^2)^2 \equiv (a)^2 \not\equiv 1 \mod{p}\]
    Then for $(a+1)^5$, we write this as the following 
    \[
        (a+1)^5 = (a+1)^4(a+1) \equiv a^2(a+1) =a^3 + 1 \equiv 1 + 1 \not\equiv 1 \mod{p}
    \]
    We know that $2 \neq 1 \mod{p}$ since $p \geqslant 3$. Finally we write 
    \[
        (a+1)^6 = (a+1)^3(a+1)^3 \equiv (-1)^2 \equiv 1 \mod{p}
    \]
    So since the smallest power at which $(a+1)$ becomes equivalent to 1 is 6, we say that
    $a+1$ is order 6 mod $p$.
\end{proof}
\end{document}