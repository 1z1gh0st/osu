\documentclass{article}

\usepackage{times}
\usepackage{amssymb, amsmath, amsthm}
\usepackage[margin=.5in]{geometry}
\usepackage{graphicx}
\usepackage[linewidth=1pt]{mdframed}

\usepackage{import}
\usepackage{xifthen}
\usepackage{pdfpages}
\usepackage{transparent}

\newcommand{\incfig}[1]{%
    \def\svgwidth{\columnwidth}
    \import{./figures/}{#1.pdf_tex}
}

\newtheorem{theorem}{Theorem}[section]
\newtheorem{lemma}{Lemma}[section]
\newtheorem*{remark}{Remark}
\theoremstyle{definition}
\newtheorem{definition}{Definition}[section]

\begin{document}

\title{Computational Number Theory - Final Exam}
\author{Philip Warton}
\date{\today}
\maketitle
\section*{Problem 1}
Compute $3^{267}$ mod 100.
We can factor 267 as follows,
\[ 
    267 = 256 + 8 + 2 + 1
\]
Then we can take powers of $3$ by squaring,
\begin{align*}
    3^1 &\equiv 3 \mod{100}\\
    3^2 &\equiv 9 \\
    3^4 &\equiv 81 \\
    3^8 &\equiv 81^2 \equiv 61 \\
    3^{16} &\equiv 61^2 \equiv 21 \\
    3^{32} &\equiv 21^2 \equiv 41 \\
    3^{64} &\equiv 41^2 \equiv 81 \\
    3^{128} &\equiv 81^2 \equiv 61 \\
    3^{256} &\equiv 21
\end{align*}
Then write our exponent as a product of these,
\begin{align*}
    3^{267} &= 3^{256 + 8 + 2 + 1} \mod{100}\\
    & = 3^{256} 3^{8} 3^{2} 3^1 \\
    &\equiv (21)(61)(9)(3) \\
    &\equiv (81)(9)(3) \\
    &\equiv (29)(3) \\
    &\equiv 87 \mod{100}
\end{align*}
\section*{Problem 2}
We want to factor 731 using Fermat factorization. We know ceil$(\sqrt{731})=28$. Then we know 
$28^2 = 224 + 560 = 784$. We write 
\begin{align*}
    28^2 - 731 &= 784 - 731 = 53 \\
    29^2 - 731 &= 841 - 731 = 110 \\
    30^2 - 731 &= 900 - 731 = 169 = 13^2
\end{align*}
So then we say that $731 = (30 - 13)(30 + 13) = (17)(43)$.
\section*{Problem 3}
We say that $x^2 \equiv 15 \mod {211}$ has no solutions by its Legendre symbol. Notice that $211 = 208 + 3 \equiv 3 \mod{4}$.
So we write,
\begin{align*}
    \left(\frac{15}{211}\right) &= \left(\frac{3}{211}\right)\left(\frac{5}{211}\right)\\
    &= (-1)\left(\frac{211}{3}\right)\left(\frac{211}{5}\right) & \text{(quadratic reciprocity)}\\
    &= (-1)\left(\frac{1}{3}\right)\left(\frac{1}{5}\right)\\
    &= -1
\end{align*}
\section*{Problem 4}
\begin{proof}
We show that $21 | n^7 - n$ for every $n \in \mathbb{N}$. First we will show that the quantity is divisble by $3$, and then by $7$,
which will of course imply that it is divisible by 21. First we rewrite
\[
    n^7 - n = n(n^6 - 1) = n(n^3 + 1)(n^3 - 1)
\]
So then if 3 divides any one of those multiplied terms it is granted that $3 | n^7 - n$. If $n \equiv 0$ mod 3 then of 
course the $n$ term is divisble by 3. If $n \equiv 1$ mod 3 then we say $n = 3k + 1$ and 
\[
    n^3 - 1 = (3k + 1)^3 - 1 = (3k)p(k) + 1 - 1 = (3)(k)p(k)
\]
Where $p(k)$ is some polynomial of $k$. We know that the only term without a $(3k)$ factor is the $1^3 = 1$ term i.e. the scalar term cubed (by distributivity of multiplication 
one could show this rigorously), giving us this result.
So in this case $3 | n^3 - 1$. Now if $3 \equiv 2$ mod 3, we have 
\[
    n^3 + 1 = (3k + 2)^3 + 1 = (3k)p(k) + 2^3 + 1 = (3)kp(k) + 9 = (3)(kp(k) + 3)
\]
So we say that $3 | n^3 + 1$. So for all possible $n$ modulo 3, we have either $3 | n,\ 3 | n^3 +1,$ or $3 | n^3 -1$.
\\\\
Now we wish to show that $7 | (n)(n^3 + 1)(n^3 - 1)$ for every $n \in \mathbb{N}$. If $n = 7k$,
clearly $7 | n$. Then if $n = 7k + 1$, we have
\[
    (7k+1)^3 - 1 = (7k)(p(k)) + 1 - 1 = (7)(kp(k))
\]
Then if $n = 7k + 2$,
\[
    (7k + 2)^3 - 1 = (7k)(p(k)) + 2^3 - 1 = (7k)(p(k)) + 7
\]
Following this argument, we can simply check that each number $0,1,2,3,\cdots,6$ cubed is equal to $\pm 1$ mod 7,
to see if 7 divides $(n^3 + 1)(n^3 -1)$.
\begin{align*}
    3^3 &= 27 \equiv -1 \mod{7}\\
    4^3 &= 64 \equiv 1 \mod{7}\\
    5^3 &= 125 \equiv -1 \mod{7}\\
    6^3 &= 216 \equiv -1 \mod{7}
\end{align*}
So for any $7k + r, 0 \leqslant r \leqslant 6$, that is for any $n$, we have $7 | (n)(n^3 + 1)(n^3 - 1)$.
So then it follows that $21 | n^7 - n$.
\end{proof}
\section*{Problem 5}
\begin{proof}
    We want to show that $q | 2^p - 1$ where $p$ is a prime equivalent to 3 mod 4. and $q = 2p + 1$.
    We write $p = 4k + 3$ and then $q = 2(4k + 3) = 8k + 7$. So then it follows that 
    \[
        \left(\frac{2}{q}\right) = 1 
    \]
    That is, $\exists x$ such that $x^2 \equiv 1$ mod $q$. Now, by Fermat's Little Theorem we have 
    \begin{align*}
        x^{q - 1}  &\equiv 1 \mod{q}\\
        x^{2(q-1)/2} &\equiv 1 \ \ \ \ \ \vdots \\
         x^{2p} &\equiv 1 \\
        (x^2)^p &\equiv 1 \\
        2^p &\equiv 1 \\
        2^p - 1 &\equiv 0 \mod{q}
    \end{align*}
    The final statement is equivalent to $q | 2^p - 1$.
\end{proof}

\end{document}