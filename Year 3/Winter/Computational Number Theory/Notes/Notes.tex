\documentclass{article}

\usepackage{times}
\usepackage{amssymb, amsmath, amsthm}
\usepackage[margin=.5in]{geometry}
\usepackage{graphicx}
\usepackage[linewidth=1pt]{mdframed}

\usepackage{import}
\usepackage{xifthen}
\usepackage{pdfpages}
\usepackage{transparent}

\newcommand{\incfig}[1]{%
    \def\svgwidth{\columnwidth}
    \import{./figures/}{#1.pdf_tex}
}

\newtheorem{theorem}{Theorem}[section]
\newtheorem{lemma}{Lemma}[section]
\newtheorem*{remark}{Remark}
\theoremstyle{definition}
\newtheorem{definition}{Definition}[section]

\begin{document}

\title{Computational Number Theory - Notes}
\author{Philip Warton}
\date{\today}
\maketitle
\section{Introduction and Divisibility}
The first important set we look at is the set of integers 
\[\mathbb{Z} = \{\cdots, -2,-1,0,1,2,\cdots \}\]
\begin{mdframed}
    \begin{definition}
        If $a,b \in \mathbb{Z}$, we say that $a$ divides $b$, denote $a | b$ if there exists some $n \in \mathbb{Z}$ such that $b = na$.
    \end{definition}
\end{mdframed}
If no such $n$ exists we say that $a$ does not divide $b$.
\\\\
\fbox{The Division Algorithm} Let $a,b \in \mathbb{Z}$ with $b \geqslant 1$.
Then there exists unique integers $q$ and $r$ such that 
\[
    a = qb + r
\]
Where $q \in \mathbb{Z}$ and $r \in \{0,1, \cdots b\}$.
\begin{proof}
    Let $S = \{a + bx | x \in \mathbb{Z} \}$. It follows that the subset 
    of nonnegative values in $S$ is bounded below, and contains some smallest nonnegative element.
    Call this $r$. Then if $r = a + bx_0$, let $q = -x_0$.
    Then of course $a = qb + r$. To show that $0 \leqslant r < b$, first note 
    that by construction $r$ must be non-negative. Then if $r \geqslant b$, it follows that we can replace 
    $bx_0$ with $b(x_0 - 1)$ resulting in a smaller non-negative element of $S$. Thuse $0 \leqslant r < b$.\\\\
    Then show uniqueness, please.
\end{proof}
\begin{mdframed}
    \begin{theorem}[Euclid]
        There are infinitely many prime numbers.
    \end{theorem}
\end{mdframed}
\begin{lemma}
    Every integer $n \geqslant 2$ is divisible by some prime.
\end{lemma}
\begin{proof}
If this lemma is false, let $n$ be the smallest integer which is not divisible by any prime.
We know that $n$ cannot be prime, since we would have $n | n$. 
So $n$ can be factored as $n = ab$ where $a,b \in \{1,2,3, \cdots , n\}$.
Then $a$ is smaller than the smallest integer that has no prime factor, thus it has a prime factor.
Then it follows that the prime factor of $a$ must be a prime factor of $n$.
\end{proof}
Now that this lemma has been proven, we can move on to prove the theorem at hand.
\begin{proof}
    Assume that there are finitely many primes.
    Let 
    \[
        N = p_1p_2, \cdots p_k + 1
    \]
    Then $N$ is divisible by a some prime $p_i$. It follows that 
    $N = p_i(m) + 1$ which means that $r \neq 0$ and $p_i$ does not divide $N$. 
\end{proof}
If $n \geqslant 2$ is composite then $n$ is divisible by some prime $p \leqslant \sqrt{n}$.
\begin{proof}
    If $x > \sqrt{n}$ and $y > \sqrt{n}$ then $n = xy > \sqrt{n}\sqrt{n} = n$ which is false.
    So either $x$ or $y$ is less than or equal to $\sqrt{n}$. Take $p$ to be a prime factor of either $x$ or $y$,
    depending on which is not larger than $\sqrt{n}$.
\end{proof}
\begin{mdframed}[]
    Siene of Eratosthenes: A method to find all primes $p$ up to some bound $N$.\\
    \fbox{1.} Write the numbers from $2$ to $N$.\\
    \fbox{2.} Starting with the smallest element $n$ still on the list. Eliminate all multiples of this number up to $N$.\\
    \fbox{3.} Let $p$ be the next smallest element remaining, and remove the previous $p$.
    \fbox{4.} Repeat steps 2 and 3 up to $\sqrt{N}$.
\end{mdframed}
\end{document}