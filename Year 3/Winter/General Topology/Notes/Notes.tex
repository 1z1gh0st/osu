\documentclass{article}

\usepackage{times}
\usepackage{amssymb, amsmath, amsthm}
\usepackage[margin=.5in]{geometry}
\usepackage{graphicx}
\usepackage[linewidth=1pt]{mdframed}

\usepackage{import}
\usepackage{xifthen}
\usepackage{pdfpages}
\usepackage{transparent}

\newcommand{\incfig}[1]{%
    \def\svgwidth{\columnwidth}
    \import{./figures/}{#1.pdf_tex}
}

\newtheorem{theorem}{Theorem}[section]
\newtheorem{lemma}{Lemma}[section]
\newtheorem*{remark}{Remark}
\theoremstyle{definition}
\newtheorem{definition}{Definition}[section]

\begin{document}

\title{General Topology - Notes}
\author{Philip Warton}
\date{\today}
\maketitle
We say that a topological space is some set $X$ equipped with some topology $\tau$ such that the following
are true (we call these the topological axioms).
\begin{align*}
    (i) & \ \ \O \in \tau, X \in \tau \\
    (ii) & \text{ The collection is closed under any union.} \\
    (iii) & \text{ The collection is closed under finite intersection. }
\end{align*}
The base of a topology is a collection of open subsets of $X$, such that 
every set in $\tau$ is a union of sets in $B$. Any base must have two properties.
First, the base elements conver $X$, that is, $\bigcup B \supset X$.
Second, let $B_1, B_2 \in B$. For every $x \in B_1 \cap B_2$ there is some element $B_3 \in B$ such
that $B_3 \subset B_1 \cap B_2$. An order relation can be defined on a product space similarly
to how we order a dictionary (expand upon this).
If we have $X_1, <_1, X_2, <_2, \cdots , X_n$. Then define 
$<$ on $X_1 \times X_2 \times \cdots \times X_n$ by 
$X_{11} \times x_{21} \times \cdots \times x_{n1} < x_{12} \times x_{22} \times \cdots \times x_{n2}$.
There is an interesting case of the 'order topology' for which these definitions may become quite useful.
s
\end{document}