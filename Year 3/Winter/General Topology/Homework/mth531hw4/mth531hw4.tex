\documentclass{article}

\usepackage{times}
\usepackage{amssymb, amsmath, amsthm}
\usepackage[margin=.5in]{geometry}
\usepackage{graphicx}
\usepackage[linewidth=1pt]{mdframed}

\usepackage{import}
\usepackage{xifthen}
\usepackage{pdfpages}
\usepackage{transparent}

\newcommand{\incfig}[1]{%
    \def\svgwidth{\columnwidth}
    \import{./figures/}{#1.pdf_tex}
}

\newtheorem{theorem}{Theorem}[section]
\newtheorem{lemma}{Lemma}[section]
\newtheorem*{remark}{Remark}
\theoremstyle{definition}
\newtheorem{definition}{Definition}[section]

\begin{document}

\title{General Topology and Fundamental Groups - Homework 4}
\author{Philip Warton}
\date{\today}
\maketitle
\section*{Problem 1}
    Let $(X,d)$ be a metric space and let $A \subset X$.
    Define the distance $d(x,A)$ of a point $x \in X$ from $A$
    by $d(x,A) = \inf\{d(x,a)|a \in A\}$.
    \subsection*{a)}
        \begin{mdframed}
            Fix $A \subset X$. The function $d(\cdot , A): X 
            \rightarrow \mathbb{R}, x \rightarrow d(x,A)$ is
            continuous.
        \end{mdframed}
        \begin{proof}
            Since we have two metric spaces, we can use the $\epsilon$-$\delta$ definition
            of continuity. That is, a function $f:X \rightarrow Y$ is continuous if and only
            if for every $\epsilon > 0$ there exists a $\delta > 0$ such that 
            \begin{align}
                d(x,y) < \delta \Longrightarrow \rho(f(x), f(y)) < \epsilon
            \end{align}
            Before our notation becomes confusing we will define our distance in the space $X$
            to be denoted by $d(x,y)$ for two points $x,y \in X$. Then we will denote our function
            described in the problem statement as $f: X \rightarrow \mathbb{R}$ where 
            \begin{align}
                f(x) = d(x,A)
            \end{align}
            Then for our metric on $\mathbb{R}$ we say that $\rho(x,y) = |x - y| \ \ \forall x,y \in
            \mathbb{R}$.
            Having made our notation follow our definition of continuity clearly we may now continue. \\\\
            Let $\epsilon > 0$ be arbitrary. Then choose $\delta = \epsilon$. Choose any point $z \in A$. 
            If $d(x,y) < \delta$, then by the reverse triangle inequality in metric spaces, we have
            \begin{align}
                |d(x,A) - d(y,A)| \leqslant |d(x,z) - d(y,z)| < d(x,y) < \delta = \epsilon
            \end{align}
            Therefore the function is continuous.
        \end{proof}
    \subsection*{b)}
        \begin{mdframed}
            The distance $d(x,A) = 0$ if and only if $x \in \overline{A}$.
        \end{mdframed}
        \begin{proof}
            \fbox{$\Rightarrow$} Assume that $d(x,A) = 0$. Then let $\epsilon > 0$ be arbitrary.
            There exists some $a \in A$ such that $d(x,a) < \epsilon$. Since $x$ is clearly a limit
            point of $A$, it follows that $x \in \overline{A}$.\\\\
            \fbox{$\Leftarrow$} Assume that $x \in \overline{A}$. Then for every $\epsilon > 0, \ \ 
            B_\epsilon(x) \cap A \neq \emptyset$. So it follows that there exists $a \in A$ such that 
            $d(x,a) < \epsilon$ for every positive radius $\epsilon$. Therefore the infimum of all strictly
            positive radii must be 0, and we say that $d(x,A) = \inf\{d(x,a) | a \in A\} = 0$.
        \end{proof}
    \subsection*{c)}
        \begin{mdframed}
            Let $A,B \subset X$ be subsets of $X$ with disjoint closures. The distance function yields 
            some Urysohn function for $A$ and $B$ directly.
        \end{mdframed}
        Define a function $f: X \rightarrow \{0,1\}$ as follows 
        \begin{align}
            f(x) =
            \begin{cases}
                0, & d(x,A) = 0\\
                1, & \text{otherwise}
            \end{cases}
        \end{align}
\section*{Problem 2}
    \begin{mdframed}
        Let $X$ be a topological space and let $D \subset \mathbb{R}$ be dense. Suppose that there
        is an open $X_r \subset X$ for each $r \in D$ such that $\overline{X}_r \subset X_{\tilde{r}}$
        when $r < \tilde{r}$, and $\bigcup_{r \in D}X_r = X$. Define $f: X \rightarrow \mathbb{R}$
        by $f(x) = \inf\{r \in D | x \in X_r\}$. The function $f$ is continuous.
    \end{mdframed}
    \begin{proof}
        Choose some open interval $(a,b) \in \mathbb{R}$. Sets of such a form are a basis for the
        standard topology on $\mathbb{R}$. Therefore if we can show that $f^{-1}((a,b))$ is an open
        set in $X$, then we say that $f$ is continuous. Define the set $X_s = \bigcup_{r < s}X_r$,
        which allows us to take such sets on any point in $\mathbb{R}$.
        So we write the following,
        \begin{align}
            f^{-1}((a,b)) &= \{x \in X | f(x) \in (a,b) \} \\
            &=\{x \in X | \inf\{r \in D | x \in X_r\} \in (a,b)\}\\
            &=\{x \in X | a < \inf\{r \in D | x \in X_r\} < b\}\\
            &=\{x \in X | a < \inf\{r \in D | x \in X_r\}\} \cap \{x \in X | b > \inf\{r \in D | x \in X_r\}\}\\
            &=\{x \in X | a \geqslant \inf\{r \in D | x \in X_r\}\}^c \cap \{x \in X | b > \inf\{r \in D | x \in X_r\}\}\\
            &= \left(\overline{X_a}\right)^c \cap X_b 
        \end{align}
        This last line is justified by the fact that these sets are nested, and that the closure of the 
        nested set still lies within any containing open set. So since the the closure of $X_a$ is
        closed, its complement is open, and clearly $X_b$, a union of open sets, is also open.
        Then we have the intersection of two open sets which is of course open.
    \end{proof}
\section*{Problem 4}
    Define a strong limit point to be a limit point $x$ of a set $A$ such that for any neighborhood
    $U$ of $x$ has infinitely many points in its intersection with $A$. Define countably compact to
    mean that any countable open cover of a topological space $X$ yields some finite sub-cover.
    \begin{mdframed}
        A singleton set in a $T_1$ space is closed.
    \end{mdframed}
    \begin{proof}
        Let $X$ be a $T_1$ topological space. Let $x \in X$.
        For every $y \neq x$, there exists some neighborhood $U(y)$ not containing $x$.
        Take the union
        \begin{align}
            \bigcup_{y \in X \setminus \{x\}}U(y) = X \setminus \{x\}
        \end{align}
        As this is a union of open sets, it is open, therefore the singleton set $\{x\}$ is closed.
    \end{proof}
    \subsection*{a)}
        \begin{mdframed}
            Assume that $X$ is a $T_1$ space. Any limit point of a set is a strong limit point of
            that set.
        \end{mdframed}
        \begin{proof}
            Let $A \subset X$ with some limit point $x \in X$. Let $U$ be some arbitrary open 
            neighborhood of $x$. Then suppose by contradiction that $U \cap A$ is finite.
            Write
            \begin{align}
                U \cap A \setminus \{x\} &= \{a_1, a_2, \cdots , a_k\} \setminus \{x\}\\
                &= \bigcup_{1 \leqslant i \leqslant k} \{a_i\} \setminus \{x\}
            \end{align}
            Since singleton sets are closed, a finite union of singleton sets will be closed. Therefore
            can take the complement of this set and it will be open. That is,
            \begin{align}
                X \setminus \left(\bigcup_{1 \leqslant i \leqslant k}\{a_i\} \setminus \{x\}\right)
            \end{align}
            However, this is an open neighborhood of $x$ that does not intersect $A$ anywhere other than at
            the point $x$ itself, therefore $x$ is not a limit point of $A$ (contradiction).
        \end{proof}
    \subsection*{b)}
        \begin{mdframed}
            Assume that $X$ is $T_1$ and assume that any infinite set in $X$ has an accumulation point.
            Then $X$ is coutably compact.
        \end{mdframed}
        \begin{proof}
            Let $\bigcup_{n \in \mathbb{N}} \mathcal{O}_n = X$ be some countable cover of $X$.
            We want to show that there exists some finite subcover of $X$.
            Suppose by contradiction that no finite subcover exists. That is, for every reordering of our
            countable cover,
            \begin{align}
                \bigcup_{i = 1}^k \mathcal{O}_i \neq X
            \end{align}
            Then for every $i \in \mathbb{N}$ there exists some point $x_i \in X \setminus \bigcup_{i = 1}^k
            \mathcal{O}_i$. Thus the set $\{x_k\}_{k \in \mathbb{N}}$ is infinite.
            For any $i \in \mathbb{N}$, the intersection $\{x_k\}_{k \in \mathbb{N}} \cap \mathcal{O}_i$ is 
            finite by definition (it will have exactly $\max\{0, i-k\}$ elements in fact).
            Then since the set $\{x_k\}_{k \in \mathbb{N}}$ is infinite, it will of course have some limit point
            $x \in X$. Then since we have an open cover of $X$, we know that for some $n \in \mathbb{N}$
            $x \in \mathcal{O}_n$. Since $x$ is a limit point of $\{x_k\}_{k \in \mathbb{N}}$ and $\mathcal{O}_n$
            is a neighborhood of $x$, it must be the case that $\mathcal{O}_n \cap \{x_k\}_{k \in \mathbb{N}}$ is
            an infinite set (contradiction).
        \end{proof}
\section*{Problem 5}
    Let $(X, \tau_1)$ be a compact Hausdorff space.
    \subsection*{a)}
        \begin{mdframed}
            Suppose $X$ is also compact in topology $\tau_2$ and assume that $\tau_2$ is finer than $\tau_1$.
            Then $\tau_1 = \tau_2$.
        \end{mdframed}
        We use the following facts without proof: A closed subset of a compact space is compact in the subspace
        topology. A compact subset of a compact Hausdorff space is closed.
        \begin{proof}
            Let $C$ be closed in $(X, \tau_2)$. Then, arbitrarily choose an open cover of $C$ in the subspace topology 
            consisting only of sets from $\tau_1$. Then clearly it must yield some finite sub-cover. That is,
            \begin{align}
                \bigcup_{\alpha \in A}U_\alpha \cap C = C \ \ \ \Longrightarrow \ \ \ \bigcup_{\alpha \in F}U_\alpha \cap C = C
            \end{align}
            It follows then that $C$ is compact in the subspace topology of $(X \cap C, \tau_1)$. Since $C$ is compact
            in this subspace, it is the case that $C$ is a closed set in $(X, \tau_1)$.
            Since any closed set in $\tau_2$ is closed in $\tau_1$, we have $\tau_2 \subset \tau_1$. Then since 
            $\tau_2$ is finer than $\tau_1$, we have $\tau_1 = \tau_2$.
        \end{proof}
    \subsection*{b)}
        \begin{mdframed}
            Assume only that $\tau_2$ is Hausdorff and finer than $\tau_1$. Then still $\tau_1 = \tau_2$.
        \end{mdframed}
        \begin{proof}
            Let $f$ be the identity map. Then $f: (X,\tau_2) \rightarrow (X,\tau_1)$ is continuous, by $\tau_2$ being finer
            than $\tau_1$. That is for every open set $O \in \tau_1, O \in \tau_2$. Then clearly the identity map
            is a bijection. So it follows that this function $f$ must be a homeomorphism since it is mapping from a
            compact Hausdorff space to a Hausdorff space. Therefore $(X, \tau_2)$ is a compact space. From there,
            the conditions of \fbox{5a} are met so we say that $\tau_1 = \tau_2$.
        \end{proof}
\end{document}