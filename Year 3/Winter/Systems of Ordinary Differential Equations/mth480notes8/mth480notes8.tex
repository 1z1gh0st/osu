\documentclass{article}

\usepackage{times}
\usepackage{amssymb, amsmath, amsthm}
\usepackage[margin=.5in]{geometry}
\usepackage{graphicx}
\usepackage[linewidth=1pt]{mdframed}

\usepackage{import}
\usepackage{xifthen}
\usepackage{pdfpages}
\usepackage{transparent}

\newcommand{\incfig}[1]{%
    \def\svgwidth{\columnwidth}
    \import{./figures/}{#1.pdf_tex}
}

\newtheorem{theorem}{Theorem}[section]
\newtheorem{lemma}{Lemma}[section]
\newtheorem*{remark}{Remark}
\theoremstyle{definition}
\newtheorem{definition}{Definition}[section]

\begin{document}

\title{Equilibria in Non-linear Systems - Notes}
\author{Philip Warton}
\date{\today}
\maketitle
We want to look at systems of ODE's that are non-linear, but that resemble linearized ones near equilibria.
Consider the syste:
\begin{align*}
    x' &= x + y^2 \\
    y' &= -y
\end{align*}
Note that when $y$ is small, $y^2$ is much smaller. So it follows that as we make $y$ very
small the system $x' = x + y^2$ will vaguely `converge' or resemble $x' = x$. So instead we
look at the system 
\begin{align*}
    x' &= x \\
    y' &= -y
\end{align*}
Then we know that the bottom equation is solved by 
\[
    y(t) = y_0 e^{-t}
\]
So we then plug this in to our non-linear system, giving us
\[
    x' = x + y_0^2 e^{-2t}
\]
Recall from good old diffEQ's that we can plug in a ``guess'' solution of $ce^{-2t}$
and yield a particular solution to this first order nonautonomous equation. Namely,
\[
    x(t) = -\frac{1}{3}y_0^2e^{-2t}
\]
So we get a general solution of
\begin{align*}
    x(t) & = \left(x_0 + \frac{1}{3}y_0^2 \right)e^t - \frac{1}{3}y_0^2 e^{-2t}\\
    y(t) & = y_0 e^{-t}
\end{align*}
\fbox{Polar Coordinates Example}
Take the following system,
\begin{align*}
    x' &= \frac{1}{2}x - y - \frac{1}{2}(x^3 + y^2 x) \\
    y' &= x + \frac{1}{2}y - \frac{1}{2}(y^3 + x^2 y)
\end{align*}
Then if you simply drop the non-linear terms you get the system,
\[
    \begin{bmatrix}
        x'(t)\\y'(t)
    \end{bmatrix} =
    \begin{bmatrix}
        \frac{1}{2}&-1\\1&\frac{1}{2}
    \end{bmatrix}
   \begin{bmatrix}
        x(t)\\y(t)
    \end{bmatrix}
\]
This clearly gives us a spiral source set of solutions, which can be checked by taking
the eigenvalues. But now we want to solve the non-linear system. Oh shit.
\end{document}