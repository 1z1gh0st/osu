\documentclass{article}

\usepackage{times}
\usepackage{amssymb, amsmath, amsthm}
\usepackage[margin=.5in]{geometry}
\usepackage{graphicx}
\usepackage[linewidth=1pt]{mdframed}

\usepackage{import}
\usepackage{xifthen}
\usepackage{pdfpages}
\usepackage{transparent}

\newcommand{\incfig}[1]{%
    \def\svgwidth{\columnwidth}
    \import{./figures/}{#1.pdf_tex}
}

\newtheorem{theorem}{Theorem}[section]
\newtheorem{lemma}{Lemma}[section]
\newtheorem*{remark}{Remark}
\theoremstyle{definition}
\newtheorem{definition}{Definition}[section]

\begin{document}

\title{Systems of ODE's - Homework 5}
\author{Philip Warton}
\date{\today}
\maketitle
\section*{Problem 8.1 (iii)}
Take the following system:
\begin{align*}
    x'&=x+y^2\\y'&=2y
\end{align*}
\subsection*{a)}
Find all of the equilibrium points and describe the behavior of the associated linearized system.\\\\
So we want solutions such that both $x' = y' = 0$.
If we want $y' = 0$ then we must have $2y = 0 \Longrightarrow y = 0$.
Then once we have $y=0$ it follows that for $x' = x + y^2 = 0$ to also 
hold we must have $x' = x + y^2 = x + 0^2 = 0$ therefore $x = 0$.
So the one and only equilibrium point is at $(0,0)$. Having found our equilibrium
point, now we wish to linearize the system, and observe the behavoir of this system.
Let
\begin{align*}
    f(x,y)&=x+y^2\\g(x,y)&=2y
\end{align*}
Then we perform a (trivial) change of variables, where 
$(a,b)$ is an equilibrium point, we let
\begin{align*}
    u&=x-a\\v&=y-b
\end{align*}
So in our case $(u,v) = (x,y)$. Then we compute
\begin{align*}
    u'&=f_x(0,0)u+f_y(0,0)v\\
    v'&=g_x(0,0)u+g_y(0,0)v\\\\
    \Longleftrightarrow
    u'&=u\\
    v'&=2v
\end{align*}
This gives us the linearized system $U' = \begin{bmatrix}
    1&0\\0&2
\end{bmatrix}U$. Clearly we have positive real and distinct eigenvalues 1 and 2.
Or we can compute our trace and determinant to be $T = 1 + 2 = 3$ and $D = 1 \cdot 2 = 2$.
Then $\Delta = T^2 - 4D = 9 - 8 = 1$. There will obviously be a nodal source behavoir
of this system if we were to draw a phase portrait.
\subsection*{b)}
For the non-linear system, notice that we can take the equilibrium solutions
$2y = 0$ and $x + y^2 = 0$. We get one solution $\langle x(t), y(t) \rangle = \langle Ce^{-t}, 0\rangle$.
Then we also get another solution that follows the parabola $x + y^2 = 0$ that must 
be oriented towards the origin. We can draw parallels between this and the nodal source 
linear equilibrium solutions, which lie upon the x and y axis. Then,
we can imagine bending the y-axis into this parabola $x + y^2 = 0$, and this will
give us a vague idea of the behavior of the non-linearized system.
\subsection*{c)}
Yes, the linearized system does resemble the non-linearized system near the origin.
Notice that the parabola $x + y^2 = 0$ is tangential to the $y$-axis as the distance from
the origin approaches 0. Then also the solution where $2y = 0$ is indentical to the 
solution to the linearized system everywhere including near the origin.
\section*{Problem 8.5}
\subsection*{a)}
We have the system of differential equations,
\begin{align*}
    x'&=x^2+y\\y'&=x-a+a
\end{align*}
We first wish to find the equilibrium points. That is, when both $x'$ and $y'$ are equal to 0.
To do this first we can write 
\[
    y' = x-y+a = 0\ \ \ \ \Longleftrightarrow \ \ \ \ y = x + a
\]
Then, we can put this into our equation for $x'$, giving us 
\begin{align*}
    x' &= x^2 + y\\
    &=x^2 + x + a
\end{align*}
Since this is simply a quadratic function, it has roots given by the quadratic formula. It 
follows that we have equilibrium points at 
\[
    \left(\frac{-1 \pm \sqrt{1-4a}}{2}, \frac{-1 \pm \sqrt{1-4a}}{2} + a\right) = \{p_1,p_2\} \subset \mathbb{R}^2
\]
Both equilibrium only exist when $a < \frac{1}{4}$.We can linearize the system by doing an exchange of variables giving us 
\[
    U' = \begin{bmatrix}
        2x&1\\1&-1
    \end{bmatrix}U
\]
Simply plug in $x_1, x_2$ to compute the system for each respective equilibrium point.
\subsection*{b)}
Now we will compute the trace, determinant, and big delta of this linear system at both
equilibrium points $p_1, p_2$. We write
\begin{align*}
    T&=2x-1\\D&=-2x-1\\ \Delta&=4x^2 + 4x + 5
\end{align*}
So for $x = \frac{-1+\sqrt{1-4a}}{2}$ (i.e. for $p_1$), we have
\begin{align*}
    T&=-2+\sqrt{1-4a}\\
    D&=-\sqrt{1-4a}\\
    \Delta&=5-4a
\end{align*}
So at $p_1$ we are guaranteed a negative determinant, which means that around this point
there will be saddle behavior.\\\\
Then if $x = \frac{-1-\sqrt{1-4a}}{2}$, or at $p_2$,
\begin{align*}
    T&=-2-\sqrt{1-4a}\\
    D&=\sqrt{1-4a}\\
    \Delta&=5-4a
\end{align*}
Here our determinant is positive, so we are in the upper half of the trace-determinant plane.
Then, our trace is guaranteed to be negative, so we are in the upper left quadrant of the trace
-determinant plane. Then if $a < \frac{5}{4}$ (which is guaranteed when $p_2$ exists) then we are guaranteed $\Delta > 0$, that is, we have 
real and distinct eigenvalues, thus we have nodal source resembling behavior around $p_2$.
\section*{Problem 8.8}
We consider the system given by
\begin{align*}
    r'&=r-r^3\\
    \theta'&=\sin^2\theta + a
\end{align*}
Clearly we have an equilibrium point at the origin.
Nearby to this point we have a spiral source for any initial value with $0 < r < 1$
. For $a \leqslant -1$, for any initial value not at the origin, we have $r \rightarrow 1$ as $t \rightarrow \infty$, meaning this 
system resembles the one presented by the hopf bifurcation. For any solution with an initial value 
of $r = 1$, we have a solution that simply travels clockwise along the unit circle. If we observe $a \in (-1,0)$,
we notice that two equilibrium points appear in our phase portrait, at $(1, 3\pi /4), (1, 7\pi / 4)$ that 
locally exibit nodal source behavior.
\end{document}