\documentclass{article}

\usepackage{times}
\usepackage{amssymb, amsmath, amsthm}
\usepackage[margin=.5in]{geometry}
\usepackage{graphicx}
\usepackage[linewidth=1pt]{mdframed}

\usepackage{import}
\usepackage{xifthen}
\usepackage{pdfpages}
\usepackage{transparent}

\newcommand{\incfig}[1]{%
    \def\svgwidth{\columnwidth}
    \import{./figures/}{#1.pdf_tex}
}

\newtheorem{theorem}{Theorem}[section]
\newtheorem{lemma}{Lemma}[section]
\newtheorem*{remark}{Remark}
\theoremstyle{definition}
\newtheorem{definition}{Definition}[section]

\begin{document}

\title{Systems of ODE's - Notes}
\author{Philip Warton}
\date{\today}
\maketitle
\section{Solutions to systems of ODE's}
Suppose we have an autonomous system 
\[x'(t) = F(x(t))\]
The solution $x = x(t) = (x_1(t), x_2(t), \cdots , x_n(t))^T$ is a function
\[X: I \rightarrow \mathbb{R}^n \]
\begin{mdframed}[]
    \begin{definition}
        An equilibrium solution of $X' = F(x)$ is a constant solution, i.e., it takes the form
        \[X(t) = X^* \in \mathbb{R}^n \ \ \ \ \forall t \in I \]
        where $x^* = (x_1^*, x_2^*, \cdots ,x_n^*)^T \in \mathbb{R}^n$ is a constant vector.
    \end{definition}
\end{mdframed}
The implication of this is that $X'(t) = 0 \Rightarrow F(x^*) = 0$. Sometimes these are called transient solutions.
\\\\
\fbox{Exponential growth/decay ODE:}
\[ x' := \frac{dx}{dt} = rx, r \in \mathbb{R} \]
This is solved using the seperation of variables technique, that is
\begin{align}
    \frac{dx}{dt} &= rx \\
    dx &= rx dt \\
    x^{-1} dx &= r dt \\
    \int x^{-1} dx &= \int r dt \\
    \ln(x) + c_0 &= rt + c_1 \\
    \ln(x) &= rt + c \\
    x &= Ke^{rt}
\end{align}
This is a one parameter $(K)$ family of solutions ($\forall K \in \mathbb{R}$ or $\mathbb{R}^+ \cup \{0\}$).
If we are given some initial value such as $x_0 = x(0)$ then we have one unique solution based on this value.
Specifically we have the solution $x(t) = x_0 e^{rt}$. If we ahve $x_0 = 0$ then $x(t) = 0$ for all $r \in \mathbb{R}$.
This constant solution is called an equilibrium or steady state solution. Equilibrium solutions or equilibria are constant solutions to the ODE.
\\\\
\fbox{Logistic Initial Value Problem}
\[x' = rx(1 - \frac{x}{K}), r > 0, K > 0\]
Where $x(0) = x_0, x_0 \in \mathbb{R}^+ \cup \{0\}$.
The solution to this can be derived using seperation of variables, giving us $x(t) = \frac{Kx_0}{x_0 + (k - x_0)e^{-rt}}$.
If $x_0 = 0$ then we have an equilibria of $x(t) = 0$. If we have $x_0 = K$ we also have a constant solution $x(t) = K$.
\section{Systems}
Let $n \in \mathbb{N}$. What are the equilibrium solutions of the linear system? That is,
\[X' = AX\]
By definition and its consequence equilibrium solutions are solutions to $AX = 0$.
If $\det(A) \neq 0$, then $X(t) = 0$ is the only equilibrium solution.
We look at planar systems of linear ODE's. Then we break it down into several cases,
based on our different eigenvalues for a $A \in M_{2 \times 2}(\mathbb{R})$ matrix.
Our main cases are 
\begin{align}
    \text{Real and distinct eigenvalues} & \\
    \text{Real repeated eigenvalues} & \\
    \text{Complex eigenvalues}
\end{align}
We often get a solution of the form $X(t) = \begin{bmatrix}
    \alpha e ^{\lambda_1 t} \\ \beta e^{\lambda_2 t}
\end{bmatrix}$.
Understanding the phase plane and portrait.
\end{document}