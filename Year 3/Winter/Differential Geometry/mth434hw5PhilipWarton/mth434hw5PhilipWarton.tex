\documentclass{article}

\usepackage{times}
\usepackage{amssymb, amsmath, amsthm}
\usepackage[margin=.5in]{geometry}
\usepackage{graphicx}
\usepackage[linewidth=1pt]{mdframed}

\usepackage{import}
\usepackage{xifthen}
\usepackage{pdfpages}
\usepackage{transparent}

\newcommand{\incfig}[1]{%
    \def\svgwidth{\columnwidth}
    \import{./figures/}{#1.pdf_tex}
}

\newtheorem{theorem}{Theorem}[section]
\newtheorem{lemma}{Lemma}[section]
\newtheorem*{remark}{Remark}
\theoremstyle{definition}
\newtheorem{definition}{Definition}[section]

\begin{document}

\title{Differential Geometry - Homework 5}
\author{Philip Warton}
\date{\today}
\maketitle
\section*{Problem 1}
Integration on the sphere. Consider $\mathbb{S}^2$, which can be viewed as the surface in $\mathbb{E}^3$ with 
the line element $ds^2 = r^2(d\theta^2+\sin^2\theta d\phi^2)$.
\subsection*{a)}
\begin{mdframed}
    Let $\omega$ be the orientation of $\mathbb{S}^2$. Determine $\int_{\mathbb{S}^2}\omega$.    
\end{mdframed}
Based on the line element, we determine that the the $d\vec{r}$ vector is,
\begin{align}
    d\vec{r} = r(d\theta \hat{\theta} + \sin\theta d\phi \hat{\phi})
\end{align}
Using this fact we can conclude that the coefficients here provide an orthonormal basis for
our space $\mathbb{S}^2$ so we write
\begin{align}
    \omega &= rd\theta \wedge r\sin\theta d\phi
\end{align}
Now we can simply compute the integral,
\begin{align}
    \int_{\mathbb{S}^2}\omega & = \int_{\mathbb{S}^2} rd\theta \wedge r\sin\theta d\phi \\
    &=\int_0^{2\pi}\int_0^\pi r^2\sin\theta d\theta d\phi \\
    &=\int_0^{2\pi}-r^2\cos\theta\bigg|_0^\pi d\phi \\
    &=\int_0^{2\pi}2r^2d\phi\\
    &=4\pi r^2
\end{align}
This seems like it is the correct answer, as it is the formula for the surface area of the
sphere. Since we are integrating the orientation over the entire sphere this appears logical.
\subsection*{b)}
\begin{mdframed}
    Let $\alpha \in \bigwedge^1(\mathbb{S}^2)$. Compute $\int_{\mathbb{S}^2}d\alpha$.
\end{mdframed}
By Stokes' theorem we have,
\begin{align}
    \int_{\mathbb{S}^2}d\alpha &= \int_{\partial\mathbb{S}^2}\alpha \\
    &=\int_{\emptyset}\alpha \\
    &= 0
\end{align}
\subsection*{c)}
\begin{mdframed}
    Find $\alpha \in \bigwedge^1(\mathbb{S}^2)$ such that $d\alpha = \omega$.
\end{mdframed}
Write $\alpha = fd\theta + gd\phi$. Then we can write 
\begin{align}
    d\alpha &=\frac{\partial f}{\partial\theta}d\theta \wedge d\phi + \frac{\partial g}{\partial \phi}d\phi \wedge d\theta
\end{align}
Choose $-r^2\cos\theta d\theta = \alpha$. Then,
\begin{align}
    d\alpha &= r^2 \sin\theta d\theta \wedge d\phi \\
    &= rd\theta \wedge r\sin\theta d\phi\\
    &= \omega
\end{align}
\subsection*{d)}
Why is this possible?
We know that $\omega$ has an anti-derivative from part \fbox{a}. So it follows that there exists some
1-form whose derivative is equal to $\omega$.
\end{document}