\documentclass{article}

\usepackage{times}
\usepackage{amssymb, amsmath, amsthm}
\usepackage[margin=.5in]{geometry}
\usepackage{graphicx}
\usepackage[linewidth=1pt]{mdframed}

\usepackage{import}
\usepackage{xifthen}
\usepackage{pdfpages}
\usepackage{transparent}

\newcommand{\incfig}[1]{%
    \def\svgwidth{\columnwidth}
    \import{./figures/}{#1.pdf_tex}
}

\newtheorem{theorem}{Theorem}[section]
\newtheorem{lemma}{Lemma}[section]
\newtheorem*{remark}{Remark}
\theoremstyle{definition}
\newtheorem{definition}{Definition}[section]

\begin{document}

\title{Differential Geoemetry - Homework 1}
\author{Philip Warton}
\date{\today}
\maketitle
Let $\vec{u} = u_x \hat{x} + u_y \hat{y} + u_z \hat{z} \in \mathbb{R}^3$. Determine two vectors
$\vec{v}$ and $\vec{w}$ such that $\vec{u} = \vec{v} \times \vec{w}$.\\\\

We know that for two vectors $\vec{v}, \vec{w} \in \mathbb{R}^3$, the cross product is defined as
\[
    \vec{v} \times \vec{w} = det\begin{bmatrix}
        \hat{x}&\hat{y}&\hat{z}\\
        v_x&v_y&v_z\\
        w_x&w_y&w_z
    \end{bmatrix} = \hat{x}(v_yw_z - v_zw_y) - \hat{y}(v_xw_z - v_zw_x) + \hat{z}(v_xw_y - v_yw_x)
\]
And, we know also that the cross product is orthogonal to both of the input vectors.
\\\\
In the case that $\vec{u} = 0$, simply choose $\vec{v} = 0 = \vec{w}$ (this is only one of many possible solutions).
Otherwise $\vec{u} \neq 0$. In this scenario, we know that for any two vectors that lie on the plane $0 = u_x x + u_y y + u_z z$
(that is, a plane whose normal vector is $\vec{u}$),
both will be orthogonal to $\vec{u}$. We choose $\vec{v_0}, \vec{w_0}$ to be two orthogonal unit vectors that lie on this plane. Then,
let $\vec{v} = \vec{v_0}$, and let $\vec{w} = ||\vec{u}|| \vec{w_0}$. As a result, we will have either $\vec{u} = \vec{v} \times \vec{w}$ or 
$-\vec{u} = \vec{v} \times \vec{w}$. This is because of the property of cross products where $||a \times b|| = ||a|| \ ||b|| 
\ |\sin \theta|$ where $\theta$
is the angle between $a$ and $b$. Since we chose two vectors orthogonal to each other $\sin \theta = 1$. In the case where $-\vec{u} = \vec{v} \times \vec{w}$, without loss of generality, relabel $\vec{v}$ and $\vec{w}$ so that they are swapped,
and then we will have $\vec{u} = \vec{v} \times \vec{w}$ by the anti-commutative property of cross products.
\end{document}