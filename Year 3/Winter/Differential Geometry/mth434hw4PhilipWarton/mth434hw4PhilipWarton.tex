\documentclass{article}

\usepackage{times}
\usepackage{amssymb, amsmath, amsthm}
\usepackage[margin=.5in]{geometry}
\usepackage{graphicx}
\usepackage[linewidth=1pt]{mdframed}

\usepackage{import}
\usepackage{xifthen}
\usepackage{pdfpages}
\usepackage{transparent}

\newcommand{\incfig}[1]{%
    \def\svgwidth{\columnwidth}
    \import{./figures/}{#1.pdf_tex}
}

\newtheorem{theorem}{Theorem}[section]
\newtheorem{lemma}{Lemma}[section]
\newtheorem*{remark}{Remark}
\theoremstyle{definition}
\newtheorem{definition}{Definition}[section]

\begin{document}

\title{Differential Geometry - Homework 3b}
\author{Philip Warton}
\date{\today}
\maketitle
\section*{Problem 1}
\subsection*{(a)}
\begin{mdframed}
    Determine the Hodge dual operator $*$ on all forms by computing its action on basis
    forms at each rank.
\end{mdframed}
Lets begin with the 0-form.
\begin{align*}
    1 \wedge * 1 &= g(1,1) dx \wedge dy \wedge dz \wedge dt \\
    \Longrightarrow *1 & = dx \wedge dy \wedge dz \wedge dt
\end{align*}
Now we do each of the 1-forms. We can determine the sign, using the inner product and counting
how many times we must commute wedges to get our LHS wedge product factored out of the RHS.
\begin{align*}
    dx \wedge * dx & = g(dx,dx) dx \wedge dy \wedge dz \wedge dt \\
    *dx & = dy \wedge dz \wedge dt \\
    *dy & = -dx \wedge dz \wedge dt \\
    *dz & = dx \wedge dy \wedge dt \\
    *dt & = dx \wedge dy \wedge dz
\end{align*}
Before moving onto 2-forms, lets compute the inner products we will need:
\begin{align*} 
    g(dx \wedge dy, dx \wedge dy) & = \det\begin{bmatrix}
        g(dx, dx) & g(dx, dy) \\
        g(dy, dx) & g(dy, dy)
    \end{bmatrix}\\
    &= \det \begin{bmatrix}
        1 & 0 \\
        0 & 1
    \end{bmatrix}\\
    &= 1
\end{align*}
\begin{align*}
    g(dx \wedge dz, dx \wedge dz) & = \det \begin{bmatrix}
        g(dx, dx) & g(dx, dz) \\
        g(dz, dx) & g(dz, dz)
    \end{bmatrix} \\
    &= \det \begin{bmatrix}
        1 & 0\\0 & 1
    \end{bmatrix}\\
    &= 1
\end{align*}
We can see that clearly any pair involving $dx,dy,dz$ will have an inner product of 1.
Any pair including $dt$ will have an inner product of $-1$. Now we move on to computing the 
Hodge operator on 2-forms:
\begin{align*}
    *(dx \wedge dy) & = dz \wedge dt \\
    *(dx \wedge dz) & = -dy \wedge dt \\
    *(dx \wedge dt) & = -dy \wedge dz \\
    *(dy \wedge dz) & = dx \wedge dt \\
    *(dy \wedge dt) & = dx \wedge dz \\
    *(dz \wedge dt) & = -dx \wedge dy
\end{align*}
Before attending to the 3-forms, we must compute the inner product of each triple.
We use the following
\begin{align*}
    g(dx \wedge dy \wedge dz,dx \wedge dy \wedge dz) = g((dx \wedge dy) \wedge dz, (dx \wedge dy) \wedge dz) & = \det \begin{bmatrix}
        g((dx \wedge dy), (dx \wedge dy)) & g((dx \wedge dy),dz)\\g(dz, (dx \wedge dy)) &g(dz, dz)
    \end{bmatrix}\\
    &= \det\begin{bmatrix}
        1&0\\0&1
    \end{bmatrix}
    & = 1
\end{align*}
We will see that any inner product involving $dt$ will gather a minus sign, while each other inner product will not.
So we compute the hodge operator on 3-forms as follows:
\begin{align*}
    *(dx \wedge dy \wedge dz) &= dt \\
    *(dx \wedge dy \wedge dt) &= dz \\
    *(dx \wedge dz \wedge dt) &= -dy \\
    *(dy \wedge dz \wedge dt) &= dx
\end{align*}
Finally for our 4-form, we compute $g(\omega, \omega) = -1$ since it will be the product of $g(dx \wedge dy \wedge dz, dx \wedge dy \wedge dz)$ and $g(dt, dt)$ which will of course be -1.
So we say that
\begin{align*}
    *(dx \wedge dy \wedge dz \wedge dt) &= -1
\end{align*}
Or equivalently
\[
    *\omega = -1
\]
\subsection*{(b)}
If we change our orientation so that $\omega = dt \wedge dx \wedge dy \wedge dz$,
by the anti-commutativity of the wedge product, we can notice that
\[
    dx \wedge dy \wedge dz \wedge dt = -dt \wedge dx \wedge dy \wedge dz
\]
So using the fact that $\alpha \wedge * \alpha = g(\alpha, \alpha) \omega$
it follows that if our bases are the same and the sign of $\omega$ is flipped,
we can simply change the sign on our Hodge operator computations.
\end{document}