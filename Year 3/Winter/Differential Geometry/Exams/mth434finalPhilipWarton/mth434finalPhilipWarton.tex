\documentclass{article}

\usepackage{times}
\usepackage{amssymb, amsmath, amsthm}
\usepackage[margin=.5in]{geometry}
\usepackage{graphicx}
\usepackage[linewidth=1pt]{mdframed}

\usepackage{import}
\usepackage{xifthen}
\usepackage{pdfpages}
\usepackage{transparent}

\newcommand{\incfig}[1]{%
    \def\svgwidth{\columnwidth}
    \import{./figures/}{#1.pdf_tex}
}

\newtheorem{theorem}{Theorem}[section]
\newtheorem{lemma}{Lemma}[section]
\newtheorem*{remark}{Remark}
\theoremstyle{definition}
\newtheorem{definition}{Definition}[section]

\begin{document}

\title{Intro to Differential Geoemetry - Final Exam}
\author{Philip Warton}
\date{\today}
\maketitle
\section*{Problem 1}
\subsection*{a)}
Express the triple product $T(\vec{u},\vec{v},\vec{w}) = (\vec{u} \times \vec{v}) \cdot \vec{w}$ in differential forms.
\\\\
If $\vec{a}$ is a vector $\vec{a} = a_x \hat x + a_y \hat y + a_z \hat z$ we say that it is 
equivalent to the differential form $a = a_x dx + a_y dy + a_z dz \in \bigwedge^1 (\mathbb{R}^3)$.
Then we say that 
\[
(\vec u \times \vec v) \cdot \vec w = (\star (u \wedge v)) \cdot \vec w = \star((\star(u \wedge v)) \wedge \star w)
\]
\subsection*{b)}
Show that $(\vec u \times \vec v) \cdot \vec w = \vec u \cdot (\vec v \times \vec w)$.
\begin{proof}
    We begin by rewriting $(\vec u \times \vec v) \cdot \vec w = \vec w \cdot (\vec u \times \vec v)$.
    This is the case because the dot product in $\mathbb{R}^3$ is of course symmetric. This can now be written 
    in differnetial froms as 
    \[
        \star ( w \wedge \star ( \star (u \wedge v)))
    \]
    Then, since $\star \star \alpha = (-1)^{k(n-k)} \alpha = \alpha$, we write 
    \begin{align*}
        (\vec u \times \vec v) \cdot \vec w &= \star ( w \wedge (u \wedge v))\\
        &=\star ( w \wedge u \wedge v) & \text{(by associativity of the wedge)}\\
        &=\star (u \wedge v \wedge w) & \text{(by antisymmetry of the wedge)}\\
        &= \star(u \wedge (v \wedge w))\\
        &= \star(u \wedge \star(\star(v \wedge w)))\\
        &= \vec u \cdot (\vec v \times \vec w)
    \end{align*}
\end{proof}
\section*{Problem 2}
\subsection*{a)}
\subsubsection*{$\beta_1$}
The form is closed. That is,
\begin{align*}
    d\beta_1 &= d(xdy \wedge dz - y dz \wedge dx) \\
    &= (1)dx \wedge dy \wedge dz - (1)dy \wedge dz \wedge dx \\
    &= dx \wedge dy \wedge dz - dx \wedge dy \wedge dz & \text{(antisymmetry)} \\
    &= 0
\end{align*}
\subsubsection*{$\beta_2$}
The form is not closed. That is,
\begin{align*}
    d\beta_2 &= d(xdy \wedge dz + y dz \wedge dx) \\
    &= (1)dx \wedge dy \wedge dz + (1)dy \wedge dz \wedge dx \\
    &= dx \wedge dy \wedge dz + dx \wedge dy \wedge dz & \text{(antisymmetry)} \\
    &= 2 dx \wedge dy \wedge dz
\end{align*}
\subsection*{b)}
\subsubsection*{$\beta_1$}
This form may be exact since $d \beta_1 = 0$, otherwise we would violate the notion of $d^2 = 0$.
We know that to produce a $xdy \wedge dz$ term we must have either a $yxdz$ or $-zxdy$ term, and similarly
to produce a $ydz \wedge dx$ term we must have either a $zydx$ or $-xydz$ term. The combination that produces
the necessary cancellations is
\[
\alpha_1 = -zxdy - zydx + C dz
\]
Where $C \in \mathbb{R}$ is a constant. Taking the derivative of this will produce $\beta_1$.
\begin{align*}
    d \alpha_1 &= d(-zxdy) - d(zydx) + dCdz \\
    &= [xdy \wedge dz - z dx \wedge dy] - [-zdx \wedge dy + y dz \wedge dx] + [0] \\
    &= xdy\wedge dz - ydz \wedge dx\\
    &= \beta_1
\end{align*}
\subsubsection*{$\beta_2$}
This form cannot be exact. Suppose that $\exists \alpha_2$ such that $d\alpha_2 = \beta_2$.
Then $d(d\alpha_2) = d^2 \alpha_2 = d\beta_2 \neq 0$ (contradiction).
\section*{Problem 3}
\subsection*{a)}
Let our space be $\bigwedge^p(M)$ where $M$ is a 3-dimensional coordinate space with coordinates $t, \psi, \phi$.
Our orthonormal basis will be determined by the coefficients of our $d\vec{r}$ vector which can be 
accurately derived by observing the line element, 
\begin{align*}
    ds^2 &= -dt^2 + r^2(d\psi^2 + \sinh^2\psi d\phi^2)\\
    \Longrightarrow d\vec{r} &= -\hat t + r \hat \psi + r\sinh \psi \hat \phi
\end{align*}
So we write our orothonormal basis as $\{dt, rd\psi, r\sinh\psi d\phi\}$. Hence the orientation 
is the wedge product of these.
To compute all $\star$ operations in this space, let us begin with the obvious 0-form,
\begin{align*}
    \star 1 &= \omega
\end{align*}
Then before moving on let us express the metric tensor as a matrix of orthonormal basis, based upon the signs observed 
in the line element. That is,
\[
    g^{ij} = \begin{bmatrix}
        -1 &0&0\\0&1&0\\0&0&1
    \end{bmatrix}
\]
Where $i,j = dt, rd\psi,r\sinh\psi d\phi$. Then we can compute the hodge duals of our basis 1-forms,
\begin{align*}
    dt \wedge \star dt &= g^{dt,dt}r^2 \sinh\psi dt \wedge d\psi \wedge d\phi \\
    \star dt &= (-1)r^2 \sinh\psi d\psi \wedge d\phi\\\\
    rd\psi \wedge \star rd\psi &= g^{rd\psi, rd\psi}r^2 \sinh\psi dt \wedge d\psi \wedge d\phi \\
    \star rd\psi &= -r\sinh\psi dt \wedge d\phi \\\\
    r\sinh\psi d\phi \wedge \star r\sinh\psi d\phi  &= g^{r\sinh\psi d\phi, r\sinh\psi d\phi }r^2 \sinh\psi dt \wedge \psi \wedge d\phi \\
    \star r\sinh\psi d\phi &= rdt \wedge d\psi
\end{align*}
Now can move on to our hodge duals of basis 2-forms, which we can derive from our hodge duals of basis 1-forms, knowing 
that $(-1)^{k(n-k)} = 1$ since $k(n-k)$ is even as long as $n$ is odd (in 3 dimensions $n = 3$). So it follows that $\star \star \alpha = \alpha$.
So immediately we have 
\begin{align*}
    \star r \sinh \psi d\phi &= rdt \wedge d\psi &\Longrightarrow& \star r dt \wedge d\psi = r \sinh\psi d\phi \\
    \star rd\psi &= -r\sinh\psi dt \wedge d\phi &\Longrightarrow& \star r \sinh\psi dt \wedge d\phi = -rd\psi \\
    \star dt &= (-1)r^2 \sinh\psi d\psi \wedge d\phi &\Longrightarrow& \star r^2\sinh\psi d\psi \wedge d\phi = -dt
\end{align*}
Then finally for our basis 3-form, we say that $\star \omega = \star \star 1 = 1$.
\subsection*{b)}
We compute the Laplacian of a function $f, \Delta f$ by the relation $\Delta f = \star d \star d f$. We write,
\begin{align*}
    df &= \frac{\partial f}{\partial t} dt + \frac{\partial f}{\partial \psi} d\psi + \frac{\partial f}{\partial \phi} d\phi \\
    \star df &=\frac{\partial f}{\partial t} \star dt + \frac{\partial f}{\partial \psi} \star d\psi + \frac{\partial f}{\partial \phi} \star d\phi \\
    d \star df &=\frac{\partial^2 f}{\partial t^2} dt \wedge \star dt + \frac{\partial^2 f}{\partial \psi^2} d\psi \wedge \star d\psi + \frac{\partial^2 f}{\partial \phi^2}d\phi \wedge \star d\phi \\
    d \star df &=(-f_{tt} + f_{\psi\psi} + f_{\phi\phi})\omega \\
    \star d\star df &= -f_{tt} + f_{\psi\psi} + f_{\phi\phi} \\
    \Delta f &= -f_{tt} + f_{\psi\psi} + f_{\phi\phi}
\end{align*}
\section*{Problem 4}
We have the line element 
\[
    ds^2 = a^2\left(\frac{dX^2 + dY^2}{Y^2}\right)
\]
Which yields the $d\vec r$ vector as 
\[
    d\vec r = \frac{a}{Y}dX \hat X + \frac{a}{Y} dY \hat{Y}
\]
We take our orthonormal basis to be,
\[\left\{\sigma^X = \frac{a}{Y} dX, \ \ \ \ \sigma^Y = \frac{a}{Y} dY\right\}\]
With $\omega = \frac{a^2}{Y^2}dX \wedge dY$. By the structure equations we can derive the following,
\begin{align*}
    d\sigma^X &= (0)dX \wedge dX - \frac{a}{Y^2} dY \wedge dX = -\frac{a}{Y^2} dY \wedge dX\\
    d\sigma^Y &= (0)dX \wedge dY - \frac{a}{Y^2} dY \wedge dX = 0
\end{align*}
So then from $0 = d\sigma^X + \omega^X_{\ Y} \wedge \sigma^Y$ it follows that 
\begin{align*}
    -\frac{a}{Y^2} dY \wedge dX &= -\omega^X_{\ Y} \wedge \frac{a}{Y}dY \\
    \frac{1}{Y} dY \wedge dX &= \omega^X_{\ Y} \wedge dY \\
    \frac{1}{Y}dX &= \omega^X_{\ Y} 
\end{align*}
And then taking the derivative we have, \[
    d\omega^X_{\ Y} = \frac{-1}{Y^2}dX \wedge dY = K \frac{a^2}{Y^2} dX \wedge dY
    \] 
Which immediately yields $K = -1 / a^2$. This result seems correct because with an extremely large scaling 
factor $a$, our curvature will become much smaller, but we will always remain in negative curvature since it is a hyperbolic surface.
\section*{Problem 5}
\subsection*{a)}
We want to compute the integral over the interior of $S^3$. First let us derive our orientation and orthonormal basis from the 
line element. We get $\{dr, r d\psi, r\sin\psi d\theta, r \sin\psi \sin\theta d\phi\}$. Then we can write 
\[
\omega = r^3\sin^2\psi\sin\theta dr \wedge d\psi \wedge d\theta \wedge d\phi
    \]
Now notice that if we take the 3-form $\alpha = \frac{1}{4}r^4 \sin^2 \psi \sin \theta d\psi \wedge d\theta \wedge d\phi$,
its derivative is equal to $\omega$. So then, by Stokes theorem, we have 
\begin{align*}
    \int_{B^4} \omega = \int_{B^4}d\alpha &= \int_{S^3} \alpha
\end{align*}
Where $B^4$ is the closed 4-ball and $S^3$ is its boundary. So we can parameterize this integral as 
\begin{align*}
    & \int_0^{2\pi} \int_0^\pi \int_0^\pi \frac{1}{4}a^4 \sin^2 \psi \sin \theta d\psi \wedge d\theta \wedge d\phi \\
    =& \frac{1}{4}a^4 \int_0^{2\pi} \int_0^\pi \int_0^\pi \sin^2 \psi \sin \theta d\psi \wedge d\theta \wedge d\phi \\
    =& \left(\frac{\pi}{2}\right)\left(\frac{1}{4}a^4\right) \int_0^{2\pi} \int_0^\pi \sin \theta d\theta \wedge d\phi \\
    =& (2)\left(\frac{\pi}{2}\right)\left(\frac{1}{4}a^4\right) \int_0^{2\pi} d\phi \\
    =& (2\pi)(2)\left(\frac{\pi}{2}\right)\left(\frac{1}{4}a^4\right) \\
    =& \frac{\pi^2 a^4}{2}
\end{align*}
\subsection*{b)}
Now we want to compute 
\[
    \int_{S^3} r \star dr
\]
First let's compute $r \star dr$, which we can do by using the properties of the hodge dual. Namely,
\begin{align*}
    dr \wedge \star dr &= g(dr, dr) \omega \\
    &= dr \wedge r^3 \sin^2\psi \sin \theta d\psi \wedge d\theta \wedge d\phi \\
    \star dr &= r^3 \sin^2\psi \sin \theta d\psi \wedge d\theta \wedge d\phi \\
    r \star dr &= r^4 \sin^2\psi \sin \theta d\psi \wedge d\theta \wedge d\phi 
\end{align*}
Then of course we can integrate over the 3-sphere just as before, and one may notice that after factoring out the first
coefficient term, the integral becomes the same,
\begin{align*}
    & \int_0^{2\pi} \int_0^\pi \int_0^\pi r^4 \sin^2 \psi \sin \theta d\psi \wedge d\theta \wedge d\phi \\
    =& r^4 \int_0^{2\pi} \int_0^\pi \int_0^\pi \sin^2 \psi \sin \theta d\psi \wedge d\theta \wedge d\phi \\
    =& \left(\frac{\pi}{2}\right)\left(r^4\right) \int_0^{2\pi} \int_0^\pi \sin \theta d\theta \wedge d\phi \\
    =& (2)\left(\frac{\pi}{2}\right)\left(r^4\right) \int_0^{2\pi} d\phi \\
    =& (2\pi)(2)\left(\frac{\pi}{2}\right)\left(r^4\right) \\
    =& 2\pi^2 r^4
\end{align*}
\section*{Problem 6}
Intuitively, one might try to first normalize the basis elements and hope they become orthogonal. But 
this does not yield a correct answer. If we choose a basis of 
\[
    \left\{\cosh X dT, \cosh XdX\right\}   
\]
We are still left with non-orthogonal, but now normal, basis elements. The goal is to find a way to eliminate the 
$-\dfrac{\sinh X}{\cosh^2X}$ term from the metric $g(d\alpha, d\beta)$ where $\alpha,\beta$ are our basis elements.
In lieu of finding this, we resort to Graham Schmidtt orthogonalization,
\begin{align*}
    u_1 &= dT \\
    u_2 &= dX - proj_{dT}dX \\
    &= dX - \frac{g(dT,dX)}{g(dT,dT)}dT \\
    &= dX - \sinh X dT
\end{align*}
Then let $e_1 = u_1 /\sqrt{g(u_1,u_1)}  = \cosh X dT$ and $e_2 = u_2 / \sqrt{g(u_2,u_2)} = dX - \sinh X dT / \sqrt{g(u_2, u_2)}$.
We can compute this by using properties of inner products we know to be true. Computing this gets us,
\begin{align*}
    g(u_2, u_2) &= g(dX - \sinh X dT, dX - \sinh X dT) \\
    &= g(dX, dX - \sinh X dT) - \sinh X g(dT, dX - \sinh X dT) \\
    &= g(dX, dX) - \sinh X g(dX, dT) - \sinh Xg(dT, dX) + \sinh^2X g(dT,dT) \\
    &= \frac{1}{\cosh^2 X} - 2\sinh X \frac{-\sinh X}{\cosh^2 X} + \sinh^2 X \frac{-1}{\cosh^2 X} \\
    &= \frac{1 + 2 \sinh^2 X - \sinh^2 X}{\cosh^2 X} \\
    &= \frac{1 + \sinh^2 X}{\cosh^2 X}\\\\
    \sqrt{g(u_2, u_2)} &= \frac{\sqrt{1 + \sinh^2 X}}{\cosh X}\}
\end{align*}
So finally, we have an orthonormal basis given by,
\[
    \left\{\cosh X dT, \frac{(\cosh X)(dX - \sinh X dT)}{\sqrt{1 + \sinh^2 X}}\right\}    
\]
\subsection*{b)}
We compute $g(e_1,e_1), g(e_2,e_2)$ in order to find the signature. Firstly,
\begin{align*}
    g(\cosh X dT, \cosh X dT) &= \cosh^2 X g(dT,dT) = -1 \\
    g\left((\frac{(\cosh X)(dX - \sinh X dT)}{\sqrt{1 + \sinh^2 X}},\frac{(\cosh X)(dX - \sinh X dT)}{\sqrt{1 + \sinh^2 X}}\right) &= \frac{\cosh^2 X}{1 + \sinh^2 X} g(dX - \sinh X dT, dX - \sinh X dT) \\
    &= \frac{\cosh^2 X}{1 + \sinh^2 X} \left(\frac{1 + \sinh^2 X}{\cosh^2 X}\right)\\
    & = 1
\end{align*}
The signature is given by the number of basis elements whose inner product is negative, so for ours we have 
a signature of $s = 1$. Or if you wanted to write it as a plus/minus ordered pair, we would write $(1,1)$ as in 1 positive, 1 negative 
respectively. Or we could even write $(-,+)$, or write out the metric as a matrix 
\[
    g = \begin{bmatrix}
        -1&0\\0&1
    \end{bmatrix}
\]
As we did for a previous problem.
\subsection*{c)}
The line element is given by 
\begin{align*}
    ds^2 &= g_{ij}dx^idx^j
\end{align*}
So we write this out as
\begin{align*}
   ds^2 &= \frac{-1}{\cosh^2 X}dT^2 + 2\left(\frac{-\sinh X}{\cosh^2X}\right)dXdT + \frac{1}{\cosh^2X}dX^2
\end{align*}
\end{document}