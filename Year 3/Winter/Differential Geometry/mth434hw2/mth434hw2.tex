\documentclass{article}

\usepackage{times}
\usepackage{amssymb, amsmath, amsthm}
\usepackage[margin=.5in]{geometry}
\usepackage{graphicx}
\usepackage[linewidth=1pt]{mdframed}

\usepackage{import}
\usepackage{xifthen}
\usepackage{pdfpages}
\usepackage{transparent}

\newcommand{\incfig}[1]{%
    \def\svgwidth{.25\columnwidth}
    \import{./figures/}{#1.pdf_tex}
}

\newtheorem{theorem}{Theorem}[section]
\newtheorem{lemma}{Lemma}[section]
\newtheorem*{remark}{Remark}
\theoremstyle{definition}
\newtheorem{definition}{Definition}[section]

\begin{document}

\title{Differential Geometry - Homework 2}
\author{Philip Warton}
\date{\today}
\maketitle
\section*{Problem 1}
\subsection*{(a)}
\fbox{Show that all 2-forms in $\mathbb{R}^3$ are decomposable.}
\begin{proof}
    Let $H \in \bigwedge^2(\mathbb{R}^3)$.
    We assume that this 2-form is defined by some vector field $\vec{H}$, and some surface $S \subset \mathbb{R}^3$.
    Given our current coordinate system, we say $H = H_x dy \wedge dz + H_y dz \wedge dx + H_z dx \wedge dy$.
    We want to show that there exist a product of 1-forms that compose $H$. Let $F, G \in \bigwedge^1(\mathbb{R}^3)$
    be 1-forms where $F = F_xdx + F_ydy + F_zdz$ and $G = G_xdx + G_ydy + G_zdz$. Then, taking their wedge product, 
    we have the following:
    \[
        F \wedge G = (F_yG_z - F_z G_y)(dy \wedge dx) - (F_xG_z - F_zG_x)(dx \wedge dz) + (F_xG_y - F_yG_x)(dx \wedge dy)
    \]
    Then, to satisfy $H = F \wedge G$, we produce a system of equations that looks like
    \begin{align}
        H_x &= F_yG_z - F_zG_y \\
        H_y &= F_zG_x - F_xG_z \\
        H_z &= F_xG_y - F_yG_x
    \end{align}
    Notice that this system of equations is equivalent to the system of equations one must solve in order to ``undo'' the
    cross product. Given that we have shown this to be possible \fbox{Homework 1}, it follows that such 1-forms $F,G$
    exist, and therefore $H = F \wedge G$ is decomposable.
\end{proof}
\subsection*{(b)}
\fbox{Provide an example of a non-decomposable $p$-form.} \\\\
We wish to find some example of a non-decomposable $p$-form in some vector space $V$.
Let us look at $\mathbb{R}^4$. Call our dimensions $x_1, x_2, x_3, x_4$. We may have some 2-form,
\[
    u = dx_1 \wedge dx_2 + dx_3 \wedge dx_4
\]
Suppose that $u = v \wedge w$ for some pair of 1-forms, $v,w \in \bigwedge^1(\mathbb{R}^4)$.
Then we say that \begin{align*}
    u \wedge u &= (v \wedge w) \wedge (v \wedge w) \\
    &= v \wedge (w \wedge v) \wedge w \\
    &= v \wedge -(v \wedge w) \wedge w \\
    &= (v \wedge -v) \wedge (-w \wedge w) \\
    &= (v \wedge v \wedge -1) \wedge (-1 \wedge w \wedge w) \\
    &= 0 \wedge 0\\
    &= 0
\end{align*}
However, we can compute this same product $u \wedge u$ as
\begin{align*}
    u \wedge u &= (dx_1 \wedge dx_2 + dx_3 \wedge dx_4) \wedge (dx_1 \wedge dx_2 + dx_3 \wedge dx_4)\\
    &= 2(dx_1 \wedge dx_2 \wedge dx_1 \wedge dx_2) + 2(dx_1 \wedge dx_2 \wedge dx_3 \wedge dx_4) \\
    &= 0 + 2(dx_1 \wedge dx_2 \wedge dx_3 \wedge dx_4)
\end{align*}
Since $u \wedge u$ cannot be both equal to 0 and not equal to 0 we say that $u$ is not decomposable into a product of 1-forms.
\subsection*{(c)}
\fbox{Is $\gamma \wedge \gamma = 0$?}\\\\
It depends on the rank of the form $\gamma$, denoted by $p$. We make the argument
 using only the anti-symmetric property of the wedge product.
Since $a \wedge b = (-1)^{p^2}(b \wedge a)$, if we let $a,b = \gamma$, then
\begin{align*}
    \gamma \wedge \gamma &= (-1)^{p^2}(\gamma \wedge \gamma) \\
\end{align*}
If $p$ is odd, then we are guaranteed that $\gamma \wedge \gamma = 0$.
Otherwise, this is not guaranteed. For example, our $2$-form example from \fbox{1b.}
serves as an example of differential form that does not equal 0 when wedged with itself.
\section*{Problem 2}
Let $\alpha = 3dx, \beta = 4dy$.
\subsection*{(a)}
\begin{figure}[ht]
    \centering
    \incfig{1}
    \caption{``Stacks'' Diagram of $\alpha$ (green) and $\beta$ (red)}
    \label{fig:1}
\end{figure}
\subsection*{(b)}
\begin{figure}[ht]
    \centering
    \incfig{2}
    \caption{``Stacks'' Diagram of $\gamma = \alpha + \beta$}
    \label{fig:2}
\end{figure}
\subsection*{(c)}
Let $\vec{v} = \langle 6, 8 \rangle \in \mathbb{R}^2$.
\begin{figure}[ht]
    \centering
    \incfig{4}
    \caption{$\vec{v}$ (blue) on $\alpha$ (green) and $\beta$ (red)}
    \label{fig:4}
\end{figure}
From this diagram \fbox{Figure 3} we can see that $\vec{v}$ crosses exactly 2 of our green stacks,
and 2 of our red stacks. So we say that $\alpha(\vec{v}) = 2 = \beta(\vec{v})$.
\begin{figure}[ht]
    \centering
    \incfig{3}
    \caption{$\vec{v}$ (blue) on $\gamma$}
    \label{fig:3}
\end{figure}
From this second diagram \fbox{Figure 4} we have $\vec{v}$ crossing exactly 2 of our stacks, which
means that $\gamma(\vec{v}) = 2$.
\subsection*{(d)}
Since $\alpha(\vec{v}) + \beta(\vec{v}) = 2 + 2 = 4 \neq 2 = \gamma(\vec{v})$, we did not obtain $\alpha(\vec{v}) + \beta(\vec{v}) = \gamma(\vec{v})$.
It feels like we should have this relationship, but it turns out not to be the case.
I believe that this is because if we were to take one ``unit'' of $\alpha,\beta,$ and $\gamma$
as vectors, we do not have $||\alpha|| + ||\beta|| = ||\gamma||$
\end{document}