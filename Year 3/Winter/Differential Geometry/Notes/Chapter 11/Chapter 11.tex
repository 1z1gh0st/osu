\documentclass{article}

\usepackage{times}
\usepackage{amssymb, amsmath, amsthm}
\usepackage[margin=.5in]{geometry}
\usepackage{graphicx}
\usepackage[linewidth=1pt]{mdframed}

\usepackage{import}
\usepackage{xifthen}
\usepackage{pdfpages}
\usepackage{transparent}

\newcommand{\incfig}[1]{%
    \def\svgwidth{\columnwidth}
    \import{./figures/}{#1.pdf_tex}
}

\newtheorem{theorem}{Theorem}[section]
\newtheorem{lemma}{Lemma}[section]
\newtheorem*{remark}{Remark}
\theoremstyle{definition}
\newtheorem{definition}{Definition}[section]

\begin{document}

\title{Differential Geometry 1 Notes - Chapter 11}
\author{Philip Warton}
\date{\today}
\maketitle
\section{Differentials and Integrands}
We will write differentiation as $f \longmapsto df$, where all our typical rules
of derivatives apply. That is,
\begin{align*}
    d(u^n) & = nu^{n-1}du, & \text{(power rule)}\\
    d(e^u) & = e^u du, & \text{(exponential)}\\
    d(\sin u) & = \cos u du, & \text{(trigonemtric derivatives)}\\
    d(uv) & = udv + vdu, & \text{(product rule)}\\
    d(u + cv) & = du + cdv, & \text{(distributive property)}
\end{align*}
Finally we write the chain rule as
\[
    \frac{df}{du} = \frac{df}{dq}\frac{dq}{du}
\]
Alternatively
\[
    df = \frac{\partial f}{\partial p}dp + \frac{\partial f}{\partial q}dq
\]

\section{Differential Forms}
Consider $\mathbb{R}^n$ with coordinates $\{x^i\}$.
Consider the formal objects $\{dx^i\}$.
Let $V$ be the almost vector space,
\[
    V = \langle dx^i \rangle = \{a_i dx^i\}
\]
\end{document}