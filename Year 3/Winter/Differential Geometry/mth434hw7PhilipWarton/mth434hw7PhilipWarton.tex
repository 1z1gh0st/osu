\documentclass{article}

\usepackage{times}
\usepackage{amssymb, amsmath, amsthm}
\usepackage[margin=.5in]{geometry}
\usepackage{graphicx}
\usepackage[linewidth=1pt]{mdframed}

\usepackage{import}
\usepackage{xifthen}
\usepackage{pdfpages}
\usepackage{transparent}

\newcommand{\incfig}[1]{%
    \def\svgwidth{\columnwidth}
    \import{./figures/}{#1.pdf_tex}
}

\newtheorem{theorem}{Theorem}[section]
\newtheorem{lemma}{Lemma}[section]
\newtheorem*{remark}{Remark}
\theoremstyle{definition}
\newtheorem{definition}{Definition}[section]

\begin{document}

\title{Differential Geometry - Homework 7}
\author{Philip Warton}
\date{\today}
\maketitle
\section*{Problem 1}
\subsection*{a)}
From \fbox{Homework 6}, we write the following elements in $\mathbb{R}^3$,
\begin{align*}
    d\hat r & = \sin\theta d\phi\hat\phi + d\theta\hat\theta\\
    d\hat\theta&=\cos\theta d\phi\hat\phi - d\theta\hat r\\
    d\hat\phi&=-\cos\theta d\phi\hat\theta - \sin\theta d\phi\hat r
\end{align*}
If we drop the $\hat r$ components, then we get the resulting elments,
\begin{align*}
    d\hat\theta&=\cos\theta d\phi\hat\phi\\
    d\hat\phi&=-\cos\theta d\phi\hat\theta
\end{align*}
Then, to get our connection components, we know that 
\[
    \omega_{ij} = g_{ik}\omega^k_{\ j} = \omega^i_{\ j}
\]
Since $d\hat e_j = \omega^i_{\ j} \hat e_i$, it follows that
\begin{align*}
    d \hat \theta &= \omega^\phi_{\ \theta} d\hat\phi\\
    \Longrightarrow \omega^\phi_{\ \theta} &= \cos\theta d\phi \\\\
    d\hat\phi &= \omega^\theta_{\ \phi} d\hat\theta \\
    \Longrightarrow \omega^\theta_{\ \phi} &= -\cos\theta d\phi
\end{align*}
So writing all our connective componenets into a matrix
we can write this out as
\[
    \begin{bmatrix}
        \omega_{\theta \theta} & \omega_{\theta \phi} \\
        \omega_{\phi \theta} & \omega_{\phi \phi}
    \end{bmatrix} =
    \begin{bmatrix}
        \omega^{\theta}_{\ \theta} & \omega^{\theta}_{\ \phi} \\
        \omega^{\phi}_{\ \theta} & \omega^{\phi}_{\ \phi}
    \end{bmatrix} =
    \begin{bmatrix}
        0 & -\cos\theta d\phi \\
        \cos\theta d\phi & 0
    \end{bmatrix}
\]
\subsection*{b)}
Then we compute $\Omega^i_{\ j} = d\omega^i_{\ j} + \omega^i_{\ k} \wedge \omega^k_{\ j}$
for this part. First note that since we have only two coordinates, any term of the form 
\[
    \omega^i_{\ k} \wedge \omega^{k}_{\ j}
\]
Is guaranteed to have one of them be $\omega^k_{\ k} = 0$, and therefore the whole term
goes to 0. Thus we have $\Omega^i_{\ j} = d\omega^i_{\ j}$. From this we get 
\begin{align*}
    \Omega_{\theta \phi} &= d\omega^\theta_{\ \phi} = d(-cos\theta d\phi) = \sin\phi d\phi \wedge d\theta\\
    \Omega_{\phi \theta} &= d\omega^\phi_{\ \theta} = d(cos \theta d\phi) = -\sin\phi d\phi \wedge d\theta
\end{align*}
Note that $\Omega_{i i} = d\omega^i_{\ i} = d(0) = 0$ is guaranteed quite easily.
 Now we simply write these $\Omega$ as a matrix,
\[
    \begin{bmatrix}
        \Omega_{\theta \theta} &\Omega_{\theta \phi}\\
        \Omega_{\phi \theta} & \Omega_{\phi \phi}
    \end{bmatrix} = \begin{bmatrix}
        0 & \sin\phi d\phi \wedge d\theta \\
        -\sin\phi d\phi \wedge d\theta & 0
    \end{bmatrix}
\]
\end{document}