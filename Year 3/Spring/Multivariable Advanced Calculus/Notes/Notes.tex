\documentclass{article}

\usepackage{times}
\usepackage{amssymb, amsmath, amsthm}
\usepackage[margin=.5in]{geometry}
\usepackage{graphicx}
\usepackage[linewidth=1pt]{mdframed}

\usepackage{import}
\usepackage{xifthen}
\usepackage{pdfpages}
\usepackage{transparent}

\newcommand{\incfig}[1]{%
    \def\svgwidth{\columnwidth}
    \import{./figures/}{#1.pdf_tex}
}

\newtheorem{theorem}{Theorem}[section]
\newtheorem{lemma}{Lemma}[section]
\newtheorem*{remark}{Remark}
\theoremstyle{definition}
\newtheorem{definition}{Definition}[section]

\begin{document}

\title{Advanced Multivariable Calculus - Notes}
\author{Philip Warton}
\date{\today}
\maketitle
\section{Introduction to $\mathbb{R}^n$}
\begin{mdframed}[]
    \begin{definition}[$n$-dimensional Vector]
        An $n$-dimensional vector is an ordered tuple
        \[
            \vec{x} = (x_1, x_2, \cdots , x_n)
        \]
        With $x_i \in \mathbb{R}$ for every $i \in \{1,2,\cdots, n\}$.
    \end{definition}
\end{mdframed}
Then it is simple to say that $\mathbb{R}^n$ is the set of all $n$-dimensional vectors.
Given some scalar constant $c \in \mathbb{R}$ and a vector $\vec{x} \in \mathbb{R}^n$ we write 
\[
    c\vec{x} = (cx_1, cx_2, \cdots , cx_n)
\]
We also define a norm on $\mathbb{R}^n$ which is given by $||\vec{x}|| = \left(\sum_{i=1}^n x_i^2\right)^{1 / 2}$.
\begin{definition}[Dot Product]
    The dot product maps from $\mathbb{R}^n \times \mathbb{R}^n$ to $\mathbb{R}$.
    Given two vectors $\vec x , \vec y \in \mathbb{R}^n$ we define their dot product to be 
    $\vec x \cdot \vec y  = \sum_{i=1}^n x_i y_i$.
\end{definition}
Then we have $||\vec x|| = \sqrt{\vec x \cdot \vec x}$.
The angle between two vectors is given by 
\[
    \theta = \cos^{-1}\left(\frac{\vec x \cdot \vec y}{||\vec x|| \ ||\vec y ||}\right)
\]
\begin{mdframed}\begin{theorem}
    [Cauchy Shwarz Inequality]
    Let $\vec x , \vec y \in \mathbb{R}^n$. Then we say
    \[
        |\vec x \cdot \vec y| \leqslant ||\vec x|| \ || \vec y ||
    \]
\end{theorem}
\end{mdframed}
\section{Functions}
    \begin{definition}[Function]
        For $m,n \in \mathbb{N}, D \subset \mathbb{R}^n$, a function $F: D \rightarrow \mathbb{R}^m$ assigns to 
        each $\vec x \in D$ a unique point $\vec y \in \mathbb{R}^m$.
        We write $F(\vec x) = \vec y$. For each $\vec x \in D$, we can write 
        \[
            \vec y = F(\vec x) = (f_1(\vec x), f_2(\vec x), \cdots , f_n(\vec x))
        \]
        Where $f_j: D \rightarrow \mathbb{R} \ \ \ \forall j, 1 \leqslant j \leqslant m$.
    \end{definition}
We can call $f_j$ the $j$-th component function of $F$.
Of course $D$ is the domain of $F$ and $F(D)$ is the image of $F$.
Now there are lots of great examples of functions that map $\mathbb{R}^m \rightarrow \mathbb{R}^n$.
But we care mostly about the following property:
\begin{mdframed}
    \begin{definition}[Continuity]
        Let $D \subset \mathbb{R}^n$. Then $f:D \rightarrow \mathbb{R}^m$ is 
        continuous at $x \in D$ if given any $\epsilon > 0$ there exists some $\delta > 0$
        such that $||\vec{x} - \vec y|| < \delta$ implies $||f(\vec x) - f(\vec y)|| < \epsilon$.
    \end{definition}
\end{mdframed}
\end{document}