\documentclass{article}

\usepackage{times}
\usepackage{amssymb, amsmath, amsthm}
\usepackage[margin=.5in]{geometry}
\usepackage{graphicx}
\usepackage[linewidth=1pt]{mdframed}

\usepackage{import}
\usepackage{xifthen}
\usepackage{pdfpages}
\usepackage{transparent}
\usepackage{bm}

\newcommand{\incfig}[1]{%
    \def\svgwidth{.5\linewidth}
    \import{./figures/}{#1.pdf_tex}
}

\newtheorem{theorem}{Theorem}[section]
\newtheorem{lemma}{Lemma}[section]
\newtheorem*{remark}{Remark}
\theoremstyle{definition}
\newtheorem{definition}{Definition}[section]

\begin{document}

\title{Advanced Multivariable Calculus - Notes}
\author{Philip Warton}
\date{\today}
\maketitle
\section{Introduction to $\mathbb{R}^n$}
\begin{mdframed}[]
    \begin{definition}[$n$-dimensional Vector]
        An $n$-dimensional vector is an ordered tuple
        \[
            \vec{x} = (x_1, x_2, \cdots , x_n)
        \]
        With $x_i \in \mathbb{R}$ for every $i \in \{1,2,\cdots, n\}$.
    \end{definition}
\end{mdframed}
Then it is simple to say that $\mathbb{R}^n$ is the set of all $n$-dimensional vectors.
Given some scalar constant $c \in \mathbb{R}$ and a vector $\vec{x} \in \mathbb{R}^n$ we write 
\[
    c\vec{x} = (cx_1, cx_2, \cdots , cx_n)
\]
We also define a norm on $\mathbb{R}^n$ which is given by $||\vec{x}|| = \left(\sum_{i=1}^n x_i^2\right)^{1 / 2}$.
\begin{definition}[Dot Product]
    The dot product maps from $\mathbb{R}^n \times \mathbb{R}^n$ to $\mathbb{R}$.
    Given two vectors $\vec x , \vec y \in \mathbb{R}^n$ we define their dot product to be 
    $\vec x \cdot \vec y  = \sum_{i=1}^n x_i y_i$.
\end{definition}
Then we have $||\vec x|| = \sqrt{\vec x \cdot \vec x}$.
The angle between two vectors is given by 
\[
    \theta = \cos^{-1}\left(\frac{\vec x \cdot \vec y}{||\vec x|| \ ||\vec y ||}\right)
\]
\begin{mdframed}\begin{theorem}
    [Cauchy Shwarz Inequality]
    Let $\vec x , \vec y \in \mathbb{R}^n$. Then we say
    \[
        |\vec x \cdot \vec y| \leqslant ||\vec x|| \ || \vec y ||
    \]
\end{theorem}
\end{mdframed}
\section{Functions}
    \begin{definition}[Function]
        For $m,n \in \mathbb{N}, D \subset \mathbb{R}^n$, a function $F: D \rightarrow \mathbb{R}^m$ assigns to 
        each $\vec x \in D$ a unique point $\vec y \in \mathbb{R}^m$.
        We write $F(\vec x) = \vec y$. For each $\vec x \in D$, we can write 
        \[
            \vec y = F(\vec x) = (f_1(\vec x), f_2(\vec x), \cdots , f_n(\vec x))
        \]
        Where $f_j: D \rightarrow \mathbb{R} \ \ \ \forall j, 1 \leqslant j \leqslant m$.
    \end{definition}
We can call $f_j$ the $j$-th component function of $F$.
Of course $D$ is the domain of $F$ and $F(D)$ is the image of $F$.
Now there are lots of great examples of functions that map $\mathbb{R}^m \rightarrow \mathbb{R}^n$.
But we care mostly about the following property:
\begin{mdframed}
    \begin{definition}[Continuity]
        Let $D \subset \mathbb{R}^n$. Then $f:D \rightarrow \mathbb{R}^m$ is 
        continuous at $x \in D$ if given any $\epsilon > 0$ there exists some $\delta > 0$
        such that $||\vec{x} - \vec y|| < \delta$ implies $||f(\vec x) - f(\vec y)|| < \epsilon$.
    \end{definition}
\end{mdframed}

\section{Integration}
\subsection{Partitions on $\mathbb{R}^n$}
Let $I = I_1 \times I_2 \times \cdots \times I_k$ be a generalized rectangle, so $I_\ell = [a_\ell, b_\ell], 1 \leqslant \ell \leqslant k$.
For each $\ell$ between $1$ and $k$, let $P_\ell$ be a partition of $I_\ell$. The collection of 
generalized rectangles 
\[
    \{ J = J_1 \times \cdots \times J_\ell \times \cdots \times J_k \ | \ J_\ell \text{ is an interval in $P_\ell$}\}
\]
Is a partition of $I$, and is denoted $P = (P_1,P_2, \cdots , P_m)$.
\begin{figure}[ht]
    \centering
    \incfig{1}
    \caption{Generalized Rectangle Partition}
    \label{fig:1}
\end{figure}
For example, we can take this rectangle with the following partition: 
$P_1 = \{x_0,x_1,x_2,x_3\}, P_2 = \{y_0,y_1,y_2,y_3,y_4\}$ \fbox{Figure 1}.\\\\
The volume $Vol(I) = \sum_{J \in P}Vol(J)$.
    Let $I$ be a generalized rectangle. Let $P$ be a partition of $I$.
    Let $f:I \rightarrow \mathbb{R}$ be bounded. For $J$ a generalized sub-rectangle in $P$,
    let 
    \[
        M(f,J) = \sup\{f(\bm x) : \bm x \in J\}
    \]
    \[
        m(f,J) = \inf\{f(\bm x) : \bm x \in J\}
    \]
    The upper sum of $f$ with respect to $P$ is \[U(f,P) = \sum_{J \in P}M(f,J)Vol(J)\]
    And the lower sum is 
    \[
        L(f,P) = \sum_{J \in P} m(f,J)Vol(J)
    \]
We wish to find an upper bound for $L(f,P)$ and a lower bound for $U(f,P)$, so we use the following lemma.
\begin{mdframed}\begin{lemma}
    Let $f:I \rightarrow \mathbb{R}$ be a bounded function on a generalized rectangle $I$.
    Suppose 
    \[
        m \leqslant f(\bm x) \leqslant M \ \ \ \ \forall \bm x \in I
    \]
    Then for every partition $P$ of $I$,
    \[
        m \cdot Vol(I) \leqslant L(f,P) \leqslant U(f,P) \leqslant M \cdot Vol(I)
    \]
\end{lemma}\end{mdframed}
Now we can write the following definition to start thinking about integration in $\mathbb{R}^n$.
    \begin{definition}
        Let $f: I \rightarrow \mathbb{R}$ be bounded, and $I$ a generalized rectangle in $\mathbb{R}^n$.
        The lower integral of $f$ on $I$ is 
        \[
            \underline{\int_{I}}f = \sup\{L(f,P) : P \text{ partition of } I\}
        \]
        The upper integral of $f$ on $I$ is 
        \[
            \overline{\int_{I}}f = \inf\{U(f,P) : P \text{ partition of } I\}
        \]
    \end{definition}
So then we also need to get a definition for integrable functions, which we say is the following:
\begin{definition}
    Let $f: I \rightarrow \mathbb{R}$ be bounded where $I$ is a generalized rectangle.
    Then $f$ is integrable on $I$ if 
    \[
        \underline{\int_I}f = \overline{\int_I}f
    \]
    In which case we write 
    \[
        \int_I f = \underline{\int_I}f = \overline{\int_I}f
    \]
\end{definition}
For an example, let $I$ be a generalized rectangle.
Define a function $f:I \rightarrow \mathbb{R}$ as follows:
\[
    f(\bm x) = \begin{cases}
        1, & \bm x \text{ has a rational coordinate } \\
        0, & \text{otherwise}
    \end{cases}
\]
Then by the density of rational and irrational numbers in $\mathbb{R}$, for any partition $P$ of $I$,
it must be the case that 
\[
    L(f,P) = \sum_{J \in P}m(f,J) Vol(J) = 0
\]
and 
\[
    U(f,P) = \sum_{J \in P}M(f,J) Vol(J) = \sum_{J \in P} Vol(J) = Vol(I) \neq 0
\]
\end{document}