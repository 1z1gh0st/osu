\documentclass{article}

\usepackage{times}
\usepackage{amssymb, amsmath, amsthm}
\usepackage[margin=.5in]{geometry}
\usepackage{graphicx}
\usepackage[linewidth=1pt]{mdframed}

\usepackage{import}
\usepackage{xifthen}
\usepackage{pdfpages}
\usepackage{transparent}
\usepackage{bm}

\newcommand{\incfig}[1]{%
    \def\svgwidth{\columnwidth}
    \import{./figures/}{#1.pdf_tex}
}

\newtheorem{theorem}{Theorem}[section]
\newtheorem{lemma}{Lemma}[section]
\newtheorem*{remark}{Remark}
\theoremstyle{definition}
\newtheorem{definition}{Definition}[section]

\begin{document}

\title{Advanced Multivariable Calculus - Assignment 3}
\author{Philip Warton}
\date{\today}
\maketitle
\section*{Problem 1}
Let $f: \mathbb{R}^2 \rightarrow \mathbb{R}$ where
\[
    f(x,y) = \begin{cases}
        \frac{xy}{x^2 + y^2}, & (x,y)= 0\\
        0, & \text{otherwise}
    \end{cases}
\]
\subsection*{a)}
\begin{mdframed}
    Compute the partial derivatives of $f$ at $(x,y) = 0$.
\end{mdframed}
For the partial derivative with respect to $x$ at 0, we write,
\begin{align*}
    \frac{\partial f}{\partial x}(0,0) &= \lim_{h \rightarrow 0} \frac{f(0 + h, 0) - f(0,0)}{h}\\
    &= \lim_{h \rightarrow 0} \frac{\frac{h(0)}{h^2 + 0^2} - 0}{h}\\
    &= \lim_{h \rightarrow 0} \frac{0}{h} \\
    &= 0
\end{align*}
And since the function is identical if we switch variables $x$ and $y$, it follows that $\frac{\partial f}{\partial y}(0,0) = 0$ as well.
\subsection*{b)}
\begin{mdframed}
    Show that $f$ is not continuous at 0 and therefore not differentiable at 0.
\end{mdframed}
\begin{proof}
    Let $(x_n,y_n) = (\frac{1}{n},\frac{1}{n})$. Then it follows that $(x_n,y_n) \rightarrow 0$ as $n \rightarrow \infty$.
    But, if we take the limit of $f(x_n,y_n)$, we find that
    \begin{align*}
        \lim_{n \rightarrow \infty} f(x_n,y_n) &= \lim_{n\rightarrow \infty}\frac{(\frac{1}{n})(\frac{1}{n})}{(\frac{1}{n})^2 + (\frac{1}{n})^2}\\
        &= \lim_{n\rightarrow \infty} \frac{(\frac{1}{n})^2}{2(\frac{1}{n})^2}\\
        &= \lim_{n \rightarrow \infty} \frac{1}{2}\\
        &= \frac{1}{2}
    \end{align*}
    Since $(x_n,y_n) \rightarrow 0$ but $f(x_n,y_n) \not \rightarrow f(0)$, we say that $f$ cannot be continuous at 0.
    Any function from $\mathbb{R}^2 \rightarrow \mathbb{R}$ that is differentiable at a point $A$, is continuous at $A$,
    so by the contrapositive of that statement it follows that $f$ cannot be differentiable at $0$.
\end{proof}
\section*{Problem 2}
    Consider the partial differential equation
    \[
        u_t + 3u_x = 0
    \]
    for a differentiable function $u(x,t)$.
    \subsection*{a)}
        \begin{mdframed}
            Suppose $f$ is a differentiable function of one variable. Show that $u(x,t) = f(x - 3t)$ satisfies the 
            partial differential equation.
        \end{mdframed}
        Let $g : \mathbb{R}^2 \rightarrow \mathbb{R}$ where $g(x,t) = x - 3t$. Then we can write $f(x-3t) = (f \circ g)(x,t)$.
        Then, we compute the partial derivatives of $u$ using the chain rule
        \begin{align*}
            u_x &= \frac{\partial}{\partial x}(f(g(x,t))) = \frac{df}{dy} \cdot \frac{\partial g}{\partial x}\\
            &= \frac{df}{dy}(1) = \frac{df}{dy} \\\\
            u_t &= \frac{\partial}{\partial t}(f(g(x,t))) = \frac{df}{dy} \cdot \frac{\partial g}{\partial t}\\
            &=  \frac{df}{dy}(-3)
        \end{align*}
        Then it follows that 
        \[
            u_t + 3u_x = \frac{df}{dy}(-3) + (3) \frac{df}{dy} = 0
        \]
        So we say that the partial differential equation is satisfied.
    \subsection*{b)}
        \begin{mdframed}
            Let $V = \frac{1}{\sqrt{10}}(3,1)$. Show that if $u(x,t)$ satisfies the differential equation then 
            the directional derivative $D_v u = 0$.
        \end{mdframed}
        \begin{proof}
            Let $u(x,t)$ be some function such that $u_t + 3u_x = 0$.
            We wish to show that $D_v u = 0$.
            So by definition of the directional derivative,
            \begin{align*}
                D_v u(x,t) &= \nabla u(x,t) \cdot V \\
                &= \langle u_x, u_t \rangle \cdot \frac{1}{\sqrt{10}}\langle 3,1\rangle \\
                &= (3u_x + u_t)\frac{1}{\sqrt{10}} \\
                &= 0
            \end{align*}
        \end{proof}
    \subsection*{c)}
        \begin{mdframed}
            Show that a line that passes through $(x,t)$ and is parallel to $V$ passes through $(x-3t,0)$.
        \end{mdframed}
        We know that there is only one line parallel to $V$ passing through $(x,t)$ by one of Euclid's geometric postulates.
        So if we take the line passing through both $(x,t)$ and $(x-3t,0)$ and show it is parallel to $V$, we have shown the desired statement.
        The slope of a line passing through both points is equal to 
        \begin{align*}
            \frac{t-0}{x - (x-3t)} & = \frac{t}{3t} = \frac{1}{3}
        \end{align*}
        Then since $V$ is a scalar multiple of $\langle 3,1\rangle $ it follows that it is parallel to this line.
    \subsection*{d)}
        \begin{mdframed}
            Show that every solution $u(x,t)$ has the property $u(x,t) = u(x - 3t, 0)$.
        \end{mdframed}
        \begin{proof}
            We know that for any solution $u(x,t)$ that its directional derivative with respect to $V$ is 0.
            Because of this, $u$ is constant on any line parallel to $V$.
            Then since for any $(x,t)$ we know that the line parallel to $V$ passing through 
            it also passes through $(x - 3t, 0)$. So $u$ is constant on this line therefore
            $u(x,t) = u(x-3t,0)$.
        \end{proof}
\section*{Problem 3}
    \begin{mdframed}
        Let $G \subset \mathbb{R}^2$ be an open set, and assume $f : G \rightarrow \mathbb{R}$ is differentiable 
        on $G$. Let $C$ be a smooth curve in $G$, given by $\gamma:(a,b) \rightarrow G$.
        Assume $f$ is constant on $C$. Prove that $\nabla f(\gamma(t)) \cdot \gamma'(t) = 0$.
    \end{mdframed}
    \begin{proof}
        Since $f$ is constant on $C$, we write $f(\gamma(t)) = c \in \mathbb{R}$ for every $t \in (a,b)$. By the chain rule we write
        \begin{align*}
            \nabla f(\gamma(t)) \cdot \gamma'(t) & = \frac{d}{dt} f(\gamma(t))\\
            &=\frac{d}{dt} c\\
            &= 0
        \end{align*}
    \end{proof}
\section*{Problem 4}
    \begin{mdframed}
        Let $f: [a,b] \rightarrow \mathbb{R}^2$ continuously on $[a,b]$ such that $f$ is differentiable
        on $(a,b)$. Then there exists some $t \in (a,b)$ such that
        \[
            ||f(b) - f(a)|| \leqslant ||f'(t)|| (b-a)
        \]
    \end{mdframed}
    \begin{proof}
        Let $z = f(b) - f(a)$ and let $\varphi = z \cdot f(t)$. Then we have $\varphi : (a,b) \rightarrow \mathbb{R}$.
        By the ordinary mean value theorem on real valued functions, we say that $\exists t \in (a,b)$ such that 
        \begin{align*}
            \varphi(b) - \varphi(a) & = \varphi ' (t)(b-a)\\
            z \cdot f(b) - z \cdot f(a) & = z \cdot f'(t) (b-a)\\
            z \cdot (f(b) - f(a)) & = z \cdot f'(t) (b-a) \\
            (f(b) - f(a)) \cdot (f(b) - f(a)) &= (f(b) - f(a)) \cdot f'(t) (b-a) \\
            ||f(b)-f(a)||^2 &= [(f_1(b) - f_1(a))f_1'(c) + (f_2(b) - f_2(a))f_2'(c)](b-a)
        \end{align*}
        Then by the mean value theorem in $\mathbb{R}$, we know that $f_i(b) - f_i(b) = f_i'(t)(b-a)$,
        but multiplying both sides by $f_i'(t)$ we get $f_i(b)f_i'(t) - f_i(a)f_i'(t) = f_i'(t)^2(b-a)$.
        Therefore we can write
        \begin{align*}
            ||f(b)-f(a)||^2 &= [f_1'(c)^2(b-a) + f_2'(c)^2(b-a)](b-a) \\
            ||f(b)-f(a)||^2 &= [f_1'(c)^2 + f_2'(c)^2](b-a)^2 \\
            ||f(b)-f(a)||^2 &= ||f'(c)||^2(b-a)^2
        \end{align*}
        Since all terms are positive, we can take the square root of each side without consequences, showing that 
        $||f(b) - f(a)|| = ||f'(t)||(b-a)$, and so the inclusive inequality will hold as well.
    \end{proof}
\section*{Problem 5}
    \begin{mdframed}
        Fix $r > 0$ and let $B_r(0)$ be the open ball in $\mathbb{R}^2$. Assume $f: B_r(0) \rightarrow \mathbb{R}
        $ is differentiable on $B_r(0)$ and assume there exists a positivie real number $M$ such that $||\nabla f(x)|| \leqslant M$
        for all $x \in B_r(0)$. Prove that for any $\bm a,\bm b \in B_r(0)$,
        \[
            |f(\bm b) - f(\bm a)| \leqslant M||\bm b-\bm a||
        \]
    \end{mdframed}
    \begin{proof}
        Let $\gamma : [0,1] \rightarrow \mathbb{R}^2$ where $\gamma(t) = \bm a + t (\bm b - \bm a)$.
        Then we write $\bm a = \gamma(0), \bm b = \gamma(1)$. Then we can rewrite 
        \[
            |f(\bm b) - f(\bm a)| = |(f\circ \gamma)(1) - (f \circ \gamma)(0)|
        \]
        By the mean value theorem, we know that $\exists c \in (0,1)$ such that 
        \[
            (f\circ \gamma)(1) - (f \circ \gamma)(0) = (f \circ \gamma)'(c)
        \]
        Then by the chain rule, we write
        \[
            (f \circ \gamma)'(c) = \nabla f(\gamma (c)) \cdot \gamma'(c)
        \]
        Then since $\gamma(t) = \bm a + t(\bm b - \bm a)$, we know that $\gamma'(t) = \bm b - \bm a$ for every
        $t$. So then we have 
        \begin{align*}
            \nabla f ( \gamma (c)) \cdot \gamma'(c) &= \nabla f ( \gamma (c)) \cdot (\bm b - \bm a)\\
            &\leqslant ||\nabla f(\gamma(c))|| \ ||\bm b - \bm a|| \\
            &\leqslant M ||\bm b - \bm a||
        \end{align*}
        And transitively we have shown that $|f(\bm b) - f(\bm a)| \leqslant M||\bm b - \bm a||$.
    \end{proof}
\begin{proof}
    Assume by contradiction that there exists some $n \in \{2,3,4,\cdots\}$ such that $n$ is neither prime 
    or a product of two or more primes. By the well-ordering principle there must exist some such $n$
    that is the smallest element of $\{2,3,4,\cdots\}$ such that it is neither prime nor the product of two or more primes.
    Call this smallest such element $n$. Then since $n$ is not prime it must be the case that $\exists a,b \in \{2,3,\cdots,n-1\}$
    such that $n = a \cdot b$. However since $n$ is the smallest natural number larger than 1 that is neither prime or a product of primes,
    it must be the case that both $a$ and $b$ are prime or a product of two or more primes. So it follows that $n$ is either prime or a product of two or more primes (contradiction).
    Therefore for every $n \in \mathbb{N}$ such that $n \geqslant 2$ we conclude that $n$ is prime or a product of two or more primes.
\end{proof}
\end{document}