\documentclass{article}

\usepackage{times}
\usepackage{amssymb, amsmath, amsthm}
\usepackage[margin=.5in]{geometry}
\usepackage{graphicx}
\usepackage[linewidth=1pt]{mdframed}

\usepackage{import}
\usepackage{xifthen}
\usepackage{pdfpages}
\usepackage{transparent}

\usepackage{bm}

\newcommand{\incfig}[1]{%
    \def\svgwidth{\columnwidth}
    \import{./figures/}{#1.pdf_tex}
}

\newtheorem{theorem}{Theorem}[section]
\newtheorem{lemma}{Lemma}[section]
\newtheorem*{remark}{Remark}
\theoremstyle{definition}
\newtheorem{definition}{Definition}[section]

\begin{document}

\title{Stochastic Elements of Mathematical Biology - Homework 3}
\author{Philip Warton}
\date{\today}
\maketitle
\section*{Problem 1}
For each $k=0,1,2,\cdots,9$ in order to compute $p_k(\bm i)$ we must first compute the likelihood of each 
possible gamete column. That is, during the process of segregation with recombination, what is the likelihood that 
a given gamete column is produced. First note that recombination only affects the probabilities of getting certain gametes 
from $i_4, i_5$. This is because for all others, the recombination produces the same two outcomes as without recomination.
For example,
\[
    \begin{pmatrix}
        A&A\\B&b
    \end{pmatrix} \mapsto_{r} \begin{pmatrix}
        A & A \\
        b & B
    \end{pmatrix}
\]
These clearly will have the same segregated gamete columns, so this process is only relavant to $i_4,i_5$.
Then for each $m \in \{0,1,\cdots,9\}$, we add the frequency of this dimension multiplied with the desired gamete column.
That is, for $p_0(\bm i)$ we can sum $\frac{i_m}{N} \cdot P_m{A \choose B}$ over $m$ where $P_m{A \choose B}$ is the likelihood
that the $m$-th gamete will produce the desired column. Given this, we write
\begin{align*}
    P{A \choose B} &= \frac{2i_0 + i_1 + i_3 + (1-r)i_4 + ri_5}{2N}\\\\
    P{A \choose b} &= \frac{i_1 + 2i_2 + ri_4 + (1-r)i_5 + 2i_7 + i_8}{2N}\\\\
    P{a \choose B} &= \frac{i_3 + ri_4 + (1-r)i_5 + 2i_7 + i_8}{2N}\\\\
    P{a \choose b} &= \frac{(1-r)i_4 + ri_5 + i_6 + i_8 + 2i_9}{2N}
\end{align*}
Since the order of columns does not matter, we have 2 options in which we can choose our columns, both with the probability 
of the likelihood of achivieng this gamete columns multiplied together. That is,
\begin{align*}
    p_0(\bm i) &= 2 \cdot P{A \choose B} \cdot P{A \choose B} &&
    p_1(\bm i) = 2 \cdot P{A \choose B} \cdot P{A \choose b} &&
    p_2(\bm i) = 2 \cdot P{A \choose b} \cdot P{A \choose b} \\\\
    p_3(\bm i) &= 2 \cdot P{A \choose B} \cdot P{a \choose B} &&
    p_4(\bm i) = 2 \cdot P{A \choose B} \cdot P{a \choose b} &&
    p_5(\bm i) = 2 \cdot P{A \choose b} \cdot P{a \choose B} \\\\
    p_6(\bm i) &= 2 \cdot P{A \choose b} \cdot P{a \choose b} &&
    p_7(\bm i) = 2 \cdot P{a \choose B} \cdot P{a \choose B} &&
    p_8(\bm i) = 2 \cdot P{a \choose B} \cdot P{a \choose b} \\\\
    p_9(\bm i) &= 2 \cdot P{a \choose b} \cdot P{a \choose b}
\end{align*}
And these are our probabilities of producing each gamete based on $\bm i$.
\\\\\\\\
\section*{Problem 2}
To compute the probability of transitioning from $\bm i$ to $\bm j$ is done using a multinomial distribution for 
our more complex Fisher-Wright model. The probability mass function for this generalization of the binomial distribution is given by 
\[
    \frac{n!}{x_1! \cdot x_2! \cdots x_k!} \cdot p_1^{x_1} \cdot p_2^{x_2} \cdots p_k^{x_k}
\]
Where $\bm x = \begin{pmatrix}
    x_1\\x_2\\ \vdots\\x_k
\end{pmatrix}$ is the desired outcome state of the probability and $\bm p = \begin{pmatrix}
    p_1\\p_2\\ \vdots \\ p_k
\end{pmatrix}$ is the probability of drawing each corresponding $x_i$.\\\\
So to write the transition probability of $p(\bm i, \bm j)$ we must know what $n, \bm x,$ and $\bm p$ are.
Firstly, we know that $n = N$, secondly, we know that $\bm x = \bm j$. Finally we know that $\bm p = \bm p(i) = (p_0(i), p_1(i),\cdots,p_9(i))$.
From here it follows simply that
\[
    p(\bm i, \bm j) = p(X(t+1) = \bm j | X(t) = \bm i) = \frac{N!}{j_0! \cdots j_9!}p_0^{j_0}(\bm i) \cdots p_9^{j_9}(\bm i)
\]
And we are done.
\end{document}