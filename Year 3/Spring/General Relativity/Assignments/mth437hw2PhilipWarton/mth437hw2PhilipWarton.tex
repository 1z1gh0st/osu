\documentclass{article}

\usepackage{times}
\usepackage{amssymb, amsmath, amsthm}
\usepackage[margin=.5in]{geometry}
\usepackage{graphicx}
\usepackage[linewidth=1pt]{mdframed}

\usepackage{import}
\usepackage{xifthen}
\usepackage{pdfpages}
\usepackage{transparent}

\newcommand{\incfig}[1]{%
    \def\svgwidth{\columnwidth}
    \import{./figures/}{#1.pdf_tex}
}

\newtheorem{theorem}{Theorem}[section]
\newtheorem{lemma}{Lemma}[section]
\newtheorem*{remark}{Remark}
\theoremstyle{definition}
\newtheorem{definition}{Definition}[section]

\begin{document}

\title{General Relativity - Homework 2}
\author{Philip Warton}
\date{\today}
\maketitle
The line element for Schwarzschild black holes is,
\[
    ds^2 = -\left(1 - \frac{2m}{r}\right)dt^2 + \frac{dr^2}{1-\frac{2m}{r}} + r^2(d\theta^2 + \sin^2\theta d\phi^2)
\]
\section*{Problem 1}
\subsection*{a)}
\begin{mdframed}
    Find an expression for the infinitesimal distance between nearby shells, assuming that only the radius changes (and that $r > 2m$).
\end{mdframed}
Let $r > 2m$. To find such an expression, we assume that we want distance given that only our radius $r$ changes, and that none of our other coordinate variables change.
If these have no change, then we say that their derivative is equal to 0.
That is, let $dt, d\theta, d\phi = 0$. Then we wish to compute $ds$ given that those non-radial derivatives are fixed as such. We write 
\begin{align*}
    ds^2 &=  -\left(1 - \frac{2m}{r}\right)0^2 + \frac{dr^2}{1-\frac{2m}{r}} + r^2(0^2 + \sin^2\theta (0^2))\\
    &= \frac{dr^2}{1-\frac{2m}{r}}\\\\
    \Longrightarrow ds &= \frac{dr}{\sqrt{1-\frac{2m}{r}}}
\end{align*}
\subsection*{b)}
\begin{mdframed}
    What happens as $r$ approaches $2m$ from above, what happens to the expression? How far away is the horizon?
    Do you ever get to the horizon?
\end{mdframed}
As $r \rightarrow 2m^+$ we say that $ds \rightarrow \frac{dr}{\sqrt{1 - 1}}$. If we fix the differential $dr = c \in \mathbb{R}$ to be constant, 
then it follows that $ds \rightarrow \infty$. This indicates that the horizon may be infinitely far away, and that one can never get to the horizon.
\\\\
However, as a corallary, what if one could construct a function of $r$ based on some parameter such that $r \rightarrow 2m^{+}$ and $dr \rightarrow 0$ at a similar rate?
Take 
\[
    r(u) = 2m + \frac{1}{u}, \ \ \ \ r'(u) = \frac{-1}{u^2}
\]
Now it follows that as $u \rightarrow \infty$ we have 
\begin{align*}
    ds &\rightarrow \lim_{u \rightarrow \infty}\frac{-u^{-2}}{\sqrt{1-(2m)/(2m + u^{-1})}}\\
    &= \lim_{u \rightarrow \infty} \frac{-1}{u^2\sqrt{1-(2m)/(2m + u^{-1})}}
\end{align*}
This is going to diverge to negative infinity somehow, which means that either I've done something wrong, or the behavior or very strange.
No matter which convergent function you plug into the derivative, I sense that it would like still diverge, and that the distance to the horizon is infinite.
\section*{Problem 2}
\begin{mdframed}
    Let $m = 5km$. Let $r \in \mathbb{R}$ be given and fixed, and $\Delta r = \frac{1}{1000} km$.
    We use the expression from \fbox{Problem 1a} to approximate the radial distance between the shells for each of the following.
\end{mdframed}
\subsection*{a)}
We approximate the change in difference by $\Delta s$ where we compute,
\begin{align*}
    \Delta s &\approx \frac{\Delta r}{\sqrt{1 - 2m/r}}\\
    &\approx \frac{1/1000}{\sqrt{1 - 2(5)/r}} \\
    &\approx \frac{1}{1000\sqrt{1- 10/r}} \\
\end{align*}
If we choose $r = 50$ then we get $\Delta s \approx \frac{\sqrt{5}}{2000} \approx 0.001118$ km.
\subsection*{b)}
If we choose $r = 15$ then we get $\Delta s \approx \frac{\sqrt{3}}{1000} \approx 0.001732$ km.
\subsection*{c)}
If we choose $r = 10.5$ then we get $\Delta s \approx \frac{\sqrt{21}}{1000} \approx 0.004583$ km.
\section*{Problem 3}
\section*{a)}
\begin{mdframed}
    Use the expression for infinitesimal distnace to determine the exact radial distance between two shells.
\end{mdframed}
We have an expression for the differential $ds$ and want the exact distance $s$. So we must integrate $ds$ with respect to $r$.
This can be computed using Wolfram$|$Alpha given the input:
\begin{verbatim}
{Integrate[1/Sqrt[1 - 2*(m/r)], r], r > 2*m}
\end{verbatim}
We get the following result for exact distance,
\[
    s = \int ds = r \sqrt{1-2m/r} + 2m \tanh^{-1}\sqrt{1- 2m/r} + c
\]
\subsection*{b)}
Decide wether the radial distance to the horizon is finite or infinite.
We use the computer solver to check the limit of $s$ as $r \rightarrow 2m^{+}$.
We input
\begin{verbatim}
Limit[r*Sqrt[1 - 2/r] + 2*ArcTanh[Sqrt[1 - 2/r]], r -> 2, Direction -> -1]
\end{verbatim}
Which gives us the result of 0, which means that the distance to the horizon is finite.
\end{document}
