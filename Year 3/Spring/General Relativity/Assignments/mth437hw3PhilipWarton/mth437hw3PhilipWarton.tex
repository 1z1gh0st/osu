\documentclass{article}

\usepackage{times}
\usepackage{amssymb, amsmath, amsthm}
\usepackage[margin=.5in]{geometry}
\usepackage{graphicx}
\usepackage[linewidth=1pt]{mdframed}

\usepackage{import}
\usepackage{xifthen}
\usepackage{pdfpages}
\usepackage{transparent}

\newcommand{\incfig}[1]{%
    \def\svgwidth{\columnwidth}
    \import{./figures/}{#1.pdf_tex}
}

\newtheorem{theorem}{Theorem}[section]
\newtheorem{lemma}{Lemma}[section]
\newtheorem*{remark}{Remark}
\theoremstyle{definition}
\newtheorem{definition}{Definition}[section]

\begin{document}

\title{General Relativity - Homework 3}
\author{Philip Warton}
\date{\today}
\maketitle
Take Schwarzschild space to be a 4-dimensional real-valued space with a line element given by
\[
    ds^2 = -\left(1-\frac{2m}{r}\right)dt^2 + \frac{1}{1-\frac{2m}{r}}dr^2 + r^2(d\theta^2 + \sin^2 \theta d\phi)
\]
\section*{Problem 1}
    \subsection*{a)}
        \begin{mdframed}
            Find the speed of a satellite orbiting a Schwarzschild black hole at constant radius $r= 6m$, as measured by a stationary (``shell”) observer at that radius.
        \end{mdframed}
        Since we have a ``shell'' observer, we know that we can measure infinitesimal space and time using,
        \[
            \sigma^t = \sqrt{1-\frac{2m}{r}} dt, \ \ \ \ \ \ \ \ \ \sigma^\phi = r d\phi
        \]
        So to get the speed $S$ of our satelite, we simply take the ratio $\dfrac{\sigma^\phi}{\sigma^t}$.
        This gives us the result 
        \begin{align*}
            S & = \frac{\sigma^\phi}{\sigma^t} \\
            &= \frac{rd\phi}{\sqrt{1-\frac{2m}{r}}dt}
        \end{align*}
        Then we know that $\dfrac{d\phi}{dt} = \dfrac{\dot \phi}{\dot t} = \Omega = \sqrt{\dfrac{m}{r^3}}$. So we can
        We write our generalized speed as 
        \begin{align*}
            S &= \frac{r\sqrt{\frac{m}{r^3}}}{\sqrt{1-\frac{2m}{r}}}\\
            &= \frac{r\sqrt{m}}{\sqrt{r^3}\sqrt{1-\frac{2m}{r}}}\\
            &=  \frac{r\sqrt{m}}{r\sqrt{r(1-\frac{2m}{r})}}\\
            &= \frac{\sqrt{m}}{\sqrt{r(1-\frac{2m}{r})}}\\
            &= \sqrt{\frac{m}{r(1- \frac{2m}{r})}}
        \end{align*}
        That the speed is a function of mass and radius makes sense.
        We plug in $r = 6m$ to get $S = \frac{1}{2}$.\\\\\\\\
    \subsection*{b)}
        \begin{mdframed}
            Is a circular orbit at $r = \frac{5}{2}m$ possible?
        \end{mdframed}
        Let us take our speed computation from part \fbox{1a} to see if this is the case.
        We have 
        \begin{align*}
            S &= \sqrt{\frac{m}{r(1- \frac{2m}{r})}}\\
            &=\sqrt{2}
        \end{align*}
        Then since $\sqrt{2} > 1$ we know that a circular orbit is not possible as a satellite cannot travel faster than lightspeed.
    \subsection*{c)}
        \begin{mdframed}
            Determine the smallest radius at which a circular orbit is possible, and the (shell) speed of a satellite in
            such an orbit.
        \end{mdframed}
        We can immediately assume $r > 2m$ since this is our horizon where the coefficients start becoming non-real.
        Also, we want to bound our speed $0 < S < 1$. So we write,
        \begin{align*}
            S &< 1 \\
            \frac{r\sqrt{\frac{m}{r^3}}}{\sqrt{1-\frac{2m}{r}}} & < 1\\
            r \sqrt{\frac{m}{r^3}} &< \sqrt{1-\frac{2m}{r}} \\
            r^2 \frac{m}{r^3} & < 1 - \frac{2m}{r} \\
            \frac{m}{r}& < 1 - \frac{2m}{r} \\
            \frac{3m}{r} & < 1 \\
            3m &< r
        \end{align*}
        So we must have a radius larger than $3m$ in order for circular orbit to be possible.
        Exactly at $r = 3m$ our satellite would have to travel at the speed of light, so we have a strict inequality and 
        we say that a ``smallest'' radius cannot analytically exist assuming that radii can be infinitely divided.
        We cannot produce a fastest possible speed without having a smallest possible radius.
\section*{Problem 2}
Imagine a beam of light in orbit around a Schwarzschild black hole at constant radius.
    \subsection*{a)}
        \begin{mdframed}
            How fast would a shell observer think the beam of light is traveling?
        \end{mdframed}
        We take the same approach as before, but first we must re-derive our $\Omega$ using the 
        proper $\dot r$ geodesic. We know that our potential function is now given by
        \[
            V = \frac{1}{2}\left(\frac{\ell^2}{r^2} - \frac{2m\ell^2}{r^3}\right) - \frac{1}{2}
        \]
        This can be rewritten as $V = \frac{\ell^2}{2r^2} - \frac{m\ell^2}{r^3} - \frac{1}{2}$. Then taking the derivative we get
        \[
            V' = \frac{dV}{dr} = -\frac{\ell^2}{r^3} + \frac{3m\ell^2}{r^4} = \frac{\ell^2(3m-r)}{r^4}
        \]
        Then we know that $V' = 0$ if and only if $\ddot r = 0$ so we use this fact to say that we must have the following
        for any circular orbit:
        \begin{align*}
            0 & = \frac{\ell^2(3m -r)}{r^4} \\
            \frac{\ell^2}{r^4} &= \frac{1}{3m -r} \\\\
            0 & = \dot r \\
            0 &= e^2 - 1 - 2V \\
            1 + 2V & = e^2 \\
            1 + \left(1-\frac{2m}{r}\right)\left(\frac{\ell^2}{r^2}\right) - 1 &= e^2 \\
            \left(1-\frac{2m}{r}\right)\left(\frac{\ell^2}{r^2}\right) &= e^2 \\
            \left(1-\frac{2m}{r}\right)\left(\frac{r^2}{3m-r}\right) &= e^2 
        \end{align*}
        Given these descriptions of $e^2$ and $\ell^2$ in terms of mass and radius, we can now write 
        \begin{align*}
            \Omega^2 &= \frac{\dot \phi^2}{\dot t^2}\\
            &= \frac{\ell^2 / r^4}{e^2 / \left(1- \frac{2m}{r}\right)^2} \\
            &= \frac{1 / (3m - r)}{\left[(1 - \frac{2m}{r})\left(r^2 / 3m - r\right)\right]/(1 - \frac{2m}{r})^2} \\
            &= \frac{1 / (3m-r)}{\left(r^2 / 3m - r\right)/(1 - \frac{2m}{r})} \\
            &= \frac{(1-\frac{2m}{r})/(3m-r)}{r^2/(3m-r)}\\
            &= \frac{1 - \frac{2m}{r}}{r^2}\\\\
            \Longrightarrow \Omega &= \frac{\dot \phi}{\dot t} = \frac{\sqrt{1-\frac{2m}{r}}}{r}
        \end{align*}
        So now we can recompute our speed by 
        \[
            S = \frac{r}{\sqrt{1-\frac{2m}{r}}} \frac{d\phi}{dt} = \frac{r}{\sqrt{1-\frac{2m}{r}}}\frac{\sqrt{1-\frac{2m}{r}}}{r} = 1
        \]
        So to the shell observer the beam of light moves at exactly 1, which is the speed of light.
    \subsection*{b)}
    \begin{mdframed}
        How fast would an observer far away think the beam of light is traveling?
    \end{mdframed}
    We know that $r = 3m$, and we know that $\dfrac{d\phi}{dt} = \frac{\sqrt{1 - \frac{2m}{r}}}{r}$.
    But the rate of change for time will appear different for the shell observer than it will for the far away 
    observer. This is given by the relationship $dt_0 = \sqrt{1-\frac{2m}{r}}dt_1$. So then we have 
    \begin{align*}
        S = \frac{rd\phi}{\sqrt{1-\frac{2m}{r}}dt_1}& = \frac{r d\phi}{dt_0}  \\
        &= \frac{\sqrt{1-\frac{2m}{r}}}{\sqrt{1-\frac{2m}{r}}}  \frac{r d\phi}{dt_0} \\
        &= \sqrt{1-\frac{2m}{r}} \frac{r d\phi}{\sqrt{1-\frac{2m}{r}}dt_0} \\
        &= \sqrt{1-\frac{2m}{r}}\sqrt{\frac{m}{r(1- \frac{2m}{r})}}\\
        &= \sqrt{\frac{m}{r}}
    \end{align*}
    Then we plug in $r = 3m$, which grants us $S = \sqrt{\frac{m}{3m}} = \frac{\sqrt{3}}{3}$.
    \subsection*{c)}
        \begin{mdframed}
            At what value(s) of $r$, if any, is such an orbit possible?
        \end{mdframed}
        As per our computations in part \fbox{Problem 1 (c)}, we say that $r = 3m$ if $S = 1$.
\end{document}