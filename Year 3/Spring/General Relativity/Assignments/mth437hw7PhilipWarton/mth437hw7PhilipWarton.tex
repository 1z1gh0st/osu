\documentclass{article}

\usepackage{times}
\usepackage{amssymb, amsmath, amsthm}
\usepackage[margin=.5in]{geometry}
\usepackage{graphicx}
\usepackage[linewidth=1pt]{mdframed}

\usepackage{import}
\usepackage{xifthen}
\usepackage{pdfpages}
\usepackage{transparent}

\newcommand{\incfig}[1]{%
    \def\svgwidth{\columnwidth}
    \import{./figures/}{#1.pdf_tex}
}

\newtheorem{theorem}{Theorem}[section]
\newtheorem{lemma}{Lemma}[section]
\newtheorem*{remark}{Remark}
\theoremstyle{definition}
\newtheorem{definition}{Definition}[section]

\begin{document}

\title{General Relativity - Homework 7}
\author{Philip Warton}
\date{\today}
\maketitle
\section{Traces}
We assume the following three equations:
\begin{align*}
\vec G &= G^i_{\ j} \sigma^j \hat e_i \\
\vec R &= R^i_{\ j}\sigma^j \hat e_i \\
G^i_{\ j} &= R^i_{\ j} - \frac{1}{2}\delta^i_{\ j}R
\end{align*}
Then to compute the $trace(G)$ we write 
\begin{align*}
    trace(G) &= \sum_{i \in I}G^i_{\ i} \\
    &= \sum_{i \in I}R^i_{\ i} - \frac{1}{2}\delta^i_{\ i}R \\
    &= \sum_{i \in I}R^i_{\ i} - \frac{1}{2}\sum_{i \in I}\delta^i_{\ i} R\\
    &= R - \frac{|I|R}{2}
\end{align*}
This is the case because $\delta^a_{\ b}$ is the Kroenecker delta we say that $\delta^i_{\ i} = 0 \ \ \forall i \in I$.
Let $n = |I|$ be our number of dimensions and we get 
\begin{align*}
    trace(G) = R\left(1 - \frac{n}{2}\right)
\end{align*}
\section{Robertson-Walker Geometry}
Firstly, the connection forms are written as
\begin{align*}
    \Omega^t_{\ r} &= \frac{\ddot a}{a}\sigma^t \wedge \sigma^r \\
    \Omega^t_{\ \theta} &= \frac{\ddot a}{a}\sigma^t \wedge \sigma^\theta \\
    \Omega^t_{\ \phi} &= \frac{\ddot a}{a}\sigma^t \wedge \sigma^\phi \\
    \Omega^r_{\ \theta} &= \frac{\dot a^2+k}{a^2}\sigma^r \wedge \sigma^\theta \\
    \Omega^r_{\ \phi} &= \frac{\dot a^2 + k}{a^2}\sigma^r \wedge \sigma^\phi \\
    \Omega^\theta_{\ \phi} &= \frac{\dot a^2 + k}{a^2}\sigma^\theta \wedge \sigma^\phi \\
\end{align*}
Then we use the following relationship to compute our Riemannian curvature tensors
\[
    \Omega^i_{\ j} = \frac{1}{2}R^i_{\ jkl}\sigma^k \wedge \sigma^l
\]
So we can write 
\begin{align*}
    \Omega^t_{\ r} &= \frac{1}{2} \left(\sum_{k \in I}\sum_{l \in I}R^t_{\ rkl}\sigma^k \wedge \sigma^l\right)
\end{align*}
However, since we know that we only have $\sigma^t \wedge \sigma^r$ on the left hand side we can eliminate any terms that 
do not have $\sigma^t \wedge \sigma^r$ on the right hand side. Thus 
\begin{align*}
    \Omega^t_{\ r} &= \frac{1}{2} \left(\sum_{k \in I}\sum_{l \in I}R^t_{\ rkl}\sigma^k \wedge \sigma^l\right) \\
    \frac{\ddot a}{a}\sigma^t \wedge \sigma^r &= \frac{1}{2}\left(R^t_{\ rtr}\sigma^t \wedge \sigma^r + R^t_{\ rrt}\sigma^r \wedge \sigma^t\right)\\
    &= \frac{1}{2}\left(R^t_{\ rtr}\sigma^t \wedge \sigma^r - R^t_{\ rrt}\sigma^t \wedge \sigma^r\right)\\
    &= \frac{1}{2}\left(R^t_{\ rtr}\sigma^t \wedge \sigma^r + R^t_{\ rtr}\sigma^t \wedge \sigma^r\right)\\
    &= \frac{1}{2}\left(2 R^t_{\ rtr}\sigma^t \wedge \sigma^r\right)\\
    &= R^t_{\ rtr} \sigma^t \wedge \sigma^r
\end{align*}
So thus it follows that $R^t_{\ rtr} = \frac{\ddot a}{a}$. Also notice that this implies that all other Riemann tensors must be equal to 0 by this argument. Similarly,
\begin{align*}
    R^t_{\ rtr} &= R^t_{\ \theta t \theta} = R^t_{\ \phi t \phi} = \frac{\ddot a}{a}\\
    R^r_{\ \theta r \theta} &= R^r_{\ \phi r \phi} = R^\theta_{\ \phi \theta \phi} = \frac{\dot a + k}{a^2}
\end{align*}
So then we can write our diagonal Ricci curvature in terms of Riemannian curvature,
\[
    R_{ii} = \sum_{k \in I}R^k_{\ iki}
\]
So we state that
\begin{align*}
    R_{tt} &= \sum_{k \in I}R^k_{\ tkt}\\
    &= \sum_{k \in I}-R^k_{\ ttk} \\
    &= \sum_{k \in I}-R^t_{\ ktk} \\
    &= -R^t_{\ ttt} - R^t_{\ rtr} - R^t_{\ \theta t \theta} - R^t_{\ \phi t \phi} \\
    &= 0 - \frac{\ddot a}{a} -\frac{\ddot a}{a}-\frac{\ddot a}{a}\\
    &= \frac{-3\ddot a}{a}
\end{align*}
For our space-like variables, $x$, we write
\begin{align*}
    R_{xx} &= \sum_{k \in I} R^k_{\ x k x} \\
    &= R^t_{\ x t x} + \sum_{k \in I \setminus \{t\}}R^k_{\ x k x} \\
    &= \frac{\ddot a}{a} + \sum_{k \in I \setminus \{t, x\}}R^k_{\ x k x}\\
    &= \frac{\ddot a}{a} + \sum_{k \in I \setminus \{t, x\}}R^x_{\ k x k}
\end{align*}
So by the Riemann curvatures computed earlier we get
\[R_{rr} = R_{\theta\theta} = R_{\phi\phi} = \frac{\ddot a}{a} + \frac{2(\dot a + k)}{a^2}\]
Computing the Ricci scalar, the sum of these diagonal components is
\begin{align*}
    R &= -R_{tt} + R_{rr} + R_{\theta\theta} + R_{\phi \phi}\\
    &= \frac{3 \ddot a}{a} + 3 \left(\frac{\ddot a}{a} + \frac{2(\dot a + k)}{a^2}\right)\\
    &= \frac{6\ddot a}{a} + \frac{6(\dot a + k)}{a^2} \\
    &= \frac{6 \ddot a a + 6 \dot a + 6k}{a^2}
\end{align*}
Then we know simply that for $x \neq t$,
\[
    R^t_{\ t} = -R_{tt} = \frac{3\ddot a}{a} \ \ \ \ \ \ \ \ \ \ \ \ R^x_{\ x} = R_{xx} = \frac{\ddot a}{a} + \frac{2(\dot a + k)}{a^2}
\]
To compute the diagonal components $G^i_{\ i}$ we simply use the Ricci terms that we have already have computed,
\begin{align*}
    G^t_{\ t} &= R^t_{\ t} - \frac{1}{2}\delta^t_{\ t} R \\
    &= \frac{3\ddot a}{a} - \frac{6 \ddot a a + 6 \dot a + 6k}{2a^2}\\
    &= \frac{-6 \dot a - 6k}{2a^2}\\
    &= \frac{-3(\dot a + k)}{a^2} \\\
    G^x_{\ x} &= R^x_{\ x} - \frac{1}{2}\delta^x_{\ x} R \\
    &= \frac{\ddot a}{a} + \frac{2(\dot a + k)}{a^2} - \frac{6 \ddot a a + 6 \dot a + 6k}{2a^2}\\
    &= \frac{2 \ddot a a}{2a^2} + \frac{4(\dot a + k)}{2a^2} - \frac{6 \ddot a a + 6 \dot a + 6k}{2a^2} \\
    &= \frac{2 \ddot a a}{2a^2} + \frac{4\dot a + 4k}{2a^2} - \frac{6 \ddot a a + 6 \dot a + 6k}{2a^2}\\
    &= \frac{-4 \ddot a a - 2 \dot a - 2k}{2a^2}\\
    &= - \frac{2 \ddot a a + \dot a + k}{2a^2}
\end{align*}
Where $x = r, \theta, \phi$. Since the Kroenecker delta is only non-zero on the diagonal, and also our Ricci components 
are only non-zero as computed, we say that no other $G^i_{\ j}$ is non-trivial.
\end{document}