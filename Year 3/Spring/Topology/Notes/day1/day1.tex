\documentclass{article}

\usepackage{times}
\usepackage{amssymb, amsmath, amsthm}
\usepackage[margin=.5in]{geometry}
\usepackage{graphicx}
\usepackage[linewidth=1pt]{mdframed}

\usepackage{import}
\usepackage{xifthen}
\usepackage{pdfpages}
\usepackage{transparent}

\newcommand{\incfig}[1]{%
    \def\svgwidth{\columnwidth}
    \import{./figures/}{#1.pdf_tex}
}

\newtheorem{theorem}{Theorem}[section]
\newtheorem{lemma}{Lemma}[section]
\newtheorem*{remark}{Remark}
\theoremstyle{definition}
\newtheorem{definition}{Definition}[section]

\begin{document}

\title{General Topology and Fundamental Group - Notes}
\author{Philip Warton}
\date{\today}
\maketitle
\section{Introduction to Algebraic Topology}
Let $Top = \{\text{all topological spaces}\}$ and let $Alg = \{ \text{all groups/rings/
}\cdots \}$. Then we have a functor $F: Top \rightarrow Alg$.
$Top$ is equipped with continuous functions. If $f: X \rightarrow Y$ is continuous then 
there exists $f_*: F(X) \rightarrow F(Y)$. The function $f_*$ must satisfy:
\begin{align}
    f = id \ \ \ &\Longrightarrow \ \ \ f_* = id \\
    f:X\rightarrow Y,g:Y\rightarrow Z \ \ \ &\Longrightarrow \ \ \ (g \circ f)_* = g_* \circ f_*
\end{align}
If $f$ is homeomorphic then $f_*$ is isomorphic.
\begin{mdframed}
    \begin{definition}[Path]
        Let $X$ be a topological space with $x_0,x_1\in X$. A path $\alpha$ is a function 
        $\alpha : I \rightarrow X$ such that $\alpha(0) = x_0, \alpha(1) = x_1$ is a path in $X$.
    \end{definition}
\end{mdframed}
The constant path $\alpha(t) = x \forall t$ can be written simply as $\alpha = x$.
\begin{mdframed}
    \begin{definition}[Path Homotopy]
        Two paths from $x_0$ to $x_1$ are said to be homotopic if $\exists F:I \times I \rightarrow X$
        such that
        \begin{align}
            &F(t,0) = \alpha(t)\\
            &F(t,1) = \beta(t)\\
            &F(0,u) = x_0 \\
            &F(1,u) = x_1
        \end{align}
        and $F$ is continuous.
    \end{definition}
\end{mdframed}
For path homotopy we write $\alpha \simeq_p^F \beta$. Now this is an equivalence relation, that is,
\begin{align}
    \alpha &\simeq_p \alpha \\
    \alpha &\simeq_p \beta \Longrightarrow \beta \simeq_p \alpha \\
    \alpha &\simeq_p \beta, \beta \simeq_p \gamma \Longrightarrow \alpha \simeq_p \beta & \text{by glueing lemma}
\end{align}
We show this now.
\begin{proof}
    Assume that $\alpha \simeq_p \beta$.
\end{proof}
We can define a concatenation of two paths $\alpha, \beta$ such that $\alpha(1) = \beta(0)$ as 
\[
    \alpha * \beta : I \rightarrow X, \alpha * \beta (t) = \begin{cases}
        x_0, & t \in [0, 1/2] \\
        \alpha(2t-1), & t \in (1/2, 1]
    \end{cases}
\]
One can show that $x_0 * \alpha \simeq_p \alpha$.
\end{document}