\documentclass{article}

\usepackage{times}
\usepackage{amssymb, amsmath, amsthm}
\usepackage[margin=.5in]{geometry}
\usepackage{graphicx}
\usepackage[linewidth=1pt]{mdframed}

\usepackage{import}
\usepackage{xifthen}
\usepackage{pdfpages}
\usepackage{transparent}

\newcommand{\incfig}[1]{%
    \def\svgwidth{.5\linewidth}
    \import{./figures/}{#1.pdf_tex}
}

\newtheorem{theorem}{Theorem}[section]
\newtheorem{lemma}{Lemma}[section]
\newtheorem*{remark}{Remark}
\theoremstyle{definition}
\newtheorem{definition}{Definition}[section]

\begin{document}

\title{General Topology and the Fundamental Group - Homework 2}
\author{Philip Warton}
\date{\today}
\maketitle
\section*{Pre-Amble}
    \begin{theorem}[First Isomorphism Theorem of Anti-Homomorphisms]
        \begin{mdframed}
            Let $H$ and $G$ be groups and let $f: G \rightarrow H$ be an anti-homomorphism.
            That is, $f(g_1\cdot g_2) = f(g_2)\cdot f(g_1)$ for every $g_1,g_2 \in G$. Then
            \begin{align*}
                (i) & \ \ \ \ker(f) \triangleleft G \\
                (ii) & \ \ \ f(G) < H \\
                (iii) & \ \ \ f(G) \cong G / \ker(f)
            \end{align*}
        \end{mdframed}
    \end{theorem}
    \begin{proof}
        Let $G$ and $H$ be groups and $f:G \rightarrow H$ be an anti-homomorphism such that $f(g_1 g_2) = f(g_2)f(g_1)$.
        \fbox{(i)} To show that $\ker(f) < G$, first we know that $e_G \in \ker(f)$ since $f(e_G) = e_H$.
        Then let $a,b \in \ker(f)$. Then we have 
        \[
            f(ab) = f(b)f(a) = e_H e_H = e_H
        \]
        Thus we say $ab \in \ker(f)$. Now to show that the subgroup is normal, let $g \in G$ and let 
        $k \in \ker(f)$ be arbitrary. Then 
        \begin{align*}
            f(gkg^{-1}) & = f(kg^{-1})f(g) \\
            &= f(g^{-1})f(k)f(g) \\
            &= f(g^{-1})e_Hf(g)\\
            &= f(g^{-1})f(g) \\
            &= f(g g^{-1}) \\
            &= f(e_G)\\
            &= e_H
        \end{align*}
        So it follows that $gkg^{-1} \in \ker(f)$ for every $g \in G$.\\\\
        \fbox{(ii)} To show that $f(G) < H$, firstly we have $e_H \in f(G)$ clearly since $f(e_G) = e_H$.
        Then let $f(a),f(b) \in f(G)$. It follows that 
        \[
            f(a)f(b) = f(ba) \in f(G)
        \]
        So we have $f(G) < H$.\\\\
        \fbox{(iii)} Let $\phi([a]) = \phi([b])$. Then we want to show that $[a] = [b]$. So we write 
        \begin{align*}
            f(a) & = f(b) \\
            f(a)f(b)^{-1} & = e \\
            f(a)f(b^{-1}) & = e \\
            f(b^{-1}a) & = e \\
            f(b^{-1}a) & = e
        \end{align*}
        So then we have $b^{-1}a \in \ker(f)$, and it follows that $a \in b\ker(f)$, thus $[a] = [b]$.
    \end{proof}
\section*{Problem 1}
    Let $f:S^1\rightarrow S^1$. Describe $f_*$, the induced homeomorphism.
    \subsection*{a)}
        \begin{mdframed}
            Let $f$ be given by the following function:
            \[
                f(z) = \begin{cases}
                    z, & Im(z) \geqslant 0 \\
                    \overline z, & Im(z) < 0
                \end{cases}
            \]
        \end{mdframed}
        First notice that $f$ folds the unit circle in half across the real axis. Any path in $\pi_1(S^1, 1)$
        can be parameterized by $g(x) = e^{ix}$ composed with some real valued function $h : [0,1] \rightarrow \mathbb{R}$
        where $h(0) = 2\pi m$ and $h(1) = 2\pi n$. The value $n - m$ gives us the path homotopy equivalence class of our path.
        That is, we can classify paths by how many times they go around, and in what direction. However, when we fold the plane
        into simply one half circle, all paths clearly become contractible to a single point \fbox{Figure 1}.
        \begin{figure}[ht]
            \centering
            \incfig{1}
            \caption{How all paths become trivial under $f_*$.}
            \label{fig:1}
        \end{figure}
    \subsection*{b)}
        \begin{mdframed}
            Let $f$ be given by the following function:
            \[
                f(z) = \begin{cases}
                    z, & Im(z) \geqslant 0 \\
                    {\overline z}^3, & Im(z) < 0
                \end{cases}
            \]
        \end{mdframed}
        Now first consider a simple example, $\gamma(t) = e^{2\pi i t}$ where $t \in [0,1]$. This would be 
        a path that travels once around the unit circle counter clock-wise. However, under $f$, this path Now
        travels half way around the circle counter clock-wise, and then turns around and travels $3\pi$ radians around 
        the circle clock-wise, and now is path homotopic to the path going once around clock-wise. If we consider $\gamma(1-t)$,
        the exact opposite will occur, making the path go from the path homotopy class that goes once counter clockwise to 
        the one that goes exactly once around counter-clockwise. Under $f_*$ the equivalence class for $n - m$ becomes the one for $m - n$,
        and contractible paths remain trivial.
\section*{Problem 2}
    Let $c_g: G \rightarrow G$ be the conjugacy of some algebraic element of a group $g \in G$, or alternatively 
    called an inner automorphism.
    \subsection*{a)}
        \begin{mdframed}
            Check that $c_g \in Aut(G)$ and that $\varphi(g) = c_g$ is an anti-homomorphism.
        \end{mdframed}
        \fbox{$c_g \in Aut(G)$} Since $g^{-1}, h, g$ all belong to $G$, it follows that $c_g(h) \in G$.
        Now let $a,b \in G$. Then
        \begin{align*}
            g^{-1} (ab) g & = g^{-1}a (id) b g \\
            &= g^{-1}a(g g^{-1})bg \\
            &=(g^{-1}ag)(g^{-1}bg)
        \end{align*}
        So obviously $c_g$ is a group homomorphism. Now we've got to show that it is both an endomorphism and a monomorphism.
        Let $a,b \in G$. Then assume that 
        \begin{align*}
            c_g(a) &= c_g(b) \\
            g^{-1}ag &= g^{-1}bg \\
            ag &= bg\\
            a & = b
        \end{align*}
        So we have a monomorphism. Pick any $a \in G$. Then if we want to find some $h \in G$  such that 
        $a = c_g(h)$ simply let $h = gag^{-1}$. Then 
        \begin{align*}
            c_g(h) &= c_g(gag^{-1})\\
            &= g^{-1}gag^{-1}g\\
            &= (id)a(id)\\
            &= a
        \end{align*}
        So we also have an endomorphism, and thus an isomorphism. Then since $c_g$ is an isomorphism from $G$ onto itself,
        by definition $c_g \in Aut(G)$.\\\\
        \fbox{$\varphi$ is an anti-homomorphism.} We take the automorphism group of $G$ to have the compositional operator.
        To show that $\varphi$ is an anti-homomorphism let $g,h \in G$. We have to demonstrate that $\varphi(gh)= \varphi(h) \circ \varphi(g)$.
        So pick any $a \in G$, and we say that
        \begin{align*}
            \varphi(gh)(a) &= c_{gh}(a)\\
            &=(gh)^{-1}a(gh)\\
            &=h^{-1}g^{-1}agh \\
            &=h^{-1}(g^{-1}ag)h\\
            &=(c_h \circ c_g)(a)\\
            &=(\varphi(g)\circ\varphi(h))(a)
        \end{align*}
        So we say that $\varphi(gh) = \varphi(h)\circ\varphi(g)$ and $\varphi$ is an anti-homomorphism.
    \subsection*{b)}
        \begin{mdframed}
            Show that $Inn(G) \triangleleft Aut(G)$.
        \end{mdframed}
        \begin{proof}
            Let $e\in G$ be the identity element. Then $c_e(a) = e^{-1}ae = a$ is an inner automorphism and 
            we say that $c_e = id_G$. So we have the identity map within $Inn(G)$.
            Then for any $g,h\in G$ we have $c_g \circ c_h = c_{h \cdot g}$ as we proved in part (a).
            So we have a subgroup of $Aut(G)$ here. Now to show that it's a normal subgroup we must show that for 
            any $f \in Aut(G)$ and for any $g \in G$ we have $(f^{-1} \circ c_g \circ f) \in Inn(G)$. 
            So pick any $a \in G$ arbitrarily, and we write 
            \begin{align*}
                (f^{-1} \circ c_g \circ f)(a) &= (f^{-1} \circ c_g)(f(a)) \\
                &=f^{-1}(g^{-1} \cdot f(a) \cdot g) \\
                &=f^{-1}(g^{-1}) \cdot f^{-1}(f(a)) \cdot f^{-1}(g) \\
                &=f^{-1}(g^{-1}) \cdot a \cdot f^{-1}(g)
            \end{align*}
            Then since $f^{-1}$ is a group homomorphsim on $G$, we know that 
            \[
                f^{-1}(g^{-1}) \cdot f^{-1}(g) = f^{-1}(g^{-1} \cdot g) = f^{-1}(e) = e
            \]
            So we know that $f^{-1}(g^{-1})$ must be the inverse of $f^{-1}(g)$. Then since $a$ is sandwiched 
            by some element of $G$ and its inverse, clearly $f^{-1} \circ c_g \circ f \in Inn(G)$, and we say that 
            $Inn(G) \triangleright Aut(G)$.
        \end{proof}
    \subsection*{c)}
        \begin{mdframed}
            Show that $\ker(\varphi) = Z(G)$ and that $Inn(G) \cong G / Z(G)$.
        \end{mdframed}
        \begin{proof}
            We know that $\ker(\varphi) = \varphi^{-1}(id_G)$. That is, the kernel of $\varphi$ is 
            the set of all elements $g \in G$ such that $c_g$ is the identity function. If this is 
            the case then $g^{-1}a g = id_g(a) = a$ for every $a \in G$. Then we can rewrite this equality 
            as $ag = ga$ multiplying both sides by $g$ on the left. So this is the set of all $g \in G$ such that 
            $g$ commutes with every element $a \in G$. By definition, this is the center, $Z(G)$.
            So by the first isomorphism theorem on anti-homomorphisms we say that the image of $\phi$ is isomorphic to $G / \ker \phi$.
            Or, alternatively,
            \[
                Inn(G) \cong G / Z(G)
            \]
        \end{proof}
\section*{Problem 3}
    \subsection*{a)}
        \begin{mdframed}
            Let $x_0, x_1 \in X$ and let $\alpha$ be a path from $x_0$ to $x_1$. Define $\hat \alpha : \pi_1(X,x_0) \rightarrow \pi_1(X,x_1)$
            by $\hat \alpha([\gamma]) = [\alpha^{-1} * \gamma * \alpha]$. Verify that the map $\hat \alpha$ depends only on the path homotopy class $\alpha \in \pi_1(X,x_0,x_1)$,
            and that, consequently, there is a well-defined map $\Theta:\pi_1(X,x_0,x_1)\rightarrow\mathcal{I}(x_0,x_1)$. Here $\mathcal{I}(x_0,x_1)$ stands for 
            the collection of group isomorphisms from $\pi_1(X,x_0)$ onto $\pi_1(X,x_1)$.
        \end{mdframed}
        \begin{proof}
            Let $[\alpha] = [\beta] \in \pi_1(X,x_0,x_1)$. We claim that if $\hat \alpha = \hat \beta$ then we have shown that $\hat \alpha$
            dependes only on the path homotopy class of $\alpha$. That is, if we do not change the path homotopy class, then we do not change the function.
            So for any $\gamma \in \pi_1(X,x_0)$ we have
            \begin{align*}
                \hat \alpha([\gamma]) &= [\alpha^{-1} * \gamma * \alpha] \\
                &= [\alpha^{-1}] * [\gamma] * [\alpha] \\
                &= [\beta^{-1}] * [\gamma] * [\beta] \\
                &= [\beta^{-1} * \gamma * \beta] \\
                &= \hat \beta(\gamma)
            \end{align*}
            So it follows that if $\hat \alpha([\gamma]) = \hat \beta([\gamma])$ for every $\gamma \in \pi_1(X,x_0)$ then 
            $\hat \alpha = \hat \beta$. Thus $\hat \alpha$ depends only on the path homotopy class of $\alpha$.
            Let $\Theta : \pi_1(X,x_0,x_1) \rightarrow \mathcal I(x_0,x_1)$ where $\Theta([\alpha]) = \hat \alpha$.
            Let us confirm that $\hat \alpha$ is guaranteed to be a group isomorphism. Let 
            \begin{align*}
                \hat\alpha([\gamma_1]) &= \hat\alpha([\gamma_2])\\
                [\alpha^{-1} * \gamma_1 * \alpha]  & =  [\alpha^{-1} * \gamma_2 * \alpha] \\
                [\alpha^{-1}] * [\gamma_1] * [\alpha] & = [\alpha^{-1}] * [\gamma_2] * [\alpha]\\
                \Longrightarrow [\gamma_1] * [\alpha] & = [\gamma_2] * [\alpha]\\
                \Longrightarrow [\gamma_1]  & = [\gamma_2] 
            \end{align*}
            So we say that $\hat \alpha$ is injective. Now let $[\gamma] \in \pi_1(X,x_1)$.
            Then the path homotopy class of $[\alpha * \gamma * \alpha^{-1}] \in \pi_1(X,x_0)$.
            So if we take $\hat\alpha([\alpha * \gamma * \alpha^{-1}])$ we get 
            \[
                \hat\alpha([\alpha * \gamma * \alpha^{-1}])  = [\alpha^{-1} * \alpha * \gamma * \alpha^{-1} * \alpha]   = [\gamma]
            \]
            So we say $\hat\alpha$ is bijective. Now we need to demonstrate that $\hat \alpha$ is a group homomorphism.
            Let $[\gamma_1],[\gamma_2] \in \pi_1(X,x_0)$. Then we say that 
            \begin{align*}
                \hat\alpha([\gamma_1] * [\gamma_2]) &= [\alpha^{-1} * [\gamma_1] * [\gamma_2] * \alpha]\\
                &=[ \alpha^{-1}] * [\gamma_1] * [\gamma_2 ]* [\alpha]\\
                &= [ \alpha^{-1}] * [\gamma_1] *e* [\gamma_2 ]* [\alpha]\\
                &=[ \alpha^{-1}] * [\gamma_1] *[\alpha * \alpha^{-1}] * [\gamma_2 ]* [\alpha]\\
                &=[ \alpha^{-1}] * [\gamma_1] *[\alpha]*[ \alpha^{-1}] * [\gamma_2 ]* [\alpha]\\
                &=[ \alpha^{-1} * \gamma_1 *\alpha]*[ \alpha^{-1} * \gamma_2 * \alpha]\\
                &= \hat\alpha([\gamma_1]) * \hat\alpha([\gamma_2])
            \end{align*}
            So we have $\hat \alpha$ is a bijective group homomorphism and therefore $\hat \alpha \in \mathcal{I}(x_0,x_1)$.
        \end{proof}
    \subsection*{b)}
        \begin{mdframed}
            Let $[\alpha], [\beta] \in \pi_1(X,x_0,x_1)$.
            Show that the composition $\Theta([\alpha])^{-1}\circ\Theta([\beta])$
            corresponds to a conjugation in $\pi_1(X,x_0)$ and conclude that $\Theta([\beta]) = \Theta([\alpha]) \circ c_{[\gamma]}$
            for some $[\gamma] \in \pi_1(X,x_0)$.
        \end{mdframed}
        \begin{proof}
            Let $[\delta] \in \pi_1(X,x_0)$. Then we write 
            \begin{align*}
                (\Theta([\alpha])^{-1} \circ \Theta([\beta]))([\delta]) & = (\hat\alpha^{-1} \circ \hat\beta)([\delta]) \\
                &=\hat\alpha^{-1}(\hat\beta([\delta])) \\
                &=\hat\alpha^{-1}([\beta^{-1} * \delta * \beta]) \\
                &=[\alpha * [\beta^{-1} * \delta * \beta] * \alpha^{-1}]\\
                &=[(\alpha * \beta^{-1}) * \delta * (\alpha * \beta^{-1})^{-1}]
            \end{align*}
            Then we know that $\alpha$ is a path from $x_0$ to $x_1$, and that $\beta^{-1}$ is a path from $x_1$ to $x_0$.
            So it follows that $[\alpha * \beta^{-1}] \in \pi_1(X,x_0)$. Thus we have a conjugation $c_{[\alpha * \beta^{-1}]} = \Theta([\alpha])^{-1} \circ \Theta([\beta])$.
            It then clearly is the case that $\Theta([\alpha]) \circ c_{[\alpha * \beta^{-1}]} = \Theta([\beta])$.
        \end{proof}
    \subsection*{c)}
        \begin{mdframed}
            Show that $\varPhi : \pi_1(X,x_0) \times \mathcal I(x_0,x_1) \rightarrow \mathcal I(x_0,x_1), \varPhi([\gamma],\varphi) = \varphi \circ c_{[\gamma]}$
            defines a right action of $\pi_1(X,x_0)$ on $\mathcal{I}(x_0,x_1)$. What is the orbit of $\Theta([\alpha]), [\alpha]\in\pi_1(X,x_0,x_1)$?
        \end{mdframed}
        \begin{proof}
            To show that $\varPhi$ is a right action, we must satisfy both identity and compatibility. \fbox{Identity}
            In order to demonstrate this let $\varphi \in\pi_1(X,x_0)$. Then we must show that 
            $\varPhi([x_0], \varphi) = \varphi$. So we write 
            \begin{align*}
                \varphi \circ c_{[x_0]} & = \varphi
            \end{align*}
            Which is clearly this case since $c_{[x_0]}(\gamma) = \gamma$ as $[x_0]$ is simply the constant trivial loop. Then 
            we also need \fbox{Compatibility}. So we must show that for any $\gamma_1, \gamma_2 \in \pi_1(X,x_0)$,
            we have $\varPhi([\gamma_1], \varPhi([\gamma_2],\varphi)) = \varPhi([\gamma_1] * [\gamma_2], \varphi)$.
            We write 
            \begin{align*}
                \varPhi([\gamma_1], \varPhi([\gamma_2],\varphi)) & =  \varPhi([\gamma_1], \varphi \circ c_{[\gamma_2]})\\
                &=\varphi \circ c_{[\gamma_2]} \circ c_{[\gamma_1]} \\
                &= \varphi \circ c_{[\gamma_1] * [\gamma_2]} \\
                &= \varPhi([\gamma_1] * [\gamma_2], \varphi)
            \end{align*}
            So we have shown that $\varPhi$ is a right group action on $\mathcal{I}(x_0,x_1)$.\\\\
            To find the orbit of $\Theta([\alpha])$ we write
            \begin{align*}
                \varPhi(\pi_1(X,x_0), \Theta([\alpha])) &= \{\varPhi([\gamma], \Theta([\alpha])) \ | \ [\gamma] \in \pi_1(X,x_0)\} \\
                &= \{\Theta([\alpha]) \circ c_{[\gamma]} \ | \ [\gamma] \in \pi_1(X,x_0)\} \\ 
            \end{align*}
        \end{proof}
    \subsection*{d)}
        -- Not Complete --
\section*{Problem 4}
        Show in detail that the following maps are covering maps.
    \subsection*{a)}
        \begin{mdframed}[]
            $p: \mathbb{R} \rightarrow S^1, p(x) = e^{i2\pi x}$
        \end{mdframed}
        \begin{proof}
            Let $y \in S^1$. We write $U(y) = \{z \ : \ |\arg(z) - \arg(y)| < \pi\}$.
            Then we write that 
            \[
                p^{-1}(U(u)) = \left\{\left(k + \frac{\arg(y)}{2\pi} - \frac{1}{4},k + \frac{\arg(y)}{2\pi} + \frac{1}{4}\right) \ | \ k \in \mathbb{Z} \right\}
            \]
            Each interval is clearly homeomorphic to the half-circle under $p$, so we have a covering map.
        \end{proof}
    \subsection*{b)}
        \begin{mdframed}[]
            $p: S^1 \rightarrow S^1, p(z) = z^n, n \in \mathbb{N}$
        \end{mdframed}
        -- Not Complete -- 
    \subsection*{c)}
    -- Not Complete --
\section*{Problem 5}
    \begin{mdframed}[]
        Let $G$ be a topological group and let $H \subset G$ be an algebraic subgroup of $G$.
        Show that $H$ and $\overline{H}$ are both topological subgroups of $G$.
    \end{mdframed}
    \begin{proof}
        We must show that $H$ is a topological group. That is, that our mulitplication and inverse functions 
        are continuous in $H$ and its subspace topology.
        Let $U \subset H$ be open. Then $\exists V \subset G$ that is open in $G$ such that $U = H \cap V$.
        Then we take the inverse, $V^{-1} \subset G$. Since $^{-1} : G \rightarrow G$ is continuous in $G$, it 
        follows that $V^{-1}$ is open in $G$. Therefore $V^{-1} \cap H$ is open in $H$. This set must also be non-empty,
        since any $x \in U$ must have an inverse element in $H$ since $H < G$ algebraicly. So $^{-1}$ is continuous in $H$.
        Now we must show that the same is true for multiplication. Let $U \subset H$ be open. Then let $h \in H$ be some element.
        Write $U = V \cap H$ where $V \subset G$ is open. Then $L_{h^{-1}}$ is continuous in $G$ so we know $L_{h^{-1}}(U)$
        will be open in $G$. Then take $H \cap L_{h^{-1}}(U)$ and it follows that this is an open set in $H$ that is non-empty.
        So we have shown that both $^{-1}$ and $m$ are continuous in $H$ so we say that $H$ is a topological subgroup of $G$.\\\\
        Having shown that this is true for any algebraic subgroup $H < G$, now we wish to show that $\overline {H}$ is a 
        subgroup of $G$ algebraically. We know we must have our identity element $e \in H \subset \overline{H} \subset G$.
        Now let $h \in \overline{H}$. For every neighborhood $U\subset G$ of $h$ we know that $U \cap H$ is non-empty.
        Then we know that $(U \cap H)^{-1}$ is an open set in $H$. Since this is true for every neighborhood of $h$
        it follows that $h^{-1}$ must be a limit point of $H$ thus $h^{-1} \in \overline H$. Now let $a,b \in \overline{H}$ be arbitrary.
        We wish to show that their product is in $\overline{H}$. So we write $(a,b) \in \overline{H} \times \overline{H}$.
        Thus 
        \[
            m(a,b) \in m(\overline{H} \times \overline{H}) = m(\overline{H} \times {H}) \subset \overline{m(H \times H)}
        \]
        So we say that every neighborhood of $ab$ intersects $m(H \times H) \subset H$ so it follows that $ab \in \overline{H}$.
        Therefore $\overline{H} < G$ and is therefore a topological subgroup of $G$.
    \end{proof}
\section*{Problem 6}
    \begin{mdframed}[]
        Let $G$ be a topological group and let $\{e\}$ be a closed set. Show that $G$ is Hausdorff.
    \end{mdframed}
    \begin{proof}
        We know that $G$ is Hausdorff or $T_2$ if and only if $\Delta$ is closed in $G \times G$.
        We have $^{-1}:G \rightarrow G$ is a continuous map, and $id:G\rightarrow G$ is continuous.
        Therefore define $f: G \times G \rightarrow G\times G$ by $f(x,y) = (x, y^{-1})$, and this will be continuous 
        since both dimensions are continuous. Then if $m:G \times G \rightarrow G$ is our multiplication operation
        we can write $m(f(x,y)) = m(x, y^{-1})$ then this, being a composition of continuous functions, is continuous.
        Now we know that ${e}$ is closed and that this function is continuous. So consider the pre-image of ${e}$ under this function.
        That is \[
            \{(x,y) \in G \times G \ | \ xy^{-1} = e\} = \{ (x,y) \in G \times G \ | \ y^{-1} = x^{-1}\} = \{(x,y) \in G \times G \ | \ y = x \} = \Delta
        \]
        So $\Delta$ must be a closed set, and we say that $G$ is Hausdorff.
    \end{proof}
\end{document}