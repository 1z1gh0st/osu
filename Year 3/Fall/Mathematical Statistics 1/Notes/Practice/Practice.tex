\documentclass{article}

\usepackage{times}
\usepackage{amssymb, amsmath, amsthm}
\usepackage[margin=.5in]{geometry}
\usepackage{graphicx}
\usepackage[linewidth=1pt]{mdframed}

\usepackage{import}
\usepackage{xifthen}
\usepackage{pdfpages}
\usepackage{transparent}

\newcommand{\incfig}[1]{%
    \def\svgwidth{\columnwidth}
    \import{./figures/}{#1.pdf_tex}
}

\newtheorem{theorem}{Theorem}[section]
\newtheorem{lemma}{Lemma}[section]
\newtheorem*{remark}{Remark}
\theoremstyle{definition}
\newtheorem{definition}{Definition}[section]

\begin{document}

\title{Practice Problems}
\author{Philip Warton}
\date{\today}
\maketitle
\section*{3.145}
    If $Y$ has a binomial distribution with $n$ trials and probability of success $p$, show that teh moment-generating function for $Y$ is
    \[
        m(t) = (pe^t + q)^n, \ \ \ \ \ \text{where $q = 1 - p$}
    \]
    \begin{proof}
        Let $Y$ be a discrete random variable that is a binomial distribution with $n$ trials and success probability $p$. Then its moment
        generating function is described by $m(t) = E[e^{tY}]$. This can be rewritten as $\sum_{k = 0}^n e^{tk} {n \choose k} p^k q^{n-k}$.
        Then we wish to show that this some ends up being equal to $(pe^t + q)^n$. To show this we compute the sum as
        \[
            m(t) = \sum_{x = 0}^n e^{tx} {n \choose x} p^x q^{n - x} = \sum_{x = 0}^n {n \choose x} (pe^t)^x q^{n-x} = (pe^t + q)^n
        \]
        This last equality comes from the binomial theorem.
    \end{proof}
\section*{3.146}
    Differentiate the moment-generating function in \fbox{3.145} to find $E(Y)$ and $E(Y^2)$. Then find $V(Y)$.\\\\
    We take the derivative of $m(t)$, and say that 
    \[
        m'(t) = \frac{d}{dt} (pe^t + q)^n = (pe^t + q)^{n-1}(pe^t)
    \]
    So then our mean is equal to $m^{(1)}(0) = Then we wish to take the second derivative of this which gives us 
    \[
        m''(t) = \frac{d}{dt} (pe^t + q)^{n-1}(pe^t) = (pe^y + q)^{n - 2}(pe^t)^2 + (pe^t + q)^{n-1}(pe^t)
    \]
\end{document}