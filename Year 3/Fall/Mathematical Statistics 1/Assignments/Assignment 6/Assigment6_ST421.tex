\documentclass{article}

\usepackage{times}
\usepackage{amssymb, amsmath, amsthm}
\usepackage[margin=.5in]{geometry}
\usepackage{graphicx}
\usepackage[linewidth=1pt]{mdframed}

\usepackage{import}
\usepackage{xifthen}
\usepackage{pdfpages}
\usepackage{transparent}

\newcommand{\incfig}[1]{%
    \def\svgwidth{\columnwidth}
    \import{./figures/}{#1.pdf_tex}
}

\newtheorem{theorem}{Theorem}[section]
\newtheorem{lemma}{Lemma}[section]
\newtheorem*{remark}{Remark}
\theoremstyle{definition}
\newtheorem{definition}{Definition}[section]

\begin{document}

\title{Mathematical Statistics - Assignment 6}
\author{Philip Warton}
\date{\today}
\maketitle
\section*{Problem 4.8}
    Suppose that $Y$ has a density function
    \[
        f(y) = 
        \begin{cases}
            k y (1-y), & 0 \leq y \leq 1\\
            0, & \text{otherwise}
        \end{cases}
    \]
    \subsection*{(a)} 
        Find some $k \in \mathbb{R}$ such that $f(y)$ is a probability
        density function. Notice that $\forall y \in [0,1],\ \ \ \ y(1-y) > 0$. This indicates
        that $k \geq 0$. The integral $\int_{-\infty}^0 f(y)dy + \int_1^\infty f(y)dy = 0$ since we know that
        the function is constantly zero on these open intervals. Then $P(-\infty, \infty) = P(0 \leq X \leq 1) = \int_0^1 f(y) dy$.
        Thus we want to find some $k$ such that $\int_0^1 k y (y-1) dy = 1$. Let us use rules of algebra and
        integration to derive the following
        \begin{align*}
            1 & = \int_0^1 ky(1-y)dy\\
            &= k \int_0^1 y-y^2 dy\\
            &= k \left(\frac{y^2}{2} - \frac{y^3}{3}\right)\bigg|_0^1\\
            &= \frac{k}{6}
        \end{align*}
        Since we must have $\frac{k}{6} = 1$, of course our answer will be $k = 6$.
    \subsection*{(b)-(e)}
        Compute various probabilities using this value of $k$ and the normalized density function $f(y)$. First
        we must find $P(.4 \leq Y \leq 1)$.
        \begin{align*}
            P(.4 \leq Y \leq 1) &= \int_{0.4}^1 6y(1-y)dy\\
            &= 6 \int_{.4}^1y - y^2dy\\
            &= 6\left[\frac{y^2}{2} - \frac{y^3}{3}\right]_{.4}^1\\
            &= 6 \left[ \frac{1}{6} - \left(\frac{2}{25} - \frac{2^3}{(3) 5^3}\right)   \right]\\
            &= .648
        \end{align*}
        The next probability to compute is $P(.4 \leq Y < 1)$. However, this probability will be the
        same as the one we just computed because $P(.4 \leq Y \leq 1) = P(.4 \leq Y < 1) + P(Y = 1)$.
        Since $P(Y = 1) = \int_1^1 f(y)dy = 0$, both probabilities must be equal. \\\\
        To compute $P(Y \leq .4 | Y \leq .8)$ by definition the probability equals 
        \[
            \frac{P(Y \leq .4 \text { and } Y \leq .8)}{P(Y \leq .8)}
        \]
        The probability in the numberator will simply be $P(Y \leq .4)$ since this event is a subset of $Y \leq .8$.
        Since the probability $P(.4 \leq Y \leq 1)$ is already known, $P(Y \leq .4) = 1 - P(Y > .4) = 1 - P(.4 \leq Y \leq 1) = .352$.
        For the denominator, the integral must be computed.
        \[
            6 \int_0^{.8} f(y) dy = 6\left[\frac{y^2}{2} - \frac{y^3}{3}\right]_0^{.8} = 6\left[\frac{4^2}{(2)5^2} - \frac{4^3}{(3)5^3}\right] = .896
        \]
        Knowing both the numerator and denominator we say $P(Y \leq .4 | Y \leq .8) = \frac{.352}{.896} = .393$.
\section*{Problem 4.14}
\section*{Problem 4.18}
\section*{Problem 4.28}
\section*{Problem 4.32}
\section*{Problem 4.40}
\section*{Problem 4.48}
\section*{Problem 4.50}
\end{document}