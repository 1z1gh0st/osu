\documentclass{article}

\usepackage{times}
\usepackage{amssymb, amsmath, amsthm}
\usepackage[margin=.5in]{geometry}
\usepackage{graphicx}
\usepackage[linewidth=1pt]{mdframed}

\usepackage{import}
\usepackage{xifthen}
\usepackage{pdfpages}
\usepackage{transparent}

\newcommand{\incfig}[1]{%
    \def\svgwidth{\columnwidth}
    \import{./figures/}{#1.pdf_tex}
}

\newtheorem{theorem}{Theorem}[section]
\newtheorem{lemma}{Lemma}[section]
\newtheorem*{remark}{Remark}
\theoremstyle{definition}
\newtheorem{definition}{Definition}[section]

\begin{document}

\title{Mathematical Statistics - Assignment 6}
\author{Philip Warton}
\date{\today}
\maketitle
\section*{Problem 4.8}
    Suppose that $Y$ has a density function
    \[
        f(y) = 
        \begin{cases}
            k y (1-y), & 0 \leq y \leq 1\\
            0, & \text{otherwise}
        \end{cases}
    \]
    \subsection*{(a)} 
        Find some $k \in \mathbb{R}$ such that $f(y)$ is a probability
        density function. Notice that $\forall y \in [0,1],\ \ \ \ y(1-y) > 0$. This indicates
        that $k \geq 0$. The integral $\int_{-\infty}^0 f(y)dy + \int_1^\infty f(y)dy = 0$ since we know that
        the function is constantly zero on these open intervals. Then $P(-\infty, \infty) = P(0 \leq X \leq 1) = \int_0^1 f(y) dy$.
        Thus we want to find some $k$ such that $\int_0^1 k y (y-1) dy = 1$. Let us use rules of algebra and
        integration to derive the following
        \begin{align*}
            1 & = \int_0^1 ky(1-y)dy\\
            &= k \int_0^1 y-y^2 dy\\
            &= k \left(\frac{y^2}{2} - \frac{y^3}{3}\right)\bigg|_0^1\\
            &= \frac{k}{6}
        \end{align*}
        Since we must have $\frac{k}{6} = 1$, of course our answer will be $k = 6$.
    \subsection*{(b)-(e)}
        Compute various probabilities using this value of $k$ and the normalized density function $f(y)$. First
        we must find $P(.4 \leq Y \leq 1)$.
        \begin{align*}
            P(.4 \leq Y \leq 1) &= \int_{0.4}^1 6y(1-y)dy\\
            &= 6 \int_{.4}^1y - y^2dy\\
            &= 6\left[\frac{y^2}{2} - \frac{y^3}{3}\right]_{.4}^1\\
            &= 6 \left[ \frac{1}{6} - \left(\frac{2}{25} - \frac{2^3}{(3) 5^3}\right)   \right]\\
            &= .648
        \end{align*}
        The next probability to compute is $P(.4 \leq Y < 1)$. However, this probability will be the
        same as the one we just computed because $P(.4 \leq Y \leq 1) = P(.4 \leq Y < 1) + P(Y = 1)$.
        Since $P(Y = 1) = \int_1^1 f(y)dy = 0$, both probabilities must be equal. \\\\
        To compute $P(Y \leq .4 | Y \leq .8)$ by definition the probability equals 
        \[
            \frac{P(Y \leq .4 \text { and } Y \leq .8)}{P(Y \leq .8)}
        \]
        The probability in the numberator will simply be $P(Y \leq .4)$ since this event is a subset of $Y \leq .8$.
        For the denominator, the integral must be computed.
        \[
            6 \int_0^{.8} f(y) dy = 6\left[\frac{y^2}{2} - \frac{y^3}{3}\right]_0^{.8} = 6\left[\frac{4^2}{(2)5^2} - \frac{4^3}{(3)5^3}\right] = .896
        \]
        Knowing both the numerator and denominator we say $P(Y \leq .4 | Y \leq .8) = \frac{.352}{.896} = .393$.
\section*{Problem 4.14}
    We have the following probability density function
    \[
        f(y) =
        \begin{cases}
            y, & 0  < y < 1\\
            2-y, & 1 \leq y < 2\\
            0, & \text{otherwise}
        \end{cases}
    \]
    \subsection*{(a)}
        We sketch the function $f(y)$.
        \begin{figure}[ht]
            \centering
            \def\svgwidth{.5\linewidth}
            \import{./figures/}{pdf-4.14.pdf_tex}
            \caption{Probability Density Function $f(y)$}
            \label{fig:pdf-4.14}
        \end{figure}
    \subsection*{(b)}
        \[
            F(y) =
            \begin{cases}
                \int_{-\infty}^y 0 dt = 0, & y \leq 0\\
                \int_{-\infty}^0 0 dt + \int_0^y t dt = \frac{t^2}{2}\bigg|_0^y = \frac{y^2}{2}, & 0 < y < 1\\
                \int_{-\infty}^0 0 dt + \int_0^y t dt + \int_1^y 2 - t dt = \frac{1}{2} + \left[2t - \frac{t^2}{2}\right]_1^y = \frac{1}{2} + \left[2y - \frac{y^2}{2} - (2 - \frac{1}{2})\right] = -\frac{y^2}{2} +2y - 1, & 1 \leq y \leq 2\\
                1, & y > 2
            \end{cases}
        \]
        \begin{figure}[ht]
            \centering
            \def\svgwidth{.5\linewidth}
            \import{./figures/}{pdf-4.14b.pdf_tex}
            \caption{The Anti-Derivative $F(y)$}
            \label{fig:pdf-4.14b}
        \end{figure}
    \subsection*{(c)}
        To find the probability that the station pumps between 8000 and 12000 gallons of gas, we say
        $P(.8 < Y < 1.2) = F(1.2) - F(.8) = -\frac{1.2^2}{2} + 2(1.2) - 1 - [\frac{.8^2}{2}] = .36$
    \subsection*{(d)}
        $P(Y > 1.5 | Y > 1) = \frac{P(Y > 1.5)}{P(Y > 1)}$. We must compute each of these in order to find the conditional probability.
        \begin{align*}
            P(Y > 1.5) &= (.5)\int_{1.5}^\infty f(y)dy = (.5)\frac{(.5)^2}{2} = (.5)\frac{.25}{2} = (.5).128 = .064\\
            P(Y > 1) &= (.5)\int_1^\infty f(y)dy = (.5) \frac{1}{2} = .25
        \end{align*}
        This is clear by the geometric argument where we are simply taking the area of square right triangles. Finally our conditional probability is going to 
        be $\frac{.064}{.25} = .256$.
\section*{Problem 4.18}
    For this problem we have the density function
    \[
        f(y) =
        \begin{cases}
            .2, & -1 < y \leq 0\\
            .2 + cy, & 0 < y \leq 1\\
            0, & \text{otherwise}
        \end{cases}
    \]
    \subsection*{(a)}
        Find the value of $c$ such that $f$ is a proper probability density function. We know that the integral
        up to $0$ will be $.2$ (area of a rectangle), which means the integral from 0 onward must be $.8$.
        We say that $.2 + \frac{c}{2} = .8 \Leftrightarrow c = 1.2$
    \subsection*{(b)}
        \[
            F(y) = 
            \begin{cases}
                0, & y \leq -1\\
                .2y|_{-1}^y = .2y + .2, & -1 < y \leq 0\\
                .2(0) + .2 + \left[.2y + \frac{1.4y^2}{2}\right]_0^y = .2 + .2y + \frac{1.2y^2}{2}, & 0 < y \leq 1\\
                1, & y > 1
            \end{cases}
        \]
    \subsection*{(c)}
        Graph $f(y), F(y)$. See \fbox{Figure 3}.
        \begin{figure}[ht]
            \centering
            \def\svgwidth{.5\linewidth}
            \import{./figures/}{pdf-4.18b.pdf_tex}
            \caption{Graphs of $f(x)$ and $F(x)$}
            \label{fig:pdf-4.18b}
        \end{figure}
    \subsection*{(d)}
        $F(-1) = 0, F(0) = .2, F(1) = 1$
    \subsection*{(e)}
        $P(0 \leq Y \leq .5) = F(.5) - F(0) = .2 + .2(.5) + \frac{1.2(.5)^2}{2} - [.2] = .25$
    \subsection*{(f)}
        $P(Y > .5 | Y > .1) = \frac{P(Y > .5)}{P(Y > .1)} = \frac{1 - P(Y < .5)}{1 - P(Y < .1)} = \frac{1 - .45}{1 -.2 + .2(.1) + \frac{1.2(.1)^2}{2}} = .67$.
\section*{Problem 4.28}
    \[
        f(y) =
        \begin{cases}
            cy^2 (1-y)^4, & 0 \leq y \leq 1\\
            0, & \text{otherwise}
        \end{cases}
    \]
    \section*{(a)}
        Find $c$ such that $f(y)$ is a probability density function.
        \[
            \int_0^1 cy^2 (1-y)^4 dy = \frac{c}{105} \Longrightarrow c = 105
        \]
    \section*{(b)}
        Find $E(Y)$.
        \[
            E(Y) = \int_{-\infty}^\infty yf(y)dy = 105 \int_{0}^1 y^3(1-y)^4dy = \frac{3}{8} = .375
        \]
\section*{Problem 4.32}
    \[
        f(y) =
        \begin{cases}
            \frac{3}{64}y^2(4-y), & 0 \leq y \leq 4\\
            0, & \text{otherwise}
        \end{cases}
    \]
    \subsection*{(a)}
        Find $E(Y)$ and $V(Y)$.
        \begin{align*}
            E(Y) &= \int_{-\infty}^\infty y f(y) dy\\
            &= \int_0^4 y f(y) dy \\
            &= \int_0^4 y\frac{3y^2(4-y)}{64} dy \\
            &= \int_0^4 \frac{3y^3(4-y)}{64} dy \\
            &= \int_0^4 \frac{12y^3 - 3y^4}{64} dy \\
            &= \frac{12y^4}{4 \cdot 64} - \frac{3y^5}{5 \cdot 64} \bigg|_0^4\\
            &= \frac{12}{5} = 2.4
        \end{align*}
        \begin{align*}
            V(Y) &= E(Y^2) - E(Y)^2\\
            &= \int_0^4 \frac{12y^4 - 3y^5}{64} dy - 2.4^2 \\
            &= .64
        \end{align*}
    \subsection*{(b)}
        If $Y$ is our weekly CPU time in hours, and the cost is \$200 dollars per hour, we cna find
        the expected value and variance of CPU cost by multiplying $V(Y)$ and $E(Y)$ by 200.
        The expected weekly cost is $200 \cdot 2.4 = 480$ dollars. The variance will be $128$ dollars.
    \subsection*{(c)}
        We expect this to occur somewhat often. For this to occur we must use at least 3 hours of CPU.
        $P(Y > 3) = \int_3^4 \frac{3}{64}y^2(4-y)dy = \frac{3}{64}\int_3^4(4y^2 - y^3)dy = \frac{67}{256} = .263$.
        This means our probability of this occurring will be more than $\frac{1}{4}$, which depending on how often is 
        defined is a fairly reasonable chance.
\section*{Problem 4.40}
    Here we have a continuous uniform distribution, which must have a constant function between the endpoints $A,B$ and must integrate to $1$ so we say,
    $P(x_i \in (A,B) \text{ for some } i \in \{1,2,3\}) = 1 - P(x_i \notin (A,B) \forall i = 1,2,3) = 1 - \left(\frac{1}{2}\right)^3 = 1 - \frac{1}{8} = \frac{7}{8} = .875$
\section*{Problem 4.48}
    \subsection*{(a)}
        Here we have another continuous uniform distribution on the interval $[0,500]$. It must be the case that
        $f(y)$ is constant and that $\int_0^{500}f(y)dy = 1$ so $f(y) = \frac{1}{500}$.
        $\frac{1}{500} \int_{475}^{500} 1 dy = \frac{1}{500}[500 - 475] = \frac{25}{500} = .05$
    \subsection*{(b)}
        We have the same probability except we are integrating over $(0, 25)$ but since the length is still 25, we get the same value.
        $.05$
    \subsection*{(c)}
        $P(Y > 250) = \int_{250}^{500} \frac{1}{500} dy = \frac{250}{500} = .5$.
\section*{Problem 4.50}
    We have success from 12:00-1:00, failure from 1:00-3:00, success from 3:00-4:00, and failure from 4:00-5:00. Our total probability is 1 distributed 
    evenly over the 5 hour period. Since $\frac{2}{5}$ of the possible time lies within the windows of success, we say the probability of success is
    $\frac{2}{5} = .4$.
\end{document}