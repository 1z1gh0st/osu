\documentclass{article}
\usepackage{times}
\usepackage{amssymb, amsmath, amsthm}
\usepackage[margin=.5in]{geometry}
\begin{document}
\section*{Problem 5}
\subsection*{(a)}
Let $T(n,k)$ denote the runtime of MergeArrays, then let $S(p,q)$ denote the runtime of Merge.
We say
\[
    T(n,k) = \sum_{i=1}^k S([i-1]n, n)    
\]
Then it follows that 
\[
    (n,k) = \sum_{i=1}^k O([i-1]n + n) = \sum_{i=1}^k O([in- n + n) = \sum_{i=1}^k O(in) 
\] 
Then we say 
\[ 
    \sum_{i=0}^k O(in) = \sum_{i=0}^k iO(n) = O(n) \sum_{i=0}^k i = O(n) \frac{k(k-1)}{2} = O(n * k^2)
\]
\subsection*{(b)}

\begin{verbatim}
procedure MergeArrays(X_1[1..n], X_2[1..n], ... , X_k[1..n]) {
    midpoint = floor(k / 2);
    left_half = MergeArrays(X_1[1..n], X_2[1..n], ... , X_midpoint[1..n]);
    right_half = MergeArrays(X_midpoint[1..n], ... , X_k[1..n]);
    return Merge(left_half, right_half);
}
\end{verbatim}
\subsection*{(c)}
Let $T(n, k)$ denote the runtime of MergeArrays, and $S(p,q)$ denote the runtime of Merge.
Then
\begin{align*}
    T(n,k) &= O(1) + T(n, \frac{k}{2}) + T(n, \frac{k}{2}) + O(\frac{nk}{2} + \frac{nk}{2}) = O(nk) + 2T(n, \frac{k}{2})\\
    &= O(nk) + 2O(\frac{nk}{2}) + 4O(\frac{nk}{4}) + \cdots + 2^{\log_2(k)}O(\frac{nk}{2^{\log_2(k)}})\\
    &= O(nk) + O(nk) +\cdots + O(nk) = \log_2(n) O(nk) = O(nk \log n)  
\end{align*}

\section*{Problem 6}
\subsection*{(a)}
\fbox{Case 1: We have no increasing exponential subsequence to which we can append $A[i]$} If this is the case then $S(i) = 1$.\\\\
\fbox{Case 2: Otherwise} There exists some previous increasing exponential subsequence to which we can append $A[i]$. Choose the longest of these, and add 1 to its length.
\[
    S(i) = \max \left\{ 1, \max_{j \in \{1,\cdots,i-1\}}\{S(j) + 1 : A[i]>2A[j]\}\right\}
\]

\subsection*{(b)}
\begin{verbatim}
function LongestExpSubsequence(array[1..n]) {
    running_max = 1
    length = [] * n
    for i in {1,2,..,n} {
        length[i] = 1
        for j in {1,2, ..,i-1} {
            if 2array[j] < array[i] {
                length[i] = max{length[i], 1 + length[j]}
            }
        }
        running_max = max{running_max, length[i]}
    }
}
\end{verbatim}

\subsection*{(c)}
What is the running time of this algorithm.\\\\
The running time of this algorithm will be $O(n^2)$ because
\end{document}