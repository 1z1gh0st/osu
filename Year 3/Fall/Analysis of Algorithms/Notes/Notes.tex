\documentclass{article}
\usepackage{listings}
\usepackage{times}
\usepackage{amssymb, amsmath, amsthm}
\usepackage[margin=.5in]{geometry}
\usepackage{graphicx}
\usepackage[linewidth=1pt]{mdframed}

\usepackage{import}
\usepackage{xifthen}
\usepackage{pdfpages}
\usepackage{transparent}

\newcommand{\incfig}[1]{%
    \def\svgwidth{\columnwidth}
    \import{./figures/}{#1.pdf_tex}
}

\newtheorem{theorem}{Theorem}[section]
\newtheorem{lemma}{Lemma}[section]
\newtheorem*{remark}{Remark}
\theoremstyle{definition}
\newtheorem{definition}{Definition}[section]

\begin{document}

\title{Analysis of Algorithms - Notes}
\author{Philip Warton}
\date{\today}
\maketitle
\section{Graph Search Algorithms}
We begin with this simple example of the ``whatever first search''. This is an algorithm that will brute force find some
path from $v \rightarrow s$ for any $s$-reachble vertex $v$.
\begin{mdframed}
    Whatever-first-search algorithm with path rememberance:
    \begin{lstlisting}[mathescape=true]
WFS(G,s) {
    Parent(s) = $\O$
    Bag = $\{(s, \O)\}$
    while Bag $\neq \O$ {
        $(v,p)$ = any vertex from Bag
            (remove $v$ from Bag)
        if $v$ is not marked {
            mark $v$
            Parent(v) = p
            for all $(v,w) \in E$ {
                add $(w,v)$ to Bag
            }
        }
    }
}
    \end{lstlisting}
\end{mdframed}
\begin{figure}[ht]
    \centering
    \incfig{wfs}
    \caption{Whatever First Search For $v_8 \rightarrow s$}
    \label{fig:wfs}
\end{figure}
Once the bag is eventually empty, we can find the path from 
$v$ to $s$ by
\[
    v \rightarrow \text{parent}(v) \rightarrow \text{parent}(\text{parent}(v)) \rightarrow \cdots \rightarrow s
\]
The WFS algorithm marks all vertices reachable from $s$.
\begin{proof}
    Let $s$ be our initial point. We use induction that is based on the shortest path $s$ to $v$. \\\\
    \fbox{Base Case:} Our vertex $v = s$, and ShortestPathLength$(v\rightarrow s) = 0$, and WFS marks it on the first iteration.\\\\
    \fbox{Inductive Step:} For any point $v$ for which the shortest path $s \leftarrow v$ is smaller than $k \in \mathbb{N}$,
    we assume that WFS has already marked $v$. Let $v$ be a point for which its minimum distance from $s$ is $k$.
    Then let $u$ be the neighbor of $v$ that lies on 
    a shortest path from $v$ to $s$. Then the length of $u \rightarrow s$ is $k-1$, and by assumption $u$ is marked.
    Since $v$ is a neighbor of $u$, $v$ will be marked as well.
\end{proof}
What kind of data structures would be good to use for Bag? If 
we use a stack, then this algorithm becomes a depth first search algorithm (DFS).
If we use a queue, then we have a breadth first search (BFS) algorithm.
If there is a weighted graph, one can use a priority queue based on edge weight,
resulting in Dijkstra's shortest path algorithm.
\end{document}