\documentclass{article}

\usepackage{times}
\usepackage{amssymb, amsmath, amsthm}
\usepackage[margin=.5in]{geometry}
\usepackage{graphicx}
\usepackage{tikz}

\usepackage{import}
\usepackage{xifthen}
\usepackage{pdfpages}
\usepackage{transparent}

\newcommand{\incfig}[1]{%
    \def\svgwidth{\columnwidth}
    \import{./figures/}{#1.pdf_tex}
}

\begin{document}

\title{Real Analysis Assignment 5}
\author{Philip Warton}
\date{\today}
\maketitle
\section*{Preamble}
\subsection*{Characterization of the Closure}
    Let $A$ be a set, and $\mathcal{L}$ be the set of limit points of $A$, that is,
    \[ \mathcal{L} = \{l \in M : \forall \epsilon > 0, \ \ \ \ B_\epsilon(l) \setminus \{l\} \cap A \neq \O \} \]
    We claim that $A \cup \mathcal{L} = \overline{A}$.
    \begin{proof}
        In order to show this we wish to show that both $A \cup \mathcal{L} \subset \overline{A}$ and $A \cup \mathcal{L} \supset \overline{A}$.\\\\
        \fbox{$\subset$} Let $x \in A \cup \mathcal{L}$ be arbitrary. If $x \in A$, then $x \in \overline{A}$ since we know that $A \subset \overline{A}$
        by definition. Otherwise we know that $x \in \mathcal{L} \setminus A$. Suppose by contradiction that $x \in \left(\overline{A}\right)^c$. Since $\overline{A}$ is closed, its complement is open.
        Then if $x$ is in the open set $ \left(\overline{A}\right)^c$, it follows that $\exists \epsilon > 0$ such that $B_\epsilon(x) \subset \left(\overline{A}\right)^c$.
        However since $x$ is a limit point of $A$, we know that every $\epsilon$-ball of $x$ intersects $A$ at some point other than $x$ and we have a
        contradiction. Therefore $x \notin \left(\overline{A}\right)^c$, and $x \in \overline{A} \Longrightarrow A \cup \mathcal{L} \subset \overline{A}$. \\\\
        \fbox{$\supset$} Let $x \in \overline{A}$ be arbitrary. Then we know that every closed set containing $A$ contains $x$. If $x \in A$, then trivially
        $x \in A \cup \mathcal{L}$. Suppose that $x \notin A$, then by contradiction suppose $x \notin \mathcal{L}$. Then $\exists \epsilon > 0$ such that
        $B_\epsilon(x) \cap A = \O$. We can exclude the possibility of having an isolated point since $x \notin A$ by assumption. Then we know that $B_\epsilon(x)^c$
        is going to be a closed set containing $A$. Since there exists a closed set containing $A$ such that $x$ does not belong to it, we say $x \notin \overline{A}$
        (contradiction).\\\\
        Having shown both inclusions, it follows that $A \cup \mathcal{L} = \overline{A}$.
    \end{proof}
\subsection*{Problem 4.46 (Characterizations of Dense Sets)}
    A set $A$ is dense in $M$ if and only if the following conditions hold:
    \begin{align*} 
        (a) & \text{ Every point in $M$ is the limit of a sequence in $A$.}\\
        (b) & \text{ For every point in $M$, every $\epsilon$-neighborhood intersects $A$.}\\
        (c) & \text{ Every non-empty open set intersects $A$. }\\
        (d) & \text{ The interior of the complement of $A$ is empty. }
    \end{align*}
    \begin{proof}
        \fbox{Dense $\Rightarrow (a)$} To show this, assume that $A$ is dense in $M$, that is, $\overline{A} = M$. Let $x \in M$ be arbitrary.
        Then $x \in \overline{A}$. Then we know that either $x \in A$ or $x \in \mathcal{L}$ where $\mathcal{L}$ is the set of limit points of $A$.
        If $x$ is a limit point of $A$, simply construct a sequence $(x_n)$ where for every $x_n$ choose some element $y \in B_{\frac{1}{n}}(x) \setminus \{x\} \cap A$
        which we know to exist by the definition of limit point. Then $\forall n \in \mathbb{N}, x_n \in A$ and $(x_n) \rightarrow A$. So if $x$ is not a 
        limit point of $A$, then $x \in A$. Simply choose the constant sequence $(x_n)$ such that $x_n = x \forall n \in \mathbb{N}$ and trivially $(x_n) \rightarrow x$
        and the sequence is contained in $A$.\\\\
        \fbox{$(a) \Rightarrow (b)$} Assume that every point in $M$ is the limit of a sequence contained in $A$.
        Let $x \in M$ be arbitrary, and let $\epsilon > 0$ be arbitrary. Since there exists a sequence in $A$
        that converges to $x$, call this sequence $x_n$, we know that this sequence is eventually in $B_\epsilon(x)$
        and therefore $B_\epsilon(x) \cap A$ is not empty.\\\\
        \fbox{$(b) \Rightarrow (c)$} Assume that $\forall x \in M, \forall \epsilon > 0,\ \ B_\epsilon(x) \cap A \neq \O$.
        Let $U \subset M$ be an open set that is non-empty. Then we know that there exists some $\epsilon$-neighborhood of
        some point $x \in M$ such that $B_\epsilon(x) \subset U$. By assumption this ball intersects $A$, therefore
        $U$ also intersects $A$.\\\\
        \fbox{$(c) \Rightarrow (d)$} Assume that every non-empty open set in $M$ intersects $A$. Suppose by contradiciton
        there exists some $x$ that belongs to the interior of $A^c$. Since the interior must be open, there exists some
        $\epsilon$-neighborhood of $x$ that is contained within the interior of $A^c$. However since this is a non-empty
        open set in $M$, it follows that $B_\epsilon(x)$ intersects $A$, and therefore cannot be contained in $int(A^c) \subset A^c$ (contradiciton).
        Since supposing that $\exists x \in int(A^c)$ leads to a contradiction, $int(A^c)$ must be empty.
        \\\\
        \fbox{$(d) \Rightarrow $ Dense} Assume that $int(A^c)$ is empty. We wish to show that $\overline{A} = M$. Since $M$ is 
        our unviersal set, we say that $\overline{A} \subset M$ by definition. To show that $\overline{A} \supset M$, let $x \in M$
        be arbitrary. Suppose by contradiction that $x \notin \overline{A}$. Then since we know the closure of $A$ is closed, its complement must be open.
        This means that $\exists B_\epsilon(x) \subset \left(\overline{A}\right)^c \subset A^c$. However this would imply that 
        $int(A^c)$ is non-empty (contradiction). Thus $\overline{A} = M$.\\\\
        Having shown each implication, we can now characterize a set as being totally dense by any one of these conditions.
    \end{proof}
\section*{Problem 5.7}
    \subsection*{Question (a)}
        If $f:M \rightarrow \mathbb{R}$ is continuous and $a \in \mathbb{R}$, show that the sets $A = \{x : f(x) > a\}$
        and $B = \{x : f(x) < a\}$ are open subsets of $M$. 
        \begin{proof}
            Since $f$ is a continuous function, we say that for any open set in $\mathbb{R}$, its pre-image
            must be open in $M$. We can write
            \[
                A = \{x \in M : f(x) \in (a, \inf)\} = f^{-1}(a,\inf) \ \ \ \ \ \ \ \ \ \ \ \ \ B = \{x \in M : f(x) \in (-\inf, a)\} = f^{-1}(-\inf,a)
            \]
            Since both intervals are open sets in $\mathbb{R}$, $A$ and $B$ are open
            in $M$.
        \end{proof}
    \subsection*{Question (b)}
        Now we must show the converse. That is, show that if $f^{-1}(-\inf, a)$ and $f^{-1}(a, \inf)$ are open in $M$
        for every $a \in \mathbb{R}$, then $f$ is continuous.
        \begin{proof}
            Let $B(a)$ denote $f^{-1}(-\inf, a)$ and $A(a)$ denote $f^{-1}(a, \inf)$. Let $U \in \mathbb{R}$ be
            an open set, we wish to show that $f^{-1}(U)$ is open in $M$. Notice that we can write any interval in $\mathbb{R}$
            as
            \[
                (x,y) = (-\inf, y) \cap (x, \inf) = B(y) \cap A(x)
            \]
            This means that $f^{-1}(x,y) = f^{-1}(B(y)) \cap f^{-1}(A(x))$. This means that the pre-image
            of the interval $(x,y)$ is a finite intersection of open sets, and therefore is open. Since any
            open set in $\mathbb{R}$ can be written as a union of open intervals (we have shown that the set
            of such intervals with rational endpoints is a valid basis for the standard topology on $\mathbb{R}$ in a previous homework,
            so it would follow that the set of all open intervals would be as well), we say
            \[
                U = \bigcup_{x \in X, y \in Y} (x, y) \ \ \ \ \ \ \ \text{and} \ \ \ \ \ \ \ \ f^{-1}(U) = f^{-1}\left(\bigcup_{x \in X, y \in Y}(x,y)\right) = \bigcup_{x \in X, y \in Y} (f^{-1}(x,y))
            \]
            Then since every set $f^{-1}(x,y)$ is open in $M$, a union of such sets must also be open. Thus
            $f^{-1}(U)$ is open in $M$ for an arbitrary open set $U \subset \mathbb{R}$, and we say that $f$ is continuous.
        \end{proof}
    \subsection*{Question (c)}
        Show that $f$ is continuous even if we take the sets $A(a)$ and $B(a)$ upon rational numbers.
        \begin{proof}
            The proof is the same, except we say that any rational interval $(p,a) = A(p) \cap B(q)$. Then it follows
            that the pre-image any interval of this form will be an open set in $M$ since $f^{-1}(p,a) = f^{-1}A(p) \cap f^{-1}B(q)$.
            Then since we know
            that the set of all rational intervals form a basis for the standard topology on $\mathbb{R}$. We write
            for any $U \subset \mathbb{R}$,
            \[
                U = \bigcup_{p \in P, q \in Q}(p,q)
            \]
            And the pre-image of $U$ is equal to the union of the all the pre-images of intervals of the form $(p,q)$,
            thus $f^{-1}(U)$ is a union of open sets in $M$ and is therefore open. Hence-ly, $f$ is continuous.
        \end{proof}
\section*{Problem 5.17}
    Let $f,g:(M,d) \rightarrow (N, \rho)$ be continuous, and let $D$ be a dense subset of $M$.
    If $f(x) = g(x)$ for every $x \in D$, show that $f(x) = g(x)$ for all $x \in M$. If $f$ is onto,
    show that $f(D)$ is dense in $N$.
    \begin{proof}
        Assume that $f(x) = g(x)$ for every $x \in D$. We wish to show that for every $x \notin D$,
        we still have $f(x) = g(x)$. Let $x \notin D$ be arbitrary. Then by our characterization of 
        dense sets, we know that there exists some sequence $(x_n) \in D$ such that $(x_n)\rightarrow x$.
        Since $f$ and $g$ are both continuous we have both
        \[
            f(x_n) \rightarrow f(x), \ \ \ \ \ \ \ \ \ \ g(x_n) \rightarrow g(x)
        \]
        Then we know that for every element of $D$, its image under $f$ and $g$ is the same, so we say that
        for every natural number $n$ we have $f(x_n) = g(x_n)$. We have two convergent sequences that are equal 
        to each other,
        so it must be the case that their limits are equal, thus $f(x) = g(x)$.
    \end{proof}
    Now that we have proved that $f(x) = g(x)$ for every point $x \in M$, we must prove that if $f$
    is onto, then $f(D)$ is dense in $N$.
    \begin{proof}
        Assume that $f$ is onto. Let $x \notin f(D)$, we wish to show that any neighborhood of $x$ intersects
        $f(D)$. Let $B_\epsilon(x)$ be arbitrary. Since $f$ is continuous, we know that $f^{-1}(B_\epsilon(x))$
        is an open set in $M$. Since $f$ is onto we know that this set is non-empty. Then since we have a
        non-empty open set in $M$ we say $f^{-1}(B_\epsilon(x)) \cap D \neq \O$. Let $y$ be an element of this intersection.
        Then $f(y) \in f(D) \cap B_\epsilon(x)$, and we say that $f(D)$ is dense in $N$.
    \end{proof}
\section*{Problem 5.56}
    Let $f:(M,d) \rightarrow (N, \rho)$.
    \subsection*{(i)}
        Provide examples that show that continuity does not imply an open map, and the converse. Let
        $f: (\mathbb{Q},d) \rightarrow (\mathbb{R},d)$ such that $f(x) = x$ (identity map). Then we say
        that $f$ is continuous but $f$ is not an open map. Since both metric spaces share the same metric,
        and $f(x) = x$, it follows that if $d(x_n,x) \rightarrow 0$ in $(\mathbb{Q}, d)$ then it would do the
        same in $(\mathbb{R}, d)$. Thus $f$ is continuous. Let $U$ be a non-empty open set in $(\mathbb{Q}, d)$.
        The image of this set $f(U)$ will be a set of rational points in $\mathbb{R}$. Since $\mathbb{Q}^c$ is 
        dense in $\mathbb{R}$, it follows that the interior of $\mathbb{Q}$ is empty, and therefore no subset
        of $\mathbb{Q}$ can be open in $\mathbb{R}$ and $f$ is not an open map.\\\\
        For an open map that fails to be continuous, let $g: (\mathbb{R},d) \rightarrow (\{0,1\},d')$ where
        $d$ is the standard metric $|x - y|$, and $d'$ is the discrete metric. Define $g$ by the rule
        \[
            g(x) =
            \begin{cases}
                0 & x \leq 0\\
                1 & x > 0
            \end{cases}
        \]
        One can show easily that this function is not continuous $\frac{1}{n} \rightarrow 0$ but $f(\frac{1}{n}) \rightarrow 1 \neq f(0)$.
        However, since $(\{0,1\}, d')$ bears the discrete metric, every set is simultaneosly open and closed, so we are
        guaranteed an open map.
    \subsection*{(ii)}
        Our previous example $g:\mathbb{R} \rightarrow \{0,1\}$ will serve as an example of a discontinous closed map.
        Since every subset of $\{0,1\}$ is both open and closed under the discrete metric, given a closed set $U \subset \mathbb{R}$,
         $g(U)$ is closed trivially.
\section*{Qualifying Exam Problem 2}
    \subsection*{(a)} 
        Given $a \in \mathbb{R}$ denote by $\{a\}$ the 
        fractional part of $a$; that is,
        \[ 
            \{a\} = \min \{a-n:n\in\mathbb{Z},n\leq a\}    
        \]
        Our first claim is that $\{a\} = a - \lfloor a \rfloor $. This proven instantly by the fact that $\lfloor a \rfloor = \max\{n \in \mathbb{Z} : n \leq a\}$.
        By subtracting the maximal such integer, we take the minimal total. Our next claim about this function is that $\{a + b\} = \{\{a\} + \{b\}\}$.
        This can be shown by noticing that the fractional part function will be integer periodic. That is, $\{n + a\} = \{a\}$ for any integer $n$. Then
        we write
        \[
            \{a + b\} = \{ \lfloor a \rfloor + \{a\} + \lfloor b \rfloor + \{b\} \} = \{\{a\} + \{b\}\} \ \ \ \ \ \ \ \ \ \ \text{since $\lfloor x \rfloor$ is an integer for any $x \in \mathbb{R}$}
        \]
        Our third claim is that for any integer $n, \{na\} = \{n\{a\}\}$. We show this with the following:
        \[
            \{n\{a\}\} = \{n (a - \lfloor a \rfloor )\} = \{na - n \lfloor a \rfloor \} = \{na\} \ \ \ \ \ \ \ \ \text{since $-n \lfloor a \rfloor$ is an integer}
        \]
        Suppose that $\alpha$ is a real irrational number. Prove that the set 
        \[
            A_\alpha = \{ \{n\alpha\} : n \in \mathbb{Z}\}
        \]
        is dense in $[0,1]$.

        \begin{proof}
            This proof will come in two parts. Firstly we show that $\forall \epsilon > 0$, there exists some $t = \{n \alpha \} \in A_\alpha$ such that 
            $0 < t < \epsilon$. Secondly we will show that any arbitrary open set intersecting $[0,1]$ will contain some element of $A_\alpha$.\\\\
            \fbox{$\exists t < \epsilon$} Let $\alpha \in \mathbb{R} \setminus \mathbb{Q}, \epsilon > 0$ be arbitrary.
            We know by the archimedean property that there exists some natural number $y$ such that $\frac{1}{y} < \epsilon$.
            Then rewrite the closed unit interval as follows:
            \[
                [0,1] = \left[\frac{0}{y}, \frac{1}{y}\right] \cup \left(\frac{1}{y}, \frac{2}{y}\right] \cup \cdots \cup \left(\frac{y-1}{y}, \frac{y}{y}\right]
            \]
            We have a partition of $[0,1]$ resulting in $y$ disjoint intervals, each of which has a length that is at most $\frac{1}{y} < \epsilon$.
            Given this fact, we know that for any $n, \{n \alpha\}$ will be contained in exactly one of these intervals. So observe the set
            \[
                \{\alpha, 2 \alpha, 3 \alpha , \cdots , y \alpha, [y+1] \alpha\}
            \]
            In particular, notice that there are $y+1$ objects, and we have only $y$ possible choices for intervals. It is guaranteed by the 
            pidgeonhole principle that there exists some interval such that the fractional part of at least two elements from the above set belong to it.
            Call these elements $y_1 \alpha, y_2 \alpha$. Since their fractional parts belong to the same interval, we say
            \begin{align*}
                \{y_1 \alpha\}, \{y_2 \alpha\} &\in \left(\frac{k}{y}, \frac{k+1}{y}\right) & \text{(we can exclude endpoints since $\alpha$ is irrational)}\\
                \Rightarrow |\{y_1 \alpha\} - \{y_2 \alpha\}| &< \frac{1}{y}\\
                |\{(y_1 - y_2) \alpha \}| = \{(y_1 - y_2) \alpha \} & < \frac{1}{y}
            \end{align*}
            Because we know that $y_1 - y_2 \in \mathbb{Z}$, it follows that $A_\alpha \ni \{(y_1 - y_2) \alpha \} < \frac{1}{y} < \epsilon$. \\\\
            \fbox{$A_\alpha$ is dense in $[0,1]$} Let $x \in [0,1], \epsilon > 0$ be arbitrary. We want to show that $B_\epsilon(x) \cap A_\alpha$ is non-empty.
            Let $0 < \delta < \frac{\epsilon}{3}$. Then it is guaranteed that there exists some interval $[(a) \delta, (a+1)\delta]$ where $a \in \{0,1,2\cdots\}$
            such that $[(a)\delta, (a+1)\delta] \subset B_\epsilon(x)$. Let $t \in A_\alpha$ such that $0 < t < \delta$.
            It follows that there exists some whole number $n$ such that $nt \in [(a)\delta, (a+1)\delta] \subset B_\epsilon(x)$. Then 
            \[
                nt = n\{m\alpha\} = \{ n m \alpha\} \in A_\alpha \cap B_\epsilon(x)
            \]
            Hence, $A_\alpha$ is dense in $[0,1]$.
            \begin{figure}[hbt!]
                \centering
                \def\svgwidth{.5\linewidth}
                \import{./figures/}{delta-partition.pdf_tex}
                \caption{$\delta$-partition of the Closed Unit Interval}
                \label{fig:delta-partition}
            \end{figure}
        \end{proof}
    \subsection*{(b)}
        A function $f: \mathbb{R} \rightarrow \mathbb{R}$ is called $p$-periodic if $f(x+p) = f(x)$ for every
        $x \in \mathbb{R}$. Prove that any continuous function that is both $1$-periodic and $\alpha$-periodic for
        some irrational $\alpha$ must be a constant function.
        \begin{proof}
            Let $f$ be a real valued continuous function that is both $\alpha$-periodic and $1$-periodic. First, notice that since 
            $f$ is $1$-periodic, $f(x) = f(nx) = f(\{x\})$ since $n$ can be any integer and there of course exists some integer $n$ such
            that $nx = \{x\}$ by definition. Then since $f$ is $\alpha$-periodic, $f(n\alpha) = f(\alpha)$ for any $n \in \mathbb{Z}$.
            Finally since $f$ is continuous, $x_n \rightarrow x \Rightarrow f(x_n) \rightarrow f(x)$.
            Let $x \in \mathbb{R}$ be arbitrary.
            By \fbox{Part (a)} we know that $\exists (x_n) \rightarrow \{x\}$ such that $x_n \in A_\alpha$ for every $n$.
            Then, since $f$ is $\alpha$-periodic, $f(x_n) = f(m_{x_n}\alpha) = f(\alpha)$ for every $n$. Thus the limit of our constant sequence will be $f(\{x\}) = f(x) = f(\alpha)$.
            Since $f(\alpha) \in \mathbb{R}$ is defined as a real constant number, and for any $x \in \mathbb{R}, f(x) = f(\alpha)$, the function $f$ is constant.
        \end{proof}

\end{document}