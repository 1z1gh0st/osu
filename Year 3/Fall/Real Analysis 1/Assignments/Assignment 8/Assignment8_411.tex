\documentclass{article}

\usepackage{times}
\usepackage{amssymb, amsmath, amsthm}
\usepackage[margin=.5in]{geometry}
\usepackage{graphicx}
\usepackage[linewidth=1pt]{mdframed}

\usepackage{import}
\usepackage{xifthen}
\usepackage{pdfpages}
\usepackage{transparent}

\newcommand{\incfig}[1]{%
    \def\svgwidth{\columnwidth}
    \import{./figures/}{#1.pdf_tex}
}

\newtheorem{theorem}{Theorem}[section]
\newtheorem{lemma}{Lemma}[section]
\newtheorem*{remark}{Remark}
\theoremstyle{definition}
\newtheorem{definition}{Definition}[section]

\begin{document}

\title{Real Analysis - Assignment 8}
\author{Philip Warton}
\date{\today}
\maketitle
    \begin{mdframed}
        A sequence of real valued functions $f_n : X \rightarrow \mathbb{R}$ is uniformly continuous if and
        only if it is uniformly Cauchy.
    \end{mdframed}
    \begin{proof}
        We must show the bi-conditional by showing that the implication holds in both directions. \\\\
        \fbox{$\Rightarrow$} Assume that $f_n$ is uniformly convergent. Then $||f_n - f||_\infty \rightarrow 0$.
        Equivalently, we say that $\sup_{x \in X} | f_n(x) - f(x) | \rightarrow 0$. Thus we say that for every $\epsilon > 0,
        \exists N \in \mathbb{N}$ such that $\forall n \geqslant N$ $\sup_{x \in X} |f_n(x) - f(x)| < \epsilon$.\\\\
        Choose some $\epsilon' > 0$ arbitrarily. Then $\exists N_{\epsilon' / 2} \in \mathbb{N}$ such that $\forall n 
        \geqslant N_{\epsilon' / 2}, \ \||f_n - f||_\infty < \epsilon' / 2$. Then choose $m,n \geqslant N_{\epsilon' / 2}$
        and it follows that
        \[
            \sup_{x \in X} |f_n(x) - f_m(x)| = \sup_{x \in X} |f_n(x) - f(x) + f(x) - f_m(x)| \leqslant \sup_{x \in X} |f_n(x) - f(x)| + \sup_{x \in X} |f_m(x) - f(x)| \leqslant 2 \epsilon ' / 2 = \epsilon'
        \]
        \fbox{$\Leftarrow$} Assume that $f_n$ is uniformly Cauchy. Then it must be the case that $f_n$ is pointwise Cauchy, and therefore pointwise convergent.
        Thus $f_n \rightarrow f$ pointwise. Suppose that this convergence is not uniform. Then $\exists \epsilon > 0$ such that $||f_n - f||_\infty \geqslant \epsilon \ \ \forall n$.
        Choose some $\epsilon > \delta > 0$ arbitrarily. Then $\exists x \in X$ such that $|f_n(x) - f(x) | > \epsilon - \delta > 0 \forall n$.
        Therefore $f_n$ is not pointwise convergent at some $x$ (contradiction). Finally $f_n$ must be uniformly convergent.
    \end{proof}
\begin{mdframed}
    Let $(X,d), (Y, \rho)$ be metric spaces. Let $f, f_n: X \rightarrow Y$ and let $f_n$ converge uniformly to $f$ on $X$.
    Show that $D(f) \subset \bigcup_{n=1}^\infty D(f_n)$.

\end{mdframed}
\begin{proof}
    Let $x \in X$ such that $x \notin D(f_n)$ for every natural number $n$. If $x$ is some isolated point, then $f(x)$  must be continuous trivially.
    Otherwise, we say that $x_n \rightarrow x \Longrightarrow f_k(x_n) \rightarrow f_k(x)$ for every natural number $k$. Choose $\frac{\epsilon}{3} > 0$ to be arbitrary.
    Then $\exists N_1$ such that $\forall n \geqslant N_1$, $\rho(f_k(x_n),f_k(x)) < \frac{\epsilon}{3}$ since $f_k$ is continuous at $x$. Since $f_k \rightarrow f$
    uniformly, $\exists N_2$ such that $\forall k \geqslant N_2$, $\rho(f_k(x), f(x)) < \frac{\epsilon}{3}$ for every point $x \in X$. Choose $N = \max\{N_1,N_2\}$. Then 
    it follows that for $k,n \geqslant N$
    \[
        \rho(f(x_n),f(x)) \leqslant \rho(f(x_n),f_k(x_n)) + \rho(f_k(x_n), f(x)) \leqslant \rho(f(x_n),f_k(x_n)) + \rho(f_k(x_n),f_k(x)) + \rho(f_k(x), f(x)) < \frac{\epsilon}{3} + \frac{\epsilon}{3} + \frac{\epsilon}{3} = \epsilon
    \] 
    Therefore $x_n \rightarrow x \Longrightarrow f(x_n) \rightarrow f(x)$ so $f$ is continuous at $x$. Thus $x \notin \bigcup_{\mathbb{N}} D(f_n)$ implies $x \notin D(f)$.
\end{proof}
\begin{mdframed}
    Let $f,f_n \in C[0,1]$ where $f_n$ converges uniformly to $f$. Then $\int_0^{1 - 1 / n} f_n \rightarrow \int_0^1 f$.
\end{mdframed}
\begin{proof}
    \[
        \left|\int_0^{1-1/n}f_n - \int_0^1 f\right| = \left|\int_0^1 f_n -f - \int_{1-1/n}^1f_n\right| \leqslant
    \]
    Yikes idk how to do this
\end{proof}

\end{document}