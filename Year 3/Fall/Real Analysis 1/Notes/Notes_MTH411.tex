\documentclass{article}

\usepackage{times}
\usepackage{amssymb, amsmath, amsthm}
\usepackage[margin=.5in]{geometry}
\usepackage{graphicx}
\usepackage{tikz}
\usepackage[linewidth=1pt]{mdframed}

\newtheorem{theorem}{Theorem}[section]
\newtheorem{lemma}{Lemma}[section]
\newtheorem*{remark}{Remark}
\theoremstyle{definition}
\newtheorem{definition}{Definition}[section]

\begin{document}

\title{MTH 411 Post Midterm Notes}
\author{Philip Warton}
\date{\today}
\maketitle
\section{Midterm Solutions and Review}
    \subsection{Let $(M,d)$ be a metric space with the discrete metric. Show that any convergent sequence is eventually constant.}
        \begin{proof}
            Let $(x_n)$ be a convergent sequence in the space.
            Choose $\epsilon = 1$.
            Our sequence will eventually be in the epsilon ball of its limit, and therefore it will be eventually constant.
        \end{proof}
    \subsection{The set $A = \{y \in M : d(x,y) \leqslant r \}$ is called the closed ball with radius $r$ about $x$.}
        \subsubsection{Show that $A$ is closed.}
            \begin{proof}
                Assume that $(y_n)$ is a convergent sequence in $A$.
                We will show that its limit is in $A$.
                Let $\epsilon > 0$ be arbitrary.
                Then,
                \[
                    d(x,y) \leqslant d(x, y_n) + d(y_n, y) \leqslant r + \epsilon
                \]
                Since this is true for any $\epsilon > 0$ we say that $d(x,y) \leqslant r$, and $y \in A$.
            \end{proof}
        \subsubsection{Give an example where $A$ is not the closure of the open ball.}
            Choose the space of integers, with an open ball radius 1 around 0.
            Then $B_1(0) = \{0\}$ is already closed, and is a proper subset of $A$.
    \subsection{If $x_n \rightarrow x$ in a metric space, show that $d(x_n,y) \rightarrow d(x,y)$.}
        \begin{proof}
            By the reverse triangle inequality and the squeeze theorem, the result follows trivially.
        \end{proof}
    \subsection{Show that the collection of polynomials with integer coefficients is countable.}
        \begin{proof}
            Let $\mathcal{P}$ be the set of all polynomials with integer coefficients, $\mathcal{P}_n$ be the set of polynomials $p(x) = \sum_{k=0}^na_kx^k$ with integer coefficients and degree at most $n$. Then
            \[
                \mathcal{P} = \bigcup_{n=0}^\infty \mathcal{P}_n
            \]
            To show that $\mathcal{P}_n$ are countable, map $\mathcal{P}_{n-1}$ onto $Z^n$ with the bijection:
            \[
                f(z_1, z_2, \cdots, z_3) = \sum_{k=1}^n z_k x^k
            \]
            Then we assume that $\mathbb{Q}^n$ is countable, and $\mathbb{Z}^n \subset \mathbb{Q}^n$ and we say that $\mathcal{P}$ must be countable.
        \end{proof}
\section{Continuity}
\section{Homeomorphisms}
\section{Connectedness}
    A space $M$ is said to be disconnected if $M = A \dot \cup B$. That is to say, if $M$ can be written as a disjoint union of open sets.
    Such a construction is called a disconnection of $M$, and $M$ is connected if it yields no disconnection.
    \begin{theorem}
        $M$ is connected if and only if $M$ contains no nontrivial clopen sets.
    \end{theorem}
\section{Completeness}
    \begin{mdframed}
        \begin{definition}[Totally Bounded]
            We define total boundedness to be the following: a set $A$ in a metric space $(M,d)$ is totally bounded $\Leftrightarrow$
            \[
                \forall \epsilon > 0, \exists n \in \mathbb{N}, x_1, \cdots , x_n \in M : A \subset \bigcup_{j=1}^n B_\epsilon(x_j)
            \]
        \end{definition}
    \end{mdframed}
    If we look at $B_1(0) \in l_1$, we find that although this set is bounded, it is not totally bounded.
    \begin{mdframed}
        \begin{theorem}
            We can characterize total boundedness by: $\forall \epsilon > 0 \exists n \in \mathbb{N}, A_1, \cdots, A_n \subset A$ such that diam$(A_j) < \epsilon, j = 1, \cdots, n$ and $A \subset \bigcup_{j=1}^n A_j$.
        \end{theorem}
    \end{mdframed}
    The property of total boundedness can be considered as a generalization of compactness.
    \begin{definition}[Bounded]
        We say that a set $A \subset M$ is bounded if there exists some ball of finite
        radius such that $A$ is contained in this ball.
    \end{definition}
    \begin{lemma} Let $(x_n)$ be a sequence in $(M,d)$ and $A = \{x_n | n \in \mathbb{N}\}$ its range.
        \begin{align*}
            (i) & \text{ if $(x_n)$ is Cauchy, then $A$ is totally bounded}\\
            (ii) & \text{ if $A$ is totally bounded, then $x_n$ has a Cauchy subsequence}
        \end{align*}
    \end{lemma}
    \begin{proof}
        \fbox{(i)} Let $\epsilon > 0$ be arbitrary. Since $(x_n)$ is Cauchy,
        we say that for some $N \in \mathbb{N}$, for every $m,n \geq N, d(x_m, x_n) < \epsilon$.
        So we say that $\bigcup_{n=1}^N B_\epsilon(x_n) \supset A$ and is a finite union of open
        balls, and is therefore open.\\\\
        \fbox{(ii)}
        If $A$ is finite, then every sequence $(x_n) \in A$ has a constant subsequence. Otherwise,
        $A$ will be infinite.
    \end{proof}
    \begin{definition}
        A metric space $(M,d)$ is complete if every Cauchy sequence in $M$ converges to a point in $M$.
    \end{definition}
    Of course the set of real numbers will be complete, however the set of rational numbers will not be complete.
    The Lebesgue space $\ell_2$ is complete. To prove this is fairly difficult.
    \begin{mdframed}
        \begin{theorem}
            For any metric space $M$, the following are equivalent
            \begin{align*}
                (i) & \text{ $M$ is complete}\\
                (ii) & \text{ The Nested Set Property holds}\\
                (iii) & \text{ The Bolzano Weirstrass Property holds. 
                That is, every totally bounded set has a limit point}\\
            \end{align*}
        \end{theorem}
    \end{mdframed}
    This is another way to characterize completeness, this time for a normed vector space.
    \begin{mdframed}
        \begin{theorem}
            A normed vector space $V$ is complete if and only If
            \[
                \sum_{n=1}^\infty ||x_n|| < \infty \Rightarrow \sum_{n=1}^\infty x_n \text { converges in $V$}    
            \]
            Every absolutely summable series in $V$ is summable.
        \end{theorem}
    \end{mdframed}
    \begin{proof}
        \fbox{$\Rightarrow$} Assume $V$ is complete, and let $(x_n) \subset V$ be such that $\sum_{n=1}^\infty ||x_n|| < \infty$.
        Let $S_n$ be the sequence of partial sums. We wish to show that $S_n$ is a cauchy sequence.
        \[
            ||S_n-S_m|| = ||\sum{k = m+1}^n x_k|| \leq \sum_{k = m + 1}^n || x_k|| \rightarrow 0
        \]
        Thus $(S_n)$ is a Cauchy sequence in $V$. Since $V$ is complete $(S_n)$ converges to $S = \sum_{k=1}^\infty x_k$.\\\\
        \fbox{$\Leftarrow$} Now assume that  $\sum ||x_n|| < \infty \Rightarrow \sum x_n$ converges in $v$ and let 
        $(x_n)$ be a Cauchy sequence in $V$. For $k = 1,2, \cdots $ let $N_k$ be such that $\forall n > m \geq N_k : d(x_n, x_m) < 2^{-k}$.
        Then let $m = N_k \Rightarrow x_n \in B_{2^{-k}}(x_{N_k}) \forall n > N_k$.
        Consider the subsequence $y_k = x_{N_k}, k \in \mathbb{N}$.
        Then $y_{k+1} = x_{N_{k+1}} \in B_{2^{-k}}(x_{N_k}) = B_{2^{-k}}(y_k)$. And $||y_{k+1} - y_k|| < 2^{-k}$.
        Hence $\sum_{k=1}^\infty ||y_{k+1} - y_k||$ converges and therefore also $\sum_{k=1}^\infty y_{k+1} - y_k$ converges.
        The partial sums for this series are $S_n = \sum_{k=1}^n y_{k+1} - y_k = y_{nn} - y_1$. Therefore the sequence 
        $(y_k) = (x_{N_k})$ converges. Thus there exists some $x \in M : x = \lim_{k\rightarrow \infty}x_{N_k}$ and $(x_n)$ is Cauchy.
    \end{proof}
    Note: Banach Space is a complete normed vector space $V$.

    \begin{definition}
        A function $f:(M,d) \rightarrow (N,s)$ is called Lipschitz if there is a constant $k < \infty$ such that
        $s(f(x), f(y)) \leq k d(x,y)$ for every $x,y \in M$.
    \end{definition}

    Immediately it should be clear that a Lipschitz mapping will be continuous.
    \begin{proof}
        Let $x_n \rightarrow x$ in $M$. Then $d(x, x_n) \rightarrow 0$. So $s(f(x), f(x_n)) < k d(x, x_n) \rightarrow 0$.
        Thus $s(f(x), f(x_n)) \rightarrow 0$ and $f$ is continuous.
    \end{proof}

    \begin{definition}
        A map $f: M \rightarrow M$ on a metric space $(M,d)$ is called a contraction if there is $0 \leq \alpha <1$ such that 
        $d(f(x), f(y)) \leq \alpha d(x,y)$.
    \end{definition}
    Since a contraction is Lipschitz with $k = \alpha$ it is continuous.


    \begin{definition}
        Let $f: M \rightarrow M$. Any $x \in M$ such that $f(x) = x$ is called a fixed point of $f$.
    \end{definition}

    \begin{mdframed}
        \begin{theorem}
            (Contraction Mapping Theorem, Banach Fixed Point Theorem) Let $(M,d)$ be a complete metric space and let 
            $f: M \rightarrow M$ be a contraction. Then, $f$ has a unique fixed point. For any $x_0 \in M$, the iteration $x_{n+1} = f(x_n)$ converges to $x$.
            One has $d(x_n, x) \leq d(x_1, x_0) \frac{\alpha^n}{1-\alpha}$.
        \end{theorem}
    \end{mdframed}
    \begin{definition}
        Let $f'(x) = f(x), f^{n+1}(x) = f(f^n(x))$, i.e. $f^n$ is the $n$-fold composition of $f$ with itself.
    \end{definition}
    \begin{proof}
        The sequence $x_n$ can be written as $x_n = f^n(x_0)$. Let $x_0 \in M$ be arbitrary.
        \begin{align*}
            d(x_{n+1}, x_n) &= d(f(x_n), f(x_{n-1}))\\
            & \leq \alpha d(x_n, x_{n-1}) = \alpha d(f(x_{n-1}), f(x_{n-2}))\\
            & \leq \alpha^2 d(x_{n-1},x_{n-2})\\
            & \vdots\\
            & \leq \alpha^n d(x_1, x_0) = c \alpha^n & c = d(x_1, x_0)
        \end{align*}
    \end{proof}

\section{Compactness}
    \begin{mdframed}
        \begin{definition}
            A metric space $(M,d)$ is said to be compact if it is both complete and totally bounded.
        \end{definition}
    \end{mdframed}

    \begin{theorem}
        $(M,d)$ is compact if and only if every seuqence has a Cauchy subsequence that converges to a point in $M$.
    \end{theorem}

    \begin{theorem}
        The image of a compact set under a continuous function is compact in metric spaces.
    \end{theorem}

    \begin{theorem}
        Let $(V, ||.||)$ and $(W, |||.|||)$ be normed vector spaces and let $T:V \rightarrow W$ be a linear map. Then the 
        following are equivalent:
        \begin{align*}
            \text{(i)} & \text{$T$ is Lipschitz} \\
            \text{(ii)} & \text{$T$ is uniformly continuous}\\
            \text{(iii)} & \text{$T$ is everywhere continuous}\\
            \text{(iv)} & \text{$T$ is continuous at $0 \in V$}\\
            \text{(v)} & \text{there is a constant $C < \infty$ such that $|||T(x)||| \leqslant C||x||$ for all $x \in V$}
        \end{align*}

    \end{theorem}

    
\end{document}