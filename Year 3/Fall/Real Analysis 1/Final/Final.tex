\documentclass{article}

\usepackage{times}
\usepackage{amssymb, amsmath, amsthm}
\usepackage[margin=.5in]{geometry}
\usepackage{graphicx}
\usepackage[linewidth=1pt]{mdframed}

\usepackage{import}
\usepackage{xifthen}
\usepackage{pdfpages}
\usepackage{transparent}

\newcommand{\incfig}[1]{%
    \def\svgwidth{\columnwidth}
    \import{./figures/}{#1.pdf_tex}
}

\newtheorem{theorem}{Theorem}[section]
\newtheorem{lemma}{Lemma}[section]
\newtheorem*{remark}{Remark}
\theoremstyle{definition}
\newtheorem{definition}{Definition}[section]

\begin{document}

\title{Real Analysis - Final Exam}
\author{Philip Warton}
\date{\today}
\maketitle
\section*{Problem 1}
\begin{mdframed}
    Let $(M,d), (N,\rho)$ be metric spaces, and let $f:M \rightarrow N$ be continuous and onto. If $(M,d)$ is
    seperable then $(N, \rho)$ is seperable.
\end{mdframed}
\begin{proof}
    Suppose $(M,d)$ is seperable. Then there exists some countable dense subset of $M, \{x_n\}_{n \in \mathbb{N}}$.
    Then for any non-empty open set in $M$, we have natural number $n$ such that $x_n$ belongs to this open set.
    We claim that $f(\{x_n\}_{n \in \mathbb{N}})$ is a countable dense subset of $N$. Being the image of a countable set,
    it is obviously countable. Suppose that this set is not dense. Then there exists some non-empty open set $V \subset N$ such that it 
    is disjoint with $f(\{x_n\}_{n \in \mathbb{N}})$. Note that since $f$ is continuous and onto, the pre-image of $V$, $U = f^{-1}(V)$
    is a non-empty open set in $M$. Then, since $f$ is onto we say $A \subset f^{-1}(f(A))$, and we have
    \begin{align*}
        \{x_n\}_{n \in \mathbb{N}} \cap U &\subset f^{-1}(f(\{x_n\}_{n \in \mathbb{N}})) \cap f^{-1}(V) \\
        &= f^{-1}(f(\{x_n\}_{n \in \mathbb{N}}) \cap V) \\
        &= f^{-1}(\O)\\
        &= \O
    \end{align*}
    However, this means that $\{x_n\}_{n \in \mathbb{N}}$ is not dense in $M$ (contradiction). Therefore it must be the case 
    that $f(\{x_n\}_{n \in \mathbb{N}})$ is dense in $N$. Thus $(N , \rho)$ is seperable.
\end{proof}
\section*{Problem 2}
\begin{mdframed}
    Let $g:[0,1] \rightarrow \mathbb{R}$ be continuous. There exists a unique continuous function $f: [0,1] \rightarrow \mathbb{R}$
    such that 
    \[
        f(x) + \int_0^x f(t) \sin(\pi t / 4) dt = g(x) \ \ \ \ \ \forall x \in [0,1]
    \]
\end{mdframed}
\begin{proof}
    Let $h: \mathbb{R} \rightarrow \mathbb{R}$ be a function in $C^{2\pi}$ that equals $g(0)$ on $[- \pi,0)$, $g(1)$ on $(1,\pi]$
    and equals $g(x)$ on $[0,1]$. Then we say that for every $\epsilon$ there exists a trigonometric polynomial such that $||T(x) - g(x)||_\infty < \epsilon$ (Weierstrass's Second Theorem).
    Perhaps it is possible to find some trigonometric polynomial that is orthoganal to $\sin(\pi x / 4)$, that is, $\int_{-\pi}^\pi f(t) \sin( \pi t / 4) dt = 0$.
    Then it may be possible to some limit of such polynomials to provide some function that is orthogonal to $\sin(t \pi / 4)$ and 
    will have the property of 
    \[
        f_n(x) + \int_0^x f_n(t) \sin( \pi t /4)dt = f_n(x) \rightarrow g(x)
    \]
I do not know how to prove this.
\end{proof}
\section*{Problem 3}
\begin{mdframed}
    Let $\mathcal{F} \subset C[0,1]$ where $\mathcal{F} = \{p \in \mathcal{P} : \max_{x \in [0,1]}|p(x)| \leqslant 2 \}$.
    The closed unit ball $B = \{f \in C[0,1] : ||f||_\infty \leqslant 1\}$ is contained in the closure of $\mathcal{F}$.
\end{mdframed}
\begin{proof}
    Let $f \in B$ be arbitrary. Then $||f||_\infty \leqslant 1$ by assumption. By the Weierstrass Approximation Theorem,
    for every $\epsilon > 0$ there exists some polynomial $p$ such that $||f-p||_\infty < \epsilon$. Let $\epsilon < 1$,
    there will exist some polynomial $p$ such that $||f - p||_\infty < \epsilon$. In other words
    \[
        \max_{x \in [0,1]}|f(x) - p(x)| < \epsilon
    \]
    Since $f$ is bounded by 1, it follows that $p$ must be bounded by $1 + \epsilon < 2$. Thus for every $\epsilon < 1$ 
    we say that $p \in \mathcal{F}$, thus as $\epsilon$ approaches 0 we can take a sequence of polynomials in $\mathcal{F}$
    and they will converge to $f$. Finally, $f$ is a limit point of $\mathcal{F}$ for any $f \in B$, and thus $B \subset \overline{\mathcal{F}}$.
\end{proof}
\section*{Problem 4}
\begin{mdframed}
    We define the colleciton of functions $\mathcal{F} \subset C[0,1]$ as 
    \[
        \mathcal{F} = \{\sin(nx) \ | \ n \in \mathbb{N}\}
    \]
\end{mdframed}
\subsection*{(a)}
    \fbox{$\mathcal{F}$ is uniformly bounded.}
    \begin{proof}
        We know that $|\sin(x)|$ is bounded by 1 on all of $\mathbb{R}$. Then for any natural number $n$,
        $nx \in \mathbb{R}$. So it follows that $|\sin(nx)| \leqslant 1$ for every natural number $n$, for every $x \in [0,1]$. Thus 
        we say that this collection of functions is uniformly bounded ($||\sin(nx)||_\infty \leqslant 1  \ \ \forall n \in \mathbb{N}$).
    \end{proof}
\subsection*{(b)}
    \fbox{$\mathcal{F}$ is not equicontinuous.}
    \begin{proof}
        Let $x = 0$, and let $\epsilon = \frac{1}{2}$.
        Then choose any stricly positive $\delta$.
        By the Archimedean Property there exists some natural number $n$ such that $\frac{ 2 \pi}{n} < \delta$.
        It follows then that since $\sin(x)$ is $2\pi$ periodic that $\sin(nx)$ is $\frac{2\pi}{n}$ periodic. Since 
        $\sin(x)$ achieves its maximum 1 within this period then there must exist some $x \in [0,\frac{2\pi}{n}]$ such that 
        $\sin(nx) = 1$. Then since $\sin(n(0)) = 0$ and there exists $y \in [0,\frac{2\pi}{n}] \subset [0, \delta)$ such that 
        $\sin(ny) = 1$, we have $|\sin(nx) - \sin(ny)| = |0 - 1| > \frac{1}{2} = \epsilon$. If we choose $\epsilon = \frac{1}{2}$,
        then for every $\delta > 0$ there is some $n \in \mathbb{N}$ such that $\sin(nx)$ is not uniformly continuous by this $\delta$, and we say
        that the collection $\mathcal{F}$ is not equicontinuous.
    \end{proof}
\subsection*{(c)}
    \fbox{$\mathcal{F}$ is not compact in $C[0,1]$.}
    \begin{proof}
        Observe the sequence $(\sin(x), \sin(2x), \sin(3x), \sin(4x), \cdots)$ we claim that there is no Cauchy subsequence, and that therefore
        the collection is not compact. Choose any $m \in \mathbb{N}$ arbitrarily. Then we say that $\sin(mx)$ is non-negative on $[0, \frac{\pi}{m}]$.
        Then choose any $n \geqslant 2m$, and we say that $\sin(nx)$ will be equal to -1 exactly at $x = \frac{3\pi}{2n} = \frac{3\pi}{4m} \in [0, \frac{\pi}{m}]$.
        It follows that since we have a non-negative function and a function that achieves -1 on this interval that 
        \[
            ||\sin(mx) - \sin(nx)||_\infty \geqslant 1
        \]
        Since any subsequence that is not eventually constant must contain some $n \geqslant 2m$ for any $m \in \mathbb{N}$ arbitrarily, there does not 
        exist any Cauchy subsequence, thus $\mathcal{F}$ is not totally bounded, and is not compact.
    \end{proof}
\section*{Problem 5}
\begin{mdframed}
    Let $a_n = \frac{1}{2^n}$ and consider the sequence of functions $f_n:[0,1] \rightarrow \mathbb{R}$ where
    \[
        f_n(x) = \begin{cases}
            \frac{1}{a_{n+2}^2}(x - a_{n+1})(a_n - x) & x \in [a_{n+1}, a_n]\\
            0 & \text{otherwise}
        \end{cases}
    \]
\end{mdframed}
\subsection*{(a)}
    \fbox{$\forall n \in \mathbb{N} \ \ \max_{x \in [0,1]} |f_n(x)| = 1$}
    \begin{proof}
        We can first restrict our domain to $[a_{n+1}, a_n]$ since any non-zero value will immediately
        have an absolute value greater than 0. Since $f_n(x)$ is a product of 3 positive terms in this domain,
        we say that $|f(x)| = f(x)$. Then by the given hint, we say that
        \[
            \max_{x \in [a_{n+1},a_n]}|f_n(x)| = (a_{n+2})^{-2} \left(\frac{a_{n+1} - a_n}{2}\right)^2 = (a_{n+2})^{-2}(a_1)^2(a_{n+1} - a_n)^2
        \]
        Then we can write this out using the definition of $a_n$, giving us 
        \begin{align*}
            \left(\frac{1}{2^{n+2}}\right)^{-2}\left(\frac{1}{2^1}\right)^2 \left(\left(\frac{1}{2^{n+2}}\right)^2 - 2\left(\frac{1}{2^{n + 1}}\right)\left(\frac{1}{2^n}\right) + \left(\frac{1}{2^n}\right)^2\right)
            &= \left(\frac{1}{2^{n+2}}\right)^{-2}\left(\frac{1}{2^1}\right)^2 \left(\left(\frac{1}{2^{n+1}}\right)^2 - \left(\frac{1}{2^{n}}\right)\left(\frac{1}{2^n}\right) + \left(\frac{1}{2^n}\right)^2\right)\\
            &= \left(\frac{1}{2^{n+2}}\right)^{-2}\left(\frac{1}{2^1}\right)^2 \left(\left(\frac{1}{2^{n+1}}\right)^2 - \left(\frac{1}{2^{n}}\right)^2 + \left(\frac{1}{2^n}\right)^2\right)\\
            &= \left(\frac{1}{2^{n+2}}\right)^{-2}\left(\frac{1}{2^1}\right)^2 \left(\frac{1}{2^{n+1}}\right)^2\\
            &= \left(\frac{1}{2^{n+2}}\right)^{-2}\left(\frac{1}{2^1}\frac{1}{2^{n+1}}\right)^2\\
            &= \left(\frac{1}{2^{n+2}}\right)^{-2}\left(\frac{1}{2^{n+2}}\right)^2\\
            &= 1
        \end{align*}
        Thus for any $n \in \mathbb{N}, \max_{x \in [0,1]}|f_n(x)| = 1$.
    \end{proof}
\subsection*{(b)}
    \fbox{The pointwise limit of $f_n(x)$ is 0 for all $x$.}
    \begin{proof}
        If $x = 0$, then for every $n \in \mathbb{N}, 0 < \frac{1}{2^{n+1}}$, so we say $f_n(x) = 0$ for all $n$,
        thus the constant sequence $(f_n(0)) = (0) \rightarrow 0$. Now let $x \in (0,1]$ be fixed. Then by the convergence of the
        geometric series, we know that $\exists n \in \mathbb{N}$ such that $\frac{1}{2^n} < x$. Thus this sequence is also eventaully the constant
        sequence $(0)$ which converges to 0. Therefore at any point $x \in [0,1]$ the point-wise limit is 0.
    \end{proof}
\subsection*{(c)}
    I invoke my proof of the following property in order to answer the question:
    \begin{mdframed}
        A sequence of real valued functions $f_n : X \rightarrow \mathbb{R}$ is uniformly continuous if and
        only if it is uniformly Cauchy.
    \end{mdframed}
    \begin{proof}
        We must show the bi-conditional by showing that the implication holds in both directions. \\\\
        \fbox{$\Rightarrow$} Assume that $f_n$ is uniformly convergent. Then $||f_n - f||_\infty \rightarrow 0$.
        Equivalently, we say that $\sup_{x \in X} | f_n(x) - f(x) | \rightarrow 0$. Thus we say that for every $\epsilon > 0,
        \exists N \in \mathbb{N}$ such that $\forall n \geqslant N$ $\sup_{x \in X} |f_n(x) - f(x)| < \epsilon$.\\\\
        Choose some $\epsilon' > 0$ arbitrarily. Then $\exists N_{\epsilon' / 2} \in \mathbb{N}$ such that $\forall n 
        \geqslant N_{\epsilon' / 2}, \ \||f_n - f||_\infty < \epsilon' / 2$. Then choose $m,n \geqslant N_{\epsilon' / 2}$
        and it follows that
        \[
            \sup_{x \in X} |f_n(x) - f_m(x)| = \sup_{x \in X} |f_n(x) - f(x) + f(x) - f_m(x)| \leqslant \sup_{x \in X} |f_n(x) - f(x)| + \sup_{x \in X} |f_m(x) - f(x)| \leqslant 2 \epsilon ' / 2 = \epsilon'
        \]
        \fbox{$\Leftarrow$} Assume that $f_n$ is uniformly Cauchy. Then it must be the case that $f_n$ is pointwise Cauchy, and therefore pointwise convergent.
        Thus $f_n \rightarrow f$ pointwise. Suppose that this convergence is not uniform. Then $\exists \epsilon > 0$ such that $||f_n - f||_\infty \geqslant \epsilon \ \ \forall n$.
        Choose some $\epsilon > \delta > 0$ arbitrarily. Then $\exists x \in X$ such that $|f_n(x) - f(x) | > \epsilon - \delta > 0 \forall n$.
        Therefore $f_n$ is not pointwise convergent at some $x$ (contradiction). Finally $f_n$ must be uniformly convergent.
    \end{proof}
    \fbox{$f_n$ does not converge uniformly.}
    \begin{proof}
        Firstly $f_n \in C[0,1]$ for every $n \in \mathbb{N}$. This is the case because the piece on $(a_{n+1},a_n)$ 
        can be expressed as a finite polynomial which is of course continuous. Then outside of this interval we have the continuous
        constant function 0. Finally at $x = a_{n+1}$ and $x = a_{n}$ both functions have a limit point at these values of $x$ and
        the value of their limits is the same, that is 
        \[
            \lim_{x \rightarrow a_n} \frac{1}{a_{n+2}^2}(x - a_{n+1})(a_n - x) = 0, \ \ \ \ \ \ \lim_{x \rightarrow a_n} 0 = 0
        \]
        And similarly,
        \[
            \lim_{x \rightarrow a_{n+1}} \frac{1}{a_{n+2}^2}(x - a_{n+1})(a_n - x) = 0, \ \ \ \ \ \ \lim_{x \rightarrow a_{n+1}} 0 = 0
        \]
        Thus $f_n \in C[0,1]$.\\\\
        Having established this, suppose it does converge uniformly, then it must be uniformly Cauchy in $C[0,1]$ ( by Uniform Convergence Def. and Uniform Convergence Uniform Cauchy Equivalence). However, 
        choose any distinct $m,n \in \mathbb{N}$, and it is guaranteed that $||f_m - f_n||_\infty \geqslant 1$. This is the case because 
        we know that $f_n$ achieves its absolute maximum at the midpoint of $a_{n}$ and $a_{n + 1}$. Since the sequence $(a_n)$ is monotone
        decreasing it must be the case that this midpoint does not lie in $[a_{m+1}, a_m]$ (that is, these intervals must overlap only at endpoints, so no midpoint will lie in two).
        At the point $x = \frac{a_{n+1} + a_n}{2}, |f_n(x) - f_m(x)| = 1$ so it follows that $||f_n - f_m||_\infty \geqslant 1$. Therefore the 
        sequence is not uniformly Cauchy, and thus not uniformly continuous.
    \end{proof}
\subsection*{(d)}
    \fbox{Let $g_n = f_{2n}$. For any $m,n \in \mathbb{N}$, $||g_m - g_n||_\infty = 1$.}
    \begin{proof}
        We have already demonstrated most of the steps in order to prove that this is true.
        Note that with our new series, we choose only every even $n \in \mathbb{N}$. This means that 
        the intervals $[a_{n+1},a_n]$ and $[a_{m+1}, a_m]$ will always be disjoint for any distinct natural 
        numbers $m,n$, as they can never share an endpoint now. Then it follows that
        \[
            ||g_m - g_n||_\infty = \max\left\{\max_{x \in [a_{n+1}, a_n]}|g_m(x) - g_n(x)|, \max_{x \in [a_{m+1}, a_m]}|g_m(x) - g_n(x)|, \max_{x \notin [a_{n+1},a_n] \cup [a_{m+1},a_m]}|g_m(x) - g_n(x)|\right\}
        \]
        This is equal to $\max\{1,1,0\}$ since the intervals are disjoint, and $g_n(x)$ achieves its maximum of 1 within the its interval $[a_{n+1},a_n]$, while $g_n$ will be the constant function 0 there.
        Therefore $||g_m - g_n||_\infty = 1$.
    \end{proof}
\subsection*{(e)}
    \fbox{The set $\mathcal{F} = \{f_n \ | \ n \in \mathbb{N}\}$ is not totally bounded in $C[0,1]$}
    \begin{proof}
        Choose the sequence $(f_1, f_2, f_3, f_4, \cdots)$. As we established in part \fbox{(c)} for any two natural numbers $n,m$
        $||f_n - f_m||_\infty \geqslant 1$. This means it is impossible to take any non-constant Cauchy subsequence of our sequence $(f_1, f_2, \cdots)$.
        Since a set is totally bounded if and only if every sequence yields some Cauchy sub-sequence, it must be the case that $\mathcal{F}$ is not totally bounded.
    \end{proof}
\end{document}