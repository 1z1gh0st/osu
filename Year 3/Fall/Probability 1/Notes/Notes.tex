\documentclass{article}

\usepackage{times}
\usepackage{amssymb, amsmath, amsthm}
\usepackage[margin=.5in]{geometry}
\usepackage{graphicx}
\usepackage[linewidth=1pt]{mdframed}

\usepackage{import}
\usepackage{xifthen}
\usepackage{pdfpages}
\usepackage{transparent}

\newcommand{\incfig}[1]{%
    \def\svgwidth{\columnwidth}
    \import{./figures/}{#1.pdf_tex}
}

\newtheorem{theorem}{Theorem}[section]
\newtheorem{lemma}{Lemma}[section]
\newtheorem*{remark}{Remark}
\theoremstyle{definition}
\newtheorem{definition}{Definition}[section]

\begin{document}

\title{Probability 1 - Lecture Notes}
\author{Philip Warton}
\date{\today}
\maketitle
\section{Markov Inequality}
    Suppose there is a distribution for which we don't know the probability mass function, and we do not know the variance, but we do know it's expectation, $E[x]$.
    What can we say about that probability?
    Can we bound it? \\\\
    \begin{mdframed}
        \begin{theorem}[Markov Inequality]
            If $X$ is a random variable that takes only non-negative values, then for any $\alpha > 0$,
            \[
                P(X \leq \alpha) \leqslant \frac{E[x]}{\alpha}
            \]
        \end{theorem}
    \end{mdframed}
    \begin{proof}
    \[
    P(X \geq \alpha) = \sum_{k : k \geq \alpha} p(\alpha) \leq \sum_{k : k \geq \alpha}\frac{k}{\alpha}p(k) = \frac{1}{\alpha}\sum_{k : k \geq \alpha}k\cdot p(k) \leq \frac{1}{\alpha} \sum_{k : k \geq 0}k \cdot p(k) = \frac{E[X]}{\alpha}
    \]
    \end{proof}
    Note that this would likely work under integration for a continuous random variable.\\\\
    \begin{mdframed}
        \begin{theorem}[Chebyshev Inequality]
            If $X$ is a random variable with a finite mean $\mu$ and variance, then for any $\kappa > 0$,
        \[
            P(|X- \mu | \geq \kappa \sigma) \leq \frac{1}{\kappa^2}
        \]
        \end{theorem}
    \end{mdframed}
\section{Continuous Random Variables}
    \begin{mdframed}
        \begin{definition}
            We say that $X$ is a continuous random variable if there exists a 
            nonnegative function $f(x)$ defined for all real $x$ such that for any $a \leq b$
            \[
                P(a \leq X \leq b) = \in_a^bf(x)dx
            \]
            Such a function $f(x)$ is the probability density function of $X$ \fbox{Figure 1}.
        \end{definition}
    \end{mdframed}
    \begin{figure}[ht]
        \centering
        \incfig{probability-density-function}
        \caption{Probability Density Function}
        \label{fig:probability-density-function}
    \end{figure}
    First notice that the prboability density function must be non-negative, because it is impossible
    to have a negative probability by definition axiomatically. There are some properties of these functions that we will
    enumerate now:
    \begin{align*}
        \text{(i) } & \int_{-\infty}^\infty f(x)dx = P(-\infty < X < \infty ) = 1 \\
        \text{(ii) } & P(X = a) = \int_a^a f(x)dx = 0 \forall a \in \mathbb{R} \\
        \text{(iii) } & P(a < X \leq b) = P(a < X < b) = P(a \leq X < b) = P(a \leq X \leq b) = \int_a^bf(x)dx
    \end{align*}
    We can restate this definition by saying, $f(x)$ is a probability density function
    $\Leftrightarrow f(x) \geq 0$ and $\int_{-\infty}^\infty f(x)dx = 1$.
    Even though $P(X = a) = 0$ for every real number $a$, since the real numbers are uncountable,
    we do not violate any of our axioms of probability. Since $P(S) = 1$ for any sample space $S$,
    it follows that $P(-\inf \leq X \leq \inf) = 1$.\\\\
    \begin{mdframed}
        \begin{definition}
            Let $X$ be a continuous random variable with density function $f(x)$. Then its expectation is
            \[
                E[X] = \int_{-\infty}^\infty x f(x) dx
            \]
        \end{definition}
    \end{mdframed}
    We carry some similar properties over from discrete expectation. Firstly,
    \[
        E[g(X)] = \int_{-\infty}^\infty g(x)f(x) dx    
    \]
    The \fbox{Markov Inequality} also will hold for continuous random variables. That is,
    \[
        P(X \geq \alpha) \leq \frac{E[X]}{\alpha}    
    \]
    \begin{proof}
        Let $\alpha > 0$. For every $\alpha \leq x < \infty, 1 \leq \frac{x}{\alpha}$. Then we say 
        \[
            P(X \geq \alpha) = \int_\alpha^\infty f(x) dx \leq \int_\alpha^\infty \frac{x}{\alpha}f(x)dx
        \]
        The right hand side is bounded by $\frac{1}{\alpha} \int_0^\infty x f(x) dx = \frac{1}{\alpha} E[X]$.
    \end{proof}
    Finally the \fbox{Chebyshev Inequality} also will hold:
    \[
        P(|X - \mu| \geq \kappa) \leq \frac{Var(x)}{\kappa}
    \]
    The proof of the Chebyshev Inequality does not change from the proof in the discrete case.
    \subsection{Exponential Random Variable}
        Let us take the example of the following function:
        \[
            f(x) = \begin{cases}
                e^{-x} & x \geq 0\\
                0 & x < 0
            \end{cases}
        \]
        We know that this function will integrate to $1$ over $\mathbb{R}$.
        Scaling, this function by $\lambda$ we get another probability density function.
        \[ f(x) = \begin{cases}\lambda e^{-\lambda x} & x \geq 0\\ 0 & x < 0 \end{cases}\]
        We obtain the same exact area, so we still have a valid probability density function so
        long as $\lambda > 0$.
        This is called an exponential random variable.
        It is a continuous analogue to the geometric random variable in the discrete case. Then
        it also carries the property of memorylessness, which means that $P(X > a + b | X > a) = P(X > b)$,
        for any $a,b \geq 0$. Generally this is because after shifting our start point to $a$, and normalizing the distribution,
        we simply get the same function again. However, we must prove this more rigorously.
        \begin{proof}
            For any $a > 0$, we can first compute the probability that $X > a$.
            \[
                P(X > a) = \int_a^{\inf} \lambda e^{- \lambda x} dx = (-e^{-\lambda x})_a^{\inf} = e^{-\lambda a}
            \]
            Then the conditional probability can be computed as follows:
            \[
                P(X > a + b | X > a) = \frac{P(X > a + b)}{P(X > a)} = \frac{e^{-\lambda (a + b)}}{e^{-\lambda a}} = e^{-\lambda b}
            \]
        \end{proof}
        We proved in class that memorylessness is unique to the exponential random variable.
        To find the expectation of such an exponential variable let $\lambda > 0$. We say that 
        \[
            E[X] = \int_{-\infty}^\infty x \cdot f(x) dx = \int_0^\infty \lambda x \cdot e^{-\lambda x}dx
        \]
        Then we must use integration by parts, which gives us $-xe^{-\lambda x}\bigg|_0^\infty - \int_0^\infty \lambda e^{-\lambda x} = \frac{1}{\lambda}$.
    \subsection{Uniform Random Variable}
        Consider a real interval $[\alpha, \beta] : \alpha < \beta$. Let $X$ be a continuous random variable with density funciton 
        \[
            f(x) =
            \begin{cases}
                \frac{1}{\beta - \alpha} & \alpha \leq x \leq \beta\\
                0 & \text{otherwise}
            \end{cases}
        \]
        With this funciton $X$ is a uniform random variable over the interval $[\alpha, beta]$.
        \[
            \int_{-\infty}^\infty f(x) dx = \int_{-\infty} ^ a 0 \cdot dx + \int_\alpha^\beta \frac{dx}{\beta - \alpha} + \int_\beta^\infty 0 \cdot dx = \left[\frac{x}{\beta - \alpha}\right]_\alpha^\beta = 1
        \]
    \subsection{Normal (Gaussian) Random Variable}
    $X$ is a normal random variable with parameters $\mu$ and $\sigma^2$ if its density function its
    \[
        f(x) = \frac{1}{\sqrt{2\pi \sigma^2}} e^{-\frac{(x-\mu)^2}{2\sigma^2}}, \ \ \ \ \ -\infty < x < \infty
    \]
    Let $\mathcal{N}(\mu, \sigma^2)$ denote a normal distribution with parameters $\mu$ and $\sigma^2$.
    \begin{figure}[ht]
        \centering
        \incfig{normal}
        \caption{Probability Density Function of Normal Distribution}
        \label{fig:normal}
    \end{figure}
    The expectation of the Normal Random Variable is $\sigma$, this can be shown by computing the integral $\int_{-\infty}^\infty x \cdot f(x) dx$.
\end{document}
