\documentclass{article}

\usepackage{times}
\usepackage{amssymb, amsmath, amsthm}
\usepackage[margin=.5in]{geometry}
\usepackage{graphicx}
\usepackage{tikz}

\begin{document}

\title{Probability 1 - Homework 5}
\author{Philip Warton}
\date{\today}
\maketitle
\section*{Problem 1}
The density function of $X$ is given by 
\[
    f(x) =
    \begin{cases}
        c / x^4, & x \geqslant 1\\
        0, & \text{otherwise}
    \end{cases}  
\]
\subsection*{(i)}
    Show that $c = 3$.
    \begin{proof}
        Since $f$ is a probability denisty function, it must be the case that $\int_\mathbb{R} f = 1$.
        Using this fact, we can show that $c$ must be equal to 3. We begin by integrating the function on
        the real line.
        \begin{align*}
            \int_\mathbb{R} f(x) dx & = \int_{-\infty}^1 f(x) dx + \int_1^\infty f(x) dx \\
            &=  \int_{-\infty}^1 0 dx + \int_1^\infty c x^{-4} dx \\
            &= \left[ -\frac{1}{3}  c x^{-3}  \right]_1^\infty \\
            &= \lim\left[ -\frac{1}{3} c n^{-3}\right] + \frac{1}{3} c 1^{-3} \\
            &= 0 + \frac{1}{3} c
        \end{align*}
        Since we know that the total probability must be equal to 1, it follows that $c = 3$.
    \end{proof}
\subsection*{(ii)}
    Compute $E[X]$ and $Var(X)$.\\\\
        Firstly we can compute $E[X]$. We know that this is equal to $\int_\mathbb{R} x f(x) dx = \int_1^\infty x 3 x^{-4} dx = \int_1^\infty 3 x^{-3} dx$.
        This integral evaluates to $\left[ -\frac{3}{2} x^{-2} \right]_1^\infty = \lim\left[-\frac{3}{2} n^{-2}\right] - \left[-\frac{3}{2} 1^{-2}\right] = \frac{3}{2}$.
        Then to compute the variance, we first wish to compute $E[X^2]$. We simply write
        \[
            E[X^2] = \int_\mathbb{R} x^2 f(x) dx = \int_1^\infty x^2 3 x^{-4} dx = \int_1^\infty 3x^{-2} dx = \left[-3x^{-1}\right]_1^\infty = 0 + 3
        \]
        Then having computed both $E[X]$ and $E[X^2]$ we know that we can compute the variance using $Var(X) = E[X^2] - E[X]^2$. This gives us $Var(X) = 3 - \frac{3^2}{2^2} = \frac{3}{4}$.
\section*{Problem 2}
    Let $Z$ be a standard normal random variable $Z \sim N(0,1)$. Compute $E[e^Z]$.\\\\
        We know that for some function we can simply integrate our density multiplied by our function to compute the expectation. So we say that 
        $E[e^Z] = \int_\mathbb{R} e^x \sqrt{2 \pi}^{-1} e^{-\frac{x^2}{2}} dx = \int_{-\infty}^\infty \sqrt{2 \pi}^{-1} e^{x - \frac{x^2}{2}}dx$.
        We can first complete the square on the term $x - \frac{x^2}{2}$ which gives us $x - \frac{x^2}{2} = (1 / 2)((x-1)^2 + 1)$. Then let $u = x-1$.
        By exchanging variables, we still have positive and negative infinite bounds, and we write $E[e^Z] = \int_\mathbb{R} \sqrt{2 \pi}^{-1} e^{(1 / 2)(u^2 + 1)} du$.
        Now we can factor out our $\sqrt{2 \pi}^{-1}$ and we can also factor out $e^{\frac{1}{2}}$. This gives us $\frac{\sqrt{e}}{\sqrt{2 \pi}} \int_\mathbb{R} e^{u^2 / 2} du$.
        Then we exchange variables again, defining $v = u / \sqrt{2}$, so that $dv = du / \sqrt{2}$. Our bounds remain unchanged as they are only scaled by some constant, 
        and we can say 
        \[
            \frac{\sqrt{2e}}{\sqrt{2\pi}} \int_\mathbb{R} e^{v^2} dv = \frac{\sqrt{2e}}{\sqrt{2\pi}} \sqrt{2\pi} = \sqrt{2e}
        \]
        Finally $E[e^Z] = \sqrt{2e}$.
\section*{Problem 3}
    Show that $Var(X + Y) = Var(X) + Var(Y)$ for two independent random variables $X$ and $Y$.
    \begin{proof}
        We should first note that $E[X]E[Y] = E[XY]$, and also that $Var(Z) = E[Z^2] + E[Z]^2$. Then we can simply use algebraic manipulations, to show that 
        this equality holds.
        \begin{align*}
            Var(X + Y) & = E[(X+Y)^2] - E[X+Y]^2\\
            &= E[X^2 + 2XY + Y^2] - (E[X] + E[Y])^2 \\
            &= E[X^2] + 2E[XY] + E[Y^2] - (E[X]^2 + 2E[X]E[Y] + E[Y^2])\\
            &= E[X^2] - E[X]^2 + E[Y^2] - E[Y]^2 + 2E[XY] - 2E[XY] \\
            &= Var(X) + Var(Y)
        \end{align*}
        And thus we have shown that variance is distributive over addition of independent random variables.
    \end{proof}
\section*{Problem 4}
    Let $X_1, X_2, X_3$ be uniform random variables on $[0,1]$. Then find the density of $Y = X_1 + X_2 + X_3$.\\\\
    Of course we will compute the cumulative distribution function $F(Y)$, and then differentiate it in order to get the density.
    We can imagine values of $Y$ as points that lie within the unit cube in $\mathbb{R}^3$. If $Y = a$ for some $a \in [0,3]$, then 
    we know that $y$ is a point in the intersection of the plane $a = x_1 + x_2 + x_3$ and the unit cube. So for the cumulative distribution
    function, we say that $P(Y < a)$ is the volume of everything underneath this plane $a = x_1 + x_2 + x_3$ (a plane with a unit vector
    as a normal, offset from the origin by $a$) intersected with the unit cube. For $a \in [0,1]$, the shape will be a right corner piece of
    the unit cube which we can compute the volume of as
    \[
        \int_0^a \int_0^{x_3} \int_0^{x_2} 1 dx_1 dx_2 dx_3 = \int_0^a \int_0^{x_3} x_2 dx_2 dx_3 = \int_0^a \frac{x_3^2}{2} dx_3 = \frac{a^3}{6}
    \]
    Then for $a \in [1,2]$, we have an odd shape. We could describe it as this right corner piece truncated at its spikes that go beyond 1. The 
    pieces that become truncated are 3 corners that are identical to these right triangle corner pieces except with length $a - 1$. So the volume of 
    each of these will be $\frac{(a - 1)^3}{6}$. So then for $a \in [1,2]$ we say $P(Y < a) = \frac{a^3}{6} - 3 \frac{a - 1)^3}{6}$. Then finally for
    $a \in [2,3]$ we have the unit cube minus the volume of a right triangle corner piece with side length $3 - a$, i.e. we have the volume $1 - \frac{(3-a)^3}{6}$.
    Then for $a < 0, P(Y < a) = 0$ and for $a > 3, P(Y < a) = 1$. So we have the cumulative distribution function 
    \[
        F_Y(y) = \begin{cases}
            0, & y < 0 \\
            \frac{y^3}{6}, & 0 \leqslant y < 1 \\
            \frac{y^3}{6} - 3 \frac{(y - 1)^3}{6}, & 1 \leqslant y < 2 \\
            1 - \frac{(3 - y)^3}{6}, & 2 \leqslant y < 3 \\
            1, & y \geqslant 3
        \end{cases}
    \]
    Then, by differentiating this function, we get our density function $f_y(y)$.
    \[
        f_y(y) = \begin{cases}
            \frac{y^2}{2}, & 0 \leqslant y < 1 \\
            \frac{y^2}{2} - (y-1)^2, & 1 \leqslant y < 2 \\
            \frac{(3-y)^2}{2}, & 2 \leqslant y < 3 \\
            0, & \text{otherwise}
        \end{cases}
    \]
\section*{Problem 5}
    The sample size $n = 1210$, with $p = \frac{1}{11}$. Then we will compute $P\{97.5 \leqslant \text{ the number of successes } \leqslant 116.5\}$.
    Then we write $\frac{97.5 - (1210 / 11)}{\sqrt{12100 / 121}} = -\frac{5}{4}$. Then to normalize our upper bound we write 
    $\frac{116.5 - (1210 / 11)}{\sqrt{12100 / 121}} = \frac{13}{20} = .65$.
    \[
        P\{97.5 \leqslant \text{ the number of successes } \leqslant 116.5\} = P(-1.25 < Z < .65) = 1 - P(Z > .65) - P(Z > 1.25) \approx 1 - .2578 - .1056 = .6366
    \]
\section*{Problem 6}
    The sample size is $n = 90000$, $p = \frac{1}{2}$. Then we want 
    \[
        P \{ 45032 \leqslant \text{ the number of heads } \leqslant 45069 \}
    \]
    Our lower bound will be $\frac{45032 - 45000}{\sqrt{22500}} = \frac{16}{75} \approx .2133$.
    Then our upper bound will be $\frac{45069 - 45000}{\sqrt{22500}} = \frac{23}{50} \approx .46$.
    Finally we can compute 
    \[
        P\{ 45032 \leqslant \text{ the number of heads } \leqslant 45069 \} \approx P(.2133 < Z < .46) = 1 - P(Z > .2133) - P(Z > .46) \approx 1 - .4168 - .3228 = .2604
    \]
\section*{Problem 7}
    With a sample size $n = 18000$ and a probability $p = \frac{1}{6}$, compute the probability of getting at least 
    3060 successful trials, $P\{3059.5 < \text{ number of successes }\}$. Then we get our lower bound and consult the $Z$-table.
    We say that $\frac{3059.5 - 3000}{\sqrt{2500}} = 1.19$. Then 
    \[
        P(Z > 1.19) \approx .1170
    \]
\end{document}