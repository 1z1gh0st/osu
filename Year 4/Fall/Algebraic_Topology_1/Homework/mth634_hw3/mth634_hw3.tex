\documentclass{article}

% -------------------------------------------- %
% ------------------PACKAGES------------------ %
% -------------------------------------------- %
\usepackage{amssymb, amsmath, amsthm}
\usepackage{bm}
\usepackage{enumerate}
\usepackage[margin=.25in]{geometry}
\usepackage{graphicx}
\usepackage{import}
\usepackage{listings}
\usepackage[linewidth=1pt]{mdframed}
\usepackage{pdfpages}
\usepackage{times}
\usepackage{transparent}
\usepackage{xifthen}
% -------------------------------------------- %

% Figure command
\newcommand{\incfig}[1]{%
    % Adjust number for defualt figure width
    \def\svgwidth{.5\columnwidth} 
    \import{./figures/}{#1.pdf_tex}
}

% Theorem command
\newtheorem{theorem}{Theorem}[section]
\newtheorem{lemma}{Lemma}[section]
\newtheorem*{remark}{Remark}
\theoremstyle{definition}
\newtheorem{definition}{Definition}[section]

\begin{document}

% -------------------------------------------- %
\title{Algebraic Topology --- Homework 3}
\author{Philip Warton}
\date{\today}
\maketitle
% -------------------------------------------- %
\section*{Problem 1}
Prove that if $A$ is a compact subspace of a Hausdorff space $X$, 
then $A$ is closed in $X$: $A \subset^{cl} X$.
\\ \hline
\subsection*{Solution}
    \begin{proof}
    Let $x \in X \setminus A$.          
    Then for any point $a$ in $A$, we have some neighborhood of $a$, $U_a$,
    that is disjoint from some neighborhood of $x$, $V_a$, since $X$ is
    Hausdorff.
    Then it follows that 
    \[
    \bigcup_{a\in A}U_a
    \] 
    is an open cover cover of $A$. Since $A$ is compact, it follows that there is some finite subcover of $A$. We will denote this by
    \[
    \bigcup_{f\in F}U_f.
    \] 
    So then since $F$ is finite, it follows that the intersection
    \[
    \bigcap_{f\in F}V_f
    \] 
    is an open neighborhood of $x$ that is disjoint from our finite cover
    of $A$, and is therefore disjoint from $A$.
    So if any arbitrary point $x \notin  A$ has a neighborhood that is 
    disjoint from $A$, then $A$ is closed.
    \end{proof}
\pagebreak
\section*{Problem 2}
Prove that if $\phi: B^n \ra X$ is a characteristic map for an $n$-cell $c^n $ in a Hausdorff space $X$, then $\bar{c}^n = \phi(B^n)$ is the closure of the open cell $c^n = \phi(\mathrm{int}(B^n))$ in $X$.
\\ \hline
\subsection*{Solution}
    \begin{proof}
    By \fbox{Problem 1}, we know that $\phi(B^n)$ is closed. This is
    because a continuous map carries compactness, and a compact subset of 
    a Hausdorff space is closed. By basic properties of sets and maps, we 
    know too that $\phi(int(B^n))\subset \phi(B^n)$.
    Suppose that $\phi(B^n)$ is not the closure of $\phi(int(B^n))$, then 
    it must be the case that $cl(\phi(int(B^n)))\subsetneq \phi(B^n)$.
    Then there must be some point $x \in \phi(B^n) \setminus cl(\phi(int(B^n)))$.
    Since $x$ is not a limit point of $\phi(int(B^n))$, we know that 
    $\phi^{-1}(x)$ is not a limit point of $int(B^n)$. This is a 
    contradiction because any point in $B^n$ is a limit point of $int(B^n)$.
    So we conclude that $\phi(B^n) = cl(c^n)$.
    \end{proof}
    What is left to show is that continuity preserves limit/accumulation points\ldots
    \\\\
Let $f:X \to Y$ be continuous.
Let $B \subset Y$ and let $y \in Y$ not be a limit point of $B$.
Then $f^{-1}(y) = x$ is not a limit point of $f^{-1}(B)$.
\begin{proof}
   Since $y$ is not a limit point of $B$, there is some open neighborhood $U$ of $y$ such that $U$ is disjoint from $B$.
   Therefore the pre-image $f^{-1}(U)$ is disjoint from the pre-image of $B$.
   So there is an open set ($f_{-1}(U)$) such that it is an open neighborhood of $y$ and is disjoint from $f^{-1}(B)$.
   Therefore the pre-image of $y$ is not a limit point of the pre-image of $B$.
\end{proof}
\pagebreak
\section*{Problem 3}
Let $\mathcal{F}$ be a family of closed subsets of a topological space $X$.
\begin{enumerate}[i.]
\item Prove that $\{A \subseteq X\ :\ A \cap F \subset^{cl} F\ \mbox{for all}\ F \in \mathcal{F}\}$ is the set of closed sets for a topology on $X$. This is called the \textbf{weak topology} on $X$ relative to $\mathcal{F}$. We write $X_w$ to refer to the set $X$ together with this weak topology.
\item Prove that if $U \subseteq X$, then $U \op X_w$ if and only if $U \cap F \subset^{op} F$ for all $F \in \mathcal{F}$.
\item Prove that for each $F \in \mathcal{F}$, the subspace topology inherited from $X_w$ is the same as the subspace topology inherited from $X$.
\item Prove that if $Y$ is any space, then a function $f:X_w \rightarrow Y$ is continuous if and only if $f|_F: F \rightarrow Y$ is continuous for all $F \in \mathcal{F}$.
\item Prove that the identity function $1_X: X_w \rightarrow X$ is continuous.
\item Give an example to show that $X_w$ need not be homeomorphic to $X$.
\end{enumerate}
\\ \hline
    \subsection*{Solution}
    We denote $\{A \subset X \ : \ A \cap F \subset^{cl} F \text{for all} F \in \mathcal{F}\} = \mathcal{A}$.
    \subsubsection*{i.} 
    \begin{proof}
    We wish to show that if we let $\mathcal{A}$ be our family of closed sets, that this will form a topology on $X$. To do this, one must show that the axioms of a topological space hold.\\\\
    \fbox{$\emptyset, X \in \mathcal{A}$} We begin with the empty set.
    For any $F \in \mathcal{F}$
    \begin{align*}
        \emptyset \cap F = \emptyset \subset^{cl} F.
    \end{align*}
    For the universal set,
    \[
    X \cap F = F \subset^{cl} F.
    \] 
    \fbox{Closure under arbitrary intersection}
    Let 
    \[
    \bigcap_{i \in I}A_i
    \] 
    be some arbitrary intersection of sets in $\mathcal{A}$.
    Then let $F \in \mathcal{F}$ be arbitrary. It follows that 
    \[
    \left(\bigcap_{i \in I}A_i\right) \cap F = \bigcap_{i\in I}\left(A_i \cap F\right)
    \] 
    Since each $A_i \cap F$ is closed in the subspace of $F$, it follows that this intersection will also be closed.\\\\
    \fbox{Closure under finite union}
    Let
    \[
    \bigcup_{j \in J}A_j 
    \] 
    Be a finite union of sets in $\mathcal{A}$. Then let $F \in \mathcal{F}$ be arbitrary, and we write 
    \[
    \left(\bigcup_{j \in J}A_j\right) \cap F =
    \bigcup_{j\in J}\left(A_j \cap F\right)
    \] 
    Then each $A_j \cap F$ is closed in $F$ so it follows that a finite union of sets closed in $F$ is closed in $F$, and therefore this union belongs in $\mathcal{A}$. Having shown these three properties, this set of closed sets forms a topology on $X$.
    \end{proof}
    \subsubsection*{ii.}
    \begin{proof}
    Suppose that $U \subset X$. Then we know that 
    \begin{align*}
    U \subset^{op} X_w &\Leftrightarrow (X \setminus U) \subset^{cl} X_w\\
    &\Leftrightarrow (X \setminus U) \in \mathcal{A}\\
&\Leftrightarrow (X \setminus U) \cap F \subset^{cl} F \ \ \ \ \forall F \in \mathcal{F}\\
&\Leftrightarrow U \cap F \subset^{op} F \ \ \ \ \forall F\in\mathcal{F}\\
    \end{align*}
    \end{proof}
    \subsubsection*{iii.}
    \begin{proof}
    Let $F \in \mathcal{F}$ be arbitrary, without loss of generality.
    To show that the topologies are the same we will demonstrate that a set is closed in one topology if and only if it is closed in the other. Let $C \subset F$ be closed under the subspace topology induced by $X_w$.
    The set $C$ is closed in $F_w$ (slight abuse of notation, $F_w$ is $F$ equipped with the subspace topology induced by $X_w$) if and only if there exists some $C'$ closed in $X_w$ such that $C' \cap F = C$. Then
    \begin{align*}
        C' \subset^{cl} X_w &\Leftrightarrow C' \cap F \subset^{cl} F \ \ \ \ \forall F \in \mathcal{F} \\
    &\Leftrightarrow C' \cap F \subset^{cl} F \text{ for our fixed $F$} \\
    &\Leftrightarrow C \subset^{cl} F
    \end{align*}
    So we conclude that $C \subset^{cl} F_w$ if and only if $C \subset^{cl} F$.
    \end{proof}
    \subsubsection*{iv.}
    \begin{proof}
    Let $Y$ be any space. Then a function $f:X_w \rightarrow Y$ is 
    continuous if and only if closed sets in $Y$ have closed pre-images.
    Closed sets $C \subset Y$ have closed pre-images in $X_w$ if and only if $f^{-1}(C) \cap F \subset^{cl} F$ for every $F \in \mathcal{F}$.
    Then $f^{-1}(C) \cap F \subset^{cl} F$ for every $F \in \mathcal{F}$ if and only if $f^{-1}\big|_F(C) \subset^{cl} F$ for every $F \in \mathcal{F}$. And this is the case if and only if $f\big|_F$ is continuous for every $F \in \mathcal{F}$.
    \end{proof}
    \subsubsection*{v.}
    \begin{proof}
    Let $C \subset^{cl} X$. Then we wish to show that its pre-image under 
    the identity function is also closed. Let $F \in \mathcal{F}$ be 
    arbitrary without loss of generality. Then by definition of subspace 
    topology $C \cap F \subset^{cl} F$.
    Then it follows that if $C \cap F \subset^{cl} F$ for every $F$,
    then $C \subset^{cl} X_w$.
    \end{proof}
    \subsubsection*{vi.}
    Take $X = \mathbb{R}$. The let $\mathcal{F} = \{\{0\}\}$.
    Then what we are left with is the discrete, weakest, topology.
    That is, every set is both open and closed. If a set does not contain 0
    , then its intersection with every $F \in \mathcal{F}$ is empty and 
    is trivially closed. And if a set contains 0 then its intersection with
    every $F \in \mathcal{F}$ is $\{0\}$ and therefore closed in $\{0\}$.
    So $\mathbb{R}_w$ and $\mathbb{R}$ are not homeomorphic because 
    $\mathbb{R}$ is connected and $\mathbb{R}_w$ is disconnected. A simple 
    disconnection for $\mathbb{R}_w$ is $\mathbb{R}_w = \{0\} \cup 
    (\mathbb{R} \setminus \{0\})$.
    Thus, we conclude that $X$ and $X_w$ need not be homeomorphic.
\pagebreak
\section*{Problem 4}
Prove that if $(X,\mathcal{C})$ is a CW complex, then the zero-skeleton $X^0$ is discrete.
\\ \hline
    \subsection*{Solution}
    % NOTES
    % Option 1 -- we violate colusre-finiteness (unlikely)
    % Option 2 -- we violate coherence (likely)
    % Option 3 -- we violate definition of CW-complex (also pretty likely)

    % As it turns out, option 3 was our friend. All cells are open + all singleton sets are closed
    % -> boom we are done
    \begin{proof}
    Suppose that $(X,\mathcal{C})$ is a CW complex.
    The 0-skeleton $X^0$ is defined as
    \[
    (X^0, \mathcal{C}^0) \text{ where } \mathcal{C}^0 = \{c \in \mathcal{C} \ : \ dim(c) \leq 0\}, X^0 = \bigcup_{dim(c) \leq 0}c
    \] 
    Since our dimension is 0, this means that all our open cells $c \in \mathcal{C}$
    are of dimension 0 or lower. Since all cells of dimension $n < 0$ 
    are non-existent (being images of the empty set). We conclude that
    all open cells $c \in \mathcal{C}$ are of dimension 0.
    \[
    \forall c \in \mathcal{C}^0, dim(c) = 0
    \] 
    Let $c \in \mathcal{C}^0$ be arbitrary.
    Then by definition of a CW complex, $\{c\}$ is open.
    By definition of a 0-cell, $c$ is simply a single point in $X$.
    Then, since $X$ is Hausdorff, it follows $\{c\}$ is closed.
    Since the singleton set of any point in the 0-skeleton is both open and closed,
    it follows that $X^0$ is discrete.
    \end{proof}
\pagebreak
\section*{Problem 5}
Prove that $H_k(B^n,S^{n-1}) \cong \left\{\begin{array}{rl} \mathbb{Z} &k=n\\0&k\neq n\end{array}\right.$.
\\ \hline
    \subsection*{Solution}
    % NOTES

    % Long exact sequence is our friend
    % Use the point + n-cell decomp on S^n
    % Show that the k-th homology on S^n is the same as on (B^n S^n-1)
    % Proof done?
    \begin{proof}
    We know that for $n > 1$, for the homology of the sphere,
    \[
        H_k(S^n) = \begin{cases}
            \mathbb{Z} & k = n \\
            0 & k \neq n
        \end{cases} 
    \] 
    Let $n,k > 1$, then we have an exact sequence 
    \[
    0 = H_k(B^n) \rightarrow^f H_k(B^n, S^{n-1}) \rightarrow^g H_{k-1}(S^{n-1}) \rightarrow^h H_{k-1}(B^n) = 0
    \] 
    Then 
    \[
    H_{k-1}(S^{n-1}) = ker(h) = im(g) \Rightarrow g \text{ is surjective}
    ,\]
    and
    \[
    ker(g) = im(f) = 0 \Rightarrow g \text{ is injective}
    .\] 
    So the middle mapping is a bijection therefore 
    \[
    H_k(B^n, S^{n-1}) \cong H_{k-1}(S^{n-1}) = \begin{cases}
        \mathbb{Z} & k-1 = n-1 \\
        0 & k-1 \neq n-1
    \end{cases}
    .\] 
    Critically $k-1=n-1$ if and only if $k=n$.
    For $k=1$,
    \[
    0 = H_1(B^n) \to^f H_1(B^n, S^{n-1}) \to^g H_0(S^{n-1}) \to^h H_0(B^n) = \mathbb{Z}
    \] 
    If $n = 1$ then $H_0(S^{0}) = \mathbb{Z} \oplus \mathbb{Z}$ so it follows that 
    \[
    \mathbb{Z} = ker(h) = im(g)
    .\] 
    Then since 
    \[
    ker(g) = imk(g) = 0
    ,\] 
    it follows that $g$ is injective, and since it's image is $\mathbb{Z}$, $f^{-1}(\mathbb{Z}) = \mathbb{Z}$.
    Therefore,
    \[
    H_1(B^1, S^0) = \mathbb{Z}
    .\] 
    If $n > 1$ then $H_0(S^n) = H_0(B^n) = \mathbb{Z}$ so $h$ is an isomorphism.
    So then $0 = ker(h) = im(g)$, and therefore $g$ is the 0 map.
    Since $im(g)$ is 0, it must be the case that $ker(f) = 0$, then $H_1(B^n, S^{n-1})$ must be 0 for $n > 1$.
    If $n=0$, then
    \[
    H_1(B^0, S^{-1}) = H_1(B^0, \emptyset) = H_1(B^0) = H_1(\{pt\}) = 0
    .\] 
    For $k=0, n = 0$, we have 
    \[
    H_0(B^0, S^{-1}) = H_0(B^0, \emptyset) = H_0(B^0) = H_0(\{pt\}) = \mathbb{Z}
    .\] 
    \end{proof}
    This is incomplete. Edge cases are handled messily and incorrectly.
    Note: shouldn't $H_k(B^1, S^0)$ be isomorphic to $H_k(S^1)$, since we are quotienting 
    the endpoints together in hand-wavey way?
\pagebreak
\section*{Problem 6}
Prove that if 
$$
C: C_m \stackrel{\partial_m}{\rightarrow} C_{m-1} \rightarrow \cdots \rightarrow C_1 \stackrel{\partial_1}{\rightarrow} C_0
$$ 
is a chain complex consisting of finitely generated abelian groups $C_0, \ldots, C_m$, then
$$
\sum_{i=0}^m (-1)^i\mathrm{rk}(C_i) = \sum_{i=0}^m (-1)^i\mathrm{rk}(H_i(C)).
$$
Use this to prove that if $(X, \mathcal{C})$ is a finite CW complex and $\alpha_i$ is the number of $i$-cells in $\mathcal{C}$, then $$\chi(X,\mathcal{C}) = \sum_{i = 0}^{\mathrm{dim}(X)} (-1)^i \alpha_i$$ is an invariant of the homotopy type of the space $X$. In particular, this number is independent of the choice of (finite) cellular decomposition $\mathcal{C}$, so we denote this number by $\chi(X)$. It is the \textbf{Euler characteristic} of $X$.
\\ \hline
    \subsection*{Solution}
    % NOTES

    % rk is rank
    % For a SES of finitely generated ab groups 0 -> A -> B -> C -> 0
    % We want to show that the alternating sum on the chain complex is equal to the alternating sum on the homology of the chain complex
    % What does the above statement really mean?


    % Induction?
    \begin{proof}
    First, notice that there are two important chain complexes
    \[
    0\to ker\partial_n \to C_n \to im\partial_n \to 0
    \] 
    and,
    \[
    0\to im\partial_{n+1}\to ker\partial_n \to H_n(C)\to 0
    .\] 
    By the rank nullity theorem
    \begin{align*}
        rk(C_n) &= rk(ker\partial_n) + rk(im\partial_n) \\
        rk(H_n(C)) &= rk(ker\partial_n) - rk(im\partial_{n+1}) \\
    \end{align*}
    Then,
    \begin{align*}
        \sum_{i=0}^{m} (-1)^i rk(C_n) &= \sum_{i=0}^{m} (-1)^i \left(rk(ker\partial_n) + rk(im\partial_n)\right) \\
        &= \sum_{i=0}^{m} (-1)^i rk(ker\partial_n) + \sum_{i=0}^{m} (-1)^i rk(im\partial_n) \\
        &= \sum_{i=0}^{m} (-1)^i rk(ker\partial_n) - \sum_{i=0}^{m} (-1)^{i-1} rk(im\partial_n) \\
        &= \sum_{i=0}^{m} (-1)^i rk(ker\partial_n) - \sum_{i=-1}^{m} (-1)^{i} rk(im\partial_{n+1}) \\
        &= \sum_{i=0}^{m} (-1)^i rk(ker\partial_n) - \sum_{i=0}^{m} (-1)^{i} rk(im\partial_{n+1}) \\
        &= \sum_{i=0}^{m} (-1)^i rk(H_n(C)) \\
    \end{align*}
    \end{proof}
    \begin{proof}
        Having proven the first property, proving that the Euler characteristic is invariant to 
        the choice of finite cell decomposition $\mathcal{C}$ becomes relatively simple.
        First, notice that the cell decomposition $\mathcal{C}$ forms a chain complex,
        \[
            0\to C_d \to^\partial C_{d-1} \to^\partial  C_{d-2} \to^\partial \cdots \to^\partial C_0 \to 0
        \] 
        Then, since we have a finite number of cells, it follows that these abelian groups are
        finitely generated, and that their rank is equal to the number of cells of that dimension.
        So we write,
        \begin{align*}
            \chi(X,\mathcal{C}) = \sum_{i=0}^{dim(X)} (-1)^i \alpha_i = \sum_{i=0}^{dim(X)} (-1)^i rk(C_i) = \sum_{i=0}^{dim(X)} (-1)^i rk(H_n(C)) = \chi(X)
        \end{align*}
    \end{proof}
\pagebreak
\section*{Problem 7}
Calculate $\chi(X)$ where
\begin{enumerate}
\item $X$ is a finite contractible CW complex
\item $X = S^n$
\end{enumerate}
\\ \hline
    \subsection*{Solution}
    \subsubsection*{1.}
    Suppose that $X$ is a finite contractible CW complex.
    That $X$ is homotopy equivalent to a singular point.
    So a cellular decomposition of a singular point is trivially,
    \[
    \mathcal{C} = \mathcal{C}^0 = \{c\}
    \] 
    where $c$ is a 0-cell.
    Then, we can simply compute,
    \[
    \chi(\{pt\},\mathcal{C}) = \sum_{i=0}^{0} (-1)^i \alpha_i = (-1)^0 \cdot 1 = 1
    \] 
    \subsubsection*{2.}
    For the $n$-sphere, it is slightly less trivial.
    We have a cellular decomposition given by one $n$-cell and one 0-cell,
    the point-$n$-cell pair that is quite useful.
    To compute $\alpha_n$, it should be clear that 
    \[
    \alpha_n = \begin{cases}
        1 & n\in\{0,n\}\\
        0 & \text{otherwise}
    \end{cases}
    \] 
    Finally,
    \begin{align*}
        \chi(X,\mathcal{C}) &= \sum_{i=0}^n (-1)^n \alpha_i \\
                            &= (-1)^0 \alpha_0 + \cdots + (-1)^n \alpha_n \\
                            &= 1 + 0 + \cdots + 0 \pm 1 \\
                            &= \begin{cases}
            2 & n \text{ is even }\\
            0 & n \text{ is odd}
        \end{cases} 
    \end{align*}
\end{document}
