\documentclass{article}

\usepackage{times}
\usepackage{amssymb, amsmath, amsthm}
\usepackage[margin=.5in]{geometry}
\usepackage{graphicx}
\usepackage[linewidth=1pt]{mdframed}

\usepackage{import}
\usepackage{xifthen}
\usepackage{pdfpages}
\usepackage{transparent}
\usepackage{bm}

\usepackage{tikz-cd}

\newcommand{\incfig}[1]{%
    \def\svgwidth{\columnwidth}
    \import{./figures/}{#1.pdf_tex}
}

\newtheorem{theorem}{Theorem}[section]
\newtheorem{lemma}{Lemma}[section]
\newtheorem*{remark}{Remark}
\theoremstyle{definition}
\newtheorem{definition}{Definition}[section]

\begin{document}

\title{Algebraic Topology - Homework 1}
\author{Philip Warton}
\date{\today}
\maketitle
\section{Problem 0.3}
    \begin{mdframed}[]
        Assume, for $n \geq 1$, that $H_i(S^n) = \mathbb{Z}$ if $i = 0, n,$ and that $H_i(S^n) = 0$ otherwise.
        Using the technique of the proof of Lemma 0.2, prove that the equator of the $n$-sphere is not a retract.
    \end{mdframed}
    \begin{proof}
        Assume that $S^{n-1}$ is the equator of the $n$-sphere. Suppose by contradiction that $S^{n-1}$ is a 
        retract of $S^n$. Then it follows that there exists some retraction $r:S^n \rightarrow S^{n-1}$.
        Then, with $i:S^{n-1} \rightarrow S^n$ being the inclusion map, and with $1$ being, of course, the 
        identity map, it follows that we would have a commutative diagram:
        \begin{center}
            \begin{tikzcd}[]
                S^{n-1} \ar[rr, "1"] \ar[rdd, "i"] && S^{n-1} \\\\
                & S^n \ar[ruu, "r"]
            \end{tikzcd}
        \end{center}
        To this diagram, we can apply our homology functor, giving us
        \begin{center}
            \begin{tikzcd}[]
                H_{n-1}(S^{n-1}) \ar[rr, "H_{n-1}(1)"] \ar[rddd, "H_{n-1}(i)"] && H_{n-1}(S^{n-1}) \\\\\\
                & H_{n-1}(S^n) \ar[ruuu, "H_{n-1}(r)"]
            \end{tikzcd}
        \end{center}
        We know by assumption that $H_{n-1}(S^n) = 0$ since $n-1 \neq n$ and that $H_n(S^{n-1}) = \mathbb{Z}$ since
        $n - 1 = n - 1$. This new diagram should continue to commute by the properties of our functor $H_{n-1}$.
        Since $H_{n-1}(S^n) = 0$, it follows that its image under $H_{n-1}(r)$ must also be zero. That is, $H_{n-1}(S^{n-1}) =0$.
        However, since this diagram commutes, it follows that the identity map takes a countable object to a trivial one (contradiction).
        Therefore, $S^{n-1}$ is not a retract of $S^n$.
    \end{proof}
\section{Problem 0.5}
    \begin{mdframed}[]
        Let $f,g : I \rightarrow I \times I$ such that $f(0) = (a, 0), f(1) = (b,1), g(0) = (0,c), g(1) = (1,d)$.
        Show that there exists some point $(s,t)$ such that $f(s) = g(t)$.
    \end{mdframed}
    \begin{proof}
        Say that its the case that $f(s) \neq g(t)$ for all $(s,t) \in I^2$. Define 
        \[
            N(s,t) = \max\{ |g_1(t) - f_1(s)| , |g_2(t) - f_2(s)|\}
        \]
        Then we define a function $F: I^2 \rightarrow I^2$ to be 
        \[
            F(s,t) = \left(
                \frac{g_1(t) - f_1(s)}{2 N(s,t)} + 1, \frac{g_2(t) - f_2(s)}{2 N(s,t)} + 1
            \right)
        \]
        Then we have that $F(I^2) \subset \partial I^2$.
        Suppose we have a fixed point $(x,y)$. Then we say that $F(s,t) = (s,t)$ therefore $(s,t) \in \partial I^2$.
        This means that $s = 0, s=1, t = 0$ or $t = 1$. Within any one of these cases, the fixed point does not hold so we 
        no fixed point for $F$ (justify). Then since $I^2$ is homeomorphic to $D^2$, we say that a contradiction is reached 
        by Brouwer fixed point.
    \end{proof}
\section{Problem 0.7}
    \begin{mdframed}[]
        Let $f \in \text{Hom}(A,B)$, and let $g,h \in \text{Hom}(B,A)$ such that $g \circ f = 1_A$ and that 
    $f \circ h = 1_B$. Then $g = h$.
    \end{mdframed}
    \begin{proof}
        We write the following:
        \begin{align*}
            f & = f \\
            f \circ 1_A &= 1_B \circ f & \text{(identity in $hom(A,A)$ and $hom(B,B)$)}\\
            f \circ (g \circ f) &= (f \circ h) \circ f & \text{(by assumption)}\\
            (f \circ g \circ f) &= (f \circ h \circ f) & \text{(associativity)}\\
            h \circ (f \circ g \circ f) & = h \circ (f \circ h \circ f) & \text{(properties of equivalence relation)}\\
            (h \circ f \circ g \circ f) & = (h \circ f \circ h \circ f) & \text{(associativity)}\\
            (h \circ f \circ g \circ f) \circ g & = (h \circ f \circ h \circ f) \circ g & \text{(properties of equivalence relation)}\\
            (h \circ f) \circ g \circ (f \circ g) &=(h \circ f) \circ h \circ (f \circ g) & \text{(associativity)}\\
            1_B \circ g \circ 1_A &= 1_B \circ h \circ 1_A & \text{(by assumption)}\\
            g &= h & \text{(identity in $hom(A,A)$ and $hom(B,B)$)}
        \end{align*}
    \end{proof}
\section{Problem 0.18}
    \begin{mdframed}[]
        For an abelian group $G$, let 
        \[
            tG = \{x \in G \ : \ \text{x has finite order}\}
        \]
        denote its torsion subgroup.
    \end{mdframed}
    \subsection*{(ii)}
        \begin{mdframed}[]
            Assume that $t$ defines a functor and that $t(f) = f\bigg|_{tG}$ for every homomorphism $f$.
            If $f$ is injective, then $t(f)$ is injective.
        \end{mdframed}
        \begin{proof}
            Let $G,H$ be abelian groups, and let $f:G \rightarrow H$ be an injection.
            Then, for any $a,b \in G$ we know that $f(a) = f(b) \Longrightarrow a = b$.
            Choose some $x,y \in tG$ such that $tf(x) = tf(y)$, or equivalently, $f\bigg|_{tG}(x) = f\bigg|_{tG}(y)$.
            Since both $x$ and $y$ belong to the torsion group, it follows that $f(x) = f(y)$.
            Then, by the injectivity of $f$, we know that $x = y$ in $G$, and since both are in $tG$, the same is true there.
        \end{proof}
    \subsection*{(iii)}
        \begin{mdframed}[]
            Give an example of a surjective homomorphism $f$ for which $t(f)$ is not surjective.
        \end{mdframed}
        Define a function $f:\mathbb{Z} \rightarrow \mathbb{Z}_2$ such that 
        \[
            f(x) = \begin{cases}
                [0] & x \text{ is even }\\
                [1] & x \text{ is odd }
            \end{cases}  
        \]
        Since $\mathbb{Z}$ contains both odd and even integers, $f$ is clearly surjective.
        It can also be demonstrated that $f$ is a group homomorphism.
        But, all non-zero integers have infinite order under addition, so we say that $t\mathbb{Z} = \{0\}$.
        Since $\mathbb{Z}_2$ is cyclic, we know that $t\mathbb{Z}_2 = \mathbb{Z}_2$.
        Take the element $[1] \in \mathbb{Z}_2$.
        Notice that it's pre-image under $f$ is confined to only odd integers, so since 0 is not odd,
        it can't have a pre-image under $t(f)$.
        Since there is an element of $t\mathbb{Z}_2$ with no pre-image under $t(f)$, the function is not surjective.
\section{Problem 0.19}
    Let $p$ be a fixed prime in $\mathbb{Z}$.
    Define a functor $f: \bm{Ab} \rightarrow \bm{Ab}$ by $F(G) = G / pG$ and 
    $F(f):x + pG \mapsto f(x) + pH$ (where $f: G \rightarrow H$ is a homomorphism).
    \subsection*{(i)}
        \begin{mdframed}[]
            Show that if $f$ is a surjection, then $F(f)$ is a surjection.
        \end{mdframed}
        \begin{proof}
            Let $y + pH \in H / pH$. Then since $y + pH \in H / pH$ it follows that $y \in H$.
            Because we know that $f$ is a surjection, it follows that if $y \in H$ there exists some $x \in G$ 
            such that $f(x) = y$. So then take the point $x + pG \in G / pG$, and of course it must be the case that 
            \[
                F(f)(x + pG) = f(x) + pH = y + pH
            \]
            And it has been proven that $F(f)$ is surjective.
        \end{proof}
    \subsection*{(ii)}
        \begin{mdframed}[]
            Give an example of a group homomorphism $f$ that is an injection such that $F(f)$ is not an injection.
        \end{mdframed}
        Let our prime number $p$ be equal to 2. Then let $G = (\{0,2\}, + \mod{4}), H = \mathbb{Z}_4$. Finally, let $f$ simply be the inclusion map. It follows that 
        $f$ is injective since it is the inclusion map. Then $F(f)(4) = F(f)(2) = 2 + 2H$, and we say that $F(f)$ is not injective. This does not hold, 
        try $p = 3, \{0,2,4,6\}, \mathbb{Z}_8$ and the inclusion map.
\section{Problem 0.20}
    \subsection*{(ii)}
        \begin{mdframed}[]
            Show that there is a contravariant functor between $Top$ and $Ring$, given by $C(X)$.
        \end{mdframed}
        \fbox{(i)} Let $X \in objTop$, then $C(X) \in objRing$ by \fbox{Part (i)}. \\\\
        \fbox{(ii)} Let $f : X \rightarrow Y$.
        Then let $u \in C(Y)$. We define $f^* : C(Y) \rightarrow C(X)$ by
        \[
            f^*(u) = u \circ f
        \]
        Then $f^*(u)$ is a continuous map from $X$ to $\mathbb{R}$. That is, if $u \in C(Y)$ then $f^*(u) \in C(X)$,
        and $f^*$ acts as a ring homomorphism. 
        To demonstrate this, let $u,v \in C(Y)$ and $h : X \rightarrow Y$ where $X,Y$ are topological spaces.
        Let $x\in X$ be arbitrary. Then we can write 
        \begin{align}
            h^*(f + g)(x) &= ((f + g) \circ h)(x) \\
            &= (f + g)(h(x)) \\
            &= f(h(x)) + g(h(x)) \\
            &= (f \circ h)(x) + (g \circ h)(x) \\
            &= h^*(f(x)) + h^*(g(x))
        \end{align}
        Finally we write
        \begin{align}
            h^*(fg)(x) &= (fg \circ h)(x)\\
            &= (fg)(h(x)) \\
            &= f(h(x))g(h(x)) \\
            &= (f\circ h)(x) \cdot (g \circ h)(x) \\
            &= h^*(g(x)) \cdot h^*(g(x))
        \end{align}
        So the ring structure is preserved, adn the property is satisfied.
        \\\\
        \fbox{(iii)} Let $f:X \rightarrow Y$ and $g: Y \rightarrow Z$ so that both are continuous and
         $g \circ f$ is defined and a morphism in $Hom(X,Z)$. Let $u \in C(Z)$ be a continuous function on $Z$.
        Then we write 
        \begin{align}
            (g \circ f)^*(u) &= u \circ (g \circ f) & \text{(definition of $(g \circ f)^*$)}\\
            &= (u \circ g) \circ f  & \text{(associativity of functional composition)}\\
            &= g^*(u) \circ f  & \text{(definition of $g^*$)} \\
            &=f^*(g^*(u)) & \text{(definition of $f^*$)}\\
            &=(f^* \circ g^*)(u) & \text{(definition of composition)}
        \end{align}\\
        \fbox{(iv)} Let $X \in objTop$. Then let $u \in C(X)$ be arbitrary. We then can write 
        \begin{align}
            (1_X)^*(u) & = u \circ 1_X & \text{(by definition of $1_X^*$)} \\
            &= u & \text{(identity in $Hom(X,X)$)} \\
            &= 1_{C(X)} \circ u & \text{(identity in $Hom(C(X),C(X)$)} \\
            &= 1_{C(X)}(u) & \text{(definition of composition)}
        \end{align}
        Since all of these properties have been shown, we conclude that we have a contravariant functor from
        the class of topological spaces to the ring of continuous maps. 
\end{document}