\documentclass{article}

\usepackage{times}
\usepackage{amssymb, amsmath, amsthm}
\usepackage[margin=.5in]{geometry}
\usepackage{graphicx}
\usepackage[linewidth=1pt]{mdframed}

\usepackage{import}
\usepackage{xifthen}
\usepackage{pdfpages}
\usepackage{transparent}

\newcommand{\incfig}[1]{%
    \def\svgwidth{\columnwidth}
    \import{./figures/}{#1.pdf_tex}
}

\newtheorem{theorem}{Theorem}[section]
\newtheorem{lemma}{Lemma}[section]
\newtheorem*{remark}{Remark}
\theoremstyle{definition}
\newtheorem{definition}{Definition}[section]

\begin{document}

\title{Algebraic Topology - Homework 2}
\author{Philip Warton}
\date{\today}
\maketitle
\section*{1.3}
    \begin{mdframed}[]
        Let $R : S^1 \rightarrow S^1$ rotate any given point by $x$ radians, then $R \simeq 1_{S^1}$.
    \end{mdframed}
    \begin{proof}
        We define the following function $F:S^1 \times X \rightarrow S^1$ and claim that it is a homotopy:
        \[
            F(p,t) = p \cdot e^{i(tx)}
        \]
        Firstly, we can trivially verify that
        \begin{align*}
            F(p,0) &= p \cdot e^0 = p \cdot 1 = p = 1_{S^1}(p) \\
            F(p,1) &= p \cdot e^{ix} = R(p)
        \end{align*}
        Since this function simply rotates the point over to $x$ radians as we vary $t$, it follows that it is 
        continuous and thus a homotopy between $R$ and $1_{S^1}$.
    \end{proof}
    \begin{mdframed}[]
        Every continuous map $f: S^1 \rightarrow S^1$ is homotopic to a continuous map $g:S^1 \rightarrow S^1$
        with $g(1) = 1$.
    \end{mdframed}
    \begin{proof}
        Let $f:S^1 \rightarrow S^1$ be continuous. Then $f(1) \in S^1$ with some corrosponding argument/angle
        $x \in [0,2\pi)$. Let $R_{\alpha} : S^1 \rightarrow S^1$ denote the function given earlier as $R$ with 
        the rotation being given in radians by $\alpha$. Then,
        \[
            (R_{-x} \circ f)(1) = R_{-x}(f(1)) = R_{-x}(e^{ix}) = e^{i(0)} = 1
        \]
        Let $g = R_{-x} \circ f$ and it follows that since $R \simeq 1_{S^1}$,
        \begin{align*}
            R_{-x} \circ f &\simeq 1_{S^1} \circ f \\
            g &\simeq f
        \end{align*}
        where $g(1) = 1$.
    \end{proof}
\section*{1.5}
    \begin{mdframed}[]
        Let $X = \{0\} \cup \{1, \frac{1}{2}, \frac{1}{3}, \cdots, \frac{1}{n}, \cdots \}$ and let $Y$ be a
        countable discrete space. Then $X$ and $Y$ do not have the same homotopy type.
    \end{mdframed}
    \begin{proof}
        Let $f: X \rightarrow Y$ be some continuous function. Then there exists some $y \in Y$ such that 
        $0 \in f^{-1}(y)$ (since $0$ must of course get mapped to some point in $Y$). Since $Y$ is a discrete 
        space it follows that $\{y\}$ is an open set. Since $f$ is continuous, $f^{-1}(\{y\})$ is an open set in 
        $X$ containing 0. Assuming that $X$ is equipped with the subspace topology from $\mathbb{R}$ it follows that 
        for any open neighborhood $U$ of 0 the followsing is true:\\\\
        There exists some $N \in \mathbb{N}$ such that for every $n \geq N$, $\frac{1}{n} \in U$.\\\\
        By this, it follows that $f^{-1}(\{y\})$ contains infinitely many points from $X$, and so it must be 
        the case that only finitely many points in $X$ are mapped to points other than $y$. Let $g:Y\rightarrow X$
        be continuous. Then we conclude that since $f(X)$ is a finite set, so too is $(g \circ f)(X)$.
        Somehow XD we conclude that $1_X \not\simeq g \circ f$ for any $f,g$ arbitrarily, and thus the two 
        spaces have two different homotopy types.
    \end{proof}
\section*{1.7}
    \begin{mdframed}[]
        Let $X = \{x,y\}$ with topology $\{X, \emptyset, \{x\}\}$, then $X$ is contractible.
    \end{mdframed}
    \begin{proof}
        Define a function $F:X \times [0,1] \rightarrow X$ by 
        \[
            F(p,t) = 
            \begin{cases}
                p, & \text{ when } t \leq \frac{1}{2}\\
                x, & \text{ when } t > \frac{1}{2}
            \end{cases}
        \]
        We can verify immediately that 
        \begin{align*}
            F(p,0) &= p = 1_X(p)\\
            F(p,1) &= x = c_x(p) & \text{(where $c_x$ is the constant map to the point $x$)}
        \end{align*}
        Let us verify that each pre-image of a neighborhood in $X$ is open in $X$.
        Firstly $F^{-1}(X) = X \times [0,1]$ since the function is well defined and surjective.
        Then we know that $F^{-1}(\{x\}) = (\{x\} \times [0,1]) \cup (\{y\} \times (\frac{1}{2}, 1])$ which is open 
        in $X \times [0,1]$. And finally $F^{-1}(\emptyset) = \emptyset$ since the function is well defined.
        So we conlcude that $F$ is continuous and a homotopy and therefore $1_X \simeq c_x$ so $X$ is contractible.
    \end{proof}
\section*{1.8}
    \begin{mdframed}[]
        There exists a continuous image of a contractible space that is not contractible.
    \end{mdframed}
    \begin{proof}
        Let $f:[0,1] \rightarrow S^1$ be given by $f(x) = e^{i(2\pi)x}$. The space $[0,1]$ is contractible
        trivially, and we claim that $f([0,1]) = S^1$ and is therefore not contractible. Let $O$ be an open
        set in $S^1$. Then it is a union of some open intervals along the circle. The pre-image of each interval
        that does not include $1$ will be of the form $(a,b)$ which is clearly open. Otherwise it will be of the 
        form $[0,a) \cup (b,1]$ and will remain open. Thus $f$ is continuous. Let $y \in S^1$ and then it can be written 
        as $y = e^{i t}$ where $t \in [0,2\pi)$. Then it follows that it has some pre-image under $f$ so the 
        function is surjective and $f([0,1]) = S^1$. We assume without proof that $S^1$ is contractible.
    \end{proof}
    \begin{mdframed}[]
        A retract of a contractible space is contractible
    \end{mdframed}
    \begin{proof}
        Let $A \subset X$ be a retract where we have a retraction $r:X \rightarrow A$ and 
        $r(a) = a$ for every element $a \in A$. Since $X$ is contractible we can write $1_X \simeq c_{x_0}$
        where $x_0 \in X$ and $c_{x_0}$ is the constant map where $c_{x_0}(x) = x_0$. Let $H : X \times [0,1] \rightarrow X$
        denote this homotopy. That is, $H(\ \cdot \ , 0) = 1_X$ and $H(\ \cdot\ , 1) = c_{x_0}$. Define a function 
        $H_A:A \times [0,1] \rightarrow A$ where $H_A = r \circ H$. Since $A \times [0,1] \subset X \times [0,1]$,
        the function $H$ remains well defined on $A \times [0,1]$. Then it follows that
        \begin{align*}
            H_A(a, 0) &= (r \circ H)(a, 0) \\
            &= r(H(a, 0)) \\
            &= r(1_X(a)) \\
            &= r(a) = a = 1_A(a) \\\\
            H_A(a, 1) &= (r \circ H)(a, 1) \\
            &= r(H(a, 1)) \\
            &= r(c_{x_0}(a)) \\
            &= r(x_0) = r_0 = c_{r_0}(a)
        \end{align*}
        Since $H_A$ is a composition of continuous functions, it itself is also continuous, and so it's a
        homology between the identity and the constant function and obviously $A$ is contracible.
    \end{proof}
\section*{4.7}
    \begin{mdframed}[]
        Compute $H_n(S^0)$ for all $n \geq 0$.
    \end{mdframed}
    We invoke \fbox{Theorem 4.13 (Rotman)} which states that we can write 
    \[
        H_n(S^0) = \sum_\lambda H_n(S^0_\lambda)
    \]
    Where each $S^0_\lambda$ is a path component of $S^0$. Since $S^0 = \{-1,1\}$, we can write
    \begin{align*}
        H_n(S^0) &= \sum_\lambda H_n(S^0_\lambda) \\
        &= H_n(\{-1\}) \oplus H_n(\{1\}) 
    \end{align*}
    Since each path component is a single-point space, we have 
    \[
        H_0(S^0) = \mathbb{Z} \oplus \mathbb{Z}, \ \ \ \ \ \ \ \ H_n(S^0) = 0
    \]
\section*{4.8}
    \begin{mdframed}[]
        Compute $H_n(X)$ for all $n \geq 0$ where $X$ is the cantor set.
    \end{mdframed}
    We invoke \fbox{Theorem 4.13 (Rotman)} once again and conclude that since the cantor set is totally disconnected
    and so each path component is a singular point. Write $X_\lambda = \{\lambda\}$ as a singleton set 
    in the cantor set. Thus,
    \[
        H_n(X) = \sum_{\lambda \in X} H_n(X_\lambda)
    \]
    And for $n = 0$, we have 
    \[
        H_0(X) = \sum_{\lambda \in X} H_0(\{pt\}) = \sum_{\lambda \in X} \mathbb{Z} = \bigoplus_{\lambda \in X}\mathbb{Z}
    \]
    And then trivially for any $n > 0$ we get 
    \[
        H_n(X) = \sum_{\lambda \in X} H_n(\{pt\}) = \sum_{\lambda \in X} 0 = 0
    \]
\section*{Other}
    \begin{mdframed}[]
        Let $0\rightarrow C'\rightarrow C \rightarrow C'' \rightarrow 0$ be a short exact sequence of chain complexes.
        We can write out the definition of the connecting homomorphisms and verify exactness of the long exact sequence.
    \end{mdframed}
    Let $f: C' \rightarrow C$ and $g : C \rightarrow C''$. We define 
    our connecting homomorphism by $h : H_n(C'') \rightarrow H_{n-1}(C')$ where $h([a]) = [c]$.
    But we must define what $[c]$ is in order for this function to be able to exist.
    Let $n \in \mathbb{N}$ be arbitrary and since $C, C',$ and $C''$ are all chain 
    complexes we can say let $[a] \in H_n(C_n'')$. That is, let $a \in C_n''$
    such that $\partial_n''(a) = 0$, and $a$ is a cycle. Since we have 
    a short exact sequence we know that $g$ is surjective, therefore $\exists b \in C_n$
    such that $g_n(b) = a$. Since we know that our diagram of chain homomorphisms on 
    chain complexes commutes it follows that 
    \[
        0 = \partial_n''(a) = \partial_n''(g_n(b)) = g_{n-1}(\partial_n(b))
    \]
    Since $\partial_n(b) \in \ker(g_{n-1})$ it follows that $\partial_n(b) \in im(f_{n-1})$.
    Thus, we have some unique element $c \in C_{n-1}'$ that satisfies $f_{n-1}(c) = \partial_n(b)$.
    Then by our commutativity we have 
    \[
        0 = \partial_{n-1}(\partial_n(b)) = \partial_{n-1}(f_{n-1}(c)) = f_{n-2}(\partial_{n-1}'(c)) 
    \]
    So then we can say $\partial_{n-1}'(c) \in \ker f_{n-2}$. Since $f$ is always injective,
    we know that this element is uniquely in the kernel of $f_{n-2}$ and so $\partial_{n-1}'(c)$
    must itself be equal to 0. Then it follows trivially that $c \in \ker \partial_{n-1}'$.
    Thus we have $c \in C_{n-1}'$ as a cycle, so our homomorphism $h([a]) = [c]$ is somewhat plausible.
    \\\\
    However for $h$ to be well defined it must be the case that we have only one element $[c]$ which 
    satisfies this property.
    Suppose that we choose $b, b'$ both to be in the pre-image of $a$ under $g_n$.
    Then we take the pre-image of their boundaries under $f_{n-1}$. That is, let 
    \[
        c,c' \in C_{n-1}' \ \ \ \ | \ \ \ \ f_n(c) = \partial_n(b), \ \ f_n(c') = \partial_n(b')
    \]
    We must show that $c,c'$ are of the same homology class in $H_{n-1}(C_{n-1}')$.
    Firstly notice that 
    \[
        g_n(b - b') = g_n(b) - g_n(b') = a - a = 0
    \]
    Then since $b - b' \in \ker(g_n) = im(f_n)$ we say there is some unique element
    $\alpha \in C_n$ such that its image under $f_n$ is $b - b'$.
    Then we can write 
    \begin{align*}
        f_{n-1}(\partial_n'(\alpha)) &= \partial_n(f_n(\alpha)) & \text{(commutativity)} \\
        &=\partial_n(b - b') \\
        &=\partial_n(b) - \partial_n(b') \\
        &=f_{n-1}(c) - f_{n-1}(c') \\
        &=f_{n-1}(c - c')
    \end{align*}
    Then since $f_{n-1}$ is injective it follows that $\partial_n'(\alpha) = c - c'$.
    So then since we have this equality, it follows naturally that $c$ and $c'$ belong to the 
    same homology class in $H_{n-1}(C')$. \\\\
    We already have short exact sequences from 
    \[
        0 \rightarrow C_n' \rightarrow C_n \rightarrow C_n'' \rightarrow 0
    \]
    So by functorality we also have a short exact sequence,
    \[
        0 \rightarrow H_n(C_k') \rightarrow H_n(C_k) \rightarrow H_n(C_k'') \rightarrow 0
    \]
    What is left to show is that we have `exactness' on our homomorphism $h:H_n(C_n'') \rightarrow H_{n-1}(C_{n-1}')$
    so that we can follow this chain of homomorphisms all the way down until 0.
    We must show that $\ker f_* = im(h)$.\\\\
    \fbox{$\subset$} Let $[c] \in \ker(f_{n-1}^*)$. Then by the funtorality of $H_n$ we say that 
    $[f_{n-1}(c)] = f_{n-1}^*([c]) = 0$. 
    Since $[f_{n-1}(c)] = 0$ we can take some cycle $b \in C_n$ and its boundary 
    will be equal to $f_{n-1}(c)$. That is, $\partial_n(b) = f_{n-1}(c)$.
    So we write that 
    \[
        \partial_n''(g_n(b)) = g_{n-1}(\partial_n(b)) = g_{n-1}(f_{n-1}(c)) = 0
    \]
    So we have that $a = g_n(b) \in \ker \partial_n''$, and thus it is a cycle in $C_n''$,
    and forms a homology class $[a] \in H_{n}(C_n'')$. So it follows that for any $[c] \in \ker(f_{n-1}^*)$
    we have some $[a] \in H_{n}(C_n'')$ such that $h([a]) = [c]$. Thus $\ker f_{n-1}^* \subset im(h)$.\\\\
    \fbox{$\supset$} Let $[a] \in H_n(C_n'')$, and so $h([a]) \in H_{n-1}(C_{n-1}')$.
    We wish to show that $h([a]) \in \ker f_{n-1}^*$. So we apply our function $h$ and write 
    \begin{align*}
        f_{n-1}^*(h([a])) &= f_{n-1}^*([c]) \\
        &= [f_{n-1}(c)] \\
        &= [\partial_n(b)] & \text{(commutativity)}\\
        &= 0 & \text{(how?)}
    \end{align*}
    So then we have shown that $h([a]) \in \ker f_{n-1}^*$ for any $[a] \in H_{n}(C_n'')$.
    Thus it follows that $\ker(f_{n-1}^*) = im(h)$. So it follows that we have a long exact 
    homology sequence 
    \[
        0 \rightarrow H_n(C_n') \rightarrow H_n(C_n) \rightarrow H_n(C_n'') \rightarrow_h H_{n-1}(C_{n-1}') \rightarrow H_{n-1}(C_{n-1}) \rightarrow \cdots \rightarrow H_0(C_0'') \rightarrow 0
    \]
\end{document}