\documentclass{article}

\usepackage{times}
\usepackage{amssymb, amsmath, amsthm}
\usepackage[margin=.5in]{geometry}
\usepackage{graphicx}
\usepackage[linewidth=1pt]{mdframed}

\usepackage{import}
\usepackage{xifthen}
\usepackage{pdfpages}
\usepackage{transparent}

\newcommand{\incfig}[1]{%
    \def\svgwidth{\columnwidth}
    \import{./figures/}{#1.pdf_tex}
}

\newtheorem{theorem}{Theorem}[section]
\newtheorem{lemma}{Lemma}[section]
\newtheorem*{remark}{Remark}
\theoremstyle{definition}
\newtheorem{definition}{Definition}[section]

\begin{document}

\title{Applied Ordinary Differential Equations - Homework 4}
\author{Philip Warton}
\date{\today}
\maketitle
\section*{9.3}
    \subsection*{4}
        \begin{mdframed}[]
            Observe the following system:
            \[
                dx / dt = (2+x)(y-x), \ \ \ \ dy/dt=(4-x)(y+x)
            \]
        \end{mdframed} 
        To have $dx / dt = 0$, it must be the case that $x = -2$ or $x = y$.
        To have $dy / dt = 0$, it must be that $x=4$ or $y=-x$.
        The only points that cause both to be 0 are $(0,0),(-2,2),(4,4)$.
        Let $F(x,y) = (2+x)(y-x), G(x,y)=(4-x)(y+x)$.
        Our next step is to find the matrix 
        \begin{align*}
            D(x,y) &= \begin{pmatrix}
               F_x & F_y \\
               G_x & G_y 
            \end{pmatrix}\\
            &= \begin{pmatrix}
                -2 - 2x + y & 2 + x \\
                4 -2x -y & 4-x
            \end{pmatrix}
        \end{align*}
        For $(0,0)$, we simply have 
        \[
            x' = 
            \begin{pmatrix}
                -2 & 2 \\
                4 & 4
            \end{pmatrix}x
        \]
        This has eigenvalues of $r = 1 \pm \sqrt{17}$. Since $1 < \sqrt{17}$ it must be the 
        case that we have one positive and one negative real eigenvalue. This means that 
        we will have a saddle point at $(0,0)$ in the linearized system, so we have a point of 
        a similar nature in our overall system. For the equilibrium point $(-2,2)$, we get
        \[
            u' = \begin{pmatrix}
                4 & 0\\
                6 & 6
            \end{pmatrix}u
        \]
        The matrix in this linearized system has two eiegenvalues $r=4, r = 6$. This means 
        that we will have an unstable source node at the point $(-2,2)$. For $(4,4)$, we get 
        \[
            v' = \begin{pmatrix}
                -6 & 6 \\
                -8 & 0
            \end{pmatrix}   v
        \]
        This system has eigenvalues of $r = -3 \pm \sqrt{39}i$. With this, it follows that we have 
        a stable spiral point, since we have a negative real part, and a non-zero complex part.
    \subsection*{5}
        \begin{mdframed}[]
            \[
                dx/dt = x-x^2-xy, \ \ \ \ dy/dt = 3y - xy - 2y^2
            \]
        \end{mdframed}
        To have $dx/dt=0$, it must be the case that $x = 0$, or that $1-x = y$. Then to have 
        $dy/dt = 0$, we need the conditions that $y = 0$ or that $3 - x = 2y$.
        The points that cause both derivatives to be 0 are $(0,0),(1,0),(0,3/2), (-1,2)$.
        As in the previous problem, let $F(x,y) = x - x^2 - xy, G(x,y) = 3y - xy -2y^2$. 
        We can write 
        \begin{align*}
            D(x,y) &= \begin{pmatrix}
                F_x & F_y \\
                G_x & G_y
            \end{pmatrix} \\
            &= \begin{pmatrix}
                1 - 2x - y & -x \\
                -y & 3 - x -4y
            \end{pmatrix} 
        \end{align*}
        Then for the point $(0,0)$, we have a linearized system given by 
        \[
            x' = \begin{pmatrix}
                1 & 0 \\
                0 & 3
            \end{pmatrix}x
        \]
        Since this is a diagonal matrix, it is clear that we have eigenvalues $r_1 = 1, r_2 = 3$.
        Both are positive and real, so it follows that at the origin we will have an 
        unstable nodal source point. Let $u = (1, 0)$, then we can write linearized system
        at this point as 
        \[
            u' = \begin{pmatrix}
                -1 & -1 \\
                0 & 2
            \end{pmatrix}   u 
        \]
        For this matrix we have eigenvalues of $r_1 = -1, r_2 = 2$.
        So this means that in the non-linear system, we are going to have a saddle point at 
        $(1,0)$. Now for the point $(0, 3/2)$, we can plug this into our jacobian to find the 
        linearized system at the point. So, let $v = \begin{pmatrix}
            x \\
            y - 3/2
        \end{pmatrix}$
        \[
            v' = \begin{pmatrix}
            -1/2 & 0 \\
            -3/2 & -2    
            \end{pmatrix}
            v
        \]
        This matrix has eigenvalues of $r_1 = -1/2, r_2 = -2$. Since we have two negative 
        real eigenvalues, we know that in the broader nonlinear system we will have a nodal sink 
        point, that is, a local stable equilibrium. Let $w = (-1,2)$, then we have linearized system 
        at $w$ given by 
        \[
            w' = \begin{pmatrix}
                1 & 1\\
                -2 & -4
            \end{pmatrix}   
            w
        \]
        This system has two distinct eigenvalues, $r = \dfrac{-3 \pm \sqrt{17}}{2}$.
        So, since $\sqrt{17} > \sqrt{16} = 4 > 3$, we know that one of these is positive,
        and the other negative, so we have another saddle point at $w$.
    \subsection*{24}
        \begin{mdframed}[]
            Consider the system 
            \[
                x' = \begin{pmatrix}
                    0 & 1 \\ -1 & 0
                \end{pmatrix}x
            \]
            and show that it has a center point at $(0,0)$ with eigenvalues $r = \pm i$.
            Then consider the system 
            \[
                x' = \begin{pmatrix}
                    \epsilon & 1 \\
                    -1 & \epsilon
                \end{pmatrix}x
            \]
            And show behavior with positive and negative $\epsilon$ is as expected.
        \end{mdframed}
        To show that the first system is as described, we wish to find the eigenvalues of the system.
        So we write 
        \begin{align*}
            0 &= \det\left( \begin{pmatrix}
                0 & 1 \\
                -1 & 0
            \end{pmatrix}-rI\right)
            \\
            0&= r^2 + 1 \\
            -1 &= r^2 \\
            \pm i &= r
        \end{align*}
        So then since we have no real part, and we do have a complex part, it must be the case 
        that we have a center point at $(0,0)$ as described in the problem statement.
        For the second given system here, we have eigenvalues given by 
        \begin{align*}
            0 &=\det\left(\begin{pmatrix}
                \epsilon & 1 \\
                -1 & \epsilon
            \end{pmatrix} - rI\right) \\
            0 &= \det \begin{pmatrix}
                \epsilon - r & 1 \\
                -1 & \epsilon - r
            \end{pmatrix} \\
            0 &= (\epsilon - r)^2 + 1\\
            -1 &= (\epsilon - r)^2 \\
            \pm i &= \epsilon - r \\
            r &= \epsilon \pm i
        \end{align*}
        So for this system we do have a non-zero real part and also an imaginary part.
        This means that if our real part, $\epsilon$, is positive, we have a spiral source 
        point at the origin. Then if $\epsilon$ is negative we must have a stable spiral sink point.
        Thus for any non-zero $\epsilon$ we still do not have a center point. We may consider $\epsilon = 0$
        to be a bifurcation point since our system goes from stable to unstable as we vary $\epsilon$, 
        and since at this point we have a qualitatively unique type of system.
\section*{9.4}
    \subsection*{1}
        \begin{mdframed}[]
            Consider the system:
            \[
                dx/dt = x(1.5 - x - 0.5y), \ \ \ \ dy/dt = y(2 - y - 0.75x)
            \]
        \end{mdframed}
        First, let us determine at which points does the system have an equilibrium 
        solution. For $dx/dt = 0$, we must have $x = 0$ or $2x = 3 - y$. Then for 
        $dy/dt = 0$, it follows that $y = 0$ or that $3x= 8 - 4y$.
        So, to satisfy both, we of course have $(0,0)$. Then, additionally, we have 
        $(0,2), (2/3,0)$, and $(4/5, 7/5)$. Let us find the jacobian of our non-linear 
        system so that it can be linearized at each point. We get the following result:
        \begin{align*}   
            D(x,y) &= \begin{pmatrix}
                F_x & F_y \\
                G_x & G_y
            \end{pmatrix} \\
            &= \begin{pmatrix}
                -2x + 1.5 -0.5y & -0.5x \\
                -0.75y & -2y + 2  - 0.75x
            \end{pmatrix}
        \end{align*}
        So for the origin $(0,0)$ we have the locally linearized system 
        \[
            x' = \begin{pmatrix}
                1.5 & 0 \\
                0 & 2
            \end{pmatrix}x
        \]
        This clearly has two positive eigenvalues, and so we will have a nodal source 
        (unstable) at the origin. Let $u = (x, y-2)$, and then we have a linearized system 
        given by 
        \[
            u' = \begin{pmatrix}
                0.5 & 0 \\
                -1.5 & -2
            \end{pmatrix}   
            u
        \]
        This system has two eigenvalues of $r_1 = 0.5, r_2 = -2$. So, with one positive 
        and one negative real eigenvalues, we must have a saddle point at $(0,2)$ in our system.
        Now let $v = (x - 2/3, y)$ and we will have the locally linearized system at $(2/3, 0)$ given by 
        \begin{align*}
            v' &= \begin{pmatrix}
                -2(\frac{2}{3}) +1.5 - 0.5(0) & -0.5(\frac{2}{3}) \\
                -0.75(0) & -2(0) + 2 -0.75(\frac{2}{3})
            \end{pmatrix} v \\
            &= \begin{pmatrix}
                \frac{-4}{3} + 1.5 & \frac{-1}{3} \\
                0 & 2 - \frac{1}{2}
            \end{pmatrix} v \\
            &= \begin{pmatrix}
                1/6 & -1/3 \\
                0 & 3/2
            \end{pmatrix}
            v
        \end{align*}
        This linearized system at $(2/3, 0)$ has two distinct eigenvalues which can be derived as 
        \begin{align*}
            \det(A - rI) &= \det\left(
                \begin{pmatrix}
                    1/6 - r & -1/3\\
                    0 & 3/2 - r
                \end{pmatrix}
            \right)\\
            &= (1/6 - r)(3/2 - r)
        \end{align*}
        And clearly this has two zeros at $r_1 = 1/6, r_2 = 3/2$ and thus we have two 
        positive real eigenvalues for our system.
        This means that at the point $(3/2, 0)$ we have an unstable nodal source system.
        Finally let $w = (4/5-x, 7/5-y)$. Then we have a linearized system at the point $w$ given by.
        \begin{align*}
            w' &= \begin{pmatrix}
                -2(4/5) + 1.5 - 0.5(7/5) & -0.5(4/5) \\
                -0.75(7/5) & -2(7/5) + 2 - 0.75(4/5)
            \end{pmatrix} w \\
            &= \begin{pmatrix}
                -4/5 & -2/5 \\
                -21/20 & -7/5
            \end{pmatrix}
            w
        \end{align*}
        The eigenvalues of this system are values of $r$ such that $\det(A - rI) = 0$.
        That is, when 
        \begin{align*}
            \det\left(
                \begin{pmatrix}
                    -4/5 - r & -2/5 \\
                    -21/20 & -7/5 - r
                \end{pmatrix}
            \right) &= 0 \\
            (-4/5 - r)(-7/5 - r) - (-2/5)(-21/20) &= 0 \\
            r^2 - 11/5 r + 28/25 - 21/50 &= 0 \\
            r^2 - 11/5 r + 56/50 - 21/50 &= 0 \\
            r^2 - 11/5 r + 35/50 &= 0 \\
            r^2 - 11/5 r + 7/10 &= 0
        \end{align*}
        And this, of course, gives us eigenvalues of 
        $r = -11/10 \pm \frac{\sqrt{51}}{10}$.
        We know that $\sqrt{51} < 11$ implies both eigenvalues are negative,
        and therefore we have a stable nodal source point at $(4/5, 7/5)$.
        \\\\
        We conclude that for the majority of starting conditions, the populations 
        wil converge to the point $(4/5, 7/5)$ because solutions will tend away 
        from these other nodes and in the direction of the stable source node.
        Or, if far away enough, the solutions will simply trend straight towards 
        the point since both $F$ and $G$ will be negative. No graph is given for this 
        system (I'm already a bit behind schedule).
\section*{9.5}
    \subsection*{1}
        Consider the system
        \[
            dx / dt = x(1.5 - 0.5y), \ \ \ \ dy/dt = y(-0.5+x)
        \]  
        Let $F(x,y) = x(1.5 - 0.5y), G(x,y) = y(-0.5+x)$.
        To find critical points, we must determine when $F = G = 0$.
        These points are given by $(0,0), (\frac{1}{2}, 3)$.
        Then we write 
        \begin{align*}
            F_x &= \frac{3}{2} - \frac{y}{2} \\
            F_y &= \frac{-x}{2} \\ 
            G_x &= y \\
            G_y &= \frac{-1}{2} + x
        \end{align*}
        Which gives way to the jacobian matrix. That is 
        \[
            D(x,y) = \begin{pmatrix}
                \frac{3}{2} - \frac{y}{2} & \frac{-x}{2} \\
                y & \frac{-1}{2} + x
            \end{pmatrix}   
        \]
        For the linearized system at the origin, we can write
        \[
            x' = \begin{pmatrix}
                \frac{3}{2} & 0 \\
                0 & \frac{-1}{2}
            \end{pmatrix}   
        \]
        This clearly has two eigenvalues $r_1 = \dfrac{3}{2}$ and $r_2 = \dfrac{-1}{2}$.
        So this gives us a saddle point at the origin for this system. 
        Now, let $u = (x - \frac{1}{2}, y - 3)$, and we have a linearized system at $u = 0$
        of 
        \begin{align*}
            u' &= \begin{pmatrix}
                \frac{3}{2} - \frac{3}{2} & \frac{-\frac{1}{2}}{2} \\
                3 & \frac{-1}{2} + \frac{1}{2}
            \end{pmatrix} u \\
            &= \begin{pmatrix}
                0 & \frac{-1}{4} \\
                3 & 0
            \end{pmatrix} u
        \end{align*}
        To determine the eigenvalues, let $\det(A - rI) = 0$. Then 
        \begin{align*}
            r^2 - (\frac{-1}{4})(3) &= 0 \\
            r^2 + \frac{3}{4} &= 0 \\
            r^2 &= \frac{-3}{4} \\
            r &= \pm \frac{i\sqrt{3}}{2} 
        \end{align*}
        Since we have no real part, this is an asymptotically stable center point at 
        $(1/2, 3)$. Since the coefficient to our real part is positive, we 
        know that this will rotate counter-clockwise.
        For the overall system, we say that if sufficiently close to 
        the point $(1/2, 3)$ the populations will give and take in a trigonmetric fashion 
        as the trajectory circles counterclockwise around the equilibrium point. 
        If we only have a population for $x$ then it will just 
        grow permanently, and if we only have a population for $y$,
        it will simply shrink permanently, this is given by the nature of a saddle point on the origin.
    \subsection*{2}
        \begin{mdframed}
            Consider the system
            \[
                dx/dt = x(1 - 0.5y), \ \ \ \ dy/dt = y(-0.25 + 0.5x)   
            \]
        \end{mdframed}
        Let $F(x,y) = x(1 - 0.5y), G(x,y) = y(-0.25 + 0.5x)$.
        We identify our equilibrium points to be the origin $(0,0)$ and 
        $(1/2, 2)$. Then we must write our jacobian matrix.
        To do so, we must get each of our desired partial derivatives:
        \begin{align*}
            F_x &= 1 - \frac{y}{2} && G_x = \frac{y}{2} \\
            F_y &= \frac{-x}{2} && G_y = \frac{-1}{4} + \frac{x}{2}
        \end{align*}
        So then you have a jacobian matrix given by 
        \[
            D(x,y) = \begin{pmatrix}
                1 - \frac{y}{2} & \frac{y}{2} \\
                \frac{-x}{2} & \frac{-1}{4} + \frac{x}{2}
            \end{pmatrix}   
        \]
        At the origin, we can linearize the system giving us 
        \[
            x' = \begin{pmatrix}
                1 & 0 \\
                0 & \frac{-1}{4}
            \end{pmatrix}  x 
        \]
        Then this has two eigenvalues, $r_1 = 1, r_2 = \frac{-1}{4}$.
        So this gives us a saddle point at the origin, with an eigenvector in the 
        $x$-direction associated with 1, and an eigenvector in the $y$-direction 
        which is associated with $\frac{-1}{4}$.
        Then for the point at $(1/2, 2)$, we have the system
        \begin{align*}
            u' = \begin{pmatrix}0 & 1 \\
            \frac{-1}{4} & 0\end{pmatrix}u
        \end{align*}
        And for this system, we can get our eigenvalues by solving 
        \begin{align*}
            r^2 - (1)(\frac{-1}{4}) &= 0 \\
            r^2 &= \frac{-1}{4} \\
            r &= \pm \frac{i}{2}
        \end{align*}
        We have a center point given by this system. We can make 
        a decent guess if it is clockwise or counter clockwise by the behavior 
        at the nearby saddle point's linearized system. 
        By this, we guess that it is probably in the counterclockwise direction.
        The behavior of the populations of this system are quite similar to the last,
        in that most starting points will ultimately spiral around our non-trivial equilibrium point.
\end{document}