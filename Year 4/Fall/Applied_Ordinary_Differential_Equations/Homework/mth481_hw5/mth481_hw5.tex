\documentclass{article} 
% -------------------------------------------- % 
% ------------------PACKAGES------------------ % 
% -------------------------------------------- % 
\usepackage{amssymb, amsmath, amsthm} 
\usepackage{bm}
\usepackage{enumerate}
\usepackage[margin=.5in]{geometry}
\usepackage{graphicx} \usepackage{import} \usepackage{listings}
\usepackage[linewidth=1pt]{mdframed}
\usepackage{pdfpages}
\usepackage{times}
\usepackage{transparent}
\usepackage{xifthen}
% -------------------------------------------- %

% Figure command
\newcommand{\incfig}[1]{%
    % Adjust number for defualt figure width
    \def\svgwidth{.5\columnwidth} 
    \import{./figures/}{#1.pdf_tex}
}

% Theorem command
\newtheorem{theorem}{Theorem}[section]
\newtheorem{lemma}{Lemma}[section]
\newtheorem*{remark}{Remark}
\theoremstyle{definition}
\newtheorem{definition}{Definition}[section]

\begin{document}

% -------------------------------------------- %
\title{Applied Ordinary Differential Equations --- Homework 5}
\author{Philip Warton}
\date{\today}
\maketitle
% -------------------------------------------- %

\section*{5.2}
    \subsection*{5.2.5}
        \begin{mdframed}
            Do the following:
            \begin{enumerate}[a.]
                \item Seek the power series relationship around the point $x_0$,
                    and figure out the recurrence relation on $a_n$.
                \item Find the first four non-zero terms in the power series'
                    for two solutions $y_1, y_2$.
                \item Show that $y_1,y_2$ form a fundamental set of
                    solutions by computing the wronskian $W[y_1,y_2](x_0)$.
                \item If possible, find the general term in each solution.
            \end{enumerate}
            \[ 
                y'' + k^2 x^2 y = 0, \ \ \ \ x_0 = 0, k \text{ is constant}
            \]
        \end{mdframed}
            \subsubsection*{a.}
                To find the power series solution around the point $x_0$,
                we must observe that $P(x) = 1, Q(x) = 0, R(X) = k^2 x^2$.
                This equation does not change when we divide by $P(X)$ since 
                it is equal to 1, so we can also write $p(x) = 0, q(x) = k^2 x^2$.
                So we write,
                \begin{align*}
                    y &= \sum_{n=0}^\infty a_n x^n \\
                    y' &= \sum_{n=1}^\infty n a_n x^{n-1} \\
                    y'' &= \sum_{n=2}^\infty n (n-1) a_n x^{n-2}
                \end{align*}
                Then if we substitute this into our original equation we get 
                \[
                    \sum_{n=2}^\infty n (n-1) a_n x^{n-2}
                    + k^2 x^2 \sum_{n=0}^\infty a_n x^n 
                    = 0
                \] 
                We can shift the index by 2 and then we have 
                \begin{align*}
                    \sum_{n=0}^\infty (n+2) (n+1) a_{n+2} x^{n}
                    + k^2 x^2 \sum_{n=0}^\infty a_n x^n 
                    &= 0 \\
                    \sum_{n=0}^\infty (n + 2) (n + 1) a_{n+2} x^n
                    + k^2 x^2 a_n x^n &= 0 \\
                    \sum_{n=0}^\infty \left( (n + 2) (n + 1) a_{n+2} 
                    + k^2 x^2 a_n\right) x^n &= 0 \\
                \end{align*}
                So for this to be true for every $x$, it follows that 
                \begin{align*}
                    (n + 2) (n + 1) a_{n+2} + k^2 x^2 a_n &= 0 \\
                    (n + 2) (n + 1) a_{n+2} &= -k^2 x^2 a_n \\
                    a_{n+2} &= \frac{-k^2 x^2}{(n+2) (n+1)} \cdot a_n
                \end{align*}
                And thus we have our recurrence relation.
            \subsubsection*{b.}
                
    \subsection*{5.2.18}
\section*{5.3}
    \subsection*{5.3.8}
    \subsection*{5.3.17}
    \subsection*{5.3.18}
\end{document}
