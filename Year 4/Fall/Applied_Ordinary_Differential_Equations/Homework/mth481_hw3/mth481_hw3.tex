\documentclass{article}

\usepackage{times}
\usepackage{amssymb, amsmath, amsthm}
\usepackage[margin=.5in]{geometry}
\usepackage{graphicx}
\usepackage[linewidth=1pt]{mdframed}

\usepackage{import}
\usepackage{xifthen}
\usepackage{pdfpages}
\usepackage{transparent}
\usepackage{bm}

\newcommand{\incfig}[1]{%
    \def\svgwidth{\columnwidth}
    \import{./figures/}{#1.pdf_tex}
}

\newtheorem{theorem}{Theorem}[section]
\newtheorem{lemma}{Lemma}[section]
\newtheorem*{remark}{Remark}
\theoremstyle{definition}
\newtheorem{definition}{Definition}[section]

\begin{document}

\title{Applied Ordinary Differential Equations - Homework 3}
\author{Philip Warton}
\date{\today}
\maketitle
\section*{Problem 1}
    \begin{mdframed}[]
        Find the general solution to the differential equation 
        \[
            \bm x' = \begin{pmatrix}
                1 & 1 \\
                4 & -2
            \end{pmatrix} \bm x + \begin{pmatrix}
                e^{-2t} \\
                -2e^t
            \end{pmatrix}
        \]
    \end{mdframed}
    First we wish to diagonalize our matrix. That is, find the eigenvalues and their corresponding eigenvectors, 
    and the inverse of the matrix where columns are eigenvectors.
    So we begin by writing our characteristic polynomial 
    \[
        (1 - \lambda)(-2 - \lambda) - (1)(4) = \lambda^2 + \lambda - 1 - 4 = \lambda^2 + \lambda - 5    
    \] 
    So this gives us eigenvalues $r_1 = 2$ and $r_2 = -3$. From there we find eigenvectors given by 
    \[
        \ker(A - 2I) = span(\begin{pmatrix}
            1 \\ 1
        \end{pmatrix})   
        \ \ \ \ \ \ \ \
        \ker(A + 3I) = span(\begin{pmatrix}
            -1 \\ 4
        \end{pmatrix})
    \]
    So then we write 
    \[
        T = \begin{pmatrix}
            1 & -1 \\
            1 & 4
        \end{pmatrix}    
        \ \ \ \ \ T^{-1} = \frac{1}{5}\begin{pmatrix}
           4 & 1 \\
           -1 & 1 
        \end{pmatrix}
    \]
    Let $\bm x = T \bm y$ and we get the following system of ODE's 
    \[
        \bm y' = \begin{pmatrix}
            2 & 0 \\
            0 & -3
        \end{pmatrix}   
        \bm y + T^{-1}\begin{pmatrix}
            e^{-2t} \\
            -2e^t
        \end{pmatrix}
    \]
    Which we can then write as 
    \begin{align*}
        y_1' &= -2y_1 + \frac{1}{5}(4e^{-2t} - 2e^t)\\
        y_2' &= 3y_2 + \frac{1}{5}(-e^{-2t} + 2e^t)
    \end{align*}
    Which is a decoupled system of non-homogenous ODE's. We then solve each of these using integrating factors
    and plug this back into $\bm x = T\bm y$.
    So then we can equivalently write 
    \begin{align*}
        y_1' - 2y_1 &= \frac{1}{5}(4e^{-2t} - 2e^t) \\
        y_2' - 3y_2 &= \frac{1}{5}(-e^{-2t} + 2e^t) \\\\
        y_1 &= e^{
            -\int (1/5) (4e^{-2t} - 2e^t)dt
        } \left[
            \int e^{
                \int (1/5) (4e^{-2t} - 2e^t)dt
            }(-2)dt + C
        \right] \\
        y_2 &=  e^{
            -\int (1/5) (-e^{-2t} + 2e^t)dt
        } \left[
            \int e^{
                \int (1/5) (-e^{-2t} + 2e^t)dt
            }(-3)dt + C
        \right] \\\\
        y_1 &= \frac{-1 + 2e^3t}{5e^2t} + c_1e^{2t} \\
        y_2 &= \frac{\frac{1}{e^{5t}}-\frac{5}{e^{2t}}}{25e^{-3t}}+c_2e^{3t}
    \end{align*}
    So then we have a solution given by 
    \[
        \bm x = T\bm y = \begin{pmatrix}
            1 & -1 \\
            1 & 4
        \end{pmatrix}   
        \begin{pmatrix}
            y_1 \\ y_2
        \end{pmatrix}
        = \begin{pmatrix}
            y_1 - y_2 \\
            y_1 + 4y_2
        \end{pmatrix}
    \]
\section*{Problem 2}
    \begin{mdframed}[]
        The equation of motion of the mass spring system is 
        \[
            m \dfrac{d^2 u}{dt^2} + c \dfrac{du}{dt} + ku = 0
        \]
        where $m, c, k$ are positive. Write this second order equaiton as a system of two first 
        order equations for $x = u$ and $y = du / dt$. Show that $x = 0, y = 0$ is a critical point 
        and analyze the nature and stability of the critical points as a function of the parameters
        $m,c,k$.
    \end{mdframed}
    So we have a system $mu'' + cu' + ku = 0$.
    Let $x = u, y = u'$ and then we can write both 
    \begin{align*}
        my' + cy + kx &= 0 \\
        my' + cx' + kx &= 0
    \end{align*}
    So it follows that 
    \begin{align*}
        y' &= \left(\frac{-k}{m}\right)x + \left(\frac{-c}{m}\right)y\\   
        x' &= y
    \end{align*}
    This means that we can write this as a system of ordinary differential equations
    given by 
    \[
        \bm x' =
        \begin{pmatrix}
            0 & 1 \\
            -k / m & -c / m
        \end{pmatrix}\bm x 
    \]
    Now let $\bm x$ be the constant function mapping $t$ to $\bm 0$. Then obviously $\bm x' = A \bm x = A \bm 0 = \bm 0$.
    This will be an equilibrium solution because as $t \rightarrow \infty$, $\bm x(t) \rightarrow \bm 0$.
    Now we can say that $x = 0, y = 0$ is a critical point. To analyze this point, we will look at techniques from 
    chapter 7.
    That is we will find the general solution of the form 
    \[
        \bm x = c_1 e^{r_1 t} \xi^{(1)} + c_2 e^{r_2 t} \xi^{(2)}   
    \]
    So we have a characteristic polynomial given by 
    \[
        (-\lambda)(-c / m - \lambda) - (1)(-k / m) = \lambda^2 + (c / m) \lambda + (k / m)   
    \]
    Which gives us two eigenvalues
    \begin{align*}
        r &= \frac{-(c/m) \pm \sqrt{(c/m)^2 - 4(k/m)}}{2} \\
        &= \frac{-c}{2m} \pm \frac{\sqrt{\frac{c^2 - 4km}{m^2}}}{2} \\
        &= \frac{-c}{2m} \pm \frac{\sqrt{c^2 - 4km}}{2m}
    \end{align*}
    So we say that $r_1 = (-c + \sqrt{c^2 - 4km})/2m, r_2 = (-c - \sqrt{c^2 - 4km})/2m$.
    With 3 parameters, $c, k, m$, we can partition $\mathbb{R}^3$ by certain bifurcation points for 
    these different parameters. We create the following table of conditions that affect the type and stability
    of our system:
    \begin{center}
        \begin{tabular}{ | l | l | l | }
            \hline
        \textbf{Parameters}       & \textbf{Type of Point}           & \textbf{Stability}             \\
            \hline
        $c^2 - 4km < 0$  &                         &                       \\
        $\ \ \ \ \ \ \ \ c < 0$  & Spiral Point            & Unstable              \\
        $ \ \ \ \ \ \ \ \ c > 0$ & Spiral Point            & Asymptotically Stable \\
        $ \ \ \ \ \ \ \ \  c = 0$ & Center                  & Stable                \\
            \hline
        $c^2 - 4km = 0$  &                         &                       \\
        $\ \ \ \ \ \ \ \ c < 0$  & Proper or improper node & Unstable              \\
        $ \ \ \ \ \ \ \ \ c > 0$ & Proper or improper node & Asymptotically Stable \\
            \hline
        $c^2 - 4km > 0$  &                         &                       \\
        $ \ \ \ \ \ \ \ \ k < 0$ & Node                    & Unstable              \\
        $ \ \ \ \ \ \ \ \ k > 0$ & Node                    & Asymptotically Stable \\
        $ \ \ \ \ \ \ \ \ k = 0$ & Saddle Point            & Unstable            \\ 
            \hline
        \end{tabular}
    \end{center}
\section*{Problem 3}
    \begin{mdframed}[]
        Consider the linear system
        \[
            \frac{dx}{dt} = a_{11}x + a_{12}y, \ \ \ \ \frac{dy}{dt} = a_{21}x + a_{22}y
        \]
        where $a_{11}, a_{12}, a_{21}, a_{22} \in \mathbb{R}$. Let $p = a_{11} + a_{22}$
        be the trace and $q = a_{11}a_{22} - a_{12}a_{21}$ be the determinant of the corresponding
        coefficient matrix. Let $\Delta = p^2 - 4q$. Show that the critical point $(0,0)$ is a 
        \begin{itemize}
            \item Node if $q > 0$ and $\Delta \geq 0$ \\
            \item Saddle point if $q < 0$ \\
            \item Spiral point if $p \neq 0$ and $\Delta < 0$ \\
            \item Center if $p = 0$ and $q > 0$
        \end{itemize}
        A theorem from linear algebra says that if $r_1$ and $r_2$ are eigenvalues of the coefficient matrix,
        then $q = r_1r_2$ and $p = r_1 + r_2$.
    \end{mdframed}
    \begin{proof}
        We can write our system as 
        \[
            \bm x' = \begin{pmatrix}
                a_{11} & a_{12} \\
                a_{21} & a_{22}
            \end{pmatrix} \bm x 
        \]
        Then we get eigenvalues given by 
        \begin{align*}
            (a_{11} - \lambda)(a_{22} - \lambda) - (a_{12})(a_{21}) &= \lambda^2 - (a_{11} + a_{22})\lambda + a_{11}a_{22} - a_{12}a_{21} \\
            &= \lambda^2 - p\lambda + q \\
            \Longrightarrow \lambda &= \frac{p \pm \sqrt{p^2 - 4q}}{2} \\
            &= \frac{p \pm \sqrt{\Delta}}{2}
        \end{align*}
        \fbox{Node}
        Suppose that $q > 0$ then by the theorem $q = r_1 r_2$ so it follows that either both $r_1,r_2 > 0$ or $r_1,r_2 < 0$.
        So it follows that if our eigenvalues are real that we will have either a source or sink node. Since $\Delta \geq 0$ 
        and our eigenvalues are given by $\lambda = (p \pm \sqrt{\Delta})/2$ we know that our eigenvalues are both real. So then 
        we have $r_1,r_2 > 0$ or $r_1, r_2 <0$ and so we are guaranteed to have a source or a sink node. \\\\
        \fbox{Saddle} Suppose that $q < 0$. Then the product of our eigenvalues is negative. Also our $\Delta = p^2 - 4q$ is guaranteed 
        to be positive so both our eigenvalues are real. So we have two real eigenvalues whose product is negative so it must be the case 
        that one is positive and one is negative. Therefore we must have a saddle point. \\\\
        \fbox{Spiral} Suppose that $\Delta < 0$, then we must have some complex part to both of our eigenvalues since they are given by $\lambda = \frac{p \pm \sqrt{\Delta}}{2}$.
        And since $p \neq 0$, it follows that our eigenvalues have both a real and a complex part, which means that we must have a spiral point either inwards 
        or outwards. \\\\
        \fbox{Center} If $p$ is equal to 0 then our eigenvalues are given by $\lambda = \pm \frac{\sqrt{\Delta}}{2}$. Then if we assume that $\Delta < 0$ 
        then it follows that our eigenvalues must have no real part and only distinct complex parts. So, therefore, we have a center point at the 
        given location.
    \end{proof}
\section*{Problem 4}
    \begin{mdframed}[]
        Continuing the previous problem, show that the critical point $(0,0)$ is 
        \begin{itemize}
            \item Asymptotically stable if $q > 0$ and $p < 0$ \\
            \item Stable if $q > 0$ and $p = 0$ \\
            \item Unstable if $q < 0$ or $p > 0$
        \end{itemize}
    \end{mdframed}
    \begin{proof}
        Recall that our eigenvalues are given by $\lambda = (p \pm \sqrt{\Delta})/2$. Now, \\\\
        \fbox{Asymptotically stable}
        Suppose that $q > 0$ and $p < 0$. Then we have $p / 2 < 0$ so if $\frac{\sqrt{\Delta}}{2}$ is a complex number then we have a spiral point 
        with a negative real part so it must spiral inward towards $(0,0)$. Then if our $\Delta > 0$ and we do not have a complex part, it follows that 
        since $q = r_1 r_2$ our eigenvalues are of the same sign. Since one of our eigenvalues is given by $\frac{p - \sqrt{\Delta}}{2}$, with $p < 0$ and 
        $\sqrt{\Delta} / 2$ being guaranteed to be postive we know that one of our eigenvalues must be negative. Then since both are the same size, both are negative 
        so we have a nodal sink point. Since both a spiral sink and a nodal sink are asymptotically stable, we must have an asymptotically stable point. \\\\
        \fbox{Stable}
        By the previous problem, the conditions $q > 0$ and $p = 0$ guarantee that our point is a center point, which means that we know it to be stable. \\\\
        \fbox{Unstable}
        Suppose that $q < 0$ or $p > 0$. If $p > 0$ then we have two cases, real and complex. In the complex case then we have a positive real part so it follows 
        that our point is a spiral source, and therefore unstable. In the real case, then we have at least one positive eigenvalue given by $r_1 = \frac{p + \sqrt{\Delta}}{2}$ 
        so it follows that we must have a saddle point or a nodal source, which are both unstable. If $q < 0$ then $r_1 r_2 < 0$. Then since $\Delta = p^2 - 4q$ it must be 
        the case that $\Delta \geq 0$ and our eigenvalues are real. This means 
        that one of our eigenvalues is positive and one is negative giving us a saddle point which is unstable. 
    \end{proof}
\end{document}