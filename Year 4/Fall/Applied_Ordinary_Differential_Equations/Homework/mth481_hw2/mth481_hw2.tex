\documentclass{article}

\usepackage{times}
\usepackage{amssymb, amsmath, amsthm}
\usepackage[margin=.5in]{geometry}
\usepackage{graphicx}
\usepackage[linewidth=1pt]{mdframed}

\usepackage{import}
\usepackage{xifthen}
\usepackage{pdfpages}
\usepackage{transparent}
\usepackage{bm}

\newcommand{\incfig}[1]{%
    \def\svgwidth{.5\linewidth}
    \import{./figures/}{#1.pdf_tex}
}

\newtheorem{theorem}{Theorem}[section]
\newtheorem{lemma}{Lemma}[section]
\newtheorem*{remark}{Remark}
\theoremstyle{definition}
\newtheorem{definition}{Definition}[section]

\begin{document}

\title{Applied Ordinary Differential Equations - Homework 2}
\author{Philip Warton}
\date{\today}
\maketitle
\section*{7.5.11}
    \begin{mdframed}[]
        Solve the given initial value problem. Describe the behavior of the solution as $t \rightarrow \infty$.
        \[
            \bm x' = 
            \begin{pmatrix}
                -2 & 1 \\
                -5 & 4
            \end{pmatrix}
            \bm x, \ \ \ \ 
            \bm x (0) =
            \begin{pmatrix}
                1 \\
                3
            \end{pmatrix}
        \]
    \end{mdframed}
    First off we want to find our eigenvalues. To do this, we just take the determinant of $A - \lambda I$, set it 
    equal to 0 and solve for $\lambda$. 
    \begin{align*}
        \det (A - \lambda I) &= \det \begin{pmatrix}
            -2 - \lambda & 1 \\
            -5 & 4 - \lambda
        \end{pmatrix} \\
        &= (-2 - \lambda)(4 - \lambda) - (1)(-5) \\
        &= \lambda^2 - 2\lambda - 8 + 5 \\
        &= \lambda^2 - 2\lambda - 3 \\
        &= (\lambda - 3)(\lambda + 1)
    \end{align*}
    So we conclude that $\lambda_1 = 3, \lambda_2 = -1$. So since we have one positive and one negative real eigenvalue,
    we know that we will have a saddle point style solution. To get the general solution, we'll solve for the eigenvectors.
    We write
    \begin{align*}
        A - \lambda_1 I &= \begin{pmatrix}
            -5 & 1 \\
            -5 & 1
        \end{pmatrix} \ \ \ \ \Longrightarrow \ \ \ \ \bm u = \begin{pmatrix}
            1 \\
            5
        \end{pmatrix} \\\\
        A - \lambda_2 I &= \begin{pmatrix}
            -1 & 1 \\
            -5 & 5
        \end{pmatrix} \ \ \ \ \Longrightarrow \ \ \ \ \bm v = \begin{pmatrix}
            1 \\
            1
        \end{pmatrix}
    \end{align*}
    This yields the general solution
    \[
        \bm x = c_1 
        \begin{pmatrix}
            1 \\
            5
        \end{pmatrix}
        e^{3t} + c_2 
        \begin{pmatrix}
            1 \\
            1
        \end{pmatrix} 
        e^{-t}
    \]
    Take the fact that $\bm x (0) = \begin{pmatrix}
        1 \\
        3
    \end{pmatrix}$ and it follows that $c_1 = c_2 = \frac{1}{2}$. So we have a solution
    \[
        \bm x = \frac{1}{2}
        \begin{pmatrix}
            1 \\
            5
        \end{pmatrix}
        e^{3t} + \frac{1}{2}
        \begin{pmatrix}
            1 \\
            1
        \end{pmatrix}
        e^{-t}
    \]
    This solution will asymptotically approach the span of $(1 \ \ 5)^T$ as $t \rightarrow \infty$.
    The general qualitative properties of the given solution can be seen in \fbox{Figure 1}.
    \begin{figure}[ht]
        \centering
        \incfig{1}
        \caption{Solution to the diffential equation given in \fbox{7.5.11}}
        \label{fig:1}
    \end{figure}
\section*{7.5.23}
    \begin{mdframed}[]
        Consider the system
        \[
            \bm x' =
            \begin{pmatrix}
                -1 & -1 \\
                -\alpha & -1
            \end{pmatrix}
            \bm x
        \]
    \end{mdframed}
    \subsection*{(a)}
        Solve the system for $\alpha = \frac{1}{2}$. Find the eigenvalues of the coefficient matrix, and classify the 
        type of equilibrium point at the origin.\\\\
        Let $\alpha = \frac{1}{2}$. Then we have a characteristic polynomial given by 
        \[
            \det (A - \lambda I) = \lambda^2 + 2 \lambda + 1 - \frac{1}{2} = \lambda^2 + 2 \lambda + \frac{1}{2}
        \]
        This gives us two eigenvalues of $\lambda_1 = -1 + \frac{\sqrt{2}}{2}, \lambda_2 = -1 - \frac{\sqrt{2}}{2}$. Since both eigenvalues 
        are negative, we say that the origin is an unstable `source' equilibrium point.
    \subsection*{(b)}
        Solve the system for $\alpha = 2$. Find the eigenvalues of the coefficient matrix, and classify the type of equilibrium point at the origin.\\\\
        Let $\alpha = 2$. Then we have a characteristic polynomial given by
        \[
            \det (A - \lambda I) = \lambda^2 + 2 \lambda + 1 - 2 = \lambda^2 + 2 \lambda - 1
        \]
        This gives us two eigenvalues of $\lambda_1 = -1 + \sqrt{2}, \lambda_2 = -1 - \sqrt{2}$.
        Since we have $\lambda_1 > 0$ and $\lambda_2 < 0$, we know that we have an unstable saddle point equilibrium.
    \subsection*{(c)}
    We can find the bifurcation point by writing
    \begin{align*}
        \det\begin{pmatrix}
            -1 & -1 \\
            - \alpha & -1
        \end{pmatrix} &= (-1 - \lambda)^2 - (-1)(-\alpha) \\
        &= \lambda^2 + 2\lambda + 1 - \alpha
    \end{align*}
    From here we can determine the eigenvalues as
    \begin{align*}
        \lambda &= \frac{-2 \pm \sqrt{2^2 - 4(1)(1-\alpha)}}{2} \\
        &= -1 \pm \frac{\sqrt{4 - 4 + 4\alpha}}{2} \\
        &= -1 \pm \sqrt{\alpha}
    \end{align*}
    The bifurcation point is exactly when one of our eigenvalues is equal to 0 because that is when one of them changes from 
    positive to negative as we vary $\alpha$. This occurs when $\alpha = 1$ which lies between $\frac{1}{2}$ and $2$ as desired.
\section*{7.6.5}
    \begin{mdframed}[]
        Compute the general solution in terms of real-valued functions of the following system of differential equations,
        \[
            x' = \begin{pmatrix}
                1 & 0 & 0 \\
                2 & 1 & -1 \\
                3 & 2 & 1
            \end{pmatrix}x
        \]
    \end{mdframed}
    We write 
    \begin{align*}
        \det\begin{pmatrix}
            1 & 0 & 0 \\
                2 & 1 & -1 \\
                3 & 2 & 1
        \end{pmatrix} &= (1 - \lambda)\left((1 - \lambda)(1 - \lambda) - (2)(-2)\right) \\
        &= (1 - \lambda)(\lambda^2 + 2\lambda + 1 + 4)
    \end{align*}
    This gives us eigenvalues of $\lambda_1 = 1, \lambda_2 = 1 - 2i, \lambda_3 = 1 + 2i$. From 
    these values we can determine that the eigenvectors are given by
    \[
        \begin{pmatrix}
            2 \\
            3 \\
            2
        \end{pmatrix}, \ \ \ \ \begin{pmatrix}
            0 \\
            -i \\
            1
        \end{pmatrix} \ \ \ \ \begin{pmatrix}
            0 \\
            i \\
            1
        \end{pmatrix}
    \]
    Then we can write the general solution,
    \[
        \bm x = c_1 \begin{pmatrix}
            2\\
            3\\
            2
        \end{pmatrix}e^t + c_2 \bm u + c_3 \bm v
    \]
    Where 
    \begin{align*}
        \bm u (t) &= e^t \left(
            \begin{pmatrix}
                0\\
                0 \\
                1
            \end{pmatrix}\cos(-2t) - \begin{pmatrix}
                0 \\
                -1 \\
                0
            \end{pmatrix}\sin(-2t)
        \right)\\
        \bm v(t) &= e^t \left(
            \begin{pmatrix}
                0\\
                0 \\
                1
            \end{pmatrix}\sin(-2t) + \begin{pmatrix}
                0 \\
                -1 \\
                0
            \end{pmatrix}\cos(-2t)
        \right)
    \end{align*}
\section*{7.6.11}
    \begin{mdframed}[]
        Determine eigenvalues in terms of $\alpha$, and then find and sketch the bevior around the bifurcation 
        point of this parameter.
        \[
            \bm x' = \begin{pmatrix}
                \alpha & 1 \\
                -1 & \alpha
            \end{pmatrix}\bm x
        \]
    \end{mdframed}
    We write 
    \begin{align*}
        \det \begin{pmatrix}
            \alpha & 1 \\
            -1 & \alpha
        \end{pmatrix} - \lambda \bm I &= 0 \\
        (\alpha - \lambda)^2 + (1)(1) &= 0 \\
        \lambda^2 - 2\alpha \lambda + \alpha^2 + 1 &= 0 \\
        \lambda^2 - 2\alpha \lambda + \alpha^2 &= 1 \\
        (\lambda - \alpha)^2 &= -1
    \end{align*}
    This will give us two eigenvalues given by 
    \begin{align*}
        \lambda &= \alpha \pm i
    \end{align*}
    The real part of these eigenvalues is positive if and only if $\alpha$ is also positive,
    giving us a bifurcation point at $\alpha = 0$. Our solutions are stable when $\alpha < 0$,
    and unstable when $\alpha > 0$. When $\alpha < 0$ our solutions will spiral toward the origin,
    and when $\alpha > 0$ they will spiral away. This is pictured in our sketches in \fbox{Figure 2}
    \begin{figure}[ht]
        \centering
        \incfig{2}
        \caption{Sketch of phase portraits for given values of $\alpha$}
        \label{fig:1}
    \end{figure}
\section*{7.7.12}
    \begin{mdframed}[]
        Show that $\Phi(t)$ does have the principle algebraic properties associated with the exponential.
    \end{mdframed}
    \begin{proof}
        We first wanna show that $\Phi(t)\Phi(s) = \Phi(t + s)$. Without loss of generality let $s$ be fixed and 
        $t$ be a variable, then we know that $\Phi(s)$ is some fixed matrix. Now let $Z(t) = \Phi(t)\Phi(s)$. Then 
        we can say that 
        \[
            Z(0) = \Phi(0)\Phi(s) = \bm I \Phi(s) = \Phi(s)
        \]
        Now let $Y(t) = \Phi(t + s)$. Then we similarly get
        \[
            Y(0) = \Phi(0 + s) = \Phi(s)
        \]
        Given that we can fix any $s$ and let $t$ vary, it follows that this will hold for any initial value
        given this system of equations. So we conclude that $\Phi(x)\Phi(y) = \Phi(x + y)$.\\\\
        Now we look at the property stating that $\Phi(t)\Phi(-t) = \bm I$. Now this follows somewhat trivially from 
        our previous result. That is,
        \[
            \Phi(t)\Phi(-t) = \Phi(t - t) = \Phi(0) = \bm I  
        \]
        So for any value $t$ we have a matrix that, right-multiplied with $\Phi(t)$, gives us the identity. This gives us 
        the following result 
        \begin{align*}
            \Phi(t)\Phi(-t) &= \bm I \\
            \Phi^{-1}(t)\Phi(t)\Phi(-t) &= \Phi^{-1}(t) \bm I \\
            \Phi(-t) &= \Phi^{-1}(t)
        \end{align*}
        And then finally, it follows obviously that 
        \begin{align*}
            \Phi(t - s) &= \Phi(t + (-s)) \\
            &= \Phi(t)\Phi(-s) \\
            &= \Phi(t)\Phi^{-1}(s)
        \end{align*}
    \end{proof}
\end{document}