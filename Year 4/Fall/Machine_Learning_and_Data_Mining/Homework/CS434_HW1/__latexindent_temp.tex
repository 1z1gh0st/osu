\documentclass{article}

\usepackage{times}
\usepackage{amssymb, amsmath, amsthm}
\usepackage[margin=.5in]{geometry}
\usepackage{graphicx}
\usepackage[linewidth=1pt]{mdframed}

\usepackage{import}
\usepackage{xifthen}
\usepackage{pdfpages}
\usepackage{transparent}

\newcommand{\incfig}[1]{%
    \def\svgwidth{\columnwidth}
    \import{./figures/}{#1.pdf_tex}
}

\newtheorem{theorem}{Theorem}[section]
\newtheorem{lemma}{Lemma}[section]
\newtheorem*{remark}{Remark}
\theoremstyle{definition}
\newtheorem{definition}{Definition}[section]

\begin{document}

\title{Machine Learning and Data Mining - Homework 1}
\author{Philip Warton}
\date{\today}
\maketitle
\section{Something}
    \subsection{Q1}
        \begin{mdframed}[]
            Let $D = \{x_1,\cdots,x_N\}$ be a dataset from $N$ poisson random variables, with a rate 
            of $\lambda \in \mathbb{R}$. Derive the Maximum Likelihood Estimation for $\lambda$.
        \end{mdframed}
        We begin by taking $\mathcal{L}(D)$. That is,
        \begin{align*}
            \mathcal{L}(D) &= P(D \ | \ \lambda) \\
            &=P(\{x_1,\cdots,x_N\} \ | \ \lambda) \\
            &=P(x_1 \ | \ \lambda )
        \end{align*}
\end{document}