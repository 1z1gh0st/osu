\documentclass{article}

\usepackage{times}
\usepackage{amssymb, amsmath, amsthm}
\usepackage[margin=.5in]{geometry}
\usepackage{graphicx}
\usepackage[linewidth=1pt]{mdframed}

\usepackage{import}
\usepackage{xifthen}
\usepackage{pdfpages}
\usepackage{transparent}

\usepackage{tikz-cd}

\newcommand{\incfig}[1]{%
    \def\svgwidth{\columnwidth}
    \import{./figures/}{#1.pdf_tex}
}

\newtheorem{theorem}{Theorem}[section]
\newtheorem{lemma}{Lemma}[section]
\newtheorem*{remark}{Remark}
\theoremstyle{definition}
\newtheorem{definition}{Definition}[section]

\begin{document}

\title{Algebraic Topology - Homework 1}
\author{Philip Warton}
\date{\today}
\maketitle
\section*{Problem 0.3}
    \begin{mdframed}[]
        Assume, for $n \geq 1$, that $H_i(S^n) = \mathbb{Z}$ if $i = 0, n,$ and that $H_i(S^n) = 0$ otherwise.
        Using the technique of the proof of Lemma 0.2, prove that the equator of the $n$-sphere is not a retract.
    \end{mdframed}
    \begin{proof}
        Assume that $S^{n-1}$ is the equator of the $n$-sphere. Suppose by contradiction that $S^{n-1}$ is a 
        retract of $S^n$. Then it follows that there exists some retraction $r:S^n \rightarrow S^{n-1}$.
        Then, with $i:S^{n-1} \rightarrow S^n$ being the inclusion map, and with $1$ being, of course, the 
        identity map, it follows that we would have a commutative diagram:
        \begin{center}
            \begin{tikzcd}[]
                S^{n-1} \ar[rr, "1"] \ar[rdd, "i"] && S^{n-1} \\\\
                & S^n \ar[ruu, "r"]
            \end{tikzcd}
        \end{center}
        To this diagram, we can apply our homology functor, giving us
        \begin{center}
            \begin{tikzcd}[]
                H_{n-1}(S^{n-1}) \ar[rr, "H_{n-1}(1)"] \ar[rddd, "H_{n-1}(i)"] && H_{n-1}(S^{n-1}) \\\\\\
                & H_{n-1}(S^n) \ar[ruuu, "H_{n-1}(r)"]
            \end{tikzcd}
        \end{center}
        We know by assumption that $H_{n-1}(S^n) = 0$ since $n-1 \neq n$ and that $H_n(S^{n-1}) = \mathbb{Z}$ since
        $n - 1 = n - 1$. This new diagram should continue to commute by the properties of our functor $H_{n-1}$.
        Since $H_{n-1}(S^n) = 0$, it follows that its image under $H_{n-1}(r)$ must also be zero. That is, $H_{n-1}(S^{n-1}) =0$.
        However, this means that our identity map 1 takes a countable algebra to a trivial one, which is a contradiction.
        therefore $S^{n-1}$ cannot be a retract of $S^n$.
    \end{proof}
\section*{Problem 0.5}
\section*{Problem 0.7}
    \begin{mdframed}[]
        Let $f \in \text{Hom}(A,B)$, and let $g,h \in \text{Hom}(B,A)$ such that $g \circ f = 1_A$ and that 
    $f \circ h = 1_B$. Then $g = h$.
    \end{mdframed}
    \begin{proof}
        Suppose that $h \neq g$,
        \begin{align*}
            h & \neq g \\
            h \circ f & \neq g \circ f \\
            f \circ h \circ f & \neq f \circ g \circ f \\
            (f \circ h) \circ f & \neq f \circ (g \circ f)\\
            1_B \circ f & \neq f \circ 1_A \\
            f & \neq f
        \end{align*}
        However, this is a contradiction, and we conclude that $h = g$.
    \end{proof}
\section*{Problem 0.18}
    \begin{mdframed}[]
        For an abelian group $G$, let 
        \[
            tG = \{x \in G \ : \ \text{x has finite order}\}
        \]
        denote its torsion subgroup.
    \end{mdframed}
    \subsection*{(ii)}
        \begin{mdframed}[]
            
        \end{mdframed}

\end{document}