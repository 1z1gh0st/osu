\documentclass{article}

\usepackage{times}
\usepackage{amssymb, amsmath, amsthm}
\usepackage[margin=.5in]{geometry}
\usepackage{graphicx}
\usepackage[linewidth=1pt]{mdframed}

\usepackage{import}
\usepackage{xifthen}
\usepackage{pdfpages}
\usepackage{transparent}

\newcommand{\incfig}[1]{%
    \def\svgwidth{\columnwidth}
    \import{./figures/}{#1.pdf_tex}
}

\newtheorem{theorem}{Theorem}[section]
\newtheorem{lemma}{Lemma}[section]
\newtheorem*{remark}{Remark}
\theoremstyle{definition}
\newtheorem{definition}{Definition}[section]

\begin{document}

\title{Algebraic Topology - Homework 2}
\author{Philip Warton}
\date{\today}
\maketitle
\section*{1.3}
    \begin{mdframed}[]
        Let $R : S^1 \rightarrow S^1$ rotate any given point by $x$ radians, then $R \simeq 1_{S^1}$.
    \end{mdframed}
    \begin{proof}
        We define the following function $F:S^1 \times X \rightarrow S^1$ and claim that it is a homotopy:
        \[
            F(p,t) = p \cdot e^{i(tx)}
        \]
        Firstly, we can trivially verify that
        \begin{align*}
            F(p,0) &= p \cdot e^0 = p \cdot 1 = p = 1_{S^1}(p) \\
            F(p,1) &= p \cdot e^{ix} = R(p)
        \end{align*}
        Since this function simply rotates the point over to $x$ radians as we vary $t$, it follows that it is 
        continuous and thus a homotopy between $R$ and $1_{S^1}$.
    \end{proof}
    \begin{mdframed}[]
        Every continuous map $f: S^1 \rightarrow S^1$ is homotopic to a continuous map $g:S^1 \rightarrow S^1$
        with $g(1) = 1$.
    \end{mdframed}
    \begin{proof}
        Let $f:S^1 \rightarrow S^1$ be continuous. Then $f(1) \in S^1$ with some corrosponding argument/angle
        $x \in [0,2\pi)$. Let $R_{\alpha} : S^1 \rightarrow S^1$ denote the function given earlier as $R$ with 
        the rotation being given in radians by $\alpha$. Then,
        \[
            (R_{-x} \circ f)(1) = R_{-x}(f(1)) = R_{-x}(e^{ix}) = e^{i(0)} = 1
        \]
        Let $g = R_{-x} \circ f$ and it follows that since $R \simeq 1_{S^1}$,
        \begin{align*}
            R_{-x} \circ f &\simeq 1_{S^1} \circ f \\
            g &\simeq f
        \end{align*}
        where $g(1) = 1$.
    \end{proof}
\section*{1.5}
    \begin{mdframed}[]
        Let $X = \{0\} \cup \{1, \frac{1}{2}, \frac{1}{3}, \cdots, \frac{1}{n}, \cdots \}$ and let $Y$ be a
        countable discrete space. Then $X$ and $Y$ do not have the same homotopy type.
    \end{mdframed}
    \begin{proof}
        Let $f: X \rightarrow Y$ be some continuous function. Then there exists some $y \in Y$ such that 
        $0 \in f^{-1}(y)$ (since $0$ must of course get mapped to some point in $Y$). Since $Y$ is a discrete 
        space it follows that $\{y\}$ is an open set. Since $f$ is continuous, $f^{-1}(\{y\})$ is an open set in 
        $X$ containing 0. Assuming that $X$ is equipped with the subspace topology from $\mathbb{R}$ it follows that 
        for any open neighborhood $U$ of 0 the followsing is true:\\\\
        There exists some $N \in \mathbb{N}$ such that for every $n \geq N$, $\frac{1}{n} \in U$.\\\\
        By this, it follows that $f^{-1}(\{y\})$ contains infinitely many points from $X$, and so it must be 
        the case that only finitely many points in $X$ are mapped to points other than $y$. Let $g:Y\rightarrow X$
        be continuous. Then we conclude that since $f(X)$ is a finite set, so too is $(g \circ f)(X)$.
        Somehow XD we conclude that $1_X \not\simeq g \circ f$ for any $f,g$ arbitrarily, and thus the two 
        spaces have two different homotopy types.
    \end{proof}
\section*{1.7}
    \begin{mdframed}[]
        Let $X = \{x,y\}$ with topology $\{X, \emptyset, \{x\}\}$, then $X$ is contractible.
    \end{mdframed}
    \begin{proof}
        Define a function $F:X \times [0,1] \rightarrow X$ by 
        \[
            F(p,t) = 
            \begin{cases}
                p, & \text{ when } t \leq \frac{1}{2}\\
                x, & \text{ when } t > \frac{1}{2}
            \end{cases}
        \]
        We can verify immediately that 
        \begin{align*}
            F(p,0) &= p = 1_X(p)\\
            F(p,1) &= x = e_x(p) & \text{(where $e_x$ is the constant map to the point $x$)}
        \end{align*}
        Let us verify that each pre-image of a neighborhood in $X$ is open in $X$.
        Firstly $F^{-1}(X) = X \times [0,1]$ since the function is well defined and surjective.
        Then we know that $F^{-1}(\{x\}) = (\{x\} \times [0,1]) \cup (\{y\} \times (\frac{1}{2}, 1])$ which is open 
        in $X \times [0,1]$. And finally $F^{-1}(\emptyset) = \emptyset$ since the function is well defined.
        So we conlcude that $F$ is a homotopy and therefore $1_X \simeq e_x$ and therefore $X$ is contractible.
    \end{proof}
\section*{1.8}
    \begin{mdframed}[]
        There exists a continuous image of a contractible space that is not contractible.
    \end{mdframed}
    \begin{proof}
        Let $f:[0,1] \rightarrow S^1$ be given by $f(x) = e^{i(2\pi)x}$. The space $[0,1]$ is contractible
        trivially, and we claim that $f([0,1]) = S^1$ and is therefore not contractible. Let $O$ be an open
        set in $S^1$. Then it is a union of some open intervals along the circle. The pre-image of each interval
        that does not include $1$ will be of the form $(a,b)$ which is clearly open. Otherwise it will be of the 
        form $[0,a) \cup (b,1]$ and will remain open. Thus $f$ is continuous. Let $y \in S^1$ and then it can be written 
        as $y = e^{i t}$ where $t \in [0,2\pi)$. Then it follows that it has some pre-image under $f$ so the 
        function is surjective and $f([0,1]) = S^1$. We assume without proof that $S^1$ is contractible.
    \end{proof}
    \begin{mdframed}[]
        A retract of a contractible space is contractible
    \end{mdframed}
\section*{4.}
\section*{4.}
\end{document}