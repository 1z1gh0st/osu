\documentclass{article}

\usepackage{times}
\usepackage{amssymb, amsmath, amsthm}
\usepackage[margin=.5in]{geometry}
\usepackage{graphicx}
\usepackage[linewidth=1pt]{mdframed}

\usepackage{import}
\usepackage{xifthen}
\usepackage{pdfpages}
\usepackage{transparent}
\usepackage{bm}

\newcommand{\incfig}[1]{%
    \def\svgwidth{.5\linewidth}
    \import{./figures/}{#1.pdf_tex}
}

\newtheorem{theorem}{Theorem}[section]
\newtheorem{lemma}{Lemma}[section]
\newtheorem*{remark}{Remark}
\theoremstyle{definition}
\newtheorem{definition}{Definition}[section]

\begin{document}

\title{Applied Ordinary Differential Equations - Homework 2}
\author{Philip Warton}
\date{\today}
\maketitle
\section*{7.5.11}
    \begin{mdframed}[]
        Solve the given initial value problem. Describe the behavior of the solution as $t \rightarrow \infty$.
        \[
            \bm x' = 
            \begin{pmatrix}
                -2 & 1 \\
                -5 & 4
            \end{pmatrix}
            \bm x, \ \ \ \ 
            \bm x (0) =
            \begin{pmatrix}
                1 \\
                3
            \end{pmatrix}
        \]
    \end{mdframed}
    First off we want to find our eigenvalues. To do this, we just take the determinant of $A - \lambda I$, set it 
    equal to 0 and solve for $\lambda$. 
    \begin{align*}
        \det (A - \lambda I) &= \det \begin{pmatrix}
            -2 - \lambda & 1 \\
            -5 & 4 - \lambda
        \end{pmatrix} \\
        &= (-2 - \lambda)(4 - \lambda) - (1)(-5) \\
        &= \lambda^2 - 2\lambda - 8 + 5 \\
        &= \lambda^2 - 2\lambda - 3 \\
        &= (\lambda - 3)(\lambda + 1)
    \end{align*}
    So we conclude that $\lambda_1 = 3, \lambda_2 = -1$. So since we have one positive and one negative real eigenvalue,
    we know that we will have a saddle point style solution. To get the general solution, we'll solve for the eigenvectors.
    We write
    \begin{align*}
        A - \lambda_1 I &= \begin{pmatrix}
            -5 & 1 \\
            -5 & 1
        \end{pmatrix} \ \ \ \ \Longrightarrow \ \ \ \ \bm u = \begin{pmatrix}
            1 \\
            5
        \end{pmatrix} \\\\
        A - \lambda_2 I &= \begin{pmatrix}
            -1 & 1 \\
            -5 & 5
        \end{pmatrix} \ \ \ \ \Longrightarrow \ \ \ \ \bm v = \begin{pmatrix}
            1 \\
            1
        \end{pmatrix}
    \end{align*}
    This yields the general solution
    \[
        \bm x = c_1 
        \begin{pmatrix}
            1 \\
            5
        \end{pmatrix}
        e^{3t} + c_2 
        \begin{pmatrix}
            1 \\
            1
        \end{pmatrix} 
        e^{-t}
    \]
    Take the fact that $\bm x (0) = \begin{pmatrix}
        1 \\
        3
    \end{pmatrix}$ and it follows that $c_1 = c_2 = \frac{1}{2}$. So we have a solution
    \[
        \bm x = \frac{1}{2}
        \begin{pmatrix}
            1 \\
            5
        \end{pmatrix}
        e^{3t} + \frac{1}{2}
        \begin{pmatrix}
            1 \\
            1
        \end{pmatrix}
        e^{-t}
    \]
    This solution will asymptotically approach the span of $(1 \ \ 5)^T$ as $t \rightarrow \infty$.
    The general qualitative properties of the given solution can be seen in \fbox{Figure 1}.
    \begin{figure}[ht]
        \centering
        \incfig{1}
        \caption{Solution to the diffential equation given in \fbox{7.5.11}}
        \label{fig:1}
    \end{figure}
\section*{Problem 7.5.23}
    \begin{mdframed}[]
        Consider the system
        \[
            \bm x' =
            \begin{pmatrix}
                -1 & -1 \\
                -\alpha & -1
            \end{pmatrix}
            \bm x
        \]
    \end{mdframed}
    \subsection*{(a)}
        Solve the system for $\alpha = \frac{1}{2}$. Find the eigenvalues of the coefficient matrix, and classify the 
        type of equilibrium point at the origin.\\\\
        Let $\alpha = \frac{1}{2}$. Then we have a characteristic polynomial given by 
        \[
            \det (A - \lambda I) = \lambda^2 + 2 \lambda + 1 - \frac{1}{2} = \lambda^2 + 2 \lambda + \frac{1}{2}
        \]
        This gives us two eigenvalues of $\lambda_1 = -1 + \frac{\sqrt{2}}{2}, \lambda_2 = -1 - \frac{\sqrt{2}}{2}$. Since both eigenvalues 
        are negative, we say that the origin is an unstable `source' equilibrium point.
    \subsection*{(b)}
        Solve the system for $\alpha = 2$. Find the eigenvalues of the coefficient matrix, and classify the type of equilibrium point at the origin.\\\\
        
\end{document}