\documentclass{article}

\usepackage{times}
\usepackage{amssymb, amsmath, amsthm}
\usepackage[margin=.5in]{geometry}
\usepackage{graphicx}
\usepackage[linewidth=1pt]{mdframed}

\usepackage{import}
\usepackage{xifthen}
\usepackage{pdfpages}
\usepackage{transparent}
\usepackage{bm}

\newcommand{\incfig}[1]{%
    \def\svgwidth{\columnwidth}
    \import{./figures/}{#1.pdf_tex}
}

\newtheorem{theorem}{Theorem}[section]
\newtheorem{lemma}{Lemma}[section]
\newtheorem*{remark}{Remark}
\theoremstyle{definition}
\newtheorem{definition}{Definition}[section]

\begin{document}

\title{Applied Ordinary Differential Equations - Homework 1}
\author{Philip Warton}
\date{\today}
\maketitle
\section*{7.5.3}
    \subsection*{(b)}
        \begin{mdframed}[]
            Find the general solution of the given system of equations 
            and describe the behavior of the solution as $t \rightarrow \infty$.
            \[
                \bm x' = \begin{pmatrix}
                    2 & -1 \\ 3 & -2
                \end{pmatrix}  \bm x
            \]
        \end{mdframed}
        We begin by writing the characteristic polynomial,
        \[
            (2 - \lambda)(-2 - \lambda) - (-1)(3) = \lambda^2
        \]
        So we have two real and distinct eigenvalues $\lambda_1 = -1, \lambda_2 = 1$.
        We can write 
        \[
            [A - \lambda_1(I)]\bm u = \begin{pmatrix}
                3 & -1 \\ 3 & -1
            \end{pmatrix} \cdot \begin{pmatrix}
                u_1 \\ u_2
            \end{pmatrix} = \begin{pmatrix}
                3u_1 -u_2 \\ 3u_1 - u_2
            \end{pmatrix} = \bm 0
        \]
        This gives us an eigenvector of $\bm u = (1,\ 3)^T$. Then we write 
        \[
            [A - \lambda_2(I)]\bm v = \begin{pmatrix}
                1 & -1 \\ 3 & -3
            \end{pmatrix} \cdot \begin{pmatrix}
                v_1 \\ v_2
            \end{pmatrix} = \begin{pmatrix}
                v_1 - v_2\\3v_1 - 3v_2
            \end{pmatrix} = \bm 0
        \]
        And then we have an eigenvector $\bm v = (1, \ \ -1)^T$.
        This leads to a general solution,
        \[
            \bm x = c_1 \begin{pmatrix}
                1 \\ 3
            \end{pmatrix}e^{-t} + c_2 \begin{pmatrix}
                1 \\ -1
            \end{pmatrix}e^{t}
        \]
\section*{7.5.7}
    \begin{mdframed}[]
        Find the general solution  of the given system of equations
        \[
            \bm x' = \begin{pmatrix}
                1 & 1 & 2 \\
                1 & 2 & 1 \\
                2 & 1 & 1
            \end{pmatrix} \bm x
        \]
    \end{mdframed}
    We begin by solving for eigenvalues,
    \begin{align*}
        \det\begin{pmatrix}
            1 - \lambda & 1 & 2 \\
            1 & 2 - \lambda & 1 \\
            2 & 1 & 1 - \lambda
        \end{pmatrix} & = (1 - \lambda)[(2 - \lambda)(1 - \lambda) - 1] - (1)[(1)(1 - \lambda) - (1)(2)] + (2)[(1)(1) - (2)(2 - \lambda)] \\
        & = -\lambda^3 + 4\lambda^2 + \lambda - 4
    \end{align*}
    This gives us three eigenvalues $\lambda_1 = -1, \lambda_2 = 1, \lambda_3 = 4$.
    After performing row reduction for the respective kernels of $A - \lambda I$.
    This yields the following eigenvectors,
    \[
        \bm u = \begin{pmatrix}
            1 \\ 0 \\ -1
        \end{pmatrix}, \ \ \bm v = \begin{pmatrix}
            1 \\ -2 \\ 1
        \end{pmatrix}, \ \ \bm w = \begin{pmatrix}
            1 \\ 1 \\ 1
        \end{pmatrix}
    \]
\section*{7.5.20}
    Consider a $2 \times 2$ system $\bm x ' = \bm{Ax}$. If we assume
    that $r_1 \neq r_2$, the general solution is $\bm x = c_1 \xi^{(1)}e^{r_1t} + c_2 \xi^{(2)}e^{r_2t}$,
    provided that $\xi^{(1)}$ and $\xi^{(2)}$ by assuming that they are
    linearly dependent and then showing that this leads to a contradiction.
    \subsection*{(a)}
        \begin{mdframed}[]
            Explain how we know that $\xi^{(1)}$ satisfies the matrix 
            equation $(A - r_1 I)\xi^{(1)} = \bm 0$. Similarly, explain why 
            $(A - r_2I)\xi^{(2)} = \bm 0$.
        \end{mdframed}
        We know that $r_i$ is the eigenvalue corresponding to the eigenvector $\xi^{(i)}$.
        We also know that $A \xi^{(i)} = r_i \xi^{(i)}$. So then we expand our term to 
        \[
            (A - r_i I)\xi^{(i)} = A\xi^{(i)} - r_i I\xi^{(i)} = r_i \xi^{(i)} - r_i \xi^{(i)} = \bm 0
        \]
    \subsection*{(b)}
        \begin{mdframed}[]
            Show that $(A - r_2I)\xi^{(1)} = (r_1 - r_2)\xi^{(1)}$.
        \end{mdframed}
        We start by expanding the LHS term to give us 
        \begin{align*}
            (A - r_2I) \xi^{(1)} &= A \xi^{(1)} - r_2I\xi^{(1)} \\
            &= r_1 \xi^{(1)} - r_2 \xi^{(1)}\\
            &=(r_1 - r_2) \xi^{(1)}
        \end{align*}
    \subsection*{(c)}
        \begin{mdframed}[]
            Suppose that $\xi^{(1)}$ and $\xi^{(2)}$ are linearly dependent.
            Then $c_1 \xi^{(1)} + c_2 \xi^{(2)} = \bm 0$ and at least one 
            of $c_1$ and $c_2$ (say, $c_1$) is not zero. Show that 
            $(A - r_2I)(c_1 \xi^{(1)} + c_2 \xi^{(2)}) = \bm 0$, and also 
            show that $(A - r_2I)(c_1 \xi^{(1)} + c_2\xi^{(2)}) = c_1(r_1 - r_2)\xi^{(1)}$.
            Hence $c_1 = 0$, which is a contradiction. Therefore, $\xi^{(1)}$ and $\xi^{(2)}$ are 
            linearly independent.
        \end{mdframed}
        We can expand a certain term as follows:
        \begin{align*}
            (A - r_2I)(c_1 \xi^{(1)} + c_2\xi^{(2)}) &= Ac_1\xi^{(1)} + Ac_2\xi^{(2)} -r_2 I c_1 \xi^{(1)} - r_2 I c_2 \xi^{(2)} \\
            &=c_1 A\xi^{(1)} + c_2A\xi^{(2)} - r_2 c_1 I \xi^{(1)} - r_2 c_2 I \xi^{(2)} \\
            &= c_1 r_1 \xi^{(1)} + c_2 r_2 \xi^{(2)} - c_1 r_2 \xi^{(1)} - c_2 r_2 \xi^{(2)} \\
            &= c_1 r_1 \xi^{(1)}  - c_1 r_2 \xi^{(1)} + c_2 r_2 \xi^{(2)} - c_2 r_2 \xi^{(2)} \\
            &= c_1 r_1 \xi^{(1)}  - c_1 r_2 \xi^{(1)} \\
            &= c_1\xi^{(1)}(r_1 - r_2)
        \end{align*}
        However, by assumption, $c_1 \xi^{(1)} + c_2 \xi^{(2)} = \bm 0$, so we know that
        \[
            (A - r_2I)(c_1 \xi^{(1)} + c_2 \xi^{(2)}) = (A - r_2I)(\bm 0) = \bm 0
        \]
        Therefore we know that $\bm 0 = c_1 \xi^{(1)}(r_1 - r_2)$, and we know that $r_1 - r_2$ is non-zero, so it follows 
        that $c_1$ must 0. This is a contradiction to our assumption, so we conclude that $\xi^{(1)}$ and $\xi^{(2)}$ are
        linearly independent.
    \subsection*{(d)}
        \begin{mdframed}[]
            Modify the argument of part (c) if we assume that $c_2 \neq 0$.
        \end{mdframed}
        The argument holds still.
    \subsection*{(e)}
        \begin{mdframed}[]
            Carry out a similar argument for the case where $A$ is $3 \times 3$.
        \end{mdframed}
        \begin{proof}
            Assume that $\xi^{(1)}, \xi^{(2)}, \xi^{(3)}$ are dependent. Then for any $i,j \in \{1,2,3\}$ we can write the following:
            Firstly, simlarly to part (a),
            \[
                (A - r_iI)\xi^{(i)} = r_i \xi^{(i)} - r_i \xi^{(1)} = 0
            \]
            Secondly, similarly to part (b),
            \[
                (A - r_iI)\xi^{(j)} = r_j\xi^{(j)} - r_i\xi^{(j)}  = (r_j - r_i)\xi^{(j)}
            \]
            Finally, assume there exists some non-trivial solution to $c_1\xi^{(1)} + c_2\xi^{(2)} + c_3\xi^{(3)} = 0$.
            Without loss of generality, suppose that $c_1 \neq 0$. Now we write that 
            \[
                (A - r_3I)(A - r_2I)(c_1\xi^{(1)} + c_2\xi^{(2)} + c_3\xi^{(3)}) = 0
            \]
            And this is true trivially because $c_1\xi^{(1)} + c_2\xi^{(2)} + c_3\xi^{(3)} = 0$ by assumption. However,
            now we can also write 
            \begin{align*}
                &(A - r_3I)(A - r_2I)(c_1\xi^{(1)} + c_2\xi^{(2)} + c_3\xi^{(3)}) \\\\
                &= (A - r_3I)(c_1 r_1 \xi^{(1)} + c_2r_2\xi^{(2)} + c_3r_3\xi^{(3)} - c_1r_2\xi^{(1)} - c_2r_2\xi^{(2)} - c_3r_2\xi^{(3)}) \\\\
                &=(A - r_3I)(c_1(r_1 - r_2)\xi^{(1)} + c_3(r_3 - r_2)\xi^{(3)}) \\\\
                &=[c_1 r_1 (r_1 - r_2)\xi^{(1)} + c_3r_3(r_3 - r_2)\xi^{(3)}]\\
                & \ -[c_1 r_3 (r_1 - r_2)\xi^{(1)} + c_3 r_3 (r_3 - r_2)\xi^{(3)}] \\\\
                &=c_1 r_1 (r_1 - r_2)\xi^{(1)} - c_1 r_3 (r_1 - r_2)\xi^{(1)} \\\\
                &=c_1(r_1 - r_3)(r_1 - r_2)\xi^{(1)}
            \end{align*}
            So from this and from our previous assertion that the same term was equal to 0, we can write,
            \[
                c_1(r_1 - r_3)(r_1 - r_2)\xi^{(1)} = 0 \Longrightarrow c_1 = 0
            \]
            This is true because we assume our eigenvalues to be distinct. However this clearly leads to the same contradiction
            as within our $2 \times 2$ case. So we conclude that our eigenvectors must be linearly independent.
        \end{proof}
\section*{7.5.21}
    \subsection*{(a)}
        Take the ordinary differntial equation 
        \[
            ay'' + by' + cy = 0
        \]
        Let $x_1 = y, x_2 = y'$.
        Then,
        \begin{align*}
            ax_2' + bx_2 + cx_1 &= 0\\
            x_2' &= -(bx_2 + cx_1)/a \\
            x_2' &= -\frac{b}{a}x_2 - \frac{c}{a}x_1
        \end{align*}
        Also,
        \begin{align*}
            ax_2' + bx_1' + cx_1 &= 0 \\
            x_1' &= -(ax_2' + cx_1)/b \\
            &= -(a(\frac{-bx_2}{a} - \frac{cx_1}{a}) + cx_1) / b \\
            &= -(-bx_2 - cx_1 + cx_1)/b \\
            &= bx_2 / b \\
            &= x_2
        \end{align*}
        So we can now write this differential equation as a system of first order ODE's of the form $\bm x' = A \bm x$. We can write 
        \[
            \bm x' = \begin{pmatrix}
                0 & 1 \\ -c/a & -b/a
            \end{pmatrix} \bm x
        \]
    \subsection*{(b)}
        To find the roots of the characteristic polynomial we want to compute $\det(A - \lambda I)=0$. That is,
        \begin{align*}
            \det \begin{pmatrix}
                0 - \lambda & 1 \\
                -c/a & -b/a - \lambda
            \end{pmatrix} &= 0\\
            (-\lambda)(-b/a - \lambda) - (1)(-c/a) &= 0\\
            \lambda(b/a) + \lambda^2 + c/a &= 0 \\\\
            a\lambda^2 + b\lambda + c &= 0
        \end{align*}
        This equation is the same as $ar^2 + br + c = 0$, the characteristic polynomial for our second order system 
        of ordinary differential equations.
\end{document}