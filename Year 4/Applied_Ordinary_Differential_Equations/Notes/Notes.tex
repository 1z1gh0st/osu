\documentclass{article}

\usepackage{times}
\usepackage{amssymb, amsmath, amsthm}
\usepackage[margin=.5in]{geometry}
\usepackage{graphicx}
\usepackage[linewidth=1pt]{mdframed}

\usepackage{import}
\usepackage{xifthen}
\usepackage{pdfpages}
\usepackage{transparent}

\newcommand{\incfig}[1]{%
    \def\svgwidth{\columnwidth}
    \import{./figures/}{#1.pdf_tex}
}

\newtheorem{theorem}{Theorem}[section]
\newtheorem{lemma}{Lemma}[section]
\newtheorem*{remark}{Remark}
\theoremstyle{definition}
\newtheorem{definition}{Definition}[section]

\begin{document}

\title{Applied Ordinary Differential Equations Notes}
\author{Philip Warton}
\date{\today}
\maketitle
\section{September 22}
We begin with the following form of a second order autonomous ODE:
\[
    ay''(t) + by'(t) + cy(t) = f(t)
\]
This equation can be transformed to a first order system of ODE's, if we define two equations to be 
\begin{align*}
    x_1(t) &= y(t) \\
    x_2(t) &= y'(t)
\end{align*}
Then the euqation can be rewritten in terms of $x_1, x_2$ as
\begin{align*}
    ax_2'(t) + bx_2(t) + cx_1(t) = f(t)
\end{align*}
Which immediately gives us this first order system of ODE's:
\begin{align*}
    x_1'(t) &= x_2(t) \\
    x_2'(t) &= -\frac{c}{a} x_1(t) - \frac{b}{a}x_2(t) + \frac{f(t)}{a}
\end{align*}
For notational purposes, write $\vec x (t) = \begin{pmatrix}
    x_1(t)\\x_2(t)
\end{pmatrix}, \vec{x'}(t) = \begin{pmatrix}
    x_1'(t)\\x_2'(t)
\end{pmatrix}$. Then we can write many first order systems of ODE's as 
\begin{mdframed}[]
    \begin{align*}
        A\vec x (t) &= \begin{pmatrix}
            a_{1 1} & a_{1 2} \\ a_{2 1} & a_{2 2}
        \end{pmatrix} \\
        &= \begin{pmatrix}
            a_{1 1}x_1(t) + a_{1 2}x_2(t) \\
            a_{2 1}x_1(t) + a_{2 2}x_2(t)
        \end{pmatrix}
    \end{align*}
\end{mdframed}
For our previous example we would have $A = \begin{pmatrix}
    0 & 1 \\ -c/a & -b/a
\end{pmatrix}$ ignoring the $f(t)$ term.\\\\
Now we study autonomous first order systems of ODE's. That is,$
    \vec{x'}(t) = A\vec x (t)$. These have solutions that can be represented as
\[
    \vec{x}(t) = e^{At} \vec{x_0}, \ \ \ \ \ \ \ e^{At} = \sum_{j=0}^{\infty} \frac{1}{j!} A^j t^j
\]
\end{document}