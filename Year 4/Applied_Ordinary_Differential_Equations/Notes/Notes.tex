\documentclass{article}

\usepackage{times}
\usepackage{amssymb, amsmath, amsthm}
\usepackage[margin=.5in]{geometry}
\usepackage{graphicx}
\usepackage[linewidth=1pt]{mdframed}

\usepackage{import}
\usepackage{xifthen}
\usepackage{pdfpages}
\usepackage{transparent}
\usepackage{bm}

\newcommand{\incfig}[1]{%
    \def\svgwidth{\columnwidth}
    \import{./figures/}{#1.pdf_tex}
}

\newtheorem{theorem}{Theorem}[section]
\newtheorem{lemma}{Lemma}[section]
\newtheorem*{remark}{Remark}
\theoremstyle{definition}
\newtheorem{definition}{Definition}[section]

\begin{document}

\title{Applied Ordinary Differential Equations Notes}
\author{Philip Warton}
\date{\today}
\maketitle
\section{September 22}
We begin with the following form of a second order autonomous ODE:
\[
    ay''(t) + by'(t) + cy(t) = f(t)
\]
This equation can be transformed to a first order system of ODE's, if we define two equations to be 
\begin{align*}
    x_1(t) &= y(t) \\
    x_2(t) &= y'(t)
\end{align*}
Then the euqation can be rewritten in terms of $x_1, x_2$ as
\begin{align*}
    ax_2'(t) + bx_2(t) + cx_1(t) = f(t)
\end{align*}
Which immediately gives us this first order system of ODE's:
\begin{align*}
    x_1'(t) &= x_2(t) \\
    x_2'(t) &= -\frac{c}{a} x_1(t) - \frac{b}{a}x_2(t) + \frac{f(t)}{a}
\end{align*}
For notational purposes, write $\vec x (t) = \begin{pmatrix}
    x_1(t)\\x_2(t)
\end{pmatrix}, \vec{x'}(t) = \begin{pmatrix}
    x_1'(t)\\x_2'(t)
\end{pmatrix}$. Then we can write many first order systems of ODE's as 
\begin{mdframed}[]
    \begin{align*}
        A\vec x (t) &= \begin{pmatrix}
            a_{1 1} & a_{1 2} \\ a_{2 1} & a_{2 2}
        \end{pmatrix} \\
        &= \begin{pmatrix}
            a_{1 1}x_1(t) + a_{1 2}x_2(t) \\
            a_{2 1}x_1(t) + a_{2 2}x_2(t)
        \end{pmatrix}
    \end{align*}
\end{mdframed}
For our previous example we would have $A = \begin{pmatrix}
    0 & 1 \\ -c/a & -b/a
\end{pmatrix}$ ignoring the $f(t)$ term.\\\\
Now we study autonomous first order systems of ODE's. That is,$
    \vec{x'}(t) = A\vec x (t)$. These have solutions that can be represented as
\[
    \vec{x}(t) = e^{At} \vec{x_0}, \ \ \ \ \ \ \ e^{At} = \sum_{j=0}^{\infty} \frac{1}{j!} A^j t^j
\]
\section{October 4}
    \subsection{Matrix Exponential}
        \begin{mdframed}(Taylor Series)
            We define the Taylor series at 0 as 
            \[
                f(x) = \sum_{j=0}^\infty \frac{f^{(j)}(0)}{j!}x^j
            \]
        \end{mdframed}
        For example, $e^x = \sum_{j=1}^\infty \frac{x^j}{j!}$.
        Now we are prepared for the matrix exponential, which we define as follows:
        \[
            e^A = \sum_{j=0}^\infty \frac{A^j}{j!}
        \]
        When $A \mapsto e^A$, we have a group homomorphism from the set 
        of $n \times n$ matrices with addition over to $n \times n$ invertible matrices with
        multiplication.
        Take the initial value problem
        \[
            \begin{cases}
                \vec x'(t) = A\vec x (t) \\
                \vec x(0) = \vec x_0
            \end{cases}
        \]
        The unique solution is $e^{At}\vec{x_0}$.
        Let $X(t) = e^{At}$. This is the matrix of fundamental solutions.
        \[
            X(0) = \begin{pmatrix}
                1 & 0 & \cdots & 0 \\
                0 & 1 & \cdots & 0 \\
                \vdots & \vdots & \ddots & \vdots \\
                0 & 0 & \cdots & 1
            \end{pmatrix} = I
        \]
        Then $X'(t) = AX(t), X(0) = I$. Therefore we get 
        $\vec x(0) = I\vec x_0 = \vec x_0$.\\\\
        Suppose $A$ is diagonalizable,
        \[
            A = U \Lambda U^{-1}, \ \ \ \ \ \ \ \ \ \ \ \text{ where } \Lambda = \begin{pmatrix}
                \lambda_1 & 0 & \cdots & 0 \\
                0 & \lambda_2 & \cdots & 0 \\
                \vdots & \vdots & \ddots & \vdots \\
                0 & 0 & \cdots & \lambda_n
            \end{pmatrix} \ \ \ \ U = (\vec u_1, \cdots , \vec u_n)
        \]
        Then we know that 
        \begin{align}
            X(t) &= e^{At} \\
            &= \sum_{j=0}^\infty \frac{A^jt^j}{j!}\\
            &= \sum_{j=0}^\infty \frac{(U\Lambda U^{-1})^jt^j}{j!} \\
            &= \sum_{j=0}^\infty \frac{U\Lambda^j U^{-1}t^j}{j!}
        \end{align}
        In the case of diagonal matrices, we get a linear composition.
        \begin{align*}
            \vec x(t) &= Ue^{\Lambda t}U^{-1}\vec{x_0}\\
            &= c_1 e^{\lambda_1 t}\vec u_1 + \cdots c_n e^{\lambda_n t}\vec u_n \\
            &= c_1 \vec \Psi_1(t) + \cdots + c_n \vec \Psi_n(t)
        \end{align*}
        When we have a diagonalizable matrix we have a nice formula for the solution.
        Now we will move on to an example:
        \begin{mdframed}[]
            Consider the following system:
            \[
                \begin{pmatrix}
                    x_1'(t)\\
                    x_2'(t)
                \end{pmatrix} = \begin{pmatrix}
                    1 & 1 \\ 0 & 1
                \end{pmatrix} \begin{pmatrix}
                    x_1(t) \\ x_2(t)
                \end{pmatrix}, \ \ \ \ \ \ \ \ \begin{pmatrix}
                    x_1(0) \\ x_2(0)
                \end{pmatrix} = \begin{pmatrix}
                    2 \\ 5
                \end{pmatrix}
            \]
            We know the solution to this is just $\vec x(t) = e^{\begin{pmatrix}1 & 1 \\0 & 1\end{pmatrix}t}\begin{pmatrix}
                2\\ 5
            \end{pmatrix}$. Now we write
            \[
                e^{\begin{pmatrix}1 & 1 \\0 & 1\end{pmatrix}t} = \sum_{j=0}^\infty \frac{\begin{pmatrix}1 & 1 \\0 & 1\end{pmatrix}^jt^j}{j!}
            \]
            Now we know that $\begin{pmatrix}1 & 1 \\0 & 1\end{pmatrix}^j = \begin{pmatrix}1 & j\\0 & 1\end{pmatrix}$.
            This means that we can actually write out what our matrix exponential is explicitly in terms of series in each component.
            We do this as follows:
            \[
                e^{\begin{pmatrix}1 & 1 \\0 & 1\end{pmatrix}t} =\begin{pmatrix}
                    \sum_{j=0}^\infty \frac{t^j}{j!} & \sum_{j=1}^\infty \frac{j t^j}{j!}\\
                    0 & \sum_{j=0}^\infty \frac{t^j}{j!} 
                \end{pmatrix} = \begin{pmatrix}
                    e^t & te^t \\
                    0 & e^t
                \end{pmatrix}
            \]
        \end{mdframed}
        The Wronskian again:
        \[
            W(t) = \det(X(t)) \ \ \ \ W'(t) = \frac{d}{dt}\det(e^{At}) = \frac{d}{dt}e^{\text{tr}(At)} = \text{tr}(A)e^{\text{tr}(A)t} = \text{tr}(A)W(t)
        \]
\end{document}