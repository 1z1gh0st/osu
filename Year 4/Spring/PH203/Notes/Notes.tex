\documentclass{article}

% -------------------------------------------- %
% ------------------PACKAGES------------------ %
% -------------------------------------------- %
\usepackage{amssymb, amsmath, amsthm}
\usepackage{bm}
\usepackage{enumerate}
\usepackage[margin=.5in]{geometry}
\usepackage{graphicx}
\usepackage{import}
\usepackage{listings}
\usepackage[linewidth=1pt]{mdframed}
\usepackage{pdfpages}
\usepackage{times}
\usepackage{transparent}
\usepackage{xifthen}
% -------------------------------------------- %

% Figure command
\newcommand{\incfig}[1]{%
    % Adjust number for defualt figure width
    \def\svgwidth{.5\columnwidth} 
    \import{./figures/}{#1.pdf_tex}
}

% Theorem command
\newtheorem{theorem}{Theorem}[section]
\newtheorem{lemma}{Lemma}[section]
\newtheorem*{remark}{Remark}
\theoremstyle{definition}
\newtheorem{definition}{Definition}[section]

\begin{document}
% -------------------------------------------- %
\title{PH203 --- Notes}
\author{Philip Warton}
\date{\today}
\maketitle
% -------------------------------------------- %
\section{Ray Optics}
Dispersion is the phenomenon that occurs because the index of refraction depends on the frequency of the light.
It occurse becuase different wavelengths refract at different angles, and is the underlying mechanics of 
the rainbow prism effect.
Recall that 
\[
    n_1 \sin \theta_1 = n_2 \sin \theta_2   
\]
and that 
\[
    \lambda_1 > \lambda_2 \Rightarrow n_1 < n_2   
.\]
\subsection{Formation of images, ray tracing}
\begin{mdframed}[]
Which of the following are sources of light in the room?
\begin{enumerate}
    \item Flashlight
    \item Laser
    \item Sun 
    \item Pencil 
    \item Cat 
    \item Your friend's nose
\end{enumerate}
Answer: all of them! If you are seeing something, light is coming off of that object.
\end{mdframed}

\end{document}
