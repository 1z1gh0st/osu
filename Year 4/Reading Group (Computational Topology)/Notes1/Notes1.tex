\documentclass{article}

\usepackage{times}
\usepackage{amssymb, amsmath, amsthm}
\usepackage[margin=.5in]{geometry}
\usepackage{graphicx}
\usepackage[linewidth=1pt]{mdframed}

\usepackage{import}
\usepackage{xifthen}
\usepackage{pdfpages}
\usepackage{transparent}

\newcommand{\incfig}[1]{%
    \def\svgwidth{\columnwidth}
    \import{./figures/}{#1.pdf_tex}
}

\newtheorem{theorem}{Theorem}[section]
\newtheorem{lemma}{Lemma}[section]
\newtheorem*{remark}{Remark}
\theoremstyle{definition}
\newtheorem{definition}{Definition}[section]

\begin{document}

\title{Computational Topology Reading Group - Notes}
\author{Philip Warton}
\date{\today}
\maketitle
\section*{Filtrations}
\begin{definition}
    Let $K$ be a simplicial complex and $A$ a set. A map $\Phi : A \rightarrow \mathbb{R}^{|K|}$
\end{definition}
Some interesting loss function and running filtration experiment
\\\\
Suppose we have a graph where each vertex is a vector of features.
For each vertex we define a function.
For $k$ we have a function.
\\\\
\section{Autoencoders}
We have an input vector $X \in \mathbb{R}^n$ and an output vector $\overline{X} \in \mathbb{R}^n$.
Then we have an intermediary vector $Z \in \mathbb{R}^k$ such that $k < n$. We compute a toplogical loss function for 
the autoencoder so that (hopefully) topological features are preserved.

\end{document}