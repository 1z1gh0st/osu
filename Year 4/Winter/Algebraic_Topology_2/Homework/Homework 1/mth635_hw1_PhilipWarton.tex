\documentclass{article}

% -------------------------------------------- %
% ------------------PACKAGES------------------ %
% -------------------------------------------- %
\usepackage{amssymb, amsmath, amsthm}
\usepackage{bm}
\usepackage{enumerate}
\usepackage[margin=.5in]{geometry}
\usepackage{graphicx}
\usepackage{import}
\usepackage{listings}
\usepackage[linewidth=1pt]{mdframed}
\usepackage{pdfpages}
\usepackage{times}
\usepackage{transparent}
\usepackage{xifthen}
% -------------------------------------------- %

% Figure command
\newcommand{\incfig}[1]{%
    % Adjust number for defualt figure width
    \def\svgwidth{.5\columnwidth} 
    \import{./figures/}{#1.pdf_tex}
}

% Theorem command
\newtheorem{theorem}{Theorem}[section]
\newtheorem{lemma}{Lemma}[section]
\newtheorem*{remark}{Remark}
\theoremstyle{definition}
\newtheorem{definition}{Definition}[section]

\begin{document}

% -------------------------------------------- %
\title{Algebraic Topology --- Homework 1}
\author{Philip Warton}
\date{\today}
\maketitle
% -------------------------------------------- %

\section*{Problem 3.1.1}
    Show that Ext$(H,G)$ is a contravariant functor of $H$ for fixed $G$, 
    and a covariant functor of $G$ for fixed $H$.
    \\
    \par\noindent\rule{\textwidth}{0.4pt}
    \subsection*{Solution}
    \fbox{Ext$(\cdot, G)$ is a contravariant functor}
    \begin{proof}
        We must show that the following are true:
        \begin{enumerate}
            \item Every $X$ is associated to some cohomology group
            \item Every $\alpha : X \rightarrow Y$ is associated to some $\alpha^* :$ Ext$(X,G) \rightarrow $Ext$(Y,G)$
            \item $1_X^* = 1_{\text{Ext}(X,G)}$ 
            \item $(\beta \circ \alpha)^* = \alpha^* \circ \beta^*$ for every $X \xrightarrow{\alpha} Y \xrightarrow{\beta} Z$
        \end{enumerate}
        For the first property, this is true by definition.
        Let $X$ be an abelian group. Then it has some free resolution $F$ given by
        \[
            0 \to F_1 \to F_0 \to X \to 0   
        .\]
        Then, $H^1(F;G) = \text{Ext}(X,G)$, and the first property has been demonstrated.
        \\\\
        Let $\alpha : X \rightarrow Y$ be arbitrary.
        Then, we can naturally extend this homomorphism to connect our 
        free resolutions $F$ and $F'$. That is,
        \begin{align*}
            0 &\to F_1 \to F_0 \to X \to 0 \\
            & \ \ \ \ \ \downarrow\alpha_1 \ \ \ \downarrow\alpha_0 \ \ \downarrow\alpha \\
            0 &\to F_1' \to F_0' \to Y \to 0
        \end{align*}
        Since every basis element of $X$ must be mapped to $Y$ under $\alpha$,
        it follows that we can simply restrict $\alpha$ to these basis elements and have a
        map $\alpha_0 : F_0 \to F_0'$. Similarly we can also extend to $\alpha_1$.
        We denote the image of these $\alpha_i$ with the image of $\alpha$, to be 
        $\alpha^*$, and we say that the second property is satisfied.
        \\\\
        The natural extension of maps goes to the identity map as well,
        and since our free resolutions are the same under the identity map it follows 
        that the cohomologies will be the same and therefore $Ext$ will 
        be under its own identity as well.
        \\\\
        Finally, let $X \xrightarrow{\alpha} Y \xrightarrow{\beta} Z$.
        Then, naturally, we have 
        \begin{align*}
            0 &\to F_1 \to F_0 \to X \to 0 \\
            & \ \ \ \ \ \downarrow\alpha_1 \ \ \ \downarrow\alpha_0 \ \ \downarrow\alpha \\
            0 &\to F_1' \to F_0' \to Y \to 0 \\
            & \ \ \ \ \ \downarrow\beta_1 \ \ \ \downarrow\beta_0 \ \ \downarrow\beta \\
            0 &\to F_1'' \to F_0'' \to Z \to 0 \\
        \end{align*}
        If we take $\beta \circ \alpha$ then we have a homomorphism from 
        $X$ to $Z$ that we extend onto their respective free resolutions. 
        Taking the induced homomorphism from that, we have a the dual, 
        a map from $H^1(Z;G) \to H^1(X;G)$. Then, for the right hand side, 
        we have two maps $\alpha^*: H^1(Y;G) \to H^1(X;G)$ and $\beta^*: H^1(Z;G) \to H^1(Y;G)$.
        Their composition will be the naturally induced homomorphism from $H^1(Z;G) \to H^1(X;G)$,
        it follows by construction that the right hand side will be equal to the left hand side. 
        Note that 
        \begin{align*}
            0 &\leftarrow F_1 \leftarrow F_0 \leftarrow X \leftarrow 0 \\
            & \ \ \ \ \ \uparrow\alpha_1^* \ \ \ \uparrow\alpha_0^* \ \ \uparrow\alpha^* \\
            0 &\leftarrow F_1' \leftarrow F_0' \leftarrow Y \leftarrow 0 \\
            & \ \ \ \ \ \uparrow\beta_1^* \ \ \ \uparrow\beta_0^* \ \ \uparrow\beta^* \\
            0 &\leftarrow F_1'' \leftarrow F_0'' \leftarrow Z \leftarrow 0 \\
        \end{align*}
        will compose to be the same as
        \begin{align*}
            0 &\leftarrow F_1 \leftarrow F_0 \leftarrow X \leftarrow 0 \\
            & \ \ \ \ \ \uparrow\gamma_1^* \ \ \ \uparrow\gamma_0^* \ \ \uparrow\gamma^* \\
            0 &\leftarrow F_1'' \leftarrow F_0'' \leftarrow Z \leftarrow 0 \\
        \end{align*}
        where $\gamma = \beta \circ \alpha$.
        With all of these properties being show, we claim that $Ext(\cdot,G)$ is a contravariant 
        functor.
    \end{proof}
    \fbox{Ext$(H,\cdot)$ is a covariant functor}
    \begin{proof}

            We can say this is a functor by simply composing maps.
            That is, if $G \to^\alpha G' \to^\beta G''$ then 
            \begin{align*}
                \text{Ext}(H,G) &= H^1(F;G) \\
                &= F_1 / Im f_1^* \\
                &= \{\text{functions from $F_1$ to $G$ vanishing on functions 
                in the image of $f_1^*$}\}
            \end{align*}
            So by extending these functions from $F_1 \to G \to G' \to G''$ the 
            functorality becomes apparent. Proof is not thorough but I am out of time.
            For the identity these sets of functions will be the same, since the go 
            to isomorphic groups.
    \end{proof}
\section*{Problem 3.1.2}
Show that the maps $G \xrightarrow{n}G$ and $H\xrightarrow{n}H$ multiplying
each element by the integer $n$ induce multiplication by $n$ in Ext$(H,G)$.
\\
\par\noindent\rule{\textwidth}{0.4pt}
\subsection*{Solution}
\fbox{$G \xrightarrow{n} G$ induces multiplication}
\begin{proof}
    We wish to show that Ext$(H,nG) = n$Ext$(H,G)$.
    We write,
    \begin{align*}
        \text{Ext}(H,nG) &= H^1(F;nG) \\
        &= \{\text{functions from $F_1$ to $nG$}\} \\
        &= n \cdot \{\text{functions from $F_1$ to $G$}\} \\
        &= n\text{Ext}(H,G)
    \end{align*}
\end{proof}
\fbox{$H \xrightarrow{n} H$ induces multiplication}
\begin{proof}
    We know that $nH$ will have the same number of generators 
    as $H$. That means that while the one to one correspondence 
    of elements of $F_0'$ and the basis of $H$ will remain the same 
    size, the number of elements not in correspondence with the basis 
    of $H$ will increase $n$-fold. That is, $F_1'$ will have $n$ times the
    cardinality as $F_1$ given the smallest possible free resolution on $H$.
    So then 
    \begin{align*}
        \text{Ext}(nH,G) &= H^1(F';nG) \\
        &= \{\text{functions from $F_1'$ to $G$}\} \\
        &= n \cdot \{\text{functions from $F_1$ to $G$}\} \\
        &= n \cdot \text{Ext}(H,G)
    \end{align*}
    This is justified by the ratio of cardinalities between $F_1$ and $F_1'$.
\end{proof}

\section*{Problem 3.1.5}
Regarding a cochain $\phi \in C^1(X;G)$ as a function from paths in $X$ to 
$G$, show that if $\phi$ is a cocycle, then 
\begin{enumerate}
    \item $\phi(f \cdot g) = \phi(f) + \phi(g),$
    \item $\phi$ takes the value 0 on constant paths,
    \item $\phi(f) = \phi(g)$ if $f \simeq g$,
    \item $\phi$ is a coboundary iff $\phi(f)$ depends only on the endpoints of $f$, for all $f$. 
\end{enumerate}
[In particular, (a) and (c) give a map $H^1(X;G) \to$ Hom$(\pi_1(X),G)$,
which the universal coefficient theorem says is an isomorphism if $X$ 
is path-connected.]
\\
\par\noindent\rule{\textwidth}{0.4pt}
\subsection*{Solution}
For \fbox{1}, let $f,g$ be paths in $X$.
We can construct a loop given by $f \cdot g \cdot (f \cdot g)^{-1}$.
So this forms the boundary of some interior in $X$.
We can use homology and say that we have a region $S$ that is the 
minimal boundary on this loop, and we have paths given by $f, g, f \cdot g$.
Then we can write that 
\[
    \phi(f) + \phi(g) - \phi(f \cdot g) = \phi(f + g - f\cdot g) = \phi(\partial S) = \delta \phi S
.\]
However, we know by assumption that $\phi$ is a cocycle, and therefore $\delta \phi S = 0$.
So we write,
\begin{align*}
    \phi(f) + \phi(g) - \phi(f \cdot g) &= \delta \phi S \\
    &= 0 \\\\
    \Longrightarrow \phi(f) + \phi(g) &= \phi(f \cdot g)
\end{align*}
For \fbox{2}, suppose that $\phi$ did not take 0 on constant paths.
Let $x_0$ be a constant path in $X$, and let $\phi(x_0) = g \neq 0 \in G$.
Then, we know that $x_0 = x_0 \cdot x_0$. From \fbox{1} it follows that 
\begin{align*}
    \phi(x_0) &= \phi(x_0 \cdot x_0) = \phi(x_0) + \phi(x_0) \\
    g & = g+ g & \text{(contradiction)}
\end{align*}
For \fbox{3} take a $S$ to be the image of the entire homotopy between 
$f$ and $g$. By the nature of homotopy and orientation, it must be the case 
that $\partial S = f - g$. Then
\[
    \phi(f) - \phi(g) = \phi(f - g) = \phi(\partial S) = \delta \phi S = 0
.\]
For \fbox{4}, we begin with the forward direction. Let $\phi$ be a coboundary.
Then let $f$  be some path in $X$. It follows that 
\[
    \phi f = \psi \partial f = \psi (f_1 - f_0)
\]
which clearly depends only on the endpoints of $f$.\\
For the reverse direction, let $X' \subset X$ be a path component.
Then for any path $\gamma$ in $X$ with enpoints $\gamma_0, \gamma_1$ define 
a function $\psi \in C^0(X;G)$ such that $\psi(\gamma_1) = \phi(f)$.
Let $f$ be an arbitrary path in $X'$, with endpoints $f_0,f_1$. Let $g$ be 
path from $f_1$ to $f_0$ that, when reversed, is path homotopic to $f$ and $h$ be a constant 
function at $f_1$. Then clearly $h \cdot g^{-1} \simeq f$.
Then we write 
\begin{align*}
    \phi(f) &= \phi(h \cdot g^{-1}) \\
    &=\phi(h) + \phi(g^{-1}) \\
    &=\phi(h) - \phi(g) \\
    &=\psi(h_1) - \psi(g_1) \\
    &= \psi(f_1) - \psi(f_0) \\
    &= \psi(f_1 - f_0) \\
    &= \psi \partial f
\end{align*}
So there does exist some $\psi$ such that $\phi = \psi \partial = \delta \psi$ and 
therefore $\phi$ is a coboundary.
\section*{Problem 3.1.6 (a)}
Directly from the definitions, compute the simplicial cohomology groups 
of $S^1 \times S^1$ with $\mathbb{Z}$ and $\mathbb{Z}_2$ coefficients,
using the $\Delta$-complex structure given in \S 2.1.
\\
\par\noindent\rule{\textwidth}{0.4pt}
\subsection*{Solution}
To compute the simplicial cohomology groups of $S^1 \times S^1$, let us recall 
the computation of the siplicial homology of the space. We have the chain group 
\[
    0 \xrightarrow{0} C_2 \xrightarrow{\partial} C_1 \xrightarrow{\partial} C_0 \xrightarrow[]{0} 0    
.\]
Then we observed that the boundary maps are defined as follows:
\begin{align*}
    \partial U &= a + b - c \\
    \partial L &= a - c + b \\
    \partial a &= v - v = 0 \\
    \partial b &= v - v = 0 \\
    \partial c &= v - v = 0
\end{align*}
Then since each cycle in $C^1$ goes to 0, we say that $\partial_1 = 0$.
Our coefficients will be equal to 0 or 1, since we are in a simplicial case.
So we can treat the $\mathbb{Z}$ and $\mathbb{Z}_2$ coefficient cases as equivalent.
Now take the dual of each chain to be denoted by itself with a 
$*$ in the superscript. That is, $U^*(U) = 1, U^*(L) = 0$ etc. Then we wish
to compute the coboundary maps of each of our duals.
Since $\partial: C_1 \to C_0$ is the 0-map, then its dual 
$\delta: C^0 \to C^1$ is also the 0-map. 
So we say that
\begin{align*}
    Im \ \delta_0 &= 0 \\
    Ker \ \delta_0 &= C^0 
\end{align*}
For $\delta_1:C^1\to C^2$,
we wish to compute how it acts upon different cochains in $C^1$.
So, we can compute $\delta$ for each of our duals.
\begin{align*}
    \delta a^* U &= a^* \partial U = a^* (a + b - c) = a^* a =1 \\
    \delta b^* U &= b^* \partial U = b^* (a + b - c) = b^* b = 1 \\
    \delta c^* U &= c^* \partial U = c^* (a + b - c) = -c^* c = -1 \\\\
    \delta a^* L &= a^* \partial L = a^* (a - c + b) = a^* a = 1 \\
    \delta b^* L &= b^* \partial L = b^* (a - c + b) = b^* b = 1 \\
    \delta c^* L &= c^* \partial L = c^* (a - c + b) = - c^* c = -1
\end{align*}
Since $\delta a^*$ is equal to 1 for both $U$ and $L$ it follows that 
$\delta a^* = U^* + L^*$ so that it has a positive coefficient acting 
upon either $U$ or $L$.
Then the same is true for $b^*$, and the opposite is true. So we say that
\begin{align*}
    \delta_1 a^* &= U^* + L^* \\
    \delta_1 b^* &= U^* + L^* \\
    \delta_1 c^* &= -U^* - L^*
\end{align*}
Since these are clearly not independent, we say that
\[
    Im \ \delta_1 = \mathbb{Z}(U^* + L^*)
.\]
Then since $\delta_1 a^* = \delta_1 b^*$ is the opposite of $\delta_1 c^*$,
it follows that 
\[
    Ker \ \delta_1 = \mathbb{Z}(a^* + c^*) \oplus \mathbb{Z}(b^* + c^*)
.\]
Then, finally for $\delta_2$, it is 0 since it is the dual of the 0 map.
Therefore its kernel is its entire domain, and its image, 0.
So at long last we can compute the cohomology,
\begin{align*}
    H^3(S^1 \times S^1; \mathbb{Z}\text{ or }\mathbb{Z}_2) &= \frac{Ker \ \delta_0}{Im \ \delta_{-1}} = \frac{C^0}{0} = C^0 = \mathbb{Z}v^* \\
    H^1(S^1 \times S^1; \mathbb{Z}\text{ or }\mathbb{Z}_2) &= \frac{Ker \ \delta_1}{Im \ \delta_0} = \frac{\mathbb{Z}(a^* + c^*) \oplus \mathbb{Z}(b^* + c^*)}{0} = \mathbb{Z}(a^* + c^*) \oplus \mathbb{Z}(b^* + c^*) \\
    H^2(S^1 \times S^1; \mathbb{Z}\text{ or }\mathbb{Z}_2) &= \frac{Ker \ \delta_2}{Im \ \delta_1} = \frac{C^2}{\mathbb{Z}(U^* + L^*)} = \frac{\mathbb{Z}U^* + \mathbb{Z}L^*}{\mathbb{Z}(U^* + L^*)} = \frac{\mathbb{Z}U^* + \mathbb{Z}(U^* + L^*)}{\mathbb{Z}(U^* + L^*)} = \mathbb{Z}U^* \\
    H^3(S^1 \times S^1; \mathbb{Z}\text{ or }\mathbb{Z}_2) &= 0 \\
    & \vdots
\end{align*}

\section*{Problem 3.1.8 (c)}
Show that if $A$ is a retract of $X$ then $H^n(X;G) \approx H^n(A;G) \oplus H^n(X,A;G)$.
\\
\par\noindent\rule{\textwidth}{0.4pt}
\subsection*{Solution}
\begin{proof}
Assume that $A$ is a retract of $X$. Then we have a short exact sequence 
\[
    0 \to H^n(A;G) \to H^n(X;G) \to H^n(X,A;G) \to 0
.\]
Also, we have some $r:X \to A$ as our retraction and similarly $i:A \to X$ as our inclusion.
Then, we have induced homomorphisms $r^*: H^n(X;G) \to H^n(A;G)$,
$i^*: H^n(A;G) \to H^n(X;G)$. So we can also use these functions to construct 
the following diagrams (which yield a split exact sequence):
\begin{align*}
    0 &\longrightarrow H^n(A;G) \longrightarrow_{i^*} H^n(X;G) \longrightarrow_{r^*} H^n(X,A;G) \longrightarrow 0 \\
0 &\longrightarrow H^n(A;G) \rightarrow H^n(A;G) \oplus H^n(X,A;G) \longrightarrow H^n(X,A;G) \longrightarrow 0
.\end{align*}
Since we have isomorphism on the left term and the right term, it follows 
that $H^n(A;G) \approx H^n(A;G) \oplus H^n(X,A;G)$.
\end{proof}
\end{document}
