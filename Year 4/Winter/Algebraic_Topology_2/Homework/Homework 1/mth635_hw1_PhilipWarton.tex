\documentclass{article}

% -------------------------------------------- %
% ------------------PACKAGES------------------ %
% -------------------------------------------- %
\usepackage{amssymb, amsmath, amsthm}
\usepackage{bm}
\usepackage{enumerate}
\usepackage[margin=.5in]{geometry}
\usepackage{graphicx}
\usepackage{import}
\usepackage{listings}
\usepackage[linewidth=1pt]{mdframed}
\usepackage{pdfpages}
\usepackage{times}
\usepackage{transparent}
\usepackage{xifthen}
% -------------------------------------------- %

% Figure command
\newcommand{\incfig}[1]{%
    % Adjust number for defualt figure width
    \def\svgwidth{.5\columnwidth} 
    \import{./figures/}{#1.pdf_tex}
}

% Theorem command
\newtheorem{theorem}{Theorem}[section]
\newtheorem{lemma}{Lemma}[section]
\newtheorem*{remark}{Remark}
\theoremstyle{definition}
\newtheorem{definition}{Definition}[section]

\begin{document}

% -------------------------------------------- %
\title{Algebraic Topology --- Homework 1}
\author{Philip Warton}
\date{\today}
\maketitle
% -------------------------------------------- %

\section*{Problem 3.1.1}
Show that Ext$(H,G)$ is a contravariant functor of $H$ for fixed $G$, 
and a covariant functor of $G$ for fixed $H$.
\\
\par\noindent\rule{\textwidth}{0.4pt}
\section*{Solution}
\fbox{Ext$(\cdot, G)$ is a contravariant functor}
\begin{proof}
    We must show that the following are true:
    \begin{enumerate}
        \item Every $X$ is associated to some cohomology group
        \item Every $\alpha : X \rightarrow Y$ is associated to some $\alpha^* :$ Ext$(X,G) \rightarrow $Ext$(Y,G)$
        \item $1_X^* = 1_{\text{Ext}(X,G)}$ 
        \item $(\beta \circ \alpha)^* = \alpha^* \circ \beta^*$ for every $X \xrightarrow{\alpha} Y \xrightarrow{\beta} Z$
    \end{enumerate}
    For the first property, this is true by definition.
    Let $X$ be an abelian group. Then it has some free resolution $F$ given by
    \[
        0 \to F_1 \to F_0 \to X \to 0   
    .\]
    Then, $H^1(F;G) = \text{Ext}(X,G)$, and the first property has been demonstrated.
    \\\\
    Let $\alpha : X \rightarrow Y$ be arbitrary.
    Then, we can naturally extend this homomorphism to connect our 
    free resolutions $F$ and $F'$. That is,
    \begin{align*}
        0 &\to F_1 \to F_0 \to X \to 0 \\
        & \ \ \ \ \ \downarrow\alpha_1 \ \ \ \downarrow\alpha_0 \ \ \downarrow\alpha \\
        0 &\to F_1' \to F_0' \to Y \to 0
    \end{align*}
    Since every basis element of $X$ must be mapped to $Y$ under $\alpha$,
    it follows that we can simply restrict $\alpha$ to these basis elements and have a
    map $\alpha_0 : F_0 \to F_0'$. Similarly we can also extend to $\alpha_1$
\end{proof}
\fbox{Ext$(H,\cdot)$ is a covariant functor}


\section*{Problem 3.1.2}
Show that the maps $G \xrightarrow{n}G$ and $H\xrightarrow{n}H$ multiplying
each element by the integer $n$ induce multiplication by $n$ in Ext$(H,G)$.

\section*{Problem 3.1.5}
Regarding a cochain $\phi \in C^1(X;G)$ as a function from paths in $X$ to 
$G$, show that if $\phi$ is a cocycle, then 
\begin{enumerate}
    \item $\phi(f \cdot g) = \phi(f) + \phi(g),$
    \item $\phi$ takes the value 0 on constant paths,
    \item $\phi(f) = \phi(g)$ if $f \simeq g$,
    \item $\phi$ is a coboundary iff $\phi(f)$ depends only on the endpoints 
    \item of $f$, for all $f$. 
\end{enumerate}
[In particular, (a) and (c) give a map $H^1(X;G) \to$ Hom$(\pi_1(X),G)$,
which the universal coefficient theorem says is an isomorphism if $X$ 
is path-connected.]

\section*{Problem 3.1.6 (a)}
Directly from the definitions, compute the simplicial cohomology groups 
of $S^1 \times S^1$ with $\mathbb{Z}$ and $\mathbb{Z}_2$ coefficients,
using the $\Delta$-complex structure given in \S 2.1.

\section*{Problem 3.1.8 (c)}
Show that if $A$ is a retract of $X$ then $H^n(X;G) \approx H^n(A;G) \oplus H^n(X,A;G)$.
\end{document}
