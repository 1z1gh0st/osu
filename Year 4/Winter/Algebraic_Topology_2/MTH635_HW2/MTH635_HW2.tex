\documentclass{article}

% -------------------------------------------- %
% ------------------PACKAGES------------------ %
% -------------------------------------------- %
\usepackage{amssymb, amsmath, amsthm}
\usepackage{bm}
\usepackage{enumerate}
\usepackage[margin=.5in]{geometry}
\usepackage{graphicx}
\usepackage{import}
\usepackage{listings}
\usepackage[linewidth=1pt]{mdframed}
\usepackage{pdfpages}
\usepackage{times}
\usepackage{transparent}
\usepackage{xifthen}
% -------------------------------------------- %

% Figure command
\newcommand{\incfig}[1]{%
    % Adjust number for defualt figure width
    \def\svgwidth{.5\columnwidth} 
    \import{./figures/}{#1.pdf_tex}
}

% Theorem command
\newtheorem{theorem}{Theorem}[section]
\newtheorem{lemma}{Lemma}[section]
\newtheorem*{remark}{Remark}
\theoremstyle{definition}
\newtheorem{definition}{Definition}[section]

\begin{document}

% -------------------------------------------- %
\title{Algebraic Topology --- Homework 2}
\author{Philip Warton}
\date{\today}
\maketitle
% -------------------------------------------- %
\section*{Problem 3.11 (a)}
Let $X$ be a Moore space $M(\mathbb{Z}_m,n)$ obtained from $S^n$ by attaching a cell $e^{n+1}$ by a map of degree $m$.
Show that the quotient map $X \to X / S^n = S^{n+1}$ induces the trivial map on $H_i(\circ;\mathbb{Z})$ for all $i$ but not on $H^{n+1}(\circ ; \mathbb{Z})$.
Deduce taht the splitting in the universal coefficient theorem for cohomology cannot be natural.
\\
\par\noindent\rule{\textwidth}{0.4pt}
\subsection*{Solution}
First we show that the trivial map is induced on the reduced homology with integer coefficients.
We say that if $i = n$ then $0 = H_n(S^{n+1}) = H_n(S^{n+1};\mathbb{Z}) \Rightarrow 0 = \tilde{H_n}(S^{n+1})$.
Since this is the target of the induced homomorphism from the quotient map, it follows that this map must necessarily be trivial.
If $i \neq n$ then we have $0 = H_i(X) = H_i(X;\mathbb{Z}) \Rightarrow 0 = \tilde{H_i}(X;\mathbb{Z})$.
Since the domain of our map is trivial, the map must be trivial.\\
Then we wish to show that we do not have the induced map be trivial on $H^{n+1}(\circ ; \mathbb{Z})$.
We have the long exact sequence
\[
    \cdots \to {H}_{n+1}(S^n) \to {H}_{n+1}(X) \to {H}_{n+1}(X / S^n) \to \cdots
\]
and its dual 
\[
    \cdots \leftarrow {H}^{n+1}(S^n;\mathbb{Z}) \leftarrow {H}^{n+1}(X;\mathbb{Z}) \leftarrow {H}^{n+1}(X / S^n;\mathbb{Z}) \leftarrow \cdots  
\]
We know that the $n+1$ cohomology of the $n$-sphere will be 0. By the universal coefficient theorem we have 
\[
    0 \to Ext(H_ns)  
\]
\section*{Problem 3.1.1}
Assuming as known the cup product structure on the torus $S^1 \times S^1$, compute the cup 
product structure in $H^*(M_g)$ for $M_g$ the closed orientable surface of genus $g$ by using the quotient map 
from $M_g$ to a wedge sum of $g$ tori, shown in the text.

\end{document}
