\documentclass{article}

% -------------------------------------------- %
% ------------------PACKAGES------------------ %
% -------------------------------------------- %
\usepackage{amssymb, amsmath, amsthm}
\usepackage{bm}
\usepackage{enumerate}
\usepackage[margin=.5in]{geometry}
\usepackage{graphicx}
\usepackage{import}
\usepackage{listings}
\usepackage[linewidth=1pt]{mdframed}
\usepackage{pdfpages}
\usepackage{times}
\usepackage{transparent}
\usepackage{xifthen}
% -------------------------------------------- %

% Figure command
\newcommand{\incfig}[1]{%
    % Adjust number for defualt figure width
    \def\svgwidth{.5\columnwidth} 
    \import{./figures/}{#1.pdf_tex}
}

% Theorem command
\newtheorem{theorem}{Theorem}[section]
\newtheorem{lemma}{Lemma}[section]
\newtheorem {remark}{Remark}
\theoremstyle{definition}
\newtheorem{definition}{Definition}[section]

\begin{document}

% -------------------------------------------- %
\title{Gr{\"o}bner Bases --- Homework 7}
\author{Philip Warton}
\date{\today}
\maketitle
% -------------------------------------------- %
\section*{Problem 1}
Let $R=\mathbb{Q}[x,y]$, and set $f=x^4y + x^2y^3 - 3x^3y - 3xy^3$, $g=x^5y - 3x^4y - 
x^2y^2 + 3xy^2$.  Use our Gr\"obner basis elimination order strategy from class to 
compute the following.  (Note: you can check your work using the gcd command in 
sagemath, but please explain the whole method and do all the relevant computations 
in sagemath for your hand in work.)
\begin{enumerate}
\item
$\text{lcm}(f,g)$
\item
$\gcd(f,g)$
\end{enumerate}
\par\noindent\rule{\textwidth}{0.4pt}
\subsection*{Solution}
First, we will compute the reduced Gr{\"o}bner basis for the ideal $\langle wf,(1-w)g \rangle$ with respect to the elimination order of $x,y$
being smaller than $w$. We will use sagemath in order to do this. We get 
\begin{verbatim}
REDUCED_GROEBNER_BASIS:
w*x^4*y - 3*w*x^3*y + w*x^2*y^3 - 3*w*x*y^3
w*x^3*y^2 - w*x^2*y^5 - 3*w*x^2*y^2 + 3*w*x*y^5 + x^6*y - 3*x^5*y - x^3*y^2 + 3*x^2*y^2
w*x^2*y^6 + w*x^2*y^2 - 3*w*x*y^6 - 3*w*x*y^2 - x^6*y^2 + 3*x^5*y^2 + x^5*y - 3*x^4*y + x^3*y^3 - 3*x^2*y^3 - x^2*y^2 + 3*x*y^2
x^7*y - 3*x^6*y + x^5*y^3 - 3*x^4*y^3 - x^4*y^2 + 3*x^3*y^2 - x^2*y^4 + 3*x*y^4
\end{verbatim}
The last polynomial is our $lcm$ since there is no $w$ variable in this polynomial. That is,
\[
    lcm(f,g)=x^7  y - 3  x^6  y + x^5  y^3 - 3  x^4  y^3 - x^4  y^2 + 3  x^3  y^2 - x^2  y^4 + 3  x  y^4
.\]
Then by our theorem, we know that $fg = lcm(f,g)gcd(f,g)$. We will use sagemath to compute $\frac{fg}{lcm(f,g)} = gcd(f,g)$.
We get as our result,
\[
    gcd(f,g)=x(x-3)y  
.\]
\section*{Problem 2}
Let $R=\mathbb{Q}[x,y]$, and set 
\begin{align*}
f_1 &= x^2\\
f_2 &= x+y\\
g_1 &= x(x+y)^2\\
g_2 &= y.\\
\end{align*}
Let $A=\langle f_1,f_2 \rangle$ and let $B=\langle g_1, g_2 \rangle$.  Use our 
Gr\"obner basis elimination order strategy from class to compute the ideal quotient
$B:A$. (Note: again you can check your work using the quotient(A) command in 
sagemath, but please explain the whole method and do all the relevant computations 
in sagemath for your hand in work.)
\par\noindent\rule{\textwidth}{0.4pt}
\subsection*{Solution}
To solve this we can first use Lemma 2.3.10 and Lemma 2.3.11 which grants us that 
\[
    B : A = \left(B : \langle f_1 \rangle\right) \cap \left(B : \langle f_2 \rangle\right)
    = \left(\frac{1}{f_1}(B \cap \langle f_1 \rangle)\right) \cap \left(\frac{1}{f_2}(B \cap \langle f_2 \rangle)\right)
.\]
Then we can use lemma 2.3.7 to get the result that 
\[
B \cap \langle f_i \rangle = \langle lcm(B,f_i) \rangle
= \langle lcm(g_1,g_2,f_i) \rangle
.\]
First, we can simply find the Gro\"obner basis, $G_1$, of $\langle wg_1, wg_2, (1-w)f_1\rangle$ and 
$G_2$ of $\langle wg_1, wg_2, (1-w)f_2\rangle$
using sagemath. This grants us 
\[
    G_1 = \{x^3, x^2y\}, \ \ \ \ \ G_2 = \{x^3 + y^3, xy + y^2\}
.\]
So then dividing by $f_1$ and $f_2$ respectively, we say
\[
 \frac{1}{f_1}(B \cap \langle f_1\rangle) = \langle x, y \rangle, \ \ \ \ \ \ \ \ \
 \frac{1}{f_2}(B \cap \langle f_2 \rangle) = \langle x^2 - yx + y^2, y \rangle
.\]
So now to intersect these two, we use sagemath to compute the Gr\"obner basis of 
$\langle wx,wy, (1-w)(x^2 - yx + y^2), (1-w)y\rangle$, which is given by
\[
    \langle x, y^2 \rangle
.\]
So, in all, $B : A = \langle x, y^2 \rangle$.
\end{document}
