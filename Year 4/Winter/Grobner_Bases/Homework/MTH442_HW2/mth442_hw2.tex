\documentclass{article}

% -------------------------------------------- %
% ------------------PACKAGES------------------ %
% -------------------------------------------- %
\usepackage{amssymb, amsmath, amsthm}
\usepackage{bm}
\usepackage{enumerate}
\usepackage[margin=.5in]{geometry}
\usepackage{graphicx}
\usepackage{import}
\usepackage{listings}
\usepackage[linewidth=1pt]{mdframed}
\usepackage{pdfpages}
\usepackage{times}
\usepackage{transparent}
\usepackage{xifthen}
% -------------------------------------------- %

% Figure command
\newcommand{\incfig}[1]{%
    % Adjust number for defualt figure width
    \def\svgwidth{.5\columnwidth} 
    \import{./figures/}{#1.pdf_tex}
}

% Theorem command
\newtheorem{theorem}{Theorem}[section]
\newtheorem{lemma}{Lemma}[section]
\newtheorem*{remark}{Remark}
\theoremstyle{definition}
\newtheorem{definition}{Definition}[section]

\begin{document}

% -------------------------------------------- %
\title{Gr{\"o}bner Bases --- Homework 2}
\author{Philip Warton}
\date{\today}
\maketitle
% -------------------------------------------- %

\section*{Problem 1}
Consider $\mathbb{Q}[x,y,z,t]$, and let 
\begin{align*}
    f_1 &= x - 2y + z + t \\
    f_2 &= x + y + 3z + t \\
    f_3 &= 2x - y -z - t \\
    f_4 &= 2x + 2y + z + t
.\end{align*}
Solve our four ``important questions/goals'' from the introduction for 
this set of polynomials.
\\
\par\noindent\rule{\textwidth}{0.4pt}
\subsection*{Solution}
\subsubsection*{1}
Given some linear function $f \in \mathbb{Q}[x,y,z,t]$, is $f \in \langle f_1,f_2,f_3,f_4\rangle$.
We will start with row reduction,
\[
    \begin{bmatrix}
        1 & -2 & 1 & 1 \\
        1 & 1 & 3 & 1 \\
        2 & -1 & -1 & -1 \\
        2 & 2 & 1 & 1
    \end{bmatrix}   
    \to 
    \begin{bmatrix}
        1 & -2 & 1 & 1 \\
        0 & 3 & 2 & 0 \\
        0 & 3 & -2 & -2 \\
        0 & 6 & -1 & -1
    \end{bmatrix} 
    \to 
    \begin{bmatrix}
        1 & -2 & 1 & 1 \\
        0 & 3 & 2 & 0 \\
        0 & 0 & -4 & 0 \\
        0 & 0 & -5 & -1 
    \end{bmatrix}
    \to 
    \begin{bmatrix}
        1 & -2 & 1 & 1 \\
        0 & 3 & 2 & 0 \\
        0 & 0 & 1 & 0 \\
        0 & 0 & 0 & 1 
    \end{bmatrix}
    \to 
    I
\]
Since we have $\langle f_1, f_2, f_3, f_4\rangle = \langle x,y,z,t\rangle$,
it follows that our ideal $I$ is equal to any linear combination of our four variables 
$x,y,z,t$. To check that some $f$ is a member of $\langle x,y,z,t \rangle$,
we wish to demonstrate that it is only a linear combination of these terms. 
That is to say, that it does not have a \textbf{constant term}.
\subsubsection*{2}
So then to write any $f$ as a linear
\subsubsection*{3}
\subsubsection*{4}
\section*{Problem 2}
Find a single generator for the ideal $I = \langle x^6 - 1, x^4 + 2x^3 + 2x^2 -2x -3\rangle$.
Is $x^5 + x^3 + x^2 - 7$ in $I$? Is $x^4 + 2x^2 - 3$ in $I$?
\\
\par\noindent\rule{\textwidth}{0.4pt}
\subsection*{Solution}
We wish to find the greatest common denominator, so we use the ``euclidean algorithm for polynomials'' to do so.
Let $f = x^6 - 1, g = $. Then in the first pass,
\begin{align*}
    x^6 - 1 & \xrightarrow{x^4 + 2x^3 + 2x^2 - 2x - 3} 2x^3 - 5x^2 - 2x +5 \\
    f &:=x^4 + 2x^3 + 2x^2 - 2x - 3 \\
    g &:= 2x^3 - 5x^2 - 2x +5
.\end{align*}
In the second pass,
\begin{align*}
    x^4 + 2x^3 + 2x^2 - 3 &\xrightarrow{ 2x^3 - 5x^2 - 2x +5} \frac{57}{4}x^2 - \frac{57}{4} \\
    f &:= 2x^3 - 5x^2 - 2x +5 \\
    g &:= \frac{57}{4}x^2 - \frac{57}{4} 
.\end{align*}
In the third pass,
\begin{align*}
    2x^3 - 5x^2 - 2x +5 &\xrightarrow{\frac{57}{4}x^2 - \frac{57}{4}} 0 \\
    f &:= \frac{57}{4}x^2 - \frac{57}{4} \\
    g &:= 0
.\end{align*}
So finally let 
\[
    f := \frac{1}{\frac{57}{4}}\left(\frac{57}{4}x^2 - \frac{57}{4}\right) = x^2 - 1
.\]
And the algorithm is completed. So, we say that $I = \langle x^2 - 1 \rangle$. \\\\
\fbox{$x^5 + x^3 + x^2 - 7$}
Recall that $f \in I = \langle g\rangle$ if and only if $f \xrightarrow{g}_+ 0$.
So we will divide the one by the other, giving us 
\[
    x^5 + x^3 + x^2 - 7 \xrightarrow{x^2 - 1} 2x - 6 \neq 0
.\]
The function \textbf{does not} belong to the ideal.\\\\
\fbox{$x^4 + 2x^2 - 3$} Again, we divide this polynomial by $x^2 - 1$ and see our 
remainder. That is,
\[
    x^4 + 2x^2 - 3 \xrightarrow{x^2 - 1} 0  
.\]
The function \textbf{does} belong to the ideal.
\end{document}
