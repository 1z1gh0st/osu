\documentclass{article}

% -------------------------------------------- %
% ------------------PACKAGES------------------ %
% -------------------------------------------- %
\usepackage{amssymb, amsmath, amsthm}
\usepackage{bm}
\usepackage{enumerate}
\usepackage[margin=.5in]{geometry}
\usepackage{graphicx}
\usepackage{import}
\usepackage{listings}
\usepackage[linewidth=1pt]{mdframed}
\usepackage{pdfpages}
\usepackage{times}
\usepackage{transparent}
\usepackage{xifthen}
% -------------------------------------------- %

% Figure command
\newcommand{\incfig}[1]{%
    % Adjust number for defualt figure width
    \def\svgwidth{.5\columnwidth} 
    \import{./figures/}{#1.pdf_tex}
}

% Theorem command
\newtheorem{theorem}{Theorem}[section]
\newtheorem{lemma}{Lemma}[section]
\newtheorem*{remark}{Remark}
\theoremstyle{definition}
\newtheorem{definition}{Definition}[section]

\begin{document}

% -------------------------------------------- %
\title{Gr{\"o}bner Bases --- Homework 4}
\author{Philip Warton}
\date{\today}
\maketitle
% -------------------------------------------- %
\section*{Problem 1}
\begin{proof}
    Let $f \in A$. Then we write $f = \sum_{i=1}^s u_i X_i$.
    Then $f \xrightarrow{X_1} f_1$ where $f_1 = f - \frac{u_1 X_1}{lt(X_1)}X_1$.
    Since $lt(X_i) = X_i$ we can say $f_1 = f - u_1X_1$. Repeat this process,
    and then $f_s = f - u_1X_1 - u_2X_2 - \cdots - u_sX_s$.
    However by assumption $f = u_1X_1 + \cdots + u_2X_2$.
    So then we have 
    \begin{align*}
        f_s &= \sum_{i=1}^s u_iX_i - \sum_{i=1}^su_i X_i \\
        &= 0
    \end{align*}
    Clearly $f \xrightarrow{G} f_s = 0$, so we have one direction complete.\\\\
    Let $f \xrightarrow{G}_+0$ by some reduction. Let each $Y_i$ be some term in $f$
    and we say that 
    \begin{align*}
        0 &= f - \sum \frac{Y_i}{lt(X_i)}X_i & \text{where this is a finite sum and all $i$ are from 1 to $s$}\\
        &= f - \sum \frac{Y_i}{X_i}X_i \\
        &= f - \sum Y_i
    \end{align*}
    Then by definition of reductions we say that $lt(X_i)$ divides $Y_i$ so $Y_i = u_iX_i$.
    Then we can substitute this for $Y_i$, giving ,
    \begin{align*}
        0 &= f - \sum Y_i \\
        0 &= f - \sum u_i X_i \\
        \sum u_i X_i &= f \\\\
        \Rightarrow f &\in A
    \end{align*}
\end{proof}
\section*{Problem 2}
First we wish to show that each class in $B$ is independent.
Let $X,X' \in \mathbb{T}^n$ such that $lp(g_i) \nmid X$ and $lp(g_i) \nmid X'$ for all $i$.
Let $X + A = X' + A$. Then wish to show that $X = X'$. For the equality of these 
cosets we can also write
\[
    \{X + g_i\}_{i = 1\cdots t} = \{X' + g_i\}_{i = 1 \cdots t}
.\]
It must be the case that some $X + g_i = X' + g_j$ then. If $i = j$,
then $X = X'$ is guaranteed. If $i \neq j$ then we have 
\begin{align*}
    X + g_i &= X' + g_j \\
    X - X' &= g_j - g_i
\end{align*}
Since $g_j - g_i \in A$ it follows that $X - X' \in A$, and therefore must reduce 
to 0 by $G$. This implies that either some $lp(g_k) \in G$ divides $X - X'$, 
in which case it would have to divide both $X$ and $X'$ leading to a contradiction,
or it is the case that $X - X' = 0 \Longrightarrow X = X'$.
\\\\
To show that $B$ generates all cosets of $A$, let $f + A \in R / A$.
Then $f \in R$ implies that $f = \sum_{i \in I} c_i X_i$ where $c_i$ belong 
to our coefficient field and $X_i \in \mathbb{T}^n$. For each 
power product that can be divided by some $lp(g_i)$, we will say 
$X_i = Y_i \in \mathbb{T}^n$. Partition our index set $I$ into 
$I_1$ and $I_2$ so that $\{Y_i\} \subset I_2$. Then,
\begin{align*}
    f &= \sum_{i \in I}c_i X_i \\
    &= \sum_{i \in I_1} c_i X_i + \sum_{i \in I_2} c_i Y_i
\end{align*}
However each term in the second sum is generated by $A$. So then it follows that for the coset of $f$,
\[
    f + A = \sum_{i \in I_1} c_i X_i + \sum_{i \in I_2} c_i Y_i + A = \sum_{i \in I_1} c_i X_i + A
\]
Where each $X_i$ cannot be divided by any $g_i$, so we say that $f + A \in \langle B \rangle$.
\end{document}
