\documentclass{article}

% -------------------------------------------- %
% ------------------PACKAGES------------------ %
% -------------------------------------------- %
\usepackage{amssymb, amsmath, amsthm}
\usepackage{bm}
\usepackage{enumerate}
\usepackage[margin=.5in]{geometry}
\usepackage{graphicx}
\usepackage{import}
\usepackage{listings}
\usepackage[linewidth=1pt]{mdframed}
\usepackage{pdfpages}
\usepackage{times}
\usepackage{transparent}
\usepackage{xifthen}
% -------------------------------------------- %

% Figure command
\newcommand{\incfig}[1]{%
    % Adjust number for defualt figure width
    \def\svgwidth{.5\columnwidth} 
    \import{./figures/}{#1.pdf_tex}
}

% Theorem command
\newtheorem{theorem}{Theorem}[section]
\newtheorem{lemma}{Lemma}[section]
\newtheorem*{remark}{Remark}
\theoremstyle{definition}
\newtheorem{definition}{Definition}[section]

\begin{document}

% -------------------------------------------- %
\title{Gr{\"o}bner Bases --- Homework 1}
\author{Philip Warton}
\date{\today}
\maketitle
% -------------------------------------------- %

\section*{Problem 1}
Let $S = \{x - y, z \} \subseteq \mathbb{R}[x,y,z]$. Describe the variety $V(S)$ both using set notation and geometrically.
\\
\par\noindent\rule{\textwidth}{0.4pt}
\subsection*{Solution}
We know that the variety $V(S)$ is the set of points such that every function in $S$ is equal to 0 at said point.
That is, in set notation,
\begin{align*}
    V(S) &= \{(x,y,z) \subset \mathbb{R}^3 \ | \ x - y = 0 = z \} \\
    &= \{(x,y,z) \subset \mathbb{R}^3 \ | \ x = y \text{ and } z = 0\}.
\end{align*}
We can also just write this as all points in $\mathbb{R}^3$ of the form $(\alpha, \alpha, 0)$ where $\alpha \in \mathbb{R}$.
Geometrically, this consists of a straight line passing through the origin that lies in the $x$-$y$ plane 
that is represented by the equation $x = y$.
\\\\
\section*{Problem 2}
Let $K$ be a field. Recall that for a set $S \subseteq K^n$, we define
\[
    I(S) = \{f \in K[x_1, \cdots, x_n] \ | \ f(\vec a) = 0 \text{ for all } \vec a \in S\}
\]
Prove that $I(S)$ is an ideal of the ring $K[x_1, \cdots, x_n]$.
\\
\par\noindent\rule{\textwidth}{0.4pt}
\subsection*{Solution}
\begin{proof}
Recall that a subring $I$ of a ring $R$ is an ideal if for every $x \in R, y \in I$,
we can say $xy \in I$.
In our case specifically, let $S \subseteq K^n$, let
\[
    f \in K[x_1, \cdots, x_n]    
,\]
and let 
\[
    g \in I(S)
.\]
Choose any $\vec a \in S$ to be arbitrary. Then 
\begin{align*}
    (fg)(\vec a) &= f(\vec a)g(\vec a) \\
    &= f(\vec a) \cdot 0 \\
    &= 0
\end{align*}
Since $(fg)(\vec a) = 0$ for any $\vec a \in S$, it follows that $fg \in I(S)$
and it must be true that $I(S)$ is an ideal.
\end{proof}
\end{document}
