\documentclass{article}

% -------------------------------------------- %
% ------------------PACKAGES------------------ %
% -------------------------------------------- %
\usepackage{amssymb, amsmath, amsthm}
\usepackage{bm}
\usepackage{enumerate}
\usepackage[margin=.5in]{geometry}
\usepackage{graphicx}
\usepackage{import}
\usepackage{listings}
\usepackage[linewidth=1pt]{mdframed}
\usepackage{pdfpages}
\usepackage{times}
\usepackage{transparent}
\usepackage{xifthen}
% -------------------------------------------- %

% Figure command
\newcommand{\incfig}[1]{%
    % Adjust number for defualt figure width
    \def\svgwidth{.5\columnwidth} 
    \import{./figures/}{#1.pdf_tex}
}

% Theorem command
\newtheorem{theorem}{Theorem}[section]
\newtheorem{lemma}{Lemma}[section]
\newtheorem*{remark}{Remark}
\theoremstyle{definition}
\newtheorem{definition}{Definition}[section]

\begin{document}

% -------------------------------------------- %
\title{Gr{\"o}bner Bases --- Homework 3}
\author{Philip Warton}
\date{\today}
\maketitle
% -------------------------------------------- %
\section*{Problem 1.4.1}
Let $f = 3x^4z - 2x^3y^4 + 7x^2y^2z^3 - 8xy^3z^3 \in \mathbb{Q}[x,y,z]$.
Determine the leading term, leading coefficient, and leading power product of $f$
with respect to deglex, lex, and degrevlex with $x > y > z$. Repeat the exercise with $x < y < z$.
\\
\par\noindent\rule{\textwidth}{0.4pt}
\subsection*{Solution}
\fbox{$x > y > z$}\\
With respect to deglex, we first note the sum of powers for each term.
\begin{align*}
    3x^4z       &\to 5 \\
    -2x^3y^4    &\to 7 \\
    7x^2y^2z^3  &\to 7 \\
    -8xy^3z^3   &\to 7
\end{align*}
So by degree we know that $3x^4z$ is smaller than our other terms.
Within the terms that have degree 7, we look at which has the highest 
power of $x$. This gives us the following ordering of terms:
\[
    3 x^4 z         <
    -8 x y^3 z^3    <
    7 x^2 y^2 z^3   <
    -2x^3y^4
.\]
So, our leading term is $-2x^3y^4$, our leading coefficient is $-2$, and 
our leading power product is $x^3y^4$.\\\\
With respect to lex ordering, we look first at our powers of $x$. This 
immediately gives us an ordering of our terms,
\[
    -8 x y^3 z^3    <
    7 x^2 y^2 z^3   <
    -2x^3y^4        <
    3 x^4 z
.\]
Thus our leading term is $3x^4$, our leading coefficient is $3$, and our leading 
power product is $x^4z$.\\\\
With respect to degrevlex, we take the degree order from degrev, 
but reverse the lexigraphical aspec. This gives us an ordering,
\[
    3 x^4 z         <
    -2x^3y^4        <
    7 x^2 y^2 z^3   <
    -8 x y^3 z^3
.\]
Therefore our leading term is $-8xy^3z^3$, our leading coefficient is $-8$,
and our leading power product $xy^3z^3$.\\\\
\fbox{$x < y < z$}\\
With respect to deglex, we take the same degree order as above.
Then we first look at the powers of $z$. So then $3x^4 < -2x^3y^4$, and 
both are smaller than the other two. Since the other two both have $z^3$,
we then look at powers of $y$. This gives us the ordering 
\[
    3x^4z < -2x^3y^4 < 7x^2y^2z^3 < -8xy^3z^3  
.\]
So we have 
\[
    lt(f) = -8xy^3z^3, \ \ \ \ lc(f) = -8, \ \ \ \ lp(f) = xy^3z^3
.\]
With respect to lexigraphical ordering, we look at the powers of $z$.
To order the two terms with $z^3$, we look at powers of $y$. This gives 
us 
\[
    -2x^3y^4 < 3x^4z < 7x^2y^2z^3 < -8xy^3z^3 
.\]
Thus,
\[
    lt(f) = -8xy^3z^3, \ \ \ \ lc(f) = -8, \ \ \ \ lp(f) = xy^3z^3   
.\]
With respect to degrevlex, we keep the usual degree order. Then we 
reverse the remaining terms, giving us
\[
    3x^4z < -8xy^3z^3 < 7x^2y^2z^3 < -2x^3y^4
.\]
Therefore 
\[
    lt(f) = -2x^3y^4, \ \ \ \ lc(f) = -2, \ \ \ \ lp(f) = x^3y^4
.\]
\section*{Problem 1.5.5}
Let $f,g,h,r,s \in R = K[x_1,\cdots,x_n]$, and let $F$ be a collection of non-zero 
polynomials in $R$. Disprove the following (by providing counterexamples).
\begin{enumerate}[(a)]
    \item If $f \xrightarrow{F}_+r$ and $g \xrightarrow{F}_+s$, then $f+g\xrightarrow{F}_+r+s$ \\
    \item If $f\xrightarrow{F}_+r$ and $g\xrightarrow{F}_+s$, then $fg\xrightarrow{F}_+rs$.
\end{enumerate}
\par\noindent\rule{\textwidth}{0.4pt}
\subsection*{Solution}
Let $x > y > z$ with degree lexigraphical ordering.
\subsubsection*{(a)}
Let 
\begin{align*}
    F &= \{x+z, y\}\\
    f &= x + y \\
    g &= -y + z
.\end{align*}
Then $f \xrightarrow{F}_+ x, g \xrightarrow{F}_+ z$.
So we say that $r + s = x + z$.
Then $f + g = x + y - y + z = x + z$. The only reduction that 
exists is $f + g \xrightarrow{F}_+ 0$. Clearly $x + z \neq 0$.
\subsubsection*{(b)}
Let
\begin{align*}
    F &= \{x,y\} \\
    f &= x + y \\
    g &= x - y
\end{align*}
By choosing reductions carefully, we say that $f \xrightarrow{y} x$ and 
$g \xrightarrow{x} -y$. So then $rs = -xy$. Then $fg = (x + y)(x - y) = x^2 - y^2$.
Clearly we do not have a way to reduce this to something with an $xy$ term.
\end{document}
